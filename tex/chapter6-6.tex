
\section{Problema 6}

Propunem în continuare o problemă care s-a dat la Olimpiada Națională de
Informatică, Suceava 1996, la clasa a XII-a. Menționăm că un singur concurent
a reușit să o ducă la bun sfârșit în timpul concursului. Problema se numește
„Cartierul Enicbo”.

{\bf ENUNȚ}: În orașul Acopan s-a construit un nou cartier. Noul cartier are
patru bulevarde paralele și un număr de $N$ străzi perpendiculare pe
ele. Există deci în total $4N$ intersecții. Furgoneta oficiului poștal trebuie
să distribuie poșta în fiecare zi; în acest scop, furgoneta pleacă de la
oficiul poștal aflat la intersecția bulevardului 1 cu strada 1 și, urmând
rețeaua stradală, trece exact o dată prin fiecare intersecție astfel încât să
încheie traseul în punctul de plecare.

Conducerea oficiului poștal roagă participanții la olimpiadă să o ajute să
afle în câte moduri distincte se poate stabili traseul furgonetei.

{\bf Intrarea}: Programul va citi de la tastatură valoarea lui $N$ ($2 \leq N
\leq 200$).

{\bf Ieșirea}: Pe ecran se va afișa soluția (numărul de trasee distincte
pentru valoarea respectivă a lui $N$).

{\bf Exemplu}: Pentru $N=3$ există 4 soluții (se citește de la tastatură
numărul 3 și se afișează pe ecran numărul 4). Iată soluțiile efective:

\newcommand\enicboContour[5]{
  \draw[grid] (0,0) grid (#1,#2);

  \enicboContourNoGrid{#3}{#4}{#5}
}

\newcommand\enicboContourNoGrid[3]{
  \def\lx{#1} \def\ly{#2}
  \foreach \x[remember=\x as \lx]/\y[remember=\y as \ly] in {#3} {
    \draw[->-] (\lx,\ly) -- (\x,\y);
  }
}

\newcommand\enicboSquares[1]{
  \foreach \x/\y in {#1} {
    \node[times] at ($(\x+0.5,\y+0.5)$) {$\times$};
  }
}

\newcommand\enicboDots[2]{
  \foreach \x in {#1} {
    \foreach \y in {#2} {
      \node at (\x,\y) {$\cdots$};
    }
  }
}

\tikzset{
  grid/.style = {gray!50,very thin},
  ->-/.style={
    thick,
    decoration={
      markings,
      mark=at position .65 with {\arrow{>},
      },
    },
    postaction={decorate},
  },
  times/.style = {
    font=\Large,
  }
}
\centeredTikzFigure[
  mat/.style = {
    matrix of nodes,
    ampersand replacement=\&,
    column sep=2em,
    row sep=2em,
  },
]{
  \matrix[mat] {
    \enicboContour{2}{3}{0}{0}{
      0/1, 0/2, 0/3, 1/3, 2/3, 2/2, 1/2, 1/1, 2/1, 2/0, 1/0, 0/0
    }
    \&
    \enicboContour{2}{3}{0}{0}{
      1/0, 2/0, 2/1, 1/1, 1/2, 2/2, 2/3, 1/3, 0/3, 0/2, 0/1, 0/0
    }
    \\
    \enicboContour{2}{3}{0}{0}{
      0/1, 1/1, 1/2, 0/2, 0/3, 1/3, 2/3, 2/2, 2/1, 2/0, 1/0, 0/0
    }
    \&
    \enicboContour{2}{3}{0}{0}{
      1/0, 2/0, 2/1, 2/2, 2/3, 1/3, 0/3, 0/2, 1/2, 1/1, 0/1, 0/0
    }
    \\
  };
}

{\bf Timp de execuție}: 30 secunde pentru un text

{\bf Timp de implementare}: 1h 30 min.

{\bf Complexitate cerută}: $O(N^2)$

{\bf REZOLVARE}: Primul lucru care ne vine în gând este „se cere numărul de
cicluri hamiltoniene într-un graf, deci problema e exponențială”. Rezolvarea
backtracking nu e deloc greu de implementat, dar nu are nici o șansă să meargă
pentru valori mari ale lui $N$. Afirmația de mai sus este corectă, dar
incompletă; din această cauză concluzia este falsă. Se scapă din vedere faptul
că graful nu este oarecare, ci are un aspect foarte particular.

Și în această problemă vom încerca să utilizăm soluțiile locale (pentru valori
mici ale lui $N$) pentru aflarea soluției globale. Respectiv, vom rezolva
problema pentru $N=2$, apoi o vom extinde pentru $N=3, 4$ și așa mai
departe. Pentru început, însă, încercăm să simplificăm enunțul, reducând
problema la una echivalentă, dar mai simplă.

Să considerăm o posibilă soluție pentru $N=5$:

\centeredTikzFigure[]{
  \enicboContour{4}{3}{0}{0}{
    0/1, 0/2, 0/3, 1/3, 1/2, 1/1, 2/1, 2/2, 2/3, 3/3, 4/3, 4/2, 3/2, 3/1, 4/1,
    4/0, 3/0, 2/0, 1/0, 0/0
  }
  \enicboSquares{0/0, 0/1, 0/2, 1/0, 2/0, 2/1, 2/2, 3/0, 3/2}
}

În loc să lucrăm cu segmente în această rețea, vom lucra cu ochiuri. Furgoneta
parcurge un ciclu, deci închide în circuitul ei un număr de ochiuri. Am marcat
aceste ochiuri cu un „$\times$” în figura de mai sus. Așadar, oricărui drum al
furgonetei i se poate atașa o matrice cu 3 linii și $N-1$ coloane, în care
unele celule sunt bifate cu „$\times$”, iar altele nu. Să vedem în primul rând
care este corespondența între numărul de circuite hamiltoniene și numărul de
matrice de acest tip.

Se observă că pentru orice circuit există un altul căruia îi este atașată
aceeași matrice. Circuitul pereche este tocmai circuitul parcurs în sens
invers, care închide în interior aceleași ochiuri de rețea:

\centeredTikzFigure[]{
  \enicboContour{4}{3}{0}{0}{
    1/0, 2/0, 3/0, 4/0, 4/1, 3/1, 3/2, 4/2, 4/3, 3/3, 2/3, 2/2, 2/1, 1/1, 1/2,
    1/3, 0/3, 0/2, 0/1, 0/0
  }
  \enicboSquares{0/0, 0/1, 0/2, 1/0, 2/0, 2/1, 2/2, 3/0, 3/2}
}

Acest lucru se întâmplă deoarece transformarea graf-matrice ignoră sensul de
parcurgere a circuitului hamiltonian. De aici rezultă că pentru a calcula
numărul de circuite hamiltoniene trebuie să calculăm numărul de matrice și
să-l înmulțim cu 2.

În continuare, să analizăm câteva proprietăți ale matricelor în discuție.

\begin{property}
  Elementele bifate cu „$\times$” în matrice formează o singură figură conexă.
\end{property}

\begin{proof}
  Dacă figura nu ar fi conexă, adică dacă ar exista mai multe figuri, ele nu
  ar putea fi înconjurate de furgonetă într-un singur drum. De remarcat că
  {\bf toate} pătratele înconjurate de furgonetă trebuie bifate cu $\times$,
  deci furgoneta nu poate înconjura pătrate nebifate.
\end{proof}

De aceea, traseul de mai jos (care prezintă o porțiune oarecare de circuit)
este imposibil.

\centeredTikzFigure[]{
  \enicboContour{3}{3}{}{}{}
  \enicboSquares{0/1, 0/2, 2/1}
  \enicboDots{-1, 4}{0.5, 1.5, 2.5}
}

Conexitatea se referă numai la vecinătatea pe latură, nu și pe colț. Spre
exemplu, figura de mai jos este incorectă, deoarece, pentru a o înconjura,
furgoneta trebuie să treacă de două ori prin punctul încercuit:

\centeredTikzFigure[]{
  \enicboContour{3}{3}{3}{1}{2/1, 1/1, 1/2, 2/2, 3/2}
  \enicboContourNoGrid{0}{3}{1/3, 1/2, 0/2}
  \enicboSquares{0/2, 1/1, 2/1}
  \enicboDots{-1, 4}{0.5, 1.5, 2.5}

  \draw (1,2) circle (0.2);
}

\begin{property}
  Elementele bifate cu „$\times$” formează o structură aciclică.
\end{property}

\begin{proof}
  Dacă structura ar fi ciclică, ar rezulta că elementele bifate cu „$\times$”
  închid între ele elemente nebifate, pe care furgoneta însă nu poate să le
  ocolească.
\end{proof}

Iată un exemplu de ciclicitate:

\centeredTikzFigure[]{
  \enicboContour{3}{3}{}{}{}
  \enicboSquares{0/0, 0/1, 0/2, 1/0, 1/2, 2/0, 2/1, 2/2}
  \enicboDots{-1, 4}{0.5, 1.5, 2.5}
  \node[times] at (1.5,1.5) {?};
}

Practic, situația de mai sus obligă furgoneta să facă două drumuri: unul pe
exterior și unul în jurul ochiului marcat cu „?”.

\begin{property}
  Nici un nod interior al rețelei nu poate avea toate cele patru ochiuri
  vecine marcate cu „$\times$”.
\end{property}

\begin{proof}
  Dacă ar exista un asemenea nod, el nu ar putea fi parcurs de furgonetă, deci
  ciclul nu ar mai fi hamiltonian. Este cazul nodului încercuit în figura
  următoare:
\end{proof}

\centeredTikzFigure[]{
  \enicboContour{3}{3}{0}{3}{1/3, 1/2, 2/2, 3/2, 3/1, 2/1, 2/0, 1/0, 0/0}
  \enicboSquares{0/0, 0/1, 0/2, 1/0, 1/1, 2/1}
  \enicboDots{-1, 4}{0.5, 1.5, 2.5}

  \draw (1,1) circle (0.2);
}

\begin{property}
  Structura elementelor bifate cu „$\times$” în cadrul matricei este
  arborescentă.
\end{property}

\begin{proof}
  Rezultă imediat din punctele anterioare: figura este conexă și aciclică.
\end{proof}

\begin{property}
  Numărul de celule bifate este $P = 2N - 1$.
\end{property}

\begin{proof}
  Folosim inducția matematică. Să presupunem că structura noastră ar avea un
  singur pătrat bifat. Atunci structura ar avea patru laturi „la
  vedere”. Traseul furgonetei care ocolește structura ar avea patru laturi. O
  structură de două pătrate (desigur lipite) va avea șase laturi la vedere:

  \centeredTikzFigure[]{
    \enicboContour{2}{1}{0}{0}{0/1, 1/1, 2/1, 2/0, 1/0, 0/0}
    \enicboSquares{0/0, 1/0}
    \node at (0.5, 1.5) {1};
    \node at (1.5, 1.5) {2};
    \node at (2.5, 0.5) {3};
    \node at (1.5, -0.5) {4};
    \node at (0.5, -0.5) {5};
    \node at (-0.5, 0.5) {6};
  }

  Să ne imaginăm acum că orice structură cu $k$ pătrate are $S_k$ laturi la
  vedere. Trebuie să demonstrăm că toate structurile cu $k+1$ pătrate au
  același număr de laturi la vedere și să aflăm efectiv acest număr,
  $S_{k+1}$. Cel de-al $k+1$-lea pătrat trebuie alipit la structura deja
  existentă în așa fel încât să nu se închidă nici un ciclu. El se va lipi
  deci de o latură la vedere a unui pătrat din structură. În acest fel, va
  dispărea o latură la vedere, dar vor apărea trei în loc. Numărul de laturi
  la vedere va crește prin urmare cu 2. Această cifră nu depinde de locul în
  care este alipit al $k+1$-lea pătrat, nici de forma structurii deja
  existente, deci am demonstrat că toate structurile arborescente cu $k$
  pătrate au același număr de laturi la vedere. Pentru a afla efectiv acest
  număr, pornim de la relațiile recurente stabilite prin inducție și eliminăm
  recurența:

  \begin{equation}
    \begin{split}
      S_{k + 1} & = S_k + 2 \\
      S_1 & = 4
    \end{split}
    \Biggr\}
    \implies S_k = 2k + 2
  \end{equation}

  Deoarece numărul total de noduri al rețelei este de $4N$, rezultă că
  structura noastră trebuie să aibă $4N$ laturi la vedere. Notând cu $P$
  numărul de pătrate bifate din matrice și rezolvând ecuația de mai jos,
  rezultă valoarea lui $P$:

  \begin{equation}
    2P + 2 = 4N \implies P = 2N - 1 = 2(N - 1) + 1
  \end{equation}

  Cum numărul de coloane al matricei este $N-1$, deducem că în medie pe
  fiecare coloană se vor afla două pătrate bifate, cu excepția uneia pe care
  se vor afla trei pătrate bifate.
\end{proof}

La nivel local, proprietatea este de asemenea respectată: numărul de pătrate
bifate din primele $k$ coloane ale matricei este $2k$, existând posibilitatea
să mai fie un pătrat suplimentar. De exemplu, în figura dată mai sus pentru
$N=5$, în primele două coloane se află patru elemente „$\times$”, deci o medie
de două pătrate pe fiecare coloană. În primele trei coloane există șapte
elemente „$\times$”, adică o medie de două pătrate pe coloană și un surplus de
un pătrat. Lăsăm ca temă cititorului să demonstreze că în primele $k$ coloane
există întotdeauna fie $2k$, fie $2k+1$ pătrate. Orice număr mai mare duce la
ciclicitatea figurii, orice număr mai mic duce la neconexitatea ei.

Pe fiecare coloană există opt combinații posibile de elemente bifate și
nebifate, pe care le vom codifica cu numere de la 0 la 7, conform unei
numărători binare:

\centeredTikzFigure[
  mat/.style = {
    matrix of nodes,
    ampersand replacement=\&,
    nodes in empty cells,
    column sep=1em,
    row 1/.style=times,
    row 2/.style=times,
    row 3/.style=times,
    row 4/.style = {
      font=\normalsize,
      nodes = {draw=none, yshift=-1.5em},
    }
  },
  times/.style = {
    nodes={draw},
    font=\Large,
    anchor=center,
    minimum height=2em,
    minimum width=2em,
  }
]{
  \matrix[mat] {
    \& \& \& \& $\times$ \& $\times$ \& $\times$ \& $\times$ \\
    \& \& $\times$ \& $\times$ \& \& \& $\times$ \& $\times$ \\
    \& $\times$ \& \& $\times$ \& \& $\times$ \& \& $\times$ \\
    0 \& 1 \& 2 \& 3 \& 4 \& 5 \& 6 \& 7 \\
  };
}

Să vedem acum care dintre aceste combinații rămân valabile. O coloană de tipul
0 nu poate exista, deoarece ea ar „rupe” matricea în două bucăți separate,
deci proprietatea de conexitate nu ar fi respectată.

Dacă pe coloana $k$ se află o combinație de tipul 3, ce s-ar putea afla pe
coloana $k+1$?

\centeredTikzFigure[]{
  \enicboContour{2}{3}{}{}{}
  \enicboSquares{0/0, 0/1, 1/1, 1/2}
  \enicboDots{-1, 3}{0.5, 1.5, 2.5}

  \node at (0.5, -0.5) {$k$};
  \node at (1.5, -0.5) {$k + 1$};

  \node at (2, 4) (a) {$A$};
  \node at (2.8, 0) (b) {$B$};

  \draw (a.west) -- (1,3);
  \draw (b.west) -- (1,1);
}

Pentru ca punctul $A$ să se afle pe traseul furgonetei, este obligatoriu să
bifăm pătratul de sub el. Pentru a menține conexitatea figurii apărute,
trebuie bifat și pătratul din centrul coloanei $k+1$. Cea de-a treia celulă a
coloanei $k+1$ nu poate fi bifată, deoarece punctul $B$ ar fi înconjurat din
patru părți de celule bifate, lucru care s-a stabilit că este imposibil. Se
vede că singura combinație posibilă pentru coloana $k+1$ este 6. Ce combinație
putem pune pe coloana $k+2$? Printr-un raționament analog, deducem că numai
combinația 3:

\centeredTikzFigure[
  label/.style={font=\small},
]{
  \enicboContour{4}{3}{}{}{}
  \enicboSquares{0/0, 0/1, 1/1, 1/2, 2/0, 2/1, 3/1, 3/2}
  \enicboDots{-1, 5}{0.5, 1.5, 2.5}

  \node[label] at (0.5, -0.5) {$k$};
  \node[label] at (1.5, -0.5) {$k + 1$};
  \node[label] at (2.5, -0.5) {$k + 2$};
  \node[label] at (3.5, -0.5) {$k + 3$};
}

Iată că, pentru a putea respecta condițiile de corectitudine a matricei, am fi
nevoiți să continuăm la nesfârșit cu coloane cu combinațiile 3-6-3-6 etc. Deci
niciuna din aceste combinații nu poate apărea în matrice.

În continuare, vom defini mai multe șiruri de forma $S_k(i,t)$, unde:

\begin{itemize}

\item $k$ este numărul unei coloane;

\item $i$ este un număr de combinație (respectiv 1, 2, 4, 5 sau 7);

\item $t$ este un număr care poate avea valoarea 0 sau 1.

\end{itemize}

$S_k(i,t)$ semnifică „numărul de matrice (corecte) cu $k$ coloane astfel încât
pe coloana cu numărul $k$ să se afle combinația $i$, iar surplusul de pătrate
bifate peste media de două pătrate pe fiecare coloană să fie $t$”. De exemplu,
$S_7(5,1)$ reprezintă numărul de matrice corecte (care respectă regulile de
construcție) cu 7 coloane, astfel încât pe ultima coloană să se afle
combinația 5 și să existe un surplus de 1 pătrat (adică numărul total de
pătrate să fie $2 \times 7 + 1 = 15$).

Facem observația că pe a $N-1$-a coloană se pot afla doar combinațiile 5 sau 7
(pentru a acoperi colțurile de NE și SE ale grafului), iar surplusul de
pătrate trebuie să fie 1 (deoarece în $N-1$ coloane trebuie să se afle
$P=2(N-1)+1$ pătrate bifate). Deci scopul nostru este să calculăm suma
$S_{N-1}(5,1) + S_{N-1}(7,1)$ și să o înmulțim cu 2 ca să aflăm numărul de
cicluri hamiltoniene.

De asemenea, remarcăm că șirurile $S_k(1,1)$, $S_k(2,1)$ și $S_k(4,1)$ nu sunt
definite. Aceasta deoarece combinațiile 1, 2 și 4 au un singur pătrat bifat pe
coloană, adică mai puțin decât media de două pătrate. Este imposibil ca după
adăugarea unei asemenea coloane să mai existe un surplus. (deoarece ar rezulta
că în primele $k$-1 coloane exista un surplus de două pătrate). La polul opus,
șirul $S_k(7,0)$ nu este definit, deoarece combinația 7 are toată coloana
bifată, adică peste medie, deci nu se poate să nu apară un surplus de pătrate
bifate.

Mai trebuie stabilite formulele de recurență între șirurile $S_k(1,0)$,
$S_k(2,0)$, $S_k(4,0)$, $S_k(5,0)$, $S_k(5,1)$ și $S_k(7,1)$. Termenii
inițiali ai recurenței sunt:

\begin{itemize}

\item $S_1(1,0)=0$ deoarece matricea nu poate începe cu combinația 1

\item $S_1(2,0)=0$ deoarece matricea nu poate începe cu combinația 2

\item $S_1(4,0)=0$ deoarece matricea nu poate începe cu combinația 4

\item $S_1(5,0)=1$ deoarece există o singură matrice de o coloană cu
  combinația 5

\item $S_1(5,1)=0$ deoarece combinația 5 are două pătrate, deci nu există
  surplus

\item $S_1(7,1)=1$ deoarece există o singură matrice de o coloană cu
  combinația 7

\end{itemize}

Pentru a stabili relația de recurență pentru șirul $S_k(1,0)$, ne întrebăm:
cărei coloane îi poate urma coloana $k$ de tip 1 astfel încât să nu mai existe
surplus? Dacă observăm că pe coloana $k$ avem un singur element bifat (deci
sub medie), rezultă că pe coloana $k-1$ exista un surplus de un pătrat. Deci
coloana $k-1$ putea fi de tipul 5 sau 7, acestea fiind singurele tipuri de
coloană după care poate exista un surplus. Rezultă formula:

\begin{equation}
  S_k(1,0)=S_{k-1}(5,1)+S_{k-1}(7,1)
\end{equation}

Printr-o simetrie perfectă se calculează aceeași formulă și pentru șirul
$S_k(4,0)$:

\begin{equation}
  S_k(4,0)=S_{k-1}(5,1)+S_{k-1}(7,1)
\end{equation}

La șirul $S_k(2,0)$, mai trebuie făcută observația că o coloană de tip 2 nu
poate urma unei coloane de tip 5, deoarece se strică conexitatea
figurii. Rezultă:

\begin{equation}
  S_k(2,0)=S_{k-1}(7,1)
\end{equation}

Șirul $S_k(5,0)$ provine din adăugarea unei coloane de tipul 5 după o coloană
de tipul 1, 4 sau 5. Coloana $k-1$ nu poate fi de tipul 2 deoarece figura
rezultată nu este conexă, nici de tipul 7 deoarece ar rezulta că în primele
$k-2$ coloane media de celule bifate este mai mică decât 2.

\begin{equation}
  S_k(5,0)=S_{k-1}(1,0)+S_{k-1}(4,0)+S_{k-1}(5,0)
\end{equation}

Șirul $S_k(5,1)$ provine din adăugarea unei coloane de tipul 5 după o coloană
de tipul 5 sau 7, deoarece tipul de coloană 5 are două pătrate bifate, deci
conservă surplusul:

\begin{equation}
  S_k(5,1)=S_{k-1}(5,1)+S_{k-1}(7,1)
\end{equation}

În sfârșit, o coloană de tip 7 poate urma oricărui tip de coloană pentru care
surplusul este 0, adică:

\begin{equation}
  S_k(7,1)=S_{k-1}(1,0)+S_{k-1}(2,0)+S_{k-1}(4,0)+S_{k-1}(5,0)
\end{equation}

Acestea sunt formulele de recurență. Rezultatul care trebuie afișat pe ecran
este $2[S_{N-1}(5,1)+S_{N-1}(7,1)]$, deoarece după $N-1$ coloane surplusul
trebuie să fie 1, iar colțurile matricei trebuie să fie bifate. Se observă că
$S_k(1,0)=S_k(4,0)=S_k(5,1)$. Practic, problema se reduce la trei
șiruri. Notăm:

\begin{equation}
  \begin{split}
    A_k & = S_k(5,0) \\
    B_k & = S_k(1,0) = S_k(4,0) = S_k(5,1) \\
    C_k & = S_k(7,1) \implies S_k(2,0)=C_{k-1}
  \end{split}
\end{equation}

De aici rezultă grupul de relații:

\begin{equation}
  \begin{cases}
    A_k & = A_{k - 1} + 2B_{k - 1} \\
    B_k & = B_{k - 1} + C_{k - 1} \\
    C_k & = A_{k - 1} + 2B_{k - 1} + C_{k - 2} = A_k + C_{k - 2}
  \end{cases}
\end{equation}

și

\begin{equation}
  \begin{cases}
    A_1 & = 1 \\
    B_1 & = 0 \\
    C_1 & = 1
  \end{cases}
\end{equation}

Noi avem nevoie de valoarea 

\begin{equation}
  2(B_{N-1} + C_{N-1}) = 2B_N
\end{equation}

Programul de mai jos nu face decât să implementeze calculul acestor șiruri
recurente. Trebuie avut grijă însă cu reprezentarea internă a numerelor,
deoarece pentru $N=200$ valorile ajung la 81 de cifre. Este deci necesară
reprezentarea numerelor ca șiruri de cifre.

\begin{lstlisting}[language=C]
#include <stdio.h>
#include <mem.h>

typedef int Huge[85];
Huge A,B,C,C2,HTemp;
int N,k;

void Atrib(Huge H, int V)
/* H <- V */
{
  memset(H,0,sizeof(Huge));
  H[0]=1;
  H[1]=V;
}

void Add(Huge A, Huge B)
/* A <- A+B */
{ int i,T=0;

  if (B[0]>A[0])
    { for (i=A[0]+1;i<=B[0];) A[i++]=0;
      A[0]=B[0];
    }
    else for (i=B[0]+1;i<=A[0];) B[i++]=0;
  for (i=1;i<=A[0];i++)
    { A[i]+=B[i]+T;
      T=A[i]/10;
      A[i]%=10;
    }
  if (T) A[++A[0]]=T;
}

void WriteHuge(Huge H)
{ int i;

  for (i=H[0];i;printf("%d",H[i--]));
  printf("\n");
}

void main(void)
{
  printf("N=");scanf("%d",&N);
  Atrib(A,1);
  Atrib(B,0);
  Atrib(C,1);
  Atrib(C2,0);
  for (k=2;k<=N;k++)
    { memmove(HTemp,C,sizeof(Huge));
      Add(A,B);Add(A,B);  /* A(k) = A(k-1) + 2*B(k-1) */
      Add(B,C);           /* B(k) = B(k-1) + C(k-1)   */
      memmove(C,A,sizeof(Huge));
      Add(C,C2);          /* C(k) = A(k) + C(k-2)     */
      memmove(C2,HTemp,sizeof(Huge)); /* noul C(K-2) */
    }
  Add(B,B);               /* Rezultatul este 2*B(n)   */
  WriteHuge(B);
}
\end{lstlisting}
