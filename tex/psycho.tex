\documentclass[12pt, twoside, a4paper]{book}

\usepackage[table]{xcolor}
\usepackage{algorithm}
\usepackage{algorithmic}
\usepackage{amssymb}
\usepackage{amsthm}
\usepackage{attrib}
\usepackage[romanian]{babel}
\usepackage{colortbl}
\usepackage{courier}
\usepackage{dashrule}
\usepackage{enumitem}
\usepackage{etoolbox}
\usepackage{float}
\usepackage{fourier-orns}
\usepackage[margin=1in]{geometry}
\usepackage[unicode]{hyperref}
\usepackage[utf8]{inputenc}
\usepackage{listings}
\usepackage{mathtools}
\usepackage{multirow}
\usepackage[markcase=noupper]{scrlayer-scrpage}
\usepackage{textcomp}
\usepackage{tikz}

\usetikzlibrary{
  arrows,
  arrows.meta,
  backgrounds,
  calc,
  decorations.markings,
  decorations.pathreplacing,
  matrix,
  positioning,
  shapes.multipart
}

%\let\MakeUppercase\relax

\title{Psihologia concursurilor de informatică}
\author{Cătălin Frâncu}
\date{}

\linespread{1.3}
\raggedbottom

\definecolor{light-gray}{gray}{0.95}

\lstset{
  aboveskip=\bigskipamount,
  backgroundcolor=\color{light-gray},
  basicstyle=\footnotesize\ttfamily,
  literate={'}{\textquotesingle}1, % otherwise apostrophes get converted
  numbers=left,
  numberstyle=\scriptsize\color{gray},
}

\tikzset{>={Triangle[length=3mm,width=2mm]}}

\newcommand{\centeredTikzFigure}[2][]{
  \begin{figure}[H]
    \centering
    \begin{tikzpicture}[#1]
      #2
    \end{tikzpicture}
  \end{figure}
}

\makeatletter
\newcommand*{\bdiv}{%
  \nonscript\mskip-\medmuskip\mkern5mu%
  \mathbin{\operator@font div}\penalty900\mkern5mu%
  \nonscript\mskip-\medmuskip
}
\makeatother

\setlength{\parskip}{0.7em}
\def \algskip {0.7em} % skip before algorithmic environments

\newcommand\Chapter[2]{
  \chapter[#1: {\itshape#2}]{#1\\[0.5ex]\Large#2}
}

\hypersetup{
  colorlinks=true,
  linkcolor=black,
  urlcolor=blue,
}

\newtheorem{property}{Proprietatea}
\newtheorem{proposition}{Propoziția}

\sloppy

\renewcommand{\algorithmicdo}{\textbf{execută}}
\renewcommand{\algorithmicelse}{\textbf{altfel}}
\renewcommand{\algorithmicend}{\textbf{sfârșit}}
\renewcommand{\algorithmicfalse}{\textbf{fals}}
\renewcommand{\algorithmicforall}{\textbf{pentru toți}}
\renewcommand{\algorithmicfor}{\textbf{pentru}}
\renewcommand{\algorithmicif}{\textbf{dacă}}
\renewcommand{\algorithmicprint}{\textbf{tipărește}}
\renewcommand{\algorithmicrepeat}{\textbf{repetă}}
\renewcommand{\algorithmicrequire}{\textbf{Intrare:}}
\renewcommand{\algorithmicreturn}{\textbf{returnează}}
\renewcommand{\algorithmicthen}{\textbf{atunci}}
\renewcommand{\algorithmicto}{\textbf{la}}
\renewcommand{\algorithmictrue}{\textbf{adevărat}}
\renewcommand{\algorithmicuntil}{\textbf{până când}}
\renewcommand{\algorithmicwhile}{\textbf{cât timp}}

\begin{document}

  \pagenumbering{Alph}
  %\pagenumbering{gobble}

  \begin{titlepage}
    \maketitle
    \thispagestyle{empty} % prevent some warnings from hyperref
  \end{titlepage}

  \textbf{Notă}: Am publicat această carte în 1997 la Editura L\&S Infomat. În
  20 de ani s-au schimbat multe. Materia predată s-a schimbat
  mult. Concursurile s-au schimbat mult. Limbajele C și C++ au înlocuit
  aproape complet limbajul Pascal la concursuri. Mai ales, eu cel de acum nu
  mai sunt de acord cu unele principii, moduri de exprimare și stiluri de
  programare din această carte. Totuși, ocazional lumea îmi mai cere o copie a
  ei și mă simt jenat să le trimit un fișier Word (apropo de principii). De
  aceea, am publicat cartea online. Am păstrat tot conținutul original,
  reparând doar greșelile evidente de tipar și convertind formatul la \LaTeX
  și imaginile la TikZ/PGF (cuvântul „migălos” abia începe să descrie aceste
  conversii). — Cătălin, București, 2 martie 2018.

  \vspace*{\fill}

  © 1997, 2018 Cătălin Frâncu

  Această operă este pusă la dispoziție sub
  \href{https://creativecommons.org/licenses/by-sa/4.0/deed.ro}{Licența
    Creative Commons Atribuire - Distribuire în condiții identice 4.0
    Internațional}.

  This work is licensed under a
  \href{https://creativecommons.org/licenses/by-sa/4.0/}{Creative Commons
    Attribution-ShareAlike 4.0 International License}.


  \newpage
  \thispagestyle{empty}

  \begin{quote}
    Această carte îi este dedicată fratelui meu Cristi, căruia îi datorez o
    mare parte din cunoștințele mele în domeniul programării. Un gând bun
    pentru familia mea și pentru prietenii mei, care mi-au luat toate
    îndatoririle de pe umeri cât timp am scris cartea de față. Fără
    înțelegerea și răbdarea lor, nu mi-aș fi putut duce munca la bun sfârșit.
  \end{quote}

  \newpage
  \thispagestyle{empty}

  \chapter*{Cuvânt înainte}
\markboth{Cuvânt înainte}{} % for page headers
\addcontentsline{toc}{chapter}{Cuvânt înainte}
\pagenumbering{arabic}

În ultimii ani s-au tipărit la noi în țară foarte multe culegeri de teorie și
probleme de programare. Fiecare din ele acoperă diverse domenii ale
informaticii. Unele își propun să inițieze cititorul în tainele diverselor
limbaje de programare, altele pun accentul mai cu seamă pe tehnicile de
programare și structurile de date folosite în rezolvarea problemelor. În
general, cele din prima categorie conțin exemple cu caracter didactic și
exerciții cu un grad nu foarte sporit de dificultate, iar celelalte
demonstrează matematic fiecare algoritm prezentat, însă neglijează partea de
implementare, considerând scrierea codului drept un ultim pas lipsit de orice
dificultate.

Desigur, fiecare din aceste cărți își are rostul ei în formarea unui elev bine
pregătit în domeniul informaticii. De altfel, citirea volumului de față
presupune cunoașterea temeinică a conținutului ambelor tipuri de materiale
enumerate mai sus. Totuși, pornind de la observația că scrierea unui program
impune atât conceperea algoritmului și demonstrarea corectitudinii, cât și
implementarea lui, ambele etape fiind complexe și nu lipsite de obstacole, am
considerat necesară scrierea unui nou volum care să trateze simultan aceste
două aspecte ale programării.

În afară de aceasta, după cum și titlul lucrării o spune, cartea se adresează
pasionaților de informatică și celor care au de gând să participe la
concursurile și olimpiadele de informatică. Concursul include apariția unui
factor suplimentar care răstoarnă multe din obișnuințele programării „la
domiciliu”: timpul. Autorul a avut la dispoziție patru ani ca să descopere pe
propria piele importanța acestui factor. Și, mai mult decât durata de timp în
sine a concursului - care la urma urmei este aceeași pentru toți concurenții -
contează capacitatea fiecăruia de a gestiona bine acest timp.

Dacă în fața calculatorului de acasă, cu o sticlă de Coca-Cola alături și
casetofonul mergând, este într-adevăr un lucru lăudabil să justificăm
matematic fiecare pas al algoritmului, să nu ne lăsăm înșelați de intuiție și
să scriem programul fără să ne grăbim, alocându-ne o jumătate din timp numai
pentru depanarea lui, în schimb în timp de concurs lucrurile stau tocmai pe
dos. De demonstrații riguroase nu se mai ocupă nimeni, intuiția este la mare
preț și de nenumărate ori este criteriul care aduce victoria, iar timpul
de-abia dacă este suficient pentru implementarea programului, despre depanare
nemaiîncăpând discuții. În multe cazuri, cele două etape ale programării -
conceperea și implementarea algoritmului - încep să se bată cap în cap. Uneori
avem la dispoziție un algoritm foarte puternic, dar nu știm cum s-ar putea
implementa, alteori acest algoritm nu face față volumului maxim de date de
intrare, iar alteori ne dăm seama că am putea foarte ușor să scriem un
program, dar nu suntem în stare să demonstrăm că el ar merge perfect. Foarte
des se renunță la implementarea algoritmilor de complexitate optimă, care sunt
alambicați și constituie adevărate focare de „bug”-uri, preferându-se un
algoritm mai lent dar care să se poată implementa rapid și fără dureri de
cap. Mulți olimpici pierd clasa a IX-a, poate chiar și pe a X-a descoperind
aceste lucruri. Cartea de față își propune să le mai ușureze drumul.

S-a presupus cunoscut limbajul de programare Pascal, cu toate instrucțiunile
și procedurile sale standard. În carte există multe surse în limbajul C
standard. Am preferat acest lucru, deși la concursuri se recomandă programarea
în Pascal, pentru că am sperat că un elev familiarizat cu limbajul Pascal va
citi fără dificultate o sursă C și pentru că am dorit ca această carte să fie
și un exercițiu de C, al cărui număr de utilizatori la nivelul liceului este
destul de redus. Surse în Pascal există numai acolo unde se urmărește punerea
în evidență a unei anumite subtilități a limbajului.

Tehnicile de programare s-au presupus cunoscute în esență, astfel încât am
trecut direct la unele optimizări, la exemple de folosire a lor și la
compararea lor, respectiv la prezentarea unor criterii în funcție de care să
optăm pentru folosirea fiecăreia. De asemenea, am renunțat la definirea
termenilor de graf, arbore, vector, matrice, listă, stivă și a tuturor
celorlalte structuri de date de bază. Am considerat interesantă prezentarea pe
larg numai a {\it heap}-urilor și a tabelelor de dispersie ({\it hash}), care
sunt mai rar folosite și de aceea mai puțin cunoscute. În sfârșit, am presupus
cunoscută noțiunea de complexitate a unui algoritm, deoarece în toate
problemele se face calculul complexității.

Cartea prezintă interes și pentru problemele pe care le cuprinde. Ele nu sunt
banale (de fapt, majoritatea au avut onoarea de a da bătăi de cap
concurenților la olimpiade) și pot fi lucrate acasă de către elevi pentru
menținerea în formă. Tocmai de aceea, am urmărit ca, ori de câte ori am propus
o problemă spre rezolvare, enunțul să fie dat în aceeași formă pe care ar fi
avut-o la un concurs: clar, detaliat, cu specificarea formatului datelor de
intrare și ieșire și cu un exemplu sau două. Singura diferență este că, de
regulă, la concursuri și olimpiade se precizează numai timpul limită admis
pentru un test; în carte am considerat folositor să se specifice și o
complexitate optimă a algoritmului care rezolvă problema, deoarece timpul de
execuție variază în funcție de resursele calculatorului și de limbajul de
programare folosit, deci e un criteriu mai puțin semnificativ. Timpul destinat
implementării unei probleme este în general egal cu cel care s-a acordat la
concursul unde a fost propusă respectiva problemă.

În sfârșit, consider că programatorii care își propun să scrie aplicații de
mari dimensiuni ar avea destul de multe lucruri de învățat din acest volum,
deoarece am inclus și detalii privind structuri de date mai neobișnuite sau
gestionarea economică a memoriei.

Sper să nu închideți această carte cu sentimentul că mai bine n-ați fi
deschis-o.

\begin{flushright}
  Cătălin Frâncu
\end{flushright}

  \Chapter{Concursul de informatică}{- de la extaz la agonie -}

 % ommit subtitle from page headers
\markboth{Capitolul 1. Concursul de informatică}{}

\begin{quote}
  Cine dorește să-și rezolve treburile la vremea potrivită, să-și împartă
  cu atenție timpul. \\
  \attrib{Plaut}
\end{quote}

Experiența demonstrează că, oricât de mare ar fi bagajul de cunoștințe
acumulat de un elev, mai e nevoie de {\it ceva} pentru a-i asigura succesul la
olimpiada de informatică. Aceasta deoarece în timp de concurs lucrurile stau
cu totul altfel decât în fața calculatorului de acasă sau de la
școală. Reușita depinde, desigur, în cea mai mare măsură de puterea fiecăruia
de a pune în practică ceea ce a învățat acasă. Numai că în acest proces
intervin o serie de factori care țin de temperament, de experiența
individuală, de numărul de ore dormite în noaptea dinaintea concursului (care
în taberele naționale este îngrijorător de mic) și așa mai departe.

Cu riscul de a cădea în demagogie, trebuie să spunem că un concurs de
informatică presupune mult mai mult decât un simplu act de prezență la locul
desfășurării ostilităților. Capitolul de față încearcă să enunțe câteva
principii ale concursului, pe care autorul și le-a însușit în cei patru ani de
liceu, atât din experiența proprie, cât și învățând de la alții. Cititorul
este liber să respingă aceste sfaturi sau să le accepte, filtrându-le prin
prisma personalității sale și alegând ceea ce i se potrivește.

\section{Înainte de concurs}

Primul și cel mai de seamă lucru pe care trebuie să îl știți este că e
important și să participi, dar e și mai important să participi onorabil, iar
dacă se poate să și câștigi. Nu trebuie să porniți la drum cu îngâmfare;
modestia e bună, dar nu trebuie în nici un caz să ducă la neîncredere în
sine. Fiecare trebuie să știe clar de ce e în stare și, mai presus de toate,
să se gândească că la urma urmei nu dificultatea concursului contează, căci
concursul, greu sau ușor, este același pentru toți. Mult mai importantă este
valoarea individuală și nu în ultimul rând pregătirea psihologică.

Autorul a fost peste măsură de surprins să constate că mulți elevi merg la
concurs fără ceas și fără hârtie de scris. Aceasta este fără îndoială o
greșeală capitală. În timpul concursului trebuie ținută o evidență drastică a
timpului scurs și a celui rămas. E drept că în general supraveghetorii anunță
din când în când timpul care a trecut, dar e bine să nu vă bazați pe nimeni și
nimic altceva decât pe voi înșivă. Unii colegi spuneau „Ei, ce nevoie am de
ceas, oricum am ceasul calculatorului la îndemână”. Așa e, dar e extrem de
incomod să te oprești mereu la jumătatea unei idei, să deschizi o sesiune DOS
din cadrul limbajului de programare și să afli cât e ceasul.

În ceea ce privește hârtia de scris, ea este în mod sigur necesară. De fapt, o
parte importantă a rezolvării unei probleme este proiectarea matematică a
algoritmului, lucru care nu se poate face decât cu creionul pe hârtie. Pe
lângă aceasta, majoritatea problemelor operează cu vectori, matrice, arbori,
grafuri etc., iar exemplele pe care este testat programul realizat trebuie
neapărat verificate „de mână”. E de preferat să aveți hârtie de matematică;
este foarte folositoare pentru problemele de geometrie analitică, precum și
pentru reprezentarea matricelor. Cantitatea depinde de imaginația
fiecăruia. În unele cazuri speciale, autorului i s-a întâmplat să umple 7-8
coli A4.

\section{În timpul concursului}

Din fericire pentru unii și din nefericire pentru alții, majoritatea
examenelor îți cer să dovedești nu că ești bine pregătit, ci că ești mai bine
pregătit decât alții. Aceasta înseamnă că și la olimpiada de informatică se
aplică legea peștelui mai mare sau, cum i se mai spune, a
concurenței. Valoarea absolută a fiecăruia nu contează chiar în totalitate,
ceea ce constituie sarea și piperul concursului. Într-adevăr, ce farmec ar
avea să mergi la un concurs la care se știe încă dinainte cine este cel mai
bun ? Este destul de amuzant să observi cum fiecare speră să prindă „o zi
bună”, iar adversarii săi „o zi proastă”.

Este ușor să fii printre cei mai buni atunci când concursul este ușor. Mai
greu e să fii cel mai bun atunci când concursul este dur, pentru că atunci
intervine - inevitabil - dramul de noroc al fiecăruia. Niciodată însă nu se
poate invoca greutatea concursului drept o scuză pentru un eventual
eșec. Concursul este la fel de greu pentru toți. Se poate întâmpla, mai ales
dacă probele durează mai multe zile (3-4) ca nici unul din concurenți să nu
acumuleze mai mult de 70-80\% din punctajul maxim. Totuși, aceasta nu înseamnă
că ei nu sunt bine pregătiți; mai mult, unul dintre ei trebuie să fie
primul. Așadar, niciodată nu trebuie adoptată o strategie de genul „problema
asta e grea și n-am s-o pot rezolva perfect, așa că nu mă mai apuc deloc de
ea”. Nu trebuie să vă impacientați dacă vi se întâmplă să nu aveți o idee
genială de rezolvare a unei probleme. Nu vă cere nimeni să faceți perfect o
problemă, ci numai să prezentați o soluție care să acumuleze cât mai multe
puncte. Evident, prima variantă este întotdeauna preferabilă, dar nu
obligatorie.

De multe ori se întâmplă ca un elev să găsească o soluție cât de cât bună la o
problemă și, măcar că știe că nu va lua punctajul maxim, ci doar o parte, să
renunțe să caute o soluție mai eficientă, deoarece timpul pierdut astfel ar
aduce un câștig prea mic și ar putea fi folosit la rezolvarea altor
probleme. Desigur, dacă nu faci toate problemele perfect, nu mai poți fi sigur
de premiul I, pentru că altcineva poate să te întreacă. Dar pe de altă parte,
locul pe care te clasezi contează numai la etapa națională a olimpiadei sau la
concursurile internaționale. În rest, important e numai să te califici, adică
să intri în primele câteva locuri.

Feriți-vă ca de foc de criza de timp. E mare păcat să ratezi o problemă
întreagă pentru că n-ai avut timp să scrii procedura de afișare a
soluției. Rezervați-vă întotdeauna timpul pe care îl socotiți necesar pentru
implementare și depanare.

Niciodată, chiar dacă concursul este ușor, nu e bine să ieșiți din sala de
concurs înainte de expirarea timpului. Oricât ați fi de convinși că ați făcut
totul perfect, mai verificați-vă; veți avea de furcă cu remușcările dacă
descoperiți după aceea că ceva, totuși, nu a mers bine. Puteți face o mulțime
de lucruri dacă mai aveți timp (deși acest lucru se întâmplă rar). Iată o
serie de metode de a exploata timpul:

\begin{itemize}

\item Verificați-vă programul cu cât mai multe teste de mici dimensiuni. Să
  presupunem că programul vostru lucrează cu vectori de maxim 10000 de
  elemente. E o idee bună să îl rulați pentru vectori de unul sau două
  elemente. Nu se știe cum pot să apară erori.

\item Treceți la polul opus și creați-vă un test de dimensiune maximă, dar cu
  o structură particulară, pentru care este ușor de calculat rezultatul și de
  mână. De exemplu, vectori de 10000 elemente cu toate elementele egale, sau
  vectori de forma (1, 2, ..., 9999, 10000). Dacă nu puteți să editați un
  asemenea fișier de mână, copiind și multiplicând blocuri, puteți scrie un
  program care să-l genereze.

\item Dacă încă v-a mai rămas timp, creați-vă un program care să genereze
  teste aleatoare. Spre exemplu, un program care să citească un număr $N$ și
  să creeze un fișier {\tt INPUT.TXT} în care să scrie $N$ numere
  aleatoare. Într-o primă fază, puteți folosi aceste teste pentru a verifica
  dacă nu cumva la valori mai mari programul nu dă eroare, nu se blochează (la
  alocarea unor zone mari de memorie) sau nu depășește limita de timp, caz în
  care mai aveți de lucru.
  
\item Dacă tot nu vă dă nimeni afară din sală, puteți scrie un alt program
  auxiliar care, primind fișierul {\tt INPUT.TXT} și fișierul {\tt OUTPUT.TXT}
  produs de programul vostru, verifică dacă ieșirea este corectă. Aceasta
  deoarece, de obicei, este mult mai ușor de verificat o soluție decât de
  produs una (sau, cum spunea Murphy, „cunoașterea soluției unei probleme
  poate ajuta în multe cazuri la rezolvarea ei”). Folosind „generatorul” de
  teste și „verificatorul”, puteți testa programul mult mai bine. De altfel,
  la multe probleme chiar testele rulate de comisia de corectare sunt create
  tot aleator.

\item În caz că ați dat o soluție euristică la o problemă NP-completă, puteți
  implementa și un backtracking ca să vedeți cât de bune sunt rezultatele
  găsite euristic. Apoi, puteți începe să modificați funcția euristică pentru
  a o face cât mai performantă.

\end{itemize}

Și, ca să nu mai lungim vorba, iată o strategie care pare să dea rezultate:

{\bf A)} Imediat ce primiți problemele, citiți toate enunțurile și faceți-vă o
idee aproximativă despre gradul de dificultate al fiecărei probleme. Neapărat
verificați dacă se dau limite pentru datele de intrare (numărul maxim de
elemente ale unui vector și valoarea maximă a acestora, numărul maxim de
noduri dintr-un graf etc.) și pentru timpii de execuție pentru fiecare
test. Dacă nu se dau, întrebați imediat. Dimensiunea input-ului poate schimba
radical dificultatea problemei. Spre exemplu, pentru un vector cu $N=100$
elemente, un algoritm $O(N^{3})$ merge rezonabil, pe când pentru $N=10000$
același algoritm ar depăși cu mult cele câteva secunde care se acordă de
obicei. Fair-play-ul cere să puneți întrebările cu voce tare, pentru ca și
ceilalți să audă; de altfel, nu aveți nici un motiv să vă feriți de ceilalți
concurenți. Cei care sunt interesați de aceste întrebări le-ar pune oricum și
ei, iar cei care nu sunt interesați vor ignora oricum răspunsul.

Dacă există probleme care cer să se găsească un optim (maxim/minim) al unei
valori, întrebați dacă se acordă punctaje parțiale pentru soluții neoptime. Și
acest fapt poate schimba natura problemei. După aceasta,

\begin{minted}{pascal}
  Nr_probleme_nerezolvate := Nr_probleme_primite;

  while (Nr_probleme_nerezolvate>0) 
    and not ('Timpul a expirat, va rugam sa salvati') do
    begin
\end{minted}

{\bf B)} Faceți o împărțire a timpului pentru problemele rămase proporțional
cu punctajul fiecărei probleme. În general problemele au punctaje egale, dar
nu totdeauna. De exemplu, dacă o problemă e cotată cu 100 puncte, iar alta cu
50, veți aloca de două ori mai mult timp primei probleme, chiar dacă nu vi se
pare prea grea. Încercați să nu depășiți niciodată limitele de timp pe care
le-ați fixat. Dacă în schimb reușiți să economisiți timp față de cât v-ați
propus, cu atât mai bine, veți face o realocare a timpului și veți avea mai
mult pentru celelalte probleme.

{\bf C)} Apucați-vă de problema \underline{cea mai simplă}, chiar dacă e
punctată mai slab. Mai bine să duceți la bun sfârșit o problemă ușoară și să
luați un punctaj mai mic, decât să vă apucați de o problemă grea și să nu
terminați niciuna. Dacă toate problemele par grele, alegeți-o pe cea din
domeniul care vă este cel mai familiar, în care ați lucrat cel mai mult. Dacă
vă este indiferent și acest lucru, alegeți o problemă unde simțiți că aveți o
idee simplă de rezolvare. Dacă, în sfârșit, nu aveți nici o idee la nici o
problemă, apucați-vă de cea mai bine punctată.

{\bf D)} Citiți din nou enunțul, de data aceasta cu mare grijă. Întrebați
supraveghetorul pentru orice nelămurire. Dacă anumite lucruri nu sunt
specificate, iar profesorul nu vă dă nici un fel de informații suplimentare,
tratați problema în cazul cel mai general. Iată mai multe exemple frecvente în
care enunțul nu este limpede:

\begin{itemize}

\item Dacă nu se precizează cât de mari pot fi întregii dintr-un vector, nu
  lucrați pe {\tt Integer}, nici pe {\tt Word}, ci pe {\tt LongInt};

\item În problemele de geometrie analitică, e bine să presupuneți că punctele
  nu au coordonate întregi, ci reale;

\item De asemenea, pătratele și dreptunghiurile nu au neapărat laturi paralele
  cu axele, ci sunt așezate oricum în plan (aceasta poate într-adevăr să
  complice extrem de mult problema; nu vă doresc să vă izbiți de o asemenea
  neclaritate...);
  
\item Dacă fișierul de intrare conține string-uri, să nu presupuneți că ele au
  maxim 255 de caractere. Mai bine scrieți propria voastră procedură de citire
  a unui string, care să citească din fișier caracter cu caracter până la {\tt
    Eoln}, decât să aveți surprize. Dacă $S$ este o variabilă de tip {\tt
    String}, {\tt ReadLn(S)} ignoră tot restul rândului care depășește
  lungimea lui $S$.
  
\item Grafurile nu sunt neorientate, ci orientate. În principiu, enunțul nu
  are voie să fie neclar în această privință, dar au existat cazuri de
  neînțelegere.

\end{itemize}

{\bf E)} Începeți să vă gândiți la algoritmi cât mai buni, estimând în același
timp și cât v-ar lua ca să-i implementați. Faceți, pentru fiecare idee care vă
vine, calculul complexității. Nu trebuie neapărat să găsiți cel mai eficient
algoritm, ci numai unul suficient de bun. În general, trebuie ca, dintre toți
algoritmii care se încadrează în timpul de rulare, să-l alegeți pe cel care
este cel mai ușor de implementat. Iată un exemplu:

\begin{itemize}

\item Să presupunem că timpul de testare este de 5 secunde, lucrați pe un
  486DX4, algoritmul vostru are complexitatea $O(N^3)$, iar $N$ este maxim
  100. Un 486 face câteva milioane de operații elementare pe secundă, să zicem
  4.000.000. Aceasta înseamnă ceva mai puține operații mai costisitoare
  (atribuiri, comparații etc.) pe secundă. Să ne oprim deci la cifra de
  1.000.000. Programul vostru are timp de rulare cubic, iar $N^3$ este maxim
  1.000.000. De aici deducem că programul ar trebui să se încadreze într-o
  secundă. Calculul nostru este grosier, dar luând și o marjă de eroare
  arhisuficientă, rezultă că programul trebuie să meargă cu ușurință în 5
  secunde, deci algoritmul este acceptabil.

\end{itemize}

{\bf F)} Dacă algoritmul găsit este greu de implementat, mai căutați un altul
o vreme. Trebuie însă ca timpul petrecut pentru găsirea unui nou algoritm plus
timpul necesar pentru scrierea programului să nu depășească timpul necesar
pentru implementarea primului algoritm, altfel nu câștigați nimic. Deci nu
exagerați cu căutările și nu încercați să reduceți dincolo de limita
imposibilului complexitatea algoritmului. Mai ales, nu uitați că programul nu
poate avea o complexitate mai mică decât dimensiunea input-ului sau a
output-ului. De exemplu, dacă programul citește sau scrie matrice de
dimensiune $N \times N$, nu are sens să vă bateți capul ca să găsiți un
algoritm mai bun decât $O(N^2)$.

{\bf G)} Dintre toate ideile de implementare găsite (care se încadrează fără
probleme în timp), o veți alege pe cea mai scurtă ca lungime de cod. De
exemplu:

\begin{itemize}

\item Dacă $N \leq 1000$ și dispuneți de doi algoritmi, unul pe care îl
  estimați cam la 200 de linii de program, de complexitate $O(N \log N)$ și
  unul de 100 de linii de complexitate $O(N^2)$, cel de-al doilea este evident
  preferabil, pentru că nu pierdeți nimic din punctaj, sau cel mult pierdeți
  un test prin cine știe ce întâmplare, în schimb câștigați timp prețios pe
  care îl puteți folosi pentru alte probleme. Bineînțeles, primul program este
  mai eficient, dar în condiții de concurs el este \underline{prea}
  eficient. Este o mândrie să faceți o problemă perfect chiar dacă ratați o
  alta, dar este un câștig și mai mare să faceți amândouă problemele suficient
  de bine.

\end{itemize}

{\bf H)} În general, pentru orice problemă există cel puțin o soluție, fie și
una slabă. Sunt numeroase cazurile când nici nu vă vine altă idee de rezolvare
decât cea slabă. De regulă, când nu aveți în minte decât o rezolvare
neeficientă a problemei, care știți că nu o să treacă toate testele (un
backtracking, sau un $O(N^5)$, $O(N^6)$ etc.), e bine să încercați următorul
lucru:

\begin{itemize}

\item Să presupunem că v-a mai rămas o oră pentru rezolvarea acestei
  probleme. Calculați cam cât timp v-ar trebui ca să implementați rezolvarea
  slabă. Să zicem 40 de minute. În acest calcul trebuie să includeți și timpul
  de depanare a programului (care variază de la persoană la persoană) și pe
  cel de testare. Dacă sunteți foarte siguri pe voi, puteți să neglijați
  timpul de testare, dar orice program trebuie testat cel puțin pe exemplul de
  pe foaie.
  
\item Mai rămân deci 20 de minute, timp în care vă puteți gândi la altceva, la
  altă soluție. Pentru a avea șanse mai mari să găsiți o altă soluție, este
  indicat să încercați să ignorați complet soluția slabă, să nu o luați ca
  punct de plecare. Încercați să vă „goliți” mintea și să găsiți ceva nou,
  altfel vă veți învârti mereu în cerc.
  
\item Dacă vă vine vreo idee mai bună, ați scăpat de griji și mergeți la
  punctul {\bf (F)}. Altfel, la expirarea timpului de 20 de minute, vă apucați
  să implementați soluția pe care o aveți, oricât de ineficientă ar fi (de
  fapt, orice soluție, oricât de ineficientă, trebuie să ia măcar o treime sau
  o jumătate din punctaj, dacă nu apar erori de implementare).
  
\item Puteți, ca o măsură extremă, să depășiți cu \underline{maxim} 5 minute
  cele 20 de minute planificate, dar de cele mai multe ori acesta e timp
  pierdut, deoarece intervine stresul și nu puteți să vă mai concentrați.

\end{itemize}

{\bf I)} Dacă ați ajuns până aici înseamnă că ați optat pentru o variantă de
implementare. Din acest moment, pentru această variantă veți scrie programul,
fără a vă mai gândi la altceva, chiar dacă pe parcurs vă vin alte idei. Iată
unele lucruri pe care e bine să le știți despre scrierea unui program:

\begin{itemize}

\item Datele de intrare se presupun a fi corecte. Aceasta este o regulă
  nescrisă (uneori) a concursului de informatică. Chiar dacă, prin absurd,
  știți sigur că datele de intrare trebuie verificate, mai bine n-o faceți,
  din mai multe motive. În primul rând, scopul cu care v-a fost dată problema
  este altul decât să se constate cine verifică mai bine datele de intrare. De
  aceea, cel mult un test sau două vor fi cu date greșite. În al doilea rând,
  nu se justifică să risipiți atâta timp numai pentru câteva puncte pe care
  le-ați putea pierde dacă nu faceți verificarea. În al treilea rând, e
  posibil să greșiți oricum problema în sine, caz în care nu mai contează dacă
  ați citit perfect datele de intrare. În sfârșit, legea lui Murphy spune că
  „oricâte teste ar efectua cineva asupra datelor de intrare, se va găsi
  cineva care să introducă date greșite”. Efortul este deci zadarnic...
  
\item Ultimul lucru, când sunteți convinși că programul este terminat și când
  v-ați hotărât să nu îl mai modificați, adăugați opțiunile de
  compilare. Puteți face aceasta apăsând {\tt Ctrl-O O}. La începutul
  programului vor apărea directivele de compilare. Setați \$B, \$I, \$R și \$S
  pe - (minus). Eventual, puteți include direct linia {\tt \{\$B-,I-,R-,S-\}}
  imediat după titlul programului. Aceasta va face compilatorul să nu mai
  evalueze complet expresiile booleene, să nu mai verifice operațiile de
  intrare/ieșire, domeniul de atribuire ({\it Range Checking}) și stiva ({\it
    Stack Checking}). Există două avantaje majore: în primul rând că programul
  merge mai repede (se câștigă câteva procente bune la viteză), iar în al
  doilea rând, psihologic vorbind, este preferabil ca un program să se
  blocheze decât să se oprească printr-un banal {\tt Range check error}. Nu vă
  grăbiți să puneți directivele de compilare încă de la început, deoarece nu
  veți mai primi mesajele corespunzătoare de eroare și vă va fi mai greu să
  depanați programul.
  
\item Pe cât este posibil, încercați să convingeți juriul să nu vă ruleze
  sursa (Pascal sau C/C++), ci direct executabilul. Merge simțitor mai
  repede. Asta numai în cazul în care vă temeți că programul ar putea să nu se
  încadreze în timp.
  
\item Dacă se poate, evitați lucrul cu pointeri. Programele care îi folosesc
  sunt mai greu de depanat și se pot bloca mult mai ușor.

\item Să presupunem că aveți de lucrat cu matrice de dimensiuni maxim $100
  \times 100$. Unii elevi au obiceiul să dimensioneze la început matricele de
  $5 \times 5$ sau $10 \times 10$, doarece sunt mai comod de evaluat cu {\it
    Evaluate} ({\tt Ctrl-F4}) sau {\it Watch} ({\tt Ctrl-F7}). Acest lucru
  este adevărat, dar există riscul ca la sfârșit să uitați să redimensionați
  matricele de $100 \times 100$. Decât să faceți o asemenea greșeală (care în
  mod sigur vă va compromite toată problema), mai bine setați dimensiunile
  corecte de la început. De altfel, ideal ar fi ca depanarea programelor să
  fie suprimată cu totul și programul să meargă din prima.
  
\item Evitați lucrul cu numere reale, dacă puteți. Operațiile în virgulă
  mobilă sunt mult mai lente. De exemplu, testul dacă un număr este prim nu va
  începe în nici un caz cu linia

  \begin{minted}{pascal}
    while i <= Sqrt(N) do
  \end{minted}

  ci cu linia

  \begin{minted}{pascal}
    while i * i <= N do
  \end{minted}

  Din punct de vedere logic, condițiile sunt perfect echivalente. Totuși,
  prima se evaluează de câteva zeci de ori mai încet decât a doua.

\item Dacă lucrați cu numere reale, nu folosiți testul

  \begin{minted}{pascal}
    R1=R2
  \end{minted}

  deoarece pot apărea erori, ci implementați o funcție:

  \begin{minted}{pascal}
    function Equal(R1,R2:Real):Boolean;
    begin
      Equal:=(Abs(R1-R2)<0.00001);
    end;
  \end{minted}

  Numărul de zerouri de după virgulă trebuie să fie suficient de mare astfel
  încât două numere diferite să nu fie tratate drept egale (se poate lucra de
  pildă cu 0.000001).

\item Tot în cazul numerelor reale, evitați pe cât posibil să faceți
  împărțiri, deoarece sunt foarte greoaie. De exemplu:

  $X/5 \leftrightarrow 0.2*X$\\
  $X/Y/Z \leftrightarrow X/(Y*Z)$

\item Optimizările de genul „{\tt X shl 1}” respectiv „{\tt X<<1}” în loc de
  „{\tt 2*X}” sunt niște artificii de cele mai multe ori neesențiale, care în
  schimb fac formulele mai lungi, greu de urmărit și pot crea complicații. Cel
  mai bine este să lucrați cu notațiile obișnuite și doar la sfârșit, dacă
  timpul de rulare trebuie redus cu orice preț, să faceți înlocuirile.

\item Alegeți-vă numele de variabile în așa fel încât programul să fie
  clar. Sunt permise mai mult de două litere! Numele fiecărei proceduri,
  funcții, variabile trebuie să-i explice clar utilitatea. E drept, lungimea
  programului crește, dar codul devine mult mai limpede și timpul de depanare
  scade foarte mult. Ca o regulă generală, claritatea programelor face mult
  mai ușoară înțelegerea lor chiar și după o perioadă mai îndelungată de timp
  (luni, ani). Nu trebuie nici să cădeți în cealaltă extremă. De exemplu, nu
  depășiți 10 caractere pentru un nume de variabilă.
  
\item Salvați programul cât mai des. Dacă vă obișnuiți, chiar la fiecare
  două-trei linii. După ce o să vă intre în reflex n-o să vă mai incomodeze cu
  nimic acest obicei, mai ales că în ziua de azi salvarea unui program de 2-3
  KB se face practic instantaneu. Au fost frecvente cazurile în care o pană de
  curent prindea pe picior greșit mulți concurenți, iar după aceea nu mai este
  absolut nimic de făcut, pentru că nimeni nu vă va crede pe cuvânt că ați
  făcut programul și că el mergea.
  
\item Obișnuiți-vă să programați modular. Faceți proceduri separate pentru
  citirea și inițializarea datelor, pentru sortare, pentru afișarea
  rezultatelor etc. În general nu se recomandă să scrieți proceduri în alte
  proceduri (adică e bine ca toate procedurile să aparțină direct de programul
  principal). Procedurile, acolo unde e posibil, nu trebuie să depășească un
  ecran, pentru a putea avea o viziune de ansamblu asupra fiecăreia în parte
  . Acest lucru ajută mult la depanare.
  
\item Rulați programul cât mai des. În primul rând după ce scrieți procedura
  de citire a datelor. Dacă e nevoie de sortarea datelor de intrare, nu strică
  să vă convingeți că programul sortează bine, rulând două-trei teste
  oarecare. E păcat să pierdeți puncte dintr-o greșeală copilărească.
  
\item O situație delicată apare când fișierul de intrare conține mai multe
  seturi de date (teste). În acest caz, atenția trebuie sporită, deoarece dacă
  la primul sau al doilea test programul vostru dă eroare și se oprește din
  execuție, veți pierde automat și toate celelalte teste care urmează. Dacă în
  fișierul de intrare exista un singur set de date, atunci pierderea din
  vedere a unui caz particular al problemei nu putea duce, în cel mai rău caz,
  decât la picarea unui test. Așa însă, picarea unui test poate atrage după
  sine picarea tuturor celor care îi urmează. Pe lângă corectitudinea strict
  necesară, programul trebuie să se încadreze și în timp pentru orice fel de
  test. Dacă la primul sau al doilea test din suită programul depășește timpul
  (sau, și mai rău, se blochează), e foarte probabil să fie oprit din execuție
  de către comisie, deci din nou veți pierde toate testele care au rămas
  neexecutate. Uneori comisia este binevoitoare și acceptă să modifice
  fișierul de date, tăind din el setul pe care programul vostru nu merge, dar
  acest lucru este greoi și nu tocmai cinstit față de ceilalți concurenți,
  deci nu vă bazați pe asta.
  
\item Tot în situația în care există mai multe seturi de date în fișierul de
  intrare, dacă ieșirea se face într-un fișier, este bine ca după afișarea
  rezultatului pentru fiecare test să actualizați fișierul de ieșire (fie prin
  comanda {\tt Flush}, fie prin două proceduri, {\tt Close} și {\tt
    Append}). În felul acesta, chiar dacă la unul din teste programul se
  blochează sau dă eroare, rezultatele deja scrise rămân scrise. Altfel, e
  posibil ca rezultatele de la testele anterioare să rămână într-un buffer în
  memorie, fără a fi „vărsate” pe disc.
  
\item Dacă fișierul de ieșire are dimensiuni foarte mari, de exemplu dacă vi
  se cer toate soluțiile, iar numărul acestora este de ordinul zecilor de mii,
  puteți avea surpriza ca timpul să nu vă ajungă pentru a le tipări pe toate
  în fișierul de ieșire. Și în acest caz, este recomandat ca după tipărirea
  fiecărei soluții să executați comanda {\tt Flush}, sau să închideți fișierul
  de ieșire și să-l redeschideți în modul {\tt Append}.
  
\end{itemize}

{\bf J)} Dacă ați trecut cu bine și de faza de scriere a programului, mai
aveți doar părțile de depanare și testare, care de multe ori se îmbină. Metoda
cea mai bună de depanare este următoarea:

\begin{itemize}

\item Începeți cu un test nici prea simplu, nici prea complicat (și ușor de
  urmărit cu creionul pe hârtie) și executați-l de la cap la coadă. Dacă merge
  perfect, treceți la teste mai complexe (\underline{minim} 4 teste și maxim
  7-8). Dacă le trece și pe acestea, puteți fi mândri. Legile lui Murphy în
  programare se aplică în continuare: „Depanarea nu poate demonstra că un
  program merge; ea poate cel mult demonstra că un program nu merge”. Totuși,
  dacă programul vostru a mers perfect pe 7-8 teste date la întâmplare, există
  șanse foarte mari să meargă pe majoritatea testelor comisiei, sau chiar pe
  toate.
  
\item Exemplul dat în enunț nu are în general nici o semnificație deosebită
  (de fapt, are mai curând darul de a semăna confuzie printre concurenți), iar
  dacă programul merge pe acest test particular, nu înseamnă că o să meargă și
  pe alte teste. În culegere nu a fost explicat pe larg algoritmul decât pe
  exemplul din enunțul fiecărei probleme, dar aceasta s-a făcut numai pentru a
  nu supraîncărca materialul.
  
\item Dacă la unul din teste programul nu merge corespunzător, rulați din nou
  testul, dar de data aceasta procedură cu procedură. După fiecare procedură
  evaluați variabilele și vedeți dacă au valorile așteptate. În felul acesta
  puteți localiza cu precizie procedura, apoi linia unde se află
  eroarea. Corectați în această manieră toate erorile, până când testul este
  trecut.
  
\item În acest moment, luați de la capăt toate testele pe care programul le-a
  trecut deja. În urma depanării, s-ar putea ca alte greșeli să iasă la
  suprafață și programul să nu mai meargă pe vechile teste.
  
\item Repetați procedeul de mai sus până când toate testele merg. Dacă vă
  obișnuiți să programați modular și îngrijit, depanarea și testarea n-ar
  trebui să dureze mai mult de 5-10 minute. Din acest moment, nu mai
  modificați nici măcar o literă în program, sau dacă țineți să o faceți,
  păstrați-vă în prealabil o copie. Nu vă bazați pe faptul că puteți să țineți
  minte modificările făcute și să refaceți oricând forma inițială a
  programului în caz că noua versiune nu va fi bună.
  
\item Dacă totuși nu-i puteți „da de cap” programului, iar timpul alocat
  problemei respective expiră, aduceți programul la o formă în care să meargă
  măcar pe o parte din teste (pe jumătate, de exemplu) și treceți la problema
  următoare.

\end{itemize}

{\bf K)}

\begin{minted}{pascal}
    Dec(Nr_probleme_nerezolvate);
  end; { while }
\end{minted}

\begin{center}
  {\Huge \decofourleft \decofourright}
\end{center}

Probabil nu veți fi de acord cu toate sfaturile date mai sus. E bine însă să
le aplicați. Scopul pentru care ele au fost incluse în această carte este de a
ajuta concurenții să se acomodeze mai ușor cu atmosfera concursului. De multe
ori, primul an de participare la olimpiadă se soldează cu un rezultat cel mult
mediu, deoarece, oricât ar spune cineva „ei, nu-i așa mare lucru să mergi la
un concurs”, experiența acumulată contează mult. De aceea, abia de la a doua
participare și uneori chiar de mai târziu încep să apară rezultatele. Intenția
autorului a fost să vă ușureze misiunea și să vă dezvăluie câteva din
dificultățile de toate felurile care apar la orice concurs, pentru a nu vă da
ocazia să le descoperiți pe propria piele. Poate că aceste ponturi vă vor fi
de folos.

  \chapter{Lucrul cu numere mari}

De multe ori, în probleme, apar situații când este nevoie să memorăm numere
întregi foarte mari (de ordinul sutelor de cifre), iar uneori trebuie să
efectuăm și operații aritmetice cu aceste numere. Iată un asemenea exemplu:

{\bf ENUNȚ}: Se dă un număr natural cu $N \leq 1000$ cifre. Se cere să se
extragă rădăcina cubică a numărului. Se garantează că numărul citit este cub
perfect.

{\bf Intrarea}: Fișierul {\tt INPUT.TXT} conține un singur rând, terminat cu
{\tt EOF}, pe care se află numărul, cifrele fiind nedespărțite.

{\bf Ieșirea}: În fișierul {\tt OUTPUT.TXT} se va tipări rădăcina cubică a
numărului, pe o singură linie terminată cu {\tt EOF}.

{\bf Exemplu:}

\texttt{
  \begin{tabular}{| l | l |}
    \hline
    {\bf INPUT.TXT} & {\bf OUTPUT.TXT} \\ \hline
    2097152 & 128 \\
    \hline
  \end{tabular}
}

{\bf Timp de implementare}: 1h 30’.

{\bf Timp de rulare}: 10 secunde.

{\bf Complexitate cerută}: $O(N^2)$.

{\bf REZOLVARE}: Problema este cât se poate de simplă din punct de vedere
matematic; dificultatea apare la implementare, atât datorită structurilor de
date necesare, cât mai ales datorită complexității cerute. Despre lucrul cu
numere întregi (chiar și {\tt Longint}) nici nu poate fi vorba, iar la lucrul
cu numere reale apar erori de calcul.

Structura de date propusă pentru abordarea acestui gen de probleme este
următoarea: numerele vor fi reprezentate printr-un vector de cifre
zecimale. Prima poziție (poziția 0) din vector este rezervată pentru a memora
numărul de cifre. Definiția C a tipului de date este:

\begin{lstlisting}[language=C]
typedef int Huge[1001];
\end{lstlisting}

Iată cum s-ar memora numărul „1997” într-un asemenea vector:

\centeredTikzFigure[
  frame/.style = {rectangle, draw=black, line width=0.2pt, minimum size=20pt},
  label/.style = {rectangle, minimum size=20pt, font=\scriptsize},
]{
  \node (L1) {H};
  \node[frame] (r1)  at ([xshift=1em] L1.east) {4};
  \node[frame, anchor=west] (r2) at (r1.east) {7};
  \node[frame, anchor=west] (r3) at (r2.east) {9};
  \node[frame, anchor=west] (r4) at (r3.east) {9};
  \node[frame, anchor=west] (r5) at (r4.east) {1};

  \node[label, anchor=south] at (r1.north) {H[0]};
  \node[label, anchor=south] at (r2.north) {H[1]};
  \node[label, anchor=south] at (r3.north) {H[2]};
  \node[label, anchor=south] at (r4.north) {H[3]};
  \node[label, anchor=south] at (r5.north) {H[4]};
}  

Se observă că vectorul este oarecum „întors pe dos”. Totuși, această formă
este cea mai convenabilă, pentru că ea permite implementarea cu o mai mare
ușurință a operațiilor aritmetice.

Mai trebuie remarcat că pe fiecare poziție $K$ în vector se află cifra care îl
reprezintă pe $10^{K-1}$ în numărul reprezentat: în $H[1]$ se află cifra
unităților ($10^0$), în $H[2]$ se află cifra zecilor ($10^1$), în $H[3]$ se
află cifra sutelor ($10^2$) ș.a.m.d. Formatul zecimal nu este cea mai fericită
(a se citi „eficientă”) alegere. El ocupă doar patru biți din cei opt ai unui
octet, deci face risipă de memorie. Dacă am alege baza de numerație 256, am
folosi la maximum memoria, și în plus operațiile ar fi cu mult mai rapide
(deoarece 256 este o putere a lui 2, înmulțirile și împărțirile la 256 sunt
simple deplasări la stânga și la dreapta ale reprezentărilor binare). Să facem
următorul calcul ca să vedem câte cifre are în baza 256 un număr care are 1000
de cifre în baza 10:

\begin{equation}
  10^{1000} =
  10 \cdot (10^3)^{333} \approx
  10 \cdot (2^{10})^{333} \approx
  10 \cdot 2^2 \cdot (2^8)^{416} =
  40 \cdot 256^{416}
\end{equation}
	
Așadar, numărul de cifre s-a redus la sub jumătate. Un algoritm liniar ar
funcționa de două ori mai repede pe reprezentări în baza 256, iar unul
pătratic ar funcționa de patru ori mai repede. Inconvenientul major este
dificultatea depanării unui program care operează într-o bază aritmetică atât
de străină nouă. Vom rămâne deci la baza 10, cu mențiunea că acei mai temerari
dintre voi pot încerca folosirea bazei 256.

Numerele reprezentate pe vectori le vom numi pur și simplu „vectori”, iar
numerele reprezentate printr-un tip ordinal de date le vom numi, printr-o
analogie ușor forțată cu matematica, „scalari”. Să vedem acum cum se
efectuează operațiile elementare pe aceste numere.

\section{Inițializarea}

Un vector poate fi inițializat în trei feluri: cu 0, cu un scalar sau cu un
alt vector.

La inițializarea cu 0, singurul lucru pe care îl avem de făcut este să setăm
numărul de cifre pe 0. De aceea, este practic inutil să implementăm această
funcție ca atare; putem folosi în loc singura instrucțiune pe care ea o
conține.

\begin{lstlisting}[language=C]
void Atrib0(Huge H)
{ H[0]=0; }
\end{lstlisting}

La inițializarea cu un scalar nenul, trebuie să așezăm fiecare cifră pe
poziția corespunzătoare, aflând în paralel și numărul de cifre. Se începe cu
cifra unităților, și la fiecare pas se pune în vector cifra cea mai puțin
semnificativă, după care numărul de reprezentat se împarte la 10
(neglijându-se restul), iar numărul de cifre se incrementează.

\begin{lstlisting}[language=C]
void AtribValue(Huge H, unsigned long X)
{ H[0]=0;
  while (X)
    { H[++H[0]]=X%10;
      X/=10;
    }
}
\end{lstlisting}

Iată, de exemplu, cum se pune pe vector numărul 195:

\centeredTikzFigure[
  col1/.style = {rectangle, minimum height=2em, text width=5em},
  col2/.style = {rectangle, minimum height=2em, text width=7em},
  frame/.style = {rectangle, draw=black, line width=0.2pt, minimum size=20pt},
]{
  % row 0
  \node[col1] (c01) {};
  \node[col2, anchor=west] (c02) at (c01.east) {};
  \node[frame] (c03)  at ([xshift=2em] c02.east) {0};
  \node[frame, anchor=west] (c04) at (c03.east) {0};
  \node[frame, anchor=west] (c05) at (c04.east) {0};
  \node[frame, anchor=west] (c06) at (c05.east) {0};

  % row 1
  \node[col1, anchor=north] (c11) at (c01.south) {$X = 195$};
  \node[col2, anchor=west] (c12) at (c11.east) {$X\,\bmod\,10 = 5$};
  \node[frame] (c13)  at ([xshift=2em] c12.east) {1};
  \node[frame, anchor=west] (c14) at (c13.east) {5};
  \node[frame, anchor=west] (c15) at (c14.east) {0};
  \node[frame, anchor=west] (c16) at (c15.east) {0};
  \node[anchor=west] (c17)  at ([xshift=2em] c16.east) {$X / 10 = 19$};

  % row 2
  \node[col1, anchor=north] (c21) at (c11.south) {$X = 19$};
  \node[col2, anchor=west] (c22) at (c21.east) {$X\,\bmod\,10 = 9$};
  \node[frame] (c23)  at ([xshift=2em] c22.east) {2};
  \node[frame, anchor=west] (c24) at (c23.east) {5};
  \node[frame, anchor=west] (c25) at (c24.east) {9};
  \node[frame, anchor=west] (c26) at (c25.east) {0};
  \node[anchor=west] (c27)  at ([xshift=2em] c26.east) {$X / 10 = 1$};

  % row 3
  \node[col1, anchor=north] (c31) at (c21.south) {$X = 1$};
  \node[col2, anchor=west] (c32) at (c31.east) {$X\,\bmod\,10 = 1$};
  \node[frame] (c33)  at ([xshift=2em] c32.east) {3};
  \node[frame, anchor=west] (c34) at (c33.east) {5};
  \node[frame, anchor=west] (c35) at (c34.east) {9};
  \node[frame, anchor=west] (c36) at (c35.east) {1};
  \node[anchor=west] (c37)  at ([xshift=2em] c36.east) {$X / 10 = 0$};

  % top label
  \node[minimum height=2em, anchor=south] (L1) at ([xshift=1em] c04.north) {H};
}  

În sfârșit, inițializarea unui vector cu altul se face printr-o simplă copiere
(se pot folosi cu succes rutine de lucru cu memoria, cum ar fi {\tt FillChar}
în Pascal sau {\tt memmove} în C). Pentru eleganță, poate fi folosită și
atribuirea cifră cu cifră:

\begin{lstlisting}[language=C]
void AtribHuge(Huge H, Huge X)
{ int i;

  for (i=0;i<=X[0];i++) H[i]=X[i];
}
\end{lstlisting}

\section{Compararea}

Pentru a compara două numere „uriașe”, începem prin a afla numărul de cifre
semnificative (deoarece, în urma anumitor operații pot rezulta zerouri
nesemnificative care „atârnă” totuși la numărul de cifre). Aceasta se face
decrementând numărul de cifre al fiecărui număr atâta timp cât cifra cea mai
semnificativă este 0. După ce ne-am asigurat asupra acestui punct, comparăm
numărul de cifre al celor două cifre. Numărul cu mai multe cifre este cel mai
mare. Dacă ambele numere au același număr de cifre, pornim de la cifra cea mai
semnificativă și comparăm cifrele celor două numere până la prima diferență
întâlnită. În acest moment, numărul a cărui cifră este mai mare este el însuși
mai mare. Dacă toate cifrele numerelor sunt egale două câte două, atunci în
mod evident numerele sunt egale.

După cum se vede, algoritmul seamănă foarte bine cu ceea ce s-a învățat la
matematică prin clasa a II-a (doar că atunci nu ni s-a spus că este vorba de
un „algoritm”). Rutina de mai jos compară două numere uriașe $H_1$ și $H_2$ și
întoarce -1, 0 sau 1, după $H_1$ este mai mic, egal sau mai mare decât $H_2$.

\begin{lstlisting}[language=C]
int Sgn(Huge H1,Huge H2)
{ int i;

  while (!H1[H1[0]] && H1[0]) H1[0]--;
  while (!H2[H2[0]] && H2[0]) H2[0]--;
  if (H1[0]!=H2[0])
    return H1[0]<H2[0] ? -1 : 1;
  i=H1[0];
  while (H1[i]==H2[i] && i) i--;
  return H1[i]<H2[i] ? -1 : H1[i]==H2[i] ? 0 : 1;
}
\end{lstlisting}

\section{Adunarea a doi vectori}

Fiind dați doi vectori, $A$ cu $M$ cifre și $B$ cu $N$ cifre, adunarea lor se
face în mod obișnuit, ca la aritmetică. Pentru a evita testele de depășire, se
recomandă să se completeze mai întâi vectorul cu mai puține cifre cu zerouri
până la dimensiunea vectorului mai mare. La sfârșit, vectorul sumă va avea
dimensiunea vectorului mai mare dintre $A$ și $B$, sau cu 1 mai mult dacă
apare transport de la cifra cea mai semnificativă. Procedura de mai jos adaugă
numărul $B$ la numărul $A$.

\begin{lstlisting}[language=C]
void Add(Huge A, Huge B)
/* A <- A+B */
{ int i,T=0;

  if (B[0]>A[0])
    { for (i=A[0]+1;i<=B[0];) A[i++]=0;
      A[0]=B[0];
    }
    else for (i=B[0]+1;i<=A[0];) B[i++]=0;
  for (i=1;i<=A[0];i++)
    { A[i]+=B[i]+T;
      T=A[i]/10;
      A[i]%=10;
    }
  if (T) A[++A[0]]=T;
}
\end{lstlisting}

\section{Scăderea a doi vectori}

Se dau doi vectori $A$ și $B$ și se cere să se calculeze diferența $A - B$. Se
presupune $B \leq A$ (acest lucru se poate testa cu funcția {\tt Sgn}). Pentru
aceasta, se pornește de la cifra unităților și se memorează la fiecare pas
„împrumutul” care trebuie efectuat de la cifra de ordin imediat superior
(împrumutul poate fi doar 0 sau 1). Deoarece $B \leq A$, avem garanția că
pentru a scădea cifra cea mai semnificativă a lui $B$ din cifra cea mai
semnificativă a lui $A$ nu e nevoie de împrumut.

\begin{lstlisting}[language=C]
void Subtract(Huge A, Huge B)
/* A <- A-B */
{ int i, T=0;

  for (i=B[0]+1;i<=A[0];) B[i++]=0;
  for (i=1;i<=A[0];i++)
    A[i]+= (T=(A[i]-=B[i]+T)<0) ? 10 : 0;
    /* Adica A[i]=A[i]-(B[i]+T);
       if (A[i]<0) T=1; else T=0;
       if (T) A[i]+=10; */
  while (!A[A[0]]) A[0]--;
}
\end{lstlisting}

\section{Înmulțirea și împărțirea cu puteri ale lui 10}

Aceste funcții sunt uneori utile. Ele pot folosi și funcțiile de înmulțire a
unui vector cu un scalar, care vor fi prezentate mai jos, dar se pot face și
prin deplasarea întregului număr spre stânga sau spre dreapta. De exemplu,
înmulțirea unui număr cu 100 presupune deplasarea lui cu două poziții înspre
cifra cea mai semnificativă și adăugarea a două zerouri la coadă. Principalul
avantaj al scrierii unor funcții separate pentru înmulțirea cu 10, 100, ...,
este că se pot folosi rutinele de acces direct al memoriei ({\tt FillChar},
respectiv {\tt memmove}). Iată funcțiile care realizează deplasarea
vectorilor, atât prin mutarea blocurilor de memorie, cât și prin atribuiri
succesive.

\begin{lstlisting}[language=C]
void Shl(Huge H, int Count)
/* H <- H*10^Count */
{ 
  /* Shifteaza vectorul cu Count pozitii */
  memmove(&H[Count+1],&H[1],sizeof(int)*H[0]);
  /* Umple primele Count pozitii cu 0 */
  memset(&H[1],0,sizeof(int)*Count);
  /* Incrementeaza numarul de cifre */
  H[0]+=Count;
}

void Shl2(Huge H, int Count)
/* H <- H*10^Count */
{ int i;

  /* Shifteaza vectorul cu Count pozitii */
  for (i=H[0];i;i--) H[i+Count]=H[i];
  /* Umple primele Count pozitii cu 0 */
  for (i=1;i<=Count;) H[i++]=0;
  /* Incrementeaza numarul de cifre */
  H[0]+=Count;
}

void Shr(Huge H, int Count)
/* H <- H/10^Count */
{ 
  /* Shifteaza vectorul cu Count pozitii */
  memmove(&H[1],&H[Count+1],sizeof(int)*(H[0]-Count));
  /* Decrementeaza numarul de cifre */
  H[0]-=Count;
}

void Shr2(Huge H, int Count)
/* H <- H/10^Count */
{ int i;

  /* Shifteaza vectorul cu Count pozitii */
  for (i=Count+1;i<=H[0];i++) H[i-Count]=H[i];
  /* Decrementeaza numarul de cifre */
  H[0]-=Count;
}
\end{lstlisting}

\section{Înmulțirea unui vector cu un scalar}

Și această operație este o simplă implementare a modului manual de efectuare a
calculului. La înmulțirea „de mână” a unui număr mare cu unul de o singură
cifră, noi parcurgem deînmulțitul de la sfârșit la început, și pentru fiecare
cifră efectuăm următoarele operații:

\begin{itemize}

\item Înmulțim cifra respectivă cu înmulțitorul;

\item Adăugăm „transportul” de la înmulțirea precedentă;

\item Separăm ultima cifră a rezultatului și o trecem la produs;

\item Celelalte cifre are rezultatului constituie transportul pentru
  următoarea înmulțire;

\item La sfârșitul înmulțirii, dacă există transport, acesta are o singură
  cifră, care se scrie înaintea rezultatului.

\end{itemize}

Exact același procedeu se poate aplica și dacă înmulțitorul are mai mult de o
cifră. Singura deosebire este că transportul poate avea mai multe cifre (poate
fi mai mare ca 9). Din această cauză, la sfârșitul înmulțirii, poate rămâne un
transport de mai multe cifre, care se vor scrie înaintea rezultatului. Iată de
exemplu cum se calculează produsul $312 \times 87$:

\begin{center}
  \begin{tabular}{lll@{\hspace{1in}}llllll}
    & & & & & {\large\bf 3} & {\large\bf 1} & {\large\bf 2} & $\times$ \\
    & & & & & & {\large\bf 8} & {\large\bf 7} & \\ \cline{4-8}

    $87 \times 2$ & $= 174$ & $= 17 \times 10 + 4$ & & & & & {\large\bf 4} & \\
    $87 \times 1 + 17$ & $= 104$ & $= 10 \times 10 + 4$ & & & & {\large\bf 4} & & \\
    $87 \times 3 + 10$ & $= 271$ & $= 27 \times 10 + 1$ & & & {\large\bf 1} & & & \\
    $87 \times 0 + 27$ & $= 27$ & $= 2 \times 10 + 7$ & & {\large\bf 7} & & & & \\
    $87 \times 0 + 2$ & $= 2$ & $= 0 \times 10 + 2$ & {\large\bf 2} & & & & &
  \end{tabular}
\end{center}

Procedura este descrisă mai jos:

\begin{lstlisting}[language=C]
void Mult(Huge H, unsigned long X)
/* H <- H*X */
{ int i;
  unsigned long T=0;

  for (i=1;i<=H[0];i++)
    { H[i]=H[i]*X+T;
      T=H[i]/10;
      H[i]=H[i]%10;
    }
  while (T) /* Cat timp exista transport */
    { H[++H[0]]=T%10;
      T/=10;
    }
}
\end{lstlisting}

\section{Înmulțirea a doi vectori}

Dacă ambele numere au dimensiuni mari și se reprezintă pe tipul de date {\tt
  Huge}, produsul lor se calculează înmulțind fiecare cifră a deînmulțitului
cu fiecare cifră a înmulțitorului și trecând rezultatul la ordinul de mărime
(exponentul lui 10) cuvenit. De fapt, același lucru îl facem și noi pe
hârtie. Considerând același exemplu, în care ambele numere sunt „uriașe”,
produsul lor se calculează de mână astfel:

\begin{center}
  \begin{tabular}{llllll}
    & & 3 & 1 & 2 & $\times$ \\
    & & & 8 & 7 \\ \cline{1-5}
    & 2 & 1 & 8 & 4 \\
    2 & 4 & 9 & 6 & \\ \cline{1-5}
    2 & 7 & 1 & 4 & 4
  \end{tabular}
\end{center}

S-a luat deci fiecare cifră a înmulțitorului și s-a efectuat produsul parțial
corespunzător, corectând la fiecare pas rezultatul prin calculul
transportului. Rezultatul pentru fiecare produs parțial s-a scris din ce în ce
mai în stânga, pentru a se alinia corect ordinele de mărime. Acest procedeu
este oarecum incomod de implementat. Se pot face însă unele observații care
ușurează mult scrierea codului:

\begin{itemize}

\item Prin înmulțirea cifrei cu ordinul de mărime $10^i$ din primul număr cu
  cifra cu ordinul de mărime $10^j$ din al doilea număr se obține o cifră
  corespunzătoare ordinului de mărime $10^{i+j}$ în rezultat (sau se obține un
  număr cu mai mult de o singură cifră, caz în care transportul merge la cifra
  corespunzătoare ordinului de mărime $10^{i + j + 1}$).
      
\item Dacă numerele au $M$ și respectiv $N$ cifre, atunci produsul lor va avea
  fie $M + N$ fie $M + N - 1$ cifre. Într-adevăr, dacă numărul $A$ are $M$
  cifre, atunci $10^{M - 1} \leq A < 10^M$ și $10^{N - 1} \leq B < 10^N$, de
  unde rezultă $10^{M + N - 2} \leq A \times B < 10^{M + N}$.

\item La calculul produselor parțiale se poate omite calculul transportului,
  acesta urmând a se face la sfârșit. Cu alte cuvinte, într-o primă fază putem
  pur și simplu să înmulțim cifră cu cifră și să adunăm toate produsele de
  aceeași putere, obținând un număr cu „cifre” mai mari ca 9, pe care îl
  aducem la forma normală printr-o singură parcurgere. Să reluăm același
  exemplu:

\end{itemize}

\begin{center}
  \begin{tabular}{l@{\hspace{0.5in}}rrrrrl}
    \multirow{2}{*}{Intrarea: Vectorii $A$ și $B$}
    & & & {\large\bf 3} & {\large\bf 1} & {\large\bf 2} & $\times$ \\
    & & & & {\large\bf 8} & {\large\bf 7} & \\ \cline{2-6}

    \multirow{2}{*}{Etapa I: Efectuarea produselor intermediare}
    & & & 21 & 7 & 14 & \\
    & & 24 & 8 & 16 & & \\ \cline{2-6}

    \multirow{2}{*}{Etapa a II-a: Adunarea produselor intermediare}
    & & 24 & 29 & 23 & 14 & \\
    & & $\overleftarrow{2}$ & $\overleftarrow{3}$ & $\overleftarrow{2}$ & $\overleftarrow{1}$ & \\
    \cline{2-6}

    Etapa a III-a: Corectarea rezultatului
    & {\large\bf 2} & {\large\bf 7} & {\large\bf 1} & {\large\bf 4} & {\large\bf 4} &
  \end{tabular}
\end{center}

Această operație efectuează $M \times N$ produse de cifre și $M + N$ (sau $M +
N - 1$, după caz) „transporturi” pentru aflarea rezultatului, deci are
complexitatea $O(M \times N)$. Iată și implementarea:

\begin{lstlisting}[language=C]
void MultHuge(Huge A, Huge B, Huge C)
/* C <- A x B */
{ int i,j,T=0;

  C[0]=A[0]+B[0]-1;
  for (i=1;i<=A[0]+B[0];) C[i++]=0;
  for (i=1;i<=A[0];i++)
    for (j=1;j<=B[0];j++)
      C[i+j-1]+=A[i]*B[j];
  for (i=1;i<=C[0];i++)
    { T=(C[i]+=T)/10;
      C[i]%=10;
    }
  if (T) C[++C[0]]=T;
}
\end{lstlisting}

Mai există o altă modalitate de a înmulți două numere de câte $N$ cifre
fiecare, care are complexitatea $O(N^{\log_2 3}) \approx O(N^{1,58}) \approx
O(N \sqrt{N})$. Ea derivă de un algoritm propus de Strassen în 1969 pentru
înmulțirea matricelor. Diferența se face simțită, ce-i drept pentru valori
mari ale lui $N$, dar constanta multiplicativă crește destul de mult și, în
plus, soluția e mai greu de implementat; de aceea nu recomandăm implementarea
ei în timpul concursului. Ideea de bază este să se micșoreze numărul de
înmulțiri și să se mărească numărul de adunări, deoarece adunarea a doi
vectori se face în $O(N)$, pe când înmulțirea se face în $O(N^2)$. Să
considerăm întregii $A$ și $B$, fiecare de câte $N$ cifre. Trebuie să-i
înmulțim într-un timp $T(N)$ mai bun decât $O(N^2)$. Să împărțim numărul $A$
în două „bucăți” $C$ și $D$, fiecare de câte $N$ / 2 cifre, iar întregul $B$
în două bucăți $E$ și $F$, tot de câte $N$ / 2 cifre (presupunem că $N$ este
par):

\centeredTikzFigure[
  frame/.style = {rectangle, draw=black, line width=0.2pt, minimum width=6em},
]{
  % row 0
  \node[minimum width=2em] (a) at (0,0) {\bf $A$};
  \node[frame] (c) at ([xshift=5em] a.east) {$C$};
  \node[frame, anchor=west] (d) at (c.east) {$D$};

  % row 0
  \node[minimum width=2em] (b) at (0,-2em) {\bf $B$};
  \node[frame] (e) at ([xshift=5em] b.east) {$E$};
  \node[frame, anchor=west] (f) at (e.east) {$F$};

  % top arrow
  \draw [<->] (3em,.5) -- node[above]{$N$} (15em,.5);

  % bottom arrows
  \draw [<->] (3em,-1.35) -- node[below]{$N/2$} (9em,-1.35);
  \draw [<->] (9em,-1.35) -- node[below]{$N/2$} (15em,-1.35);
}  

Atunci se poate scrie relația

\begin{align}
  \begin{split}
    A \times B & = (C \times 10^{N / 2} + D) \times (E \times 10^{N / 2} + F) \\
    & = CE \times 10^{N} + (CF + DE) \times 10^{N / 2} + DF
  \end{split}
\end{align}

Pentru a putea calcula produsul $A \times B$ avem, prin urmare, nevoie de
patru produse parțiale, de trei adunări și de două înmulțiri cu puteri ale lui
10. Adunările și înmulțirile cu puteri ale lui 10 se fac în timp liniar. Dacă
efectuăm cele patru produse parțiale prin patru înmulțiri, rezultă formula
recurentă de calcul

\begin{equation}
  T(N) = 4 T(N / 2) + O(N)
\end{equation}

care duce prin eliminarea recurenței la $T(N) \in O(N^2)$. Cu alte cuvinte,
încă n-am câștigat nimic. Trebuie să reușim cumva să reducem numărul de
înmulțiri de la 4 la 3, chiar dacă prin aceasta vom mări numărul de adunări
necesare. Să definim produsul

\begin{align}
  \begin{split}
    G & = (C + D) \times (E + F) \\
    & = CE + CF + DE + DF \\
    & = CE + DF + (CF + DE)
  \end{split}
\end{align}

Atunci putem scrie:

\begin{equation}
  A \times B = CE \times 10^{N} + (G - CE - DF) \times 10^{N / 2} + DF
\end{equation}
	
Pentru această variantă, folosim doar trei înmulțiri, și chiar dacă avem
nevoie de șase adunări și scăderi și două înmulțiri cu puteri ale lui 10,
complexitatea se va reduce la $O(N^{\log_2 3})$. În cazul în care $N$ este o
putere a lui 2, împărțirea în două a numerelor se poate face fără probleme la
fiecare pas, până se ajunge la numere de o singură cifră, care se înmulțesc
direct. În cazul în care $N$ nu este o putere a lui 2, este comod să se
completeze numerele cu zerouri până la o putere a lui 2. În funcțiile descrise
mai jos, {\tt MultRec} nu face decât înmulțirea recursivă, pe când {\tt
  MultHuge2} se ocupă și de corectarea numărului de cifre (incrementarea până
la o putere a lui 2). Pentru calculul produselor $C \times E$ și $D \times F$,
procedura {\tt MultRec} se autoapelează; pentru calcularea produsului $(C+D)
\times (E+F)$, însă, este nevoie să fie apelată procedura {\tt MultHuge2},
deoarece prin cele două adunări poate să apară o creștere a numărului de cifre
al factorilor, care în acest caz trebuie readuși la un număr de cifre egal cu
o putere a lui 2.

\begin{lstlisting}[language=C]
void MultHuge2(Huge A, Huge B, Huge P);

void MultRec(Huge A, Huge B, Huge P)
{ Huge C,D,E,F,CE,DF;

  if (A[0]==1)
    { P[1]=A[1]*B[1];
      P[0]=(P[2]=P[1]/10)>0 ? 2 : 1;
      P[1]%=10;
    }
    else { P[0]=0;
           AtribHuge(C,A);Shr(C,A[0]/2);
           AtribHuge(D,A);D[0]=A[0]/2;
           AtribHuge(E,B);Shr(E,B[0]/2);
           AtribHuge(F,B);F[0]=B[0]/2;
           MultRec(C,E,CE);MultRec(D,F,DF);
           Add(C,D);Add(E,F);
           MultHuge2(C,E,P);
           Subtract(P,CE);Subtract(P,DF);
           Shl(P,A[0]/2);
           Shl(CE,A[0]);Add(P,CE);
           Add(P,DF);
         }
}

void MultHuge2(Huge A, Huge B, Huge P)
/* P <- A x B, varianta N^(lg 3) */
{ int i,j,NDig=A[0]>B[0] ? A[0] : B[0],Needed=1,SaveA,SaveB;

  /* Corecteaza numarul de cifre */
  while (Needed<NDig) Needed<<=1;
  SaveA=A[0];SaveB=B[0];A[0]=B[0]=Needed;
  for (i=SaveA+1;i<=Needed;) A[i++]=0;
  for (i=SaveB+1;i<=Needed;) B[i++]=0;
  MultRec(A,B,P);

  /* Restaureaza numarul de cifre */
  A[0]=SaveA;B[0]=SaveB;
  while (!P[P[0]] && P[0]>1) P[0]--;
}
\end{lstlisting}

\section{Împărțirea unui vector la un scalar}

Ne propunem să scriem o funcție care să împartă numărul $A$ de tip {\tt Huge}
la scalarul $B$, să rețină valoarea câtului tot în numărul $A$ și să întoarcă
restul (care este o variabilă scalară). Să pornim de la un exemplu particular
și să generalizăm apoi procedeul: să calculăm câtul și restul împărțirii lui
1997 la 7. Cu alte cuvinte, să găsim acele numere $C$ de tip {\tt Huge} și $R
\in \{0, 1, 2, 3, 4, 5, 6\}$ cu proprietatea că $1997 = 7 \times C + R$.

\begin{center}
  \begin{tabular}{
      llll
      llllll}
    1 & 9 & 9 & 7 & & \multicolumn{1}{|l}{7} & & & & \\ \cline{6-9}
    0 & & & & & \multicolumn{1}{|l}{0} & 2 & 8 & 5 & $\leftarrow$ câtul \\ \cline{1-1}
    1 & 9 & & & & & & & & \\
    1 & 4 & & & & & & & & \\ \cline{1-2}
    & 5 & 9 & & & & & & & \\
    & 5 & 6 & & & & & & & \\ \cline{2-3}
    & & 3 & 7 & & & & & & \\
    & & 3 & 5 & & & & & & \\ \cline{3-4}
    & & & 2 & $\leftarrow$ restul & & & & &
  \end{tabular}
\end{center}

La fiecare pas se coboară câte o cifră de la deîmpărțit alături de numărul
deja existent (care inițial este 0), apoi rezultatul se împarte la împărțitor
(7 în cazul nostru). Câtul este întotdeauna o cifră și se va depune la
sfârșitul câtului împărțirii, iar restul va fi folosit pentru următoarea
împărțire. Restul care rămâne după ultima împărțire este tocmai $R$ pe care îl
căutăm. Procedeul funcționează și atunci când deîmpărțitul are mai multe
cifre. La sfârșit trebuie să decrementăm corespunzător numărul de cifre al
câtului, prin neglijarea zerourilor create la începutul numărului. Numărul
maxim de cifre al câtului este egal cu cel al deîmpărțitului.

\begin{lstlisting}[language=C]
unsigned long Divide(Huge A, unsigned long X)
/* A <- A/X si intoarce A%X */
{ int i;
  unsigned long R=0;

  for (i=A[0];i;i--)
    { A[i]=(R=10*R+A[i])/X;
      R%=X;
    }
  while (!A[A[0]] && A[0]>1) A[0]--;
  return R;
}
\end{lstlisting}

Dacă dorim numai să aflăm restul împărțirii, nu mai avem nevoie decât să
recalculăm restul la fiecare pas, fără a mai modifica vectorul $A$:

\begin{lstlisting}[language=C]
unsigned long Mod(Huge A, unsigned long X)
/* Intoarce A%X */
{ int i;
  unsigned long R=0;

  for (i=A[0];i;i--)
    R=(10*R+A[i])%X;
  return R;
}
\end{lstlisting}

\section{Împărțirea a doi vectori}

Dacă se dau doi vectori $A$ și $B$ și se cere să se afle câtul $C$ și restul
$R$, etapele de parcurs sunt aceleași ca la punctul precedent, numai că
operatorii „/” și „\%” trebuie implementați de utilizator, ei nefiind definiți
pentru vectori. Cu alte cuvinte, după ce „coborâm” la fiecare pas următoarea
cifră de la deîmpărțit, trebuie să aflăm cea mai mare cifră $X$ astfel încât
împărțitorul să se cuprindă de $X$ ori în restul de la momentul
respectiv. Acest lucru se face cel mai comod prin adunări repetate: pornim cu
cifra $X$ = 0 și o incrementăm, micșorând concomitent restul, până când restul
care rămâne este prea mic. Să efectuăm aceeași împărțire, $1997:7$, considerând
că ambele numere sunt reprezentate pe tipul {\tt Huge}.

\centeredTikzFigure[
  scale=0.8,
  every node/.style={scale=0.8},
  frame/.style = {rectangle, draw=black, line width=0.2pt, minimum width=22em},
  digit/.style = {rectangle, minimum width = 2em, font=\bf\Large}
]{
  % box 1
  \node[frame] (box1) { 
    \begin{tabular}{ll}
      \multicolumn{2}{l}{Restul înainte de coborârea cifrei: 0} \\
      \multicolumn{2}{l}{Restul după coborârea cifrei: $10 \times 0 + 1 = 1$} \\
      $7 \times 1 = 7$ & prea mare \\
      Restul rămas: 1
    \end{tabular}
  };

  % box 2
  \node[frame] (box2) at (0,-4) { 
    \begin{tabular}{ll}
      \multicolumn{2}{l}{Restul înainte de coborârea cifrei: 1} \\
      \multicolumn{2}{l}{Restul după coborârea cifrei: $10 \times 1 + 9 = 19$} \\
      $7 \times 1 = 7$ & $19 - 7 = 12$ \\
      $7 \times \mathbf{2} = 14$ & $12 - 7 = 5$ \\
      $7 \times 3 = 21$ & prea mare \\
      Restul rămas: 5
    \end{tabular}
  };

  % box 3
  \node[frame] (box3) at (10.5,-6.2) { 
    \begin{tabular}{ll}
      \multicolumn{2}{l}{Restul înainte de coborârea cifrei: 5} \\
      \multicolumn{2}{l}{Restul după coborârea cifrei: $10 \times 5 + 9 = 59$} \\
      $7 \times 1 = 7$ & $59 - 7 = 52$ \\
      $\cdots$ \\
      $7 \times \mathbf{8} = 56$ & $10 - 7 = 3$ \\
      $7 \times 9 = 63$ & prea mare \\
      Restul rămas: 3
    \end{tabular}
  };

  % box 4
  \node[frame] (box4) at (10.5,-1) { 
    \begin{tabular}{ll}
      \multicolumn{2}{l}{Restul înainte de coborârea cifrei: 3} \\
      \multicolumn{2}{l}{Restul după coborârea cifrei: $10 \times 3 + 7 = 37$} \\
      $7 \times 1 = 7$ & $37 - 7 = 30$ \\
      $\cdots$ \\
      $7 \times \mathbf{5} = 35$ & $9 - 7 = 2$ \\
      $7 \times 6 = 42$ & prea mare \\
      Restul rămas: 2
    \end{tabular}
  };

  % digits
  \node[digit] (d1) at (4, 3.5) {1};
  \node[digit, anchor=west] (d2) at (d1.east) {9};
  \node[digit, anchor=west] (d3) at (d2.east) {9};
  \node[digit, anchor=west] (d4) at (d3.east) {7};
  \node[digit, anchor=west] (quotient) at (d4.east) {: 7 =};
  \node[digit, anchor=north] (d5) at (d1.south) {0};  
  \node[digit, anchor=west] (d6) at (d5.east) {2};
  \node[digit, anchor=west] (d7) at (d6.east) {8};
  \node[digit, anchor=west] (d8) at (d7.east) {5};
  \node[digit, anchor=west] (remainder) at (d8.east) {rest 2};

  %arrows
  \draw[-] (d5.south) -- (d5 |- box1.north);
  \draw[-] (d6.south) |- (box2.east);
  \draw[-] (d7.south) |- (box3.west);
  \draw[-] (d8.south) -- (d8 |- box4.north);
}

Cazul cel mai defavorabil (când $X = 9$) presupune 9 scăderi și 10 comparații,
cazul cel mai favorabil (când $X = 0$) presupune numai o comparație, deci
cazul mediu presupune 4 scăderi și 5 comparații. Căutarea lui $X$ se poate
face și binar, prin înjumătățirea intervalului, ceea ce reduce timpul mediu de
căutare la aproximativ 3 comparații și trei înmulțiri, dar codul se complică
nejustificat de mult (de cele mai multe ori).

\begin{lstlisting}[language=C]
void DivideHuge(Huge A, Huge B, Huge C, Huge R)
/* A/B = C rest R */
{ int i;

  R[0]=0;C[0]=A[0];
  for (i=A[0];i;i--)
    { Shl(R,1);R[1]=A[i];
      C[i]=0;
      while (Sgn(B,R)!=1)
        { C[i]++;
          Subtract(R,B);
        }
    }
  while (!C[C[0]] && C[0]>1) C[0]--;
}
\end{lstlisting}

\section{Extragerea rădăcinii cubice}

Vom sări peste prezentarea algoritmului de extragere a rădăcinii pătrate, pe
care îl vom lăsa ca temă cititorului, și ne vom îndrepta atenția asupra celui
de extragere a rădăcinii cubice, care este puțin mai complicat, dar care poate
fi ușor extins pentru rădăcini de orice ordin. Problema este exact cea din
enunț, așa că vom porni de la exemplul dat. Să notăm $A = 2.097.152$ și $X =
\sqrt[3]{A}$. Cum se află $X$?

O primă variantă ar fi căutarea binară a rădăcinii, prin înjumătățirea
intervalului. Inițial se pornește cu intervalul (1,$A$), deoarece rădăcina
cubică se află undeva între 1 și A (evident, încadrarea este mai mult decât
acoperitoare; ea ar putea fi mai limitativă, dar nu ar reduce timpul de lucru
decât cu câteva iterații). La fiecare pas, intervalul va fi înjumătățit. Cum,
probabil că știți deja; se ia jumătatea intervalului, se ridică la puterea a
treia și se compară cu $A$. Dacă este mai mare, înseamnă că rădăcina trebuie
căutată în jumătatea inferioară a intervalului. Dacă este mai mică, vom
continua căutarea în jumătatea superioară a intervalului. Dacă cele două
numere sunt egale, înseamnă că am găsit tocmai ce ne interesa. Prima variantă
a pseudocodului este:

\vspace{\algskip}
\begin{algorithmic}[1]
  \STATE {\bf citește} $A$ cu $N$ cifre
  \STATE $Lo \leftarrow 1, Hi \leftarrow A, X \leftarrow 0$
  \WHILE{$X = 0$}
  \STATE $Mid \leftarrow (Lo+Hi)/2$
  \IF {$Mid3 < A$}
  \STATE $Lo \leftarrow Mid+1$
  \ELSIF{$Mid3 > A$}
  \STATE $Hi \leftarrow Mid-1$
  \ELSE
  \STATE $X \leftarrow Mid$
  \ENDIF
  \ENDWHILE
\end{algorithmic}

În cazul cel mai rău, algoritmul de mai sus efectuează $\log_2 A$ înjumătățiri
de interval, fiecare din ele presupunând o adunare, o împărțire la 2 și o
ridicare la cub. Dintre aceste operații, cea mai costisitoare este ridicarea
la cub, $O(N^2)$. Complexitatea totală este prin urmare $O(N^2 \log_2
A)$. Deoarece $A$ are ordinul de mărime $10^N$, rezultă complexitatea $O(N^3
\log 10)= O(N^3)$, adică mai proastă decât cea cerută (de altfel, un algoritm
cu această complexitate nici nu s-ar încadra în timp pentru $N$ = 1000). Dacă
timpul ne permite, trebuie să căutăm altă metodă.

În exemplul ales, să observăm că $10^6 \leq A < 10^9$, de unde deducem că
$10^2 \leq X < 10^3$. Cu alte cuvinte, $X$ are 3 cifre. În cazul general, dacă
$A$ are $N$ cifre, atunci $X$ are $\lceil N / 3 \rceil$ cifre (prin $\lceil N
/ 3 \rceil$ se înțelege „cel mai mic întreg mai mare sau egal cu $N /
3$”). Care ar putea fi prima cifră a lui $X$ ? Dacă $X$ începe cu cifra 2,
atunci $200 \leq X < 300 \implies 8.000.000 \leq 2.097.152 < 27.000.000$, ceea
ce este fals. Cu atât mai puțin poate prima cifră a lui $X$ să fie mai mare ca
2. Rezultă că prima cifră a lui $X$ este 1. De altfel, pentru a afla acest
lucru, putem să și neglijăm ultimele 6 cifre ale lui $A$. Ne interesează doar
prima cifră, cea a milioanelor, iar prima cifră a lui $X$ o alegem în așa fel
încât cubul ei să fie mai mic sau egal cu 2.

Ce putem spune despre a doua cifră? Dacă ar fi 3, atunci $130 \leq X < 140
\implies 2.197.000 \leq 2.097.152 < 2.744.000$, fals (deci cifra este cel mult
2). Dacă ar fi 1, atunci $110 \leq X < 120 \implies 1.331.000 \leq 2.097.152 <
1.728.000$, fals. Rezultă că a doua cifră este obligatoriu 2. Analog, putem
neglija ultimele trei cifre ale lui $A$, iar a doua cifră a lui $X$ este cel
mai mare $C$ pentru care $\overline{1C}^3 \leq 2097$. Pentru a afla ultima
cifră, aplicăm același raționament: Dacă ar fi 9, atunci $X$ = 129 și ar
rezulta $2.146.688 = X^3 = 2.097.152$, absurd. Dacă considerăm că cifra este
8, atunci calculul se verifică. Am aflat așadar că $X$ = 128.

Procedeul general este următorul: dându-se un număr $A$ cu $N$ cifre, îl
completăm cu zerouri nesemnificative până când $N$ se divide cu 3 (poate fi
necesar să adăugăm maxim două zerouri). Numărul de cifre semnificative ale
rădăcinii cubice este $M = N / 3$. Aflăm pe rând fiecare cifră, începând cu
cea mai semnificativă. Să presupunem că am aflat cifrele $X_{M}, X_{M-1},
\dots, X_{K+1}$. Cifra $X_K$ este cea mai mare cifră pentru care numărul
$\overline{X_{M}X_{M-1}{\dots}X_{K+1}X_{K}00{\dots}00}^3 \leq A$. Cifra $X_K$
este unică, deoarece există, în general, mai multe cifre care verifică
proprietatea cerută, dar una singură este „cea mai mare”. O a doua versiune a
pseudocodului este deci:

\vspace{\algskip}
\begin{algorithmic}[1]
\STATE {\bf citește} $A$ cu $N$ cifre
\STATE $X \leftarrow 0, T \leftarrow 0$
\WHILE{$N \bmod 3 \neq 0$}
\STATE adaugă un 0 nesemnificativ
\ENDWHILE
\FOR {$i = 1$ \TO $N/3$}
\STATE adaugă la $T$ următoarele 3 cifre din $A$
\STATE adaugă la $X$ cea mai mare cifră astfel încât $X^3 \leq T$
\ENDFOR
\end{algorithmic}

Să evaluăm complexitatea acestei versiuni. Linia 1 se execută în timp liniar,
$O(N)$. Liniile 2-5 se execută în timp constant. Linia 7 presupune adăugarea
unei cifre ($O(N)$), iar linia 8 un număr constant de ridicări la cub, așadar
înmulțiri ($O(N^2)$). Deoarece liniile 7 și 8 se execută de $N/3$ ori (linia
6), rezultă o complexitate de $O(N^3)$. Iată că nici acest algoritm nu a adus
îmbunătățiri și pare și ceva mai greu de implementat. El poate fi totuși
modificat pentru a-i scădea complexitatea la $O(N^2)$.

Principalul neajuns al său este efectuarea ridicării la cub, care se face în
$O(N^2)$. Dacă am putea să-l aflăm la fiecare pas pe $X^3$ fără a efectua
înmulțiri, adică în timp liniar, atunci întregul algoritm ar avea complexitate
pătratică. Bineînțeles, prima întrebare care vine pe buzele cititorului este
„cum să ridicăm la cub fără să facem înmulțiri?”. Să nu uităm însă ceva: că
noi nu-l cunoaștem numai pe $X$. Îl cunoaștem și pe $X$ de la pasul anterior,
care avea o cifră mai puțin (îl vom boteza $OldX$). Să presupunem că, printr-o
metodă oarecare, am reușit să-l ridicăm pe $OldX$ la puterile a doua și a
treia (și am obținut numerele $OldX2$ și $OldX3$). Cum putem, cunoscând aceste
trei numere, precum și noua cifră ce se va adăuga la sfârșitul lui $X$ (să-i
spunem $C$), să-l aflăm pe $X$, pătratul și cubul său? Nu e prea greu:

\begin{equation}
  X = 10 \cdot OldX + C
\end{equation}

\begin{align}
  \begin{split}
    X^2 & = (10 \cdot OldX + C)^2 = 100 \cdot OldX^2 + 20 \cdot C \cdot OldX + C^2 \\
    & = 100 \cdot OldX^2 + (20 \cdot C) \cdot OldX + C^2
  \end{split}
\end{align}

\begin{align}
  \begin{split}
    X^3 & = (10 \cdot OldX + C)^3 = 1000 \cdot OldX^3 + 300 \cdot C \cdot OldX^2 + 30 \cdot C^2 \cdot OldX + C^3 \\
    & = 1000 \cdot OldX^3 + (300 \cdot C) \cdot OldX^2 + (30 \cdot C^2) \cdot OldX + C^3
  \end{split}
\end{align}

Iată așadar că pentru a afla noile valori ale puterilor 1, 2 și 3 ale lui $X$,
folosindu-le pe cele vechi, nu avem nevoie decât de adunări și de înmulțiri cu
numere mici (de ordinul miilor). Toate aceste operații se fac în timp liniar,
deci am reușit să găsim un algoritm pătratic. Iată mai jos sursa C:

\begin{lstlisting}[language=C]
void FindDigit(Huge L,Huge NewL2,Huge NewL3,
     Huge OldL,Huge OldL2,Huge OldL3,Huge Target)
{ Huge Aux;

  L[1]=10;
  do
    { L[1]--;
      /* Trebuie calculat L^3. Se stiu OldL (L/10)
         si noua cifra L[1]. Deci (OldL*10+L[1])^3=?
         Se aplica binomul lui Newton. */
      AtribHuge(NewL3,OldL3);Shl(NewL3,3);
      AtribHuge(Aux,OldL2);Mult(Aux,300*L[1]);
      Add(NewL3,Aux);
      AtribHuge(Aux,OldL);Mult(Aux,30*L[1]*L[1]);
      Add(NewL3,Aux);
      AtribValue(Aux,L[1]*L[1]*L[1]);
      Add(NewL3,Aux);
    }
  while (Sgn(NewL3,Target)==1);
  /* Aceeasi operatie pentru L^2 */
  AtribHuge(NewL2,OldL2);Shl(NewL2,2);
  AtribHuge(Aux,OldL);Mult(Aux,20*L[1]);
  Add(NewL2,Aux);
  AtribValue(Aux,L[1]*L[1]);
  Add(NewL2,Aux);
  /* Noile valori devin 'vechi' */
  AtribHuge(OldL2,NewL2);
  AtribHuge(OldL,L);
  AtribHuge(OldL3,NewL3);
}

void CubeRoot(Huge A, Huge X)
{ Huge Target,OldX,OldX2,OldX3,NewX2,NewX3;
  int i;

  /* Se initializeaza vectorii cu 0 (nici o cifra) */
  OldX[0]=OldX2[0]=OldX3[0]=X[0]=0;
  for (i=1;i<=(A[0]+2)/3;i++)
    { AtribHuge(Target,A);
      Shr(Target,3*((A[0]+2)/3-i));
      Shl(X,1);
      FindDigit(X,NewX2,NewX3,OldX,OldX2,OldX3,Target);
    }
}
\end{lstlisting}

Acum nu mai avem decât să scriem rutinele de intrare/ieșire și programul
principal:

\begin{lstlisting}[language=C]
#include <stdio.h>
#include <mem.h>
#define NMax 1000
typedef int Huge[NMax+3];
Huge A,X; /* A[0] si X[0] indica numarul de cifre */

void ReadData(void)
{ FILE *F=fopen("input.txt","rt");
  int C,i;

  A[0]=0;
  do A[++A[0]]=(C=fgetc(F))-'0';
  while (C!=EOF);
  A[0]--;
  fclose(F);
  /* Intoarce vectorul pe dos */
  for (i=1;i<=A[0]/2;i++)
    A[i]=(A[i]^A[A[0]+1-i])^(A[A[0]+1-i]=A[i]);
}

void WriteSolution(void)
{ FILE *F=fopen("output.txt","wt");
  int i=X[0];

  while (!X[i]) i--;
  while (i) fputc(X[i--]+'0',F);
  fclose(F);
}

void main(void)
{
  ReadData();
  CubeRoot(A,X);
  WriteSolution();
}
\end{lstlisting}

Pentru a extinde această metodă la rădăcini de orice ordin $K$, trebuie numai
să ținem cont de expresia binomului lui Newton:

\begin{align}
  \begin{split}
    X^p & = (10 \cdot OldX + C)^p \\
    & = \sum_{i=0}^{p} \mathbf{C}_p^i \cdot 10^i \cdot OldX^i \cdot C^{p-i}
  \end{split}
\end{align}

Presupunând că avem calculate toate puterile de la 1 la $p$ ale lui $OldX$, se
poate calcula noua valoare a lui $X^p$ folosind numai adunări și înmulțiri cu
scalari. În felul acesta se pot calcula în timp liniar valorile lui $X, X^2,
X^3, \dots, X^K.$

  \chapter{Lucrul cu structuri mari de date}

De multe ori în practică, atât la concursuri cât și atunci când scriem
programe care vehiculează un volum mai mare de date, avem nevoie de structuri
de date de mari dimensiuni. După cum se știe, însă, compilatorul Borland
Pascal nu permite definirea de structuri de date mai mari de 64 KB. Ce facem
dacă, spre exemplu, avem nevoie de un vector cu sute de mii de elemente sau de
o matrice pătratică de $400 \times 400$ elemente ?

O primă soluție este să schimbăm limbajul de programare folosit și să ne
orientăm spre C / C++ sau alte limbaje în care gestiunea datelor voluminoase
se face mai comod pentru utilizator. De fapt, programele scrise „la domiciliu”
se scriu mai degrabă în C decât în Pascal, deoarece codul generat este mai
eficient. Din nefericire, compilatorul de C este destul de lent, cel puțin în
comparație cu cel de Pascal și, deși există destui concurenți care lucrează în
C la olimpiadă, Pascal-ul mi se pare o alegere mai bună atunci când timpul de
implementare contează decisiv.

În aceste condiții, se impune găsirea unor modalități de a ne supune rigorilor
limbajului Pascal și de a „înghesui” cumva datele în memorie. Chiar dacă
dispunem de memorie extinsă, segmentarea la 64 KB a datelor ridică destul de
multe probleme. Vom trata pe rând câteva cazuri.

\section{Vectori de tip boolean de mari dimensiuni}

În Borland Pascal, variabilele de tip logic ({\tt Boolean}) se reprezintă pe
un octet. Se știe însă că variabilele booleene pot lua doar două valori, deci
un singur bit ar fi suficient pentru a le reprezenta. Motivul acestei aparente
„risipe” de memorie este viteza de rulare a programului. Regiștrii lucrează la
nivel de octet, iar operațiile la nivel de bit sunt mai costisitoare din punct
de vedere al timpului. În plus, cei șapte biți care ar fi economisiți nu ar
putea fi folosiți decât cel mult pentru alte variabile booleene.

În unele cazuri, însă, comprimarea variabilelor logice la un singur bit este
necesară, această misiune revenindu-i programatorului. Iată un exemplu de
problemă de concurs:

{\bf ENUNȚ}: Se dau $N - 1$ numere naturale distincte cuprinse între 1 și
$N$. Să se tipărească cel de-al $N$-lea număr (cel care lipsește).

{\bf Intrarea} se face din fișierul {\tt INPUT.TXT}, care conține pe prima
linie valoarea lui $N$ ($N \leq 500.000$), iar pe următoarele $N - 1$ linii
câte un număr cuprins între 1 și $N$.

{\bf Ieșirea}: pe ecran se va tipări numărul care lipsește.

{\bf Exemplu}: Pentru fișierul de intrare:

\begin{verbatim}
  5
  3
  2
  5
  1
\end{verbatim}

rezultatul tipărit pe ecran va fi 4.

{\bf Complexitate cerută}: $O(N)$.

{\bf Timp de implementare}: 30 minute.

{\bf REZOLVARE}: O soluție foarte elegantă a problemei este următoarea: se
știe că suma primelor $N$ numere naturale este

\begin{equation}
  S_N = \frac{N(N + 1)}{2}
\end{equation}

Se calculează suma $S'$ a numerelor din fișierul de intrare, iar numărul lipsă
este $K = S_N - S'$. Această metodă are un dezavantaj care o face
inutilizabilă: $S_{500.000}$ este aproximativ $125 \cdot 10^9$, adică un număr
mult prea mare chiar și pentru tipul de date {\tt Longint}. Ar trebui să se
memoreze numerele foarte mari pe un vector de cifre, apoi ar trebui scrise
funcțiile pentru adunarea și scăderea unor asemenea numere (vezi capitolul al
II-lea), lucru destul de incomod dacă se ține seama de timpul de implementare.

Pentru a memora tot vectorul în memorie este nevoie de $500.000 \times 4 =
2.000.000$ octeți, adică foarte mult, iar găsirea valorii care lipsește ar
presupune în cel mai bun caz o sortare în $O(N \log N)$, care ar depăși
complexitatea cerută. Ar mai fi soluția de a căuta pe rând fiecare număr în
fișier și de a tipări primul număr pe care nu-l găsim. În cazul cel mai rău,
însă, algoritmul ar face $N$ parcurgeri ale fișierului de intrare și $N^2$
comparații, deci ar depăși de asemenea complexitatea cerută, ca să nu mai
vorbim de timpul de rulare.

Rezolvarea cea mai la îndemână este de a construi un vector de variabile
booleene cu $N$ elemente. Se face apoi citirea datelor și se bifează în vector
fiecare număr citit. Apoi se parcurge vectorul și se caută singurul element
nebifat. Această versiune face o singură parcurgere a fișierului, adică
minimul posibil.

Mai rămâne de văzut cum operăm cu un vector de 500.000 de variabile
booleene. Dacă fiecare variabilă ar ocupa un octet, necesarul de memorie ar fi
de 500.000 de octeți, care sunt greu de găsit în memoria de bază. Dacă însă
alocăm câte un bit pentru fiecare element, necesarul de memorie este 500.000 /
8 = 62.500 octeți, adică o sumă rezonabilă care, mai mult, încape într-un
singur segment de memorie și poate fi alocată static fără probleme. Cum se
poate accesa și modifica valoarea unui bit din acest vector comprimat? Vom
numerota vectorul nostru începând de la 0, deci ultimul său element va fi
$V[62.499]$. Primii săi opt biți (biții 0...7, adică octetul $V[0]$) vor fi
variabilele logice atașate numerelor 1...8, următorii opt biți (biții 8...15,
adică octetul $V[1]$) vor fi variabilele logice atașate numerelor 9...16, și
așa mai departe. Ultimii opt biți (biții 499.992...499.999, adică octetul
$V[62.499]$) vor fi variabilele logice atașate numerelor 499.993...500.000. În
general, bitul cu numărul $X$ indică dacă numărul $X + 1$ a fost găsit sau
nu. Iată cum ar putea arăta vectorul la un moment oarecare al citirii din
fișier:

\centeredTikzFigure[
  byte/.style = {matrix of nodes, ampersand replacement=\&, anchor=west, nodes=bit},
  bit/.style = {rectangle, draw, minimum width=1em, minimum height=1em},
  bitLabel/.style = {font=\scriptsize, anchor=north, yshift=-2.5em},
]{
  % bytes and \cdots
  \matrix[byte] (m1) {
    1 \& 0 \& 0 \& 0 \& 1 \& 0 \& 1 \& 0 \\
  };
  \node[anchor=west] (byteDots) at (m1.east) {$\cdots$};
  \matrix[byte] (m2) at (byteDots.east) {
    0 \& 1 \& 0 \& 1 \& 0 \& 0 \& 1 \& 1 \\
  };
  \matrix[byte] (m3) at ([xshift=1em]m2.east) {
    1 \& 0 \& 0 \& 1 \& 0 \& 0 \& 0 \& 1 \\
  };

  % byte labels
  \node[anchor=south] at (m1.north) {$V[62.499]$};
  \node[anchor=south] at (m2.north) {$V[1]$};
  \node[anchor=south] at (m3.north) {$V[0]$};

  % bit labels
  \node[bitLabel] (l1) at (m1-1-1.south) {bitul 499.999};
  \node[bitLabel] at (m1-1-4.south east) {$\cdots$};
  \node[bitLabel] (l2) at (m1-1-8.south) {bitul 499.992};

  \node[bitLabel] (l3) at (m2-1-1.south) {bitul 15};
  \node[bitLabel] at (m2-1-4.south east) {$\cdots$};
  \node[bitLabel] (l4) at (m2-1-8.south) {bitul 8};

  \node[bitLabel] (l5) at (m3-1-1.south) {bitul 7};
  \node[bitLabel] at (m3-1-4.south east) {$\cdots$};
  \node[bitLabel] (l6) at (m3-1-8.south) {bitul 0};

  % arrows
  \draw[->] (l1.north) -- (m1-1-1.south);
  \draw[->] (l2.north) -- (m1-1-8.south);
  \draw[->] (l3.north) -- (m2-1-1.south);
  \draw[->] (l4.north) -- (m2-1-8.south);
  \draw[->] (l5.north) -- (m3-1-1.south);
  \draw[->] (l6.north) -- (m3-1-8.south);
}

Vectorul este reprezentat „pe dos”, ceea ce ar putea să pară ciudat la prima
vedere. Am făcut însă acest lucru deoarece, în cadrul octetului, biții sunt
numerotați în ordine crescătoare de la dreapta spre stânga și am ținut să
păstrăm aceeași ordine și pentru numerotarea octeților în vector.

Primul octet semnifică că numărul 1 a fost întâlnit, numerele 2, 3 și 4 nu au
fost întâlnite încă, numărul 5 a fost găsit etc. Trebuie acum să vedem cum se
face efectiv modificarea vectorului. Inițial toți biții au valoarea 0. Se
citește din fișier un număr $X$ și trebuie ca al $X-1$-lea bit să fie setat pe
1.

Mai întâi trebuie aflat în ce octet se află al $X-1$-lea bit. Se observă
imediat că în octetul $Oct = (X-1) \bdiv 8$. De exemplu, biții 0..7 se află în
octetul 0, biții 7..15 în octetul 1 ș.a.m.d. Mai e nevoie să știm al câtulea
bit este bitul nostru în cadrul octetului. El va fi pe poziția $B = (X-1)
\bmod 8$ (numărătoarea începe de la 0). În sfârșit, trebuie să setăm bitul
respectiv la valoarea 1. În problema de mai sus, singura operație necesară
este modificarea unui bit din 0 în 1. Vom trata însă cazul cel mai general,
când se cere setarea unui bit la o anumită valoare (0 sau 1) fără a se ști ce
valoare a avut el înainte.

Să pornim de la un exemplu particular urmând ca apoi să generalizăm
rezultatul. Se cere să se seteze bitul 2 al octetului $A = 00110010$ la
valoarea 1. Pentru aceasta, putem folosi o mască logică în care numai bitul 2
este setat pe 1, iar ceilalți sunt 0, adică masca $M = 00000100$. Între
octeții $A$ și $M$ se poate face acum un SAU logic:

\begin{verbatim}
  A  00110010 SAU
  M  00000100
     --------
  B  00110110
\end{verbatim}

Se observă că $B$ = $A$ cu excepția bitului 2, care a luat valoarea 1. Acest
fapt este ușor de explicat: 0 este element neutru în raport cu operația SAU
($0\; \mathrm{SAU}\; X = X,\; \forall X$) și deci biții de valoare 0 din $M$
nu modifică biții corespunzători din $A$. În schimb, $1\; \mathrm{SAU}\; X =
1,\; \forall X$, deci bitul 2 din $B$ va lua valoarea 1 indiferent de valoarea
bitului corespunzător din $A$.

Revenind la cazul nostru, octetul {\tt Oct} are o valoare oarecare cuprinsă
între 0 și 255, iar noi trebuie să-i setăm bitul $B$ ($0 \leq B \leq 7$) la
valoarea 1. Masca $M$ va avea toți biții de valoare 0, cu excepția bitului $B$
care va avea valoarea 1. Aceasta înseamnă ca masca $M$ are valoarea $M = 2^B$
= {\tt 1 shl B}. Operația pe care o avem de făcut este:

\begin{minted}{pascal}
V[Oct] := V[Oct] or (1 shl B)
\end{minted}

Să luăm acum un exemplu pentru a vedea cum se setează un bit oarecare la
valoarea 0. Fie $A = 01101101$. Se cere să setăm bitul 5 la valoarea 0. De
data aceasta, vom folosi o mască în care toți biții sunt 1, mai puțin al 5-lea
și vom aplica operația ȘI logic:

\begin{verbatim}
  A 01101101 ȘI
  M 11011111
    --------
  B 01001101
\end{verbatim}

Se observă că $B = A$ cu excepția bitului 5, care a trecut de la valoarea 1 la
valoarea 0. Aceasta deoarece 1 este element neutru în raport cu operația ȘI
($1\; \mathrm{\text{ȘI}}\; X = X,\; \forall X$) și deci biții de valoare 1 din
$M$ nu modifică biții corespunzători din $A$. În schimb, $0\;
\mathrm{\text{ȘI}}\; X = 0,\; \forall X$, deci bitul 5 din $B$ va lua valoarea
0 indiferent de valoarea bitului corespunzător din $A$.

Să ne întoarcem la problema noastră: trebuie setat bitul $B$ din octetul {\tt
  Oct} la valoarea 0. Pentru a construi masca $M$ remarcăm că un octet cu toți
biții de valoare 1 este egal cu 255. Din 255 trebuie să scădem ({\tt 1 shl
  B}), deci operația necesară este:

\begin{minted}{pascal}
V[Oct] := V[Oct] and (255 - (1 shl Bit))
\end{minted}

Mai avem nevoie și să testăm valoarea unui bit. Să luăm de exemplu $A =
00101111$ și să aflăm ce valoare au biții 3 și 7. Folosim măștile $M_3 =
00001000$ și $M_7 = 10000000$, aplicând de fiecare dată operația ȘI logic.

\begin{verbatim}
  A  00101111 ȘI                A   00101111 ȘI
  M3 00001000                   M7  10000000
     --------                       --------
     00001000                       00000000  
\end{verbatim}

În ce caz rezultatul poate fi diferit de 0? Șapte dintre biții măștii sunt 0,
deci biții corespunzători din B vor fi oricum 0. Cel de-al optulea depinde de
bitul respectiv din $A$: dacă acesta este 0, rezultatul va fi 0, dacă este 1,
rezultatul va fi diferit de 0. Revenind la problemă, testul care trebuie făcut
este:

\begin{minted}{pascal}
V[Oct] and (1 shl Bit) <>  0
\end{minted}

Cel mai bine este să se creeze o „interfață” care să aibă implementate cele
două funcții (setarea și testarea unui bit). După aceasta nu se va mai accesa
direct vectorul, ci numai prin intermediul acestor funcții. În felul acesta,
dacă vor apărea erori de funcționare din cauza lucrului cu vectorul, vom ști
unde să le căutăm.

Inițializarea vectorului se poate face prin două metode: una, mai elegantă,
constă în folosirea funcției de atribuire pentru fiecare element al său în
parte. A doua, mai rapidă, constă în suprascrierea cu valoarea 0 a tuturor
celor 62.500 de elemente ale vectorului $V$, printr-un acces direct al
memoriei (procedura {\tt FillChar} din Pascal).

În program, pentru creșterea vitezei, s-au făcut următoarele modificări:

\begin{itemize}

\item $(X-1) \bdiv 8$ înseamnă $(X-1) \bdiv 2^3$, adică {\tt (X-1) shr 3};

\item $(X-1) \bmod 8$ reprezintă ultimii 3 biți din reprezentarea binară a
  lui $X - 1$, adică {\tt (X-1) and 7} (deoarece 7 în binar este 00000111).

\item {\tt 255 - (1 shl A) = 255 xor (1 shl A)}. Cu alte cuvinte, schimbarea
  unui singur bit din 1 în 0 se poate face atât prin scădere, cât și printr-un
  SAU exclusiv, a doua variantă fiind mai rapidă.

\end{itemize}

\inputminted{pascal}{src/chapter3-1.pas}

\section{Vectori de dimensiuni mari cu elemente de valori mici}

Metoda descrisă la punctul anterior se poate aplica și în alte două cazuri
particulare. Dacă elementele vectorului de care avem nevoie pot lua nu doar
două valori, ci patru valori (0, 1, 2, 3), atunci ele se pot reprezenta pe doi
biți. În concluzie, într-un octet putem „înghesui” patru elemente ale
vectorului, iar pe un segment de memorie se pot reprezenta peste 250.000
elemente. În cazul în care elementele vectorului iau valori între 0 și 15, ele
se reprezintă pe 4 biți (în limba engeză, grupul de 4 biți poartă un nume
special - {\it nibble}), adică două elemente pe un octet, iar pe un segment
încap peste 125.000 elemente. Vom trata numai al doilea caz, lăsându-l pe
primul ca temă.

Vom folosi de asemenea un vector $V$ cu 62.500 elemente numerotate cu începere
de la 0. Octetul 0 va fi folosit pentru a memora primele două elemente din
vectorul dat, octetul 1 va memora al treilea și al patrulea element
etc. Octetul 62.499 va memora elementele 124.999 și 125.000. Să vedem de
exemplu cum se memorează vectorul (5, 12, 4, 11, 0, 15).

\centeredTikzFigure[
  byte/.style = {matrix of nodes, ampersand replacement=\&, anchor=west, nodes=bit},
  bit/.style = {rectangle, draw, minimum width=1em, minimum height=1em},
  byteLabel/.style = {anchor=south, yshift=0.5em},
  accolade/.style = {decorate, decoration={brace, amplitude=0.5em, raise=0.8em, mirror}},
  nlabel/.style = {below=1.5em, font=\scriptsize},
]{
  % bytes
  \matrix[byte] (m1) {
    1 \& 1 \& 1 \& 1 \& 0 \& 0 \& 0 \& 0 \\
  };
  \matrix[byte] (m2) at (m1.east) {
    1 \& 0 \& 1 \& 1 \& 0 \& 1 \& 0 \& 0 \\
  };
  \matrix[byte] (m3) at (m2.east) {
    1 \& 1 \& 0 \& 0 \& 0 \& 1 \& 0 \& 1 \\
  };

  % byte labels
  \node[byteLabel] at (m1.north) {$V[2]$};
  \node[byteLabel] at (m2.north) {$V[1]$};
  \node[byteLabel] at (m3.north) {$V[0]$};

  % nibble dividers
  \draw[line width=1pt] (m1.north) -- (m1.south);
  \draw[line width=1pt] (m2.north) -- (m2.south);
  \draw[line width=1pt] (m3.north) -- (m3.south);

  % nibble labels and decorations
  \draw[accolade] (m1-1-1.south west) to node[nlabel] {nibble 5} (m1-1-4.south east);
  \draw[accolade] (m1-1-5.south west) to node[nlabel] {nibble 4} (m1-1-8.south east);
  \draw[accolade] (m2-1-1.south west) to node[nlabel] {nibble 3} (m2-1-4.south east);
  \draw[accolade] (m2-1-5.south west) to node[nlabel] {nibble 2} (m2-1-8.south east);
  \draw[accolade] (m3-1-1.south west) to node[nlabel] {nibble 1} (m3-1-4.south east);
  \draw[accolade] (m3-1-5.south west) to node[nlabel] {nibble 0} (m3-1-8.south east);
}

Avem nevoie de aceleași operații ca și în cazul precedent:

\begin{itemize}

\item Setarea valorii unui element, indiferent de valoarea sa precedentă;

\item Aflarea valorii unui element;

\item Inițializarea vectorului.

\end{itemize}

Pentru a seta valoarea elementului cu numărul $X$ ($1 \leq X \leq 125.000$),
trebuie alterat nibble-ul cu numărul $X-1$. În ce octet se află acest nibble?
În octetul $Oct = (X-1) \bdiv 2$. Ce poziție ocupă nibble-ul în cadrul
octetului? Poziția $Nib = (X-1) \bmod 2$. Să luăm acum un caz particular și să
vedem cum se face modificarea propriu-zisă, urmând ca apoi să generalizăm.

Se dă octetul $A = 01110010$ și se cere ca nibble-ul 1 (cel din stânga) să fie
setat la valoarea 13 (în binar 1101). Se observă că nibble-ul stâng are deja o
altă valoare, deci în primul rând trebuie „curățată zona”, respectiv nibble-ul
trebuie adus la valoarea 0. Cum? Probabil ați învățat deja, cu o mască
logică. Cum toți patru biții trebuie puși pe 0, iar ceilalți patru trebuie să
rămână nealterați, folosim masca $M_1 = 00001111$ și aplicăm operatorul ȘI
logic:

\begin{verbatim}
  A   01110010 ȘI
  M1  00001111
      --------
  A’  00000010.
\end{verbatim}

Așadar nibble-ul 0 a rămas nemodificat, iar nibble-ul 1 are valoarea 0. Acum
putem aduna pur și simplu nibble-ul de valoare 13. Pentru aceasta, luăm
numărul 13 (în binar 1101) și îl deplasăm spre stânga cu 4 poziții, pentru a-l
alinia în dreptul nibble-ului 1, după care efectuăm un SAU logic (se poate
face și adunarea, dar ea este mai lentă din punct de vedere al
calculatorului).

\begin{verbatim}
  A’  00000010 SAU
  M2  11010000
      --------
  B   11010010
\end{verbatim}

Să presupunem acum că voiam să setăm nibble-ul drept la aceeași valoare,
13. Ce aveam de făcut? „Curățam” jumătatea dreaptă a octetului printr-un ȘI
logic cu masca $M_1 = 11110000$, apoi făceam un SAU logic cu nibble-ul 13:

\begin{verbatim}
  A   01110010 ȘI
  M1  11110000
      --------
  A’  01110000 SAU
  M2  00001101
      --------
  B   01111101
\end{verbatim}

Metoda generală este deci următoarea. Se construiește masca $M_1$ cu care se
setează pe 0 nibble-ul dorit. Masca se obține scăzând din octetul „plin”
11111111 (zecimal 255, hexazecimal \$FF) valoarea 1111 (zecimal 15,
hexazecimal \$0F), deplasată la stânga cu 4 poziții dacă {\tt Nib = 1}. O
primă formă a instrucțiunii de construire a măștii ar fi:

\begin{minted}{pascal}
if Nib=1 then M1:=$FF - ($0F shl 4)
         else M1:=$FF - $0F
\end{minted}

Pentru a evita instrucțiunea {\tt if}, destul de mare consumatoare de timp, se
poate calcula direct

\begin{minted}{pascal}
M1:=$FF xor ($0F shl (Nib shl 2))
\end{minted}

deoarece {\tt Nib shl 2} este 0 dacă $Nib = 0$ și 4 dacă $Nib = 1$. S-a
înlocuit scăderea cu operația SAU exclusiv, pentru motivul arătat la punctul
1.

Apoi se adună, după aceeași metodă, valoarea $K$ dorită ($0 \leq K \leq 15$):

\begin{minted}{pascal}
V[Oct] = (V[Oct] and ($FF xor ($0F shl (Nib shl 2))))
         or (K shl (Nib shl 2))
\end{minted}

Să vedem cum se află valoarea unui nibble. Să presupunem că se dă octetul $A =
01111010$ și se cere să se afle nibble-ul 1 (cel din stânga). Pentru aceasta,
se face un ȘI logic cu masca $M = 11110000$ (deoarece ultimii patru biți nu
interesează), iar rezultatul se deplasează spre dreapta cu 4 biți:

\begin{verbatim}
  A   01111010 ȘI
  M   11110000
      --------
  A’  01110000 >>>>
      --------
  B   00000111
\end{verbatim}

Deci nibble-ul stâng are valoarea 7. Revenind la cazul nostru, valoarea $K$ a
nibble-ului $Nib$ din octetul $Oct$ este:

\begin{minted}{pascal}
K = (V[Oct] and ($0F shl (Nib shl 2))) shr (Nib shl 2)
\end{minted}

Se recomandă și în acest caz scrierea unor funcții și folosirea vectorului $V$
numai prin intermediul acestor funcții. În cazul inițializării vectorului,
dacă toate elementele trebuie puse la valoarea 0, se poate folosi totuși
procedura {\tt FillChar}, care e mult mai rapidă decât apelarea funcțiilor
noastre pentru fiecare element în parte.

Prezentăm mai jos numai un program demonstrativ despre modul de lucru cu
aceste funcții:

\inputminted{pascal}{src/chapter3-2.pas}

\section{Alocarea dinamică a matricelor de mari dimensiuni}

Să presupunem că avem nevoie de o matrice cu 400 linii și 400 coloane cu
elemente de tip {\tt Integer}. Necesarul de memorie este deci de $400 \times
400 \times 2 = 320.000$ octeți, adică mult mai mult decât un segment. O cale
foarte comodă de a rezolva această dificultate este de a declara matricea
drept un vector de pointeri la vectori, ca în figura de mai jos:

% Outputs a row. Overwrites object IDs.
\newcommand{\ptrToArray}[2]{
  \node[leader] (h1) at (#1) {#2};
  \node[cell] (c1) at (h1.east) {};
  \matrix[mat] (m1) at (c1.east) {
    \ \& \ \& \ \\
  };
  \draw[->] (c1.center) -- (m1-1-1.west);
  \node[anchor=west] (dots1) at (m1.east) {$\cdots$};
  \node[cell] (t1) at (dots1.east) {};
}
\centeredTikzFigure[
  leader/.style = {minimum width=4em, anchor=north, yshift=-1em},
  mat/.style = {matrix of nodes, ampersand replacement=\&, anchor=west, xshift=2em, nodes=cell},
  cell/.style = {rectangle, draw, minimum width=2em, minimum height=2em, anchor=west},
  cellLabel/.style = {font=\scriptsize, anchor=west},
]{
  % first row
  \node (placeholder) {};
  \ptrToArray{placeholder}{$A[1]$}

  % column headers
  \node[anchor=south] at (m1-1-1.north) {1};
  \node[anchor=south] at (m1-1-2.north) {2};
  \node[anchor=south] at (m1-1-3.north) {3};
  \node[anchor=south] at (t1.north) {400};

  % more rows
  \ptrToArray{h1.south}{$A[2]$}
  \ptrToArray{h1.south}{$A[3]$}

  % skip
  \node[leader] (h1) at (h1.south) {$\vdots$};

  % last row
  \ptrToArray{h1.south}{$A[400]$}
}

Dimensiunea vectorului de pointeri este de $400 \times 4 = 1.600$ octeți (un
pointer se reprezintă pe 4 octeți). Dimensiunea unei linii din matrice este de
$400 \times 2 = 800$ octeți. Așadar, fiecare structură de date încape pe un
segment. Vectorii sunt alocați dinamic la intrarea în program.

Metoda este comod de implementat (practic singura diferență este că elementul
de la coordonatele $(i,j)$ din matrice nu va mai fi adresat cu {\tt A[i,j]},
ci cu {\tt A[i]\^{}[j]}) și nu este consumatoare de timp (adresarea unui
element mai presupune, pe lângă cele două incrementări datorate indicilor, și
o indirectare datorată pointerului).

Iată un exemplu demonstrativ de folosire a acestei structuri de date, care
calculează suma numerelor naturale de la 1 la 100.

\inputminted{pascal}{src/chapter3-3.pas}

\section{Fragmentarea matricelor de mari dimensiuni}

O altă soluție pentru a reprezenta în memorie structuri de date care depășesc
un segment este de a le fragmenta în bucăți mai mici, fiecare din acestea
nedepășind un segment. Pentru a accesa un element oarecare al structurii,
depistăm întâi fragmentul din care el face parte, apoi îl localizăm în cadrul
fragmentului.

Să considerăm același exemplu al unei matrice $A$ de dimensiuni $400 \times
400$ cu elemente de tip {\tt Integer}. Spațiul necesar este de 320.000 octeți,
adică mai mult de 4 segmente de câte 64 KB și mai puțin de 5. Să spunem că am
dori să fragmentăm această matrice în 5 matrice $B[0], B[1], B[2], B[3],
B[4]$, fiecare având câte 400 linii și 80 de coloane (vom numerota liniile de
la 0 la 399 și coloanele de la 0 la 79).  Cele 5 matrice se vor aloca dinamic,
fiecare încăpând pe un segment (așadar vectorul $B$ este un vector de pointeri
la matrice).

Unde se va regăsi elementul $A[i,j]$ ? Primele 80 de coloane din matricea $A$
se vor afla în matricea $B[0]$, următoarele 80 de coloane din $A$ se vor afla
în matricea $B[1]$ ș.a.m.d. Coloana $j$ a matricei $A$ se va afla prin urmare
în matricea $B[j \bdiv 80]$. Linia $i$ va fi aceeași, iar coloana pe care se
află elementul $A[i,j]$ în cadrul matricei $B[j \bdiv 80]$ este $j \bmod 80$.

Acum se vede de ce, pentru a calcula mai rapid câtul și restul împărțirilor,
este bine ca numărul de coloane la care se face fragmentarea să fie o putere a
lui 2, astfel încât operațiile $\bdiv$ și $\bmod$ să se poată înlocui
printr-un {\tt shr} și un {\tt and}. Pentru problema noastră, cea mai
rezonabilă cifră este 64, ceea ce înseamnă că avem nevoie de $\lceil 400/64
\rceil = 7$ fragmente (prin rotunjire în sus a rezultatului). De fapt, prin
alocarea a 7 segmente (numerotate de la 0 la 6) se creează $7 \times 64 = 448$
de coloane, cu 48 mai mult decât era necesar. Se pierde deci o cantitate de
memorie de $48 \times 400 \times 2 = 38.400$ octeți. În cazul în care această
memorie nu este vitală, se poate face „risipă”, câștigându-se în schimb
viteză.

Iată cum ar arăta matricea fragmentată la 64 de coloane:

\centeredTikzFigure[
  % array style
  arrayCell/.style = {rectangle, draw, minimum width=2.5em, minimum height=2.5em},
  array/.style = {
    matrix of nodes,
    ampersand replacement=\&,
    nodes=arrayCell,
  },
  % matrix style
  cell/.style = {rectangle, draw, minimum width=2.5em, minimum height=2.5em, anchor=west},
  noBorder/.style = {nodes={draw=none}},
  tall/.style = {minimum height=4em},
  wide/.style = {nodes={minimum width=4em}},
  hspace1/.style = {nodes={minimum width=4em}},
  hspace2/.style = {nodes={minimum width=8em}},
  mat/.style = {
    matrix of nodes,
    ampersand replacement=\&,
    scale=0.5,
    every node/.style={scale=0.5},
    nodes=cell,
    row 1/.style=noBorder,
    row 5/.style={noBorder, tall},
    row 7/.style=noBorder,
    column 1/.style=noBorder,
    column 5/.style={noBorder, wide},
    column 7/.style={noBorder, hspace1},
    column 11/.style={noBorder, wide},
    column 13/.style={noBorder, hspace2},
    column 17/.style={noBorder, wide},
    column 20/.style={noBorder, wide},
  },
]{
  % array
  \matrix[array] (a) {
    $B[0]$ \& $B[1]$ \& $B[2]$ \& $B[3]$ \& $B[4]$ \& $B[5]$ \& $B[6]$ \\
  };

  % matrix -- all chunks in one matrix
  \matrix[mat, anchor=north] (m) at ([yshift=-3em]a.south) {
    \  \& 0 \& 1 \& 2 \& $\cdots$ \& 63 \& \  \&
    0 \& 1 \& 2 \& $\cdots$ \& 63 \& \  \&
    0 \& 1 \& 2 \& $\cdots$ \& 15 \& 16 \& $\cdots$ \& 63 \\

    0 \& \  \& \  \& \  \& $\cdots$ \& \  \& \  \&
    \  \& \  \& \  \& $\cdots$ \& \  \& \  \&
    \  \& \  \& \  \& $\cdots$ \& \  \& \  \& $\cdots$ \& \ \\

    1 \& \  \& \  \& \  \& $\cdots$ \& \  \& \  \&
    \  \& \  \& \  \& $\cdots$ \& \  \& \  \&
    \  \& \  \& \  \& $\cdots$ \& \  \& \  \& $\cdots$ \& \ \\

    2 \& \  \& \  \& \  \& $\cdots$ \& \  \& \  \&
    \  \& \  \& \  \& $\cdots$ \& \  \& \  \&
    \  \& \  \& \  \& $\cdots$ \& \  \& \  \& $\cdots$ \& \ \\

    $\vdots$ \& $\vdots$ \& $\vdots$ \& $\vdots$ \& \  \& $\vdots$ \& \  \&
    $\vdots$ \& $\vdots$ \& $\vdots$ \& \  \& $\vdots$ \& \  \&
    $\vdots$ \& $\vdots$ \& $\vdots$ \& \  \& $\vdots$ \& $\vdots$ \& \  \& $\vdots$ \\

    399 \& \  \& \  \& \  \& $\cdots$ \& \  \& \  \&
    \  \& \  \& \  \& $\cdots$ \& \  \& \  \&
    \  \& \  \& \  \& $\cdots$ \& \  \& \  \& $\cdots$ \& \ \\

    \  \& 0 \& 1 \& 2 \& $\cdots$ \& 63 \& \  \&
    64 \& 65 \& 66 \& $\cdots$ \& 127 \& \  \&
    384 \& 385 \& 386 \& $\cdots$ \& 399 \& 400 \& $\cdots$ \& 447 \\
  };

  % some matrix borders in the background
  \begin{scope}[on background layer]
    \draw[densely dotted, fill=gray!20] (m-1-19.north west) rectangle (m-7-21.south east);
    \draw (m-2-2.north west) rectangle (m-6-6.south east);
    \draw (m-2-8.north west) rectangle (m-6-12.south east);
    \draw (m-2-14.north west) rectangle (m-6-18.south east);
    \draw (m-2-19.north west) rectangle (m-6-21.south east);
  \end{scope}

  % arrows
  \draw[->] (a-1-1.south) -- (m-1-5.north west);
  \draw[->] (a-1-2.south) -- (m-1-11.north west);
  \draw[->] (a-1-7.south) -- (m-1-17.north west);
}

Prezentăm mai jos un scurt exemplu de fragmentare, care face același lucru ca
și programul de la punctul anterior. S-au scris, ca și în cazurile precedente,
două funcții, una pentru modificarea unui element și una pentru aflarea
valorii lui. De asemenea, {\tt X div 64} a fost înlocuit peste tot cu {\tt X
  shr 6}, iar {\tt X mod 64} cu {\tt X and 63}.

\inputminted{pascal}{src/chapter3-4.pas}

  \chapter{Heap-uri și tabele de dispersie}

Vom încheia prezentarea structurilor de date mai speciale cu două structuri
care se fac folositoare în problemele de căutare și sortare. Ele nu sunt
dificil de implementat și se mulează peste orice structuri de date care conțin
multe înregistrări ce pot fi ordonate după anumite criterii.

\section{Heap-uri}

Să pornim de la o problemă interesantă mai mult din punct de vedere teoretic:

{\bf ENUNȚ}: Un vector se numește $k$-sortat dacă orice element al său se
găsește la o distanță cel mult egală cu $k$ de locul care i s-ar cuveni în
vectorul sortat. Iată un exemplu de vector 2-sortat cu 5 elemente:

\begin{align}
V & = (6 \quad 2 \quad 7 \quad 4 \quad 10) \\
V_{sortat} & = (2 \quad 4 \quad 6 \quad 7 \quad 10)
\end{align}

Se observă că elementele 4 și 6 se află la două poziții distanță de locul lor
în vectorul sortat, elementele 2 și 7 se află la o poziție distanță, iar
elementul 10 se află chiar pe poziția corectă. Distanța maximă este 2, deci
vectorul $V$ este 2-sortat. Desigur, un vector $k$-sortat este în același timp
și un vector ($k+1$)-sortat, și un vector ($k+2$)-sortat etc., deoarece, dacă
orice element se află la o distanță mai mică sau egală cu $k$ de locul
potrivit, cu atât mai mult el se va afla la o distanță mai mică sau egală cu
$k+1$, $k+2$ etc. În continuare, când vom spune că vectorul este $k$-sortat,
ne vom referi la cel mai mic $k$ pentru care afirmația este adevărată. Prin
urmare, un vector cu $N$ elemente poate fi $N$-1 sortat în cazul cel mai
defavorabil. Mai facem precizarea că un vector 0-sortat este un vector sortat
în înțelesul obișnuit al cuvântului, deoarece fiecare element se află la o
distanță egală cu 0 de locul său.

Problema este: dându-se un vector $k$-sortat cu $N$ elemente numere întregi,
se cere să-l sortăm într-un timp mai bun decât $O(N \log N)$.

{\bf Intrarea}: Fișierul {\tt INPUT.TXT} conține pe prima linie valorile lui
$N$ și $K$ ($2 \leq K < N$ și $N \leq 10000$), despărțite printr-un spațiu. Pe
a doua linie se dau cele $N$ elemente ale vectorului, despărțite prin spații.

{\bf Ieșirea}: În fișierul {\tt OUTPUT.TXT} se vor tipări pe o singură linie
elementele vectorului sortat, separate prin spații.

{\bf Exemplu}:

\texttt{
  \begin{tabular}{| l | l |}
    \hline
        {\bf INPUT.TXT} & {\bf OUTPUT.TXT} \\ \hline
        \begin{tabular}[t]{l}
          5 2\\
          6 2 7 4 10
        \end{tabular}
        &
        2 4 6 7 10 \\
    \hline
  \end{tabular}
}

{\bf Timp de implementare}: 45 minute.

{\bf Timp de rulare}: 5 secunde.

{\bf Complexitate cerută}: $O(N \log K)$.

{\bf REZOLVARE}: Vom începe prin a defini noțiunea de {\it heap}. Un heap
(engl. {\it grămadă}) este un vector care poate fi privit și ca un arbore
binar, așa cum se vede în figura de mai jos:

\centeredTikzFigure[
  level/.style={sibling distance=20em/#1},
  every node/.style = {circle, draw, minimum size=2.2em},
  every label/.style = {draw=white, font=\scriptsize, text=gray, minimum size=1em},
  edge from parent/.style={draw,<-},
]{
  \node [label=north:1] {12}
  child { node [label=north:2] {10}
    child { node [label=north:4] {10}
      child { node [label=north:8] {2} }
      child { node [label=north:9] {8} }
    }
    child { node [label=north:5] {7}
      child { node [label=north:10] {1} }
      child { node [label=north:11] {4} }
    }
  }
  child { node [label=north:3] {11}
    child { node [label=north:6] {9}
      child { node [label=north:12] {3} }
      child[missing] {node {}}
    }
    child { node [label=north:7] {3} }
  };
}

Lângă fiecare nod din arbore se află câte un număr, reprezentând poziția în
vector pe care ar avea-o nodul respectiv. Pentru cazul considerat, vectorul
echivalent ar fi:

\begin{equation}
  V = (12 \quad 10 \quad 11 \quad 10 \quad 7 \quad 9 \quad
  3 \quad 2 \quad 8 \quad 1 \quad 4 \quad 3)
\end{equation}

Se observă că nodurile sunt parcurse de la stânga la dreapta și de sus în
jos. O proprietate necesară pentru ca un arbore binar să se poată numi heap
este ca toate nivelele să fie complete, cu excepția ultimului, care se
completează începând de la stânga și continuând până la un punct. De aici de
ducem că înălțimea unui heap cu $N$ noduri este

\begin{equation}
  h = \lfloor \log_2 N \rfloor
\end{equation}

(reamintim că înălțimea unui arbore este adâncimea maximă a unui nod,
considerând rădăcina drept nodul de adâncime 0). Reciproc, numărul de noduri
ale unui heap de înălțime $h$ este:

\begin{equation}
  N \in [2^h, 2^{h+1} - 1]
\end{equation}

Tot din această organizare mai putem deduce că tatăl unui nod $k>1$ este nodul
$\lceil k/2 \rceil$, iar fiii nodului $k$ sunt nodurile $2k$ și $2k+1$. Dacă
$2k=N$, atunci nodul $2k+1$ nu există, iar nodul $k$ are un singur fiu; dacă
$2k>N$, atunci nodul $k$ este frunză și nu are nici un fiu. Exemple: tatăl
nodului 5 este nodul 2, iar fiii săi sunt nodurile 10 și 11. Tatăl nodului 6
este nodul 3, iar unicul său fiu este nodul 12. Tatăl nodului 9 este nodul 4,
iar fii nu are, fiind frunză în heap.

Dar cea mai importantă proprietate a heap-ului, cea care îl face util în
operațiile de căutare, este aceea că valoarea oricărui nod este mai mare sau
egală cu valoarea oricărui fiu al său. După cum se vede mai sus, nodul 2 are
valoarea 10, iar fiii săi - nodurile 4 și 5 - au valorile 10 și respectiv
7. Întrucât operatorul $\geq$ este tranzitiv, putem trage concluzia că un nod
este mai mare sau egal decât oricare din nepoții săi și, generalizând, va
rezulta că orice nod este mai mare sau egal decât toate nodurile din
subarborele a cărui rădăcină este el.

Această afirmație nu decide în nici un fel între valorile a două noduri
dispuse astfel încât nici unul nu este descendent al celuilalt. Cu alte
cuvinte, nu înseamnă că orice nod de pe un nivel mic are valoare mai mare
decât nodurile mai apropiate de frunze. Este cazul nodului 7, care are
valoarea 3 și este mai mic decât nodul 9 de valoare 8, care este totuși așezat
mai jos în heap. În orice caz, o primă concluzie care rezultă din această
proprietate este că rădăcina are cea mai mare valoare din tot heap-ul.

Structura de heap permite efectuarea multor operații într-un timp foarte bun:

\begin{itemize}
  \item Căutarea maximului în $O(1)$;
  \item Crearea unei structuri de heap dintr-un vector oarecare în $O(N)$;
  \item Eliminarea unui element în $O(\log N)$;
  \item Inserarea unui element în $O(\log N)$;
  \item Sortarea în $O(N \log N)$
  \item Căutarea (singura care nu este prea eficientă) în $O(N)$.
\end{itemize}

Desigur, toate aceste operații se fac menținând permanent structura de heap a
arborelui, adică respectând modul de repartizare a nodurilor pe nivele și
„înălțarea” elementelor de valoare mai mare. Este de la sine înțeles că datele
nu se vor reprezenta în memorie în forma arborescentă, ci în cea
vectorială. Să le analizăm pe rând.

\subsection{Căutarea maximului}

Practic operația aceasta nu are de făcut decât să întoarcă valoarea primului
element din vector:

\begin{lstlisting}[language=C]
typedef int Heap[10001];
void Max(Heap H, int N)
{
  return H[1];
}
\end{lstlisting}

\subsection{Crearea unei structuri de heap dintr-un vector oarecare}

Pentru a discuta acest aspect, vom vorbi mai întâi despre două proceduri
specifice heap-urilor, {\it Sift} (engl. {\it a cerne}) și {\it Percolate}
(engl. {\it a se infiltra}). Să presupunem că un vector are o structură de
heap, cu excepția unui nod care este mai mic decât unul din fiii săi. Este
cazul nodului 3 din figura de mai jos, care are o valoare mai mică decât nodul
6:

\centeredTikzFigure[
  scale=0.75,
  level/.style={sibling distance=20em/#1},
  every node/.style = {scale=0.75, circle, draw, minimum size=2.2em},
  every label/.style = {draw=white, font=\scriptsize, text=gray, minimum size=1em},
  edge from parent/.style={draw,<-},
]{
  \node [label=north:1] {12}
  child { node [label=north:2] {10}
    child { node [label=north:4] {10}
      child { node [label=north:8] {2} }
      child { node [label=north:9] {8} }
    }
    child { node [label=north:5] {7}
      child { node [label=north:10] {1} }
      child { node [label=north:11] {4} }
    }
  }
  child { node[fill=gray!50] [label=north:3] {3}
    child { node [label=north:6] {11}
      child { node [label=north:12] {9} }
      child[missing] {node {}}
    }
    child { node [label=north:7] {5} }
  };
}

Ce e de făcut? Desigur, nodul va trebui coborât în arbore, iar în locul lui
vom aduce alt nod, mai exact unul dintre fiii săi. Întrebarea este: care din
fiii săi? Dacă vom aduce nodul 7 în locul lui, acesta fiind mai mic decât
nodul 6, inegalitatea se va păstra. Trebuie deci să schimbăm nodul 3 cu nodul
6:

\centeredTikzFigure[
  scale=0.75,
  level/.style={sibling distance=20em/#1},
  every node/.style = {scale=0.75, circle, draw, minimum size=2.2em},
  every label/.style = {draw=white, font=\scriptsize, text=gray, minimum size=1em},
  edge from parent/.style={draw,<-},
]{
  \node [label=north:1] {12}
  child { node [label=north:2] {10}
    child { node [label=north:4] {10}
      child { node [label=north:8] {2} }
      child { node [label=north:9] {8} }
    }
    child { node [label=north:5] {7}
      child { node [label=north:10] {1} }
      child { node [label=north:11] {4} }
    }
  }
  child { node [label=north:3] {11}
    child { node[fill=gray!50] [label=north:6] {3}
      child { node [label=north:12] {9} }
      child[missing] {node {}}
    }
    child { node [label=north:7] {5} }
  };
}

Problema nu este însă rezolvată, deoarece noul nod 6, proaspăt „retrogradat”,
este încă mai mic decât fiul său, nodul 12. De data aceasta avem un singur
fiu, deci o singură opțiune: schimbăm nodul 6 cu nodul 12 și obținem o
structură de heap corectă:

\centeredTikzFigure[
  scale=0.75,
  level/.style={sibling distance=20em/#1},
  every node/.style = {scale=0.75, circle, draw, minimum size=2.2em},
  every label/.style = {draw=white, font=\scriptsize, text=gray, minimum size=1em},
  edge from parent/.style={draw,<-},
]{
  \node [label=north:1] {12}
  child { node [label=north:2] {10}
    child { node [label=north:4] {10}
      child { node [label=north:8] {2} }
      child { node [label=north:9] {8} }
    }
    child { node [label=north:5] {7}
      child { node [label=north:10] {1} }
      child { node [label=north:11] {4} }
    }
  }
  child { node [label=north:3] {11}
    child { node [label=north:6] {9}
      child { node[fill=gray!50] [label=north:12] {3} }
      child[missing] {node {}}
    }
    child { node [label=north:7] {5} }
  };
}

Procedura {\tt Sift} primește ca parametri un heap $H$ cu $N$ noduri și un nod
$K$ și presupune că ambii subarbori ai nodului $K$ au structură de heap
corectă. Misiunea ei este să „cearnă” nodul $K$ până la locul potrivit,
interschimbând mereu nodul cu cel mai mare fiu al său până când nodul nu mai
are fii (ajunge pe ultimul nivel în arbore) sau până când fiii săi au valori
mai mici decât el.

\begin{lstlisting}[language=C]
void Sift(Heap H, int N, int K)
{ int Son;

  /* Alege un fiu mai mare ca tatal */
  if (K<<1<=N)
    { Son=K<<1;
      if (K<<1<N && H[(K<<1)+1]>H[(K<<1)])
        Son++;
      if (H[Son]<=H[K]) Son=0;
    }
    else Son=0;
  while (Son)
    { /* Schimba H[K] cu H[Son] */
      H[K]=(H[K]^H[Son])^(H[Son]=H[K]);
      K=Son;
      /* Alege un alt fiu */
      if (K<<1<=N)
        { Son=K<<1;
          if (K<<1<N && H[(K<<1)+1]>H[(K<<1)])
            Son++;
          if (H[Son]<=H[K]) Son=0;
        }
        else Son=0;
    }
}
\end{lstlisting}

Procedura {\tt Percolate} se va ocupa tocmai de fenomenul invers. Să
presupunem că un heap are o „defecțiune” în sensul că observăm un nod care are
o valoare mai mare decât tatăl său. Atunci, va trebui să interschimbăm cele
două noduri. Este cazul nodului 10 din figura care urmează. Deoarece fiul care
trebuie urcat este mai mare ca tatăl, care la rândul lui (presupunând că
restul heap-ului e corect) este mai mare decât celălalt fiu al său, rezultă că
după interschimbare fiul devenit tată este mai mare decât ambii săi
fii. Trebuie totuși să privim din nou în sus și să continuăm să urcăm nodul în
arbore fie până ajungem la rădăcină, fie până îi găsim un tată mai mare ca
el. Iată ce se întâmplă cu nodul 10:

\centeredTikzFigure[
  scale=0.75,
  level/.style={sibling distance=20em/#1},
  every node/.style = {scale=0.75, circle, draw, minimum size=2.2em},
  every label/.style = {draw=white, font=\scriptsize, text=gray, minimum size=1em},
  edge from parent/.style={draw,<-},
]{
  \node [label=north:1] {12}
  child { node [label=north:2] {7}
    child { node [label=north:4] {6}
      child { node [label=north:8] {2} }
      child { node [label=north:9] {5} }
    }
    child { node [label=north:5] {4}
      child { node[fill=gray!50] [label=north:10] {10} }
      child { node [label=north:11] {1} }
    }
  }
  child { node [label=north:3] {11}
    child { node [label=north:6] {9}
      child { node [label=north:12] {3} }
      child[missing] {node {}}
    }
    child { node [label=north:7] {3} }
  };
}

\centeredTikzFigure[
  scale=0.75,
  level/.style={sibling distance=20em/#1},
  every node/.style = {scale=0.75, circle, draw, minimum size=2.2em},
  every label/.style = {draw=white, font=\scriptsize, text=gray, minimum size=1em},
  edge from parent/.style={draw,<-},
]{
  \node [label=north:1] {12}
  child { node [label=north:2] {7}
    child { node [label=north:4] {6}
      child { node [label=north:8] {2} }
      child { node [label=north:9] {5} }
    }
    child { node[fill=gray!50] [label=north:5] {10}
      child { node [label=north:10] {4} }
      child { node [label=north:11] {1} }
    }
  }
  child { node [label=north:3] {11}
    child { node [label=north:6] {9}
      child { node [label=north:12] {3} }
      child[missing] {node {}}
    }
    child { node [label=north:7] {3} }
  };
}

\centeredTikzFigure[
  scale=0.75,
  level/.style={sibling distance=20em/#1},
  every node/.style = {scale=0.75, circle, draw, minimum size=2.2em},
  every label/.style = {draw=white, font=\scriptsize, text=gray, minimum size=1em},
  edge from parent/.style={draw,<-},
]{
  \node [label=north:1] {12}
  child { node[fill=gray!50] [label=north:2] {10}
    child { node [label=north:4] {6}
      child { node [label=north:8] {2} }
      child { node [label=north:9] {5} }
    }
    child { node [label=north:5] {7}
      child { node [label=north:10] {4} }
      child { node [label=north:11] {1} }
    }
  }
  child { node [label=north:3] {11}
    child { node [label=north:6] {9}
      child { node [label=north:12] {3} }
      child[missing] {node {}}
    }
    child { node [label=north:7] {3} }
  };
}

\begin{lstlisting}[language=C]
void Percolate(Heap H, int N, int K)
{ int Key;

  Key = H[K];
  while ((K>1) && (Key > H[K>>1]))
    { H[K]=H[K>>1];
      K>>=1;
    }
  H[K] = Key;
}
\end{lstlisting}

Acum ne putem ocupa efectiv de construirea unui heap. Am spus că procedura
{\tt Sift} presupune că ambii fii ai nodului pentru care este ea apelată au
structură de heap. Aceasta înseamnă că putem apela procedura {\tt Sift} pentru
orice nod imediat superior nivelului frunzelor, deoarece frunzele sunt arbori
cu un singur nod, și deci heap-uri corecte. Apelând procedura {\tt Sift}
pentru toate nodurile de deasupra frunzelor, vom obține deja o structură mai
organizată, asigurându-ne că pe ultimele două nivele avem de-a face numai cu
heap-uri. Apoi apelăm aceeași procedură pentru nodurile de pe al treilea nivel
începând de la frunze, apoi pentru cele de deasupra lor și așa mai departe
până ajungem la rădăcină. În acest moment, heap-ul este construit. Iată cum
funcționează algoritmul pentru un arbore total dezorganizat:

\centeredTikzFigure[
  scale=0.75,
  level/.style={sibling distance=20em/#1},
  every node/.style = {scale=0.75, circle, draw, minimum size=2.2em},
  edge from parent/.style={draw,<-},
  caption/.style = {draw=none, rectangle},
]{
  \node (t) {2}
  child { node {8}
    child { node {10}
      child { node {1} }
      child { node {5} }
    }
    child { node {3}
      child { node {4} }
      child { node {11} }
    }
  }
  child { node {6}
    child { node {7}
      child { node {9} }
      child[missing] {node {}}
    }
    child { node {12} }
  };

  \node[caption, anchor=north] at ([yshift=-11em]t.south)
       {Nivelul frunzelor este organizat};
}

\centeredTikzFigure[
  scale=0.75,
  level/.style={sibling distance=20em/#1},
  every node/.style = {scale=0.75, circle, draw, minimum size=2.2em},
  edge from parent/.style={draw,<-},
  caption/.style = {draw=none, rectangle},
]{
  \node (t) {2}
  child { node {8}
    child { node {10}
      child { node {1} }
      child { node {5} }
    }
    child { node {11}
      child { node {4} }
      child { node {3} }
    }
  }
  child { node {6}
    child { node {9}
      child { node {7} }
      child[missing] {node {}}
    }
    child { node {12} }
  };

  \node[caption, anchor=north] at ([yshift=-11em]t.south)
       {Ultimele două nivele sunt organizate};
}

\centeredTikzFigure[
  scale=0.75,
  level/.style={sibling distance=20em/#1},
  every node/.style = {scale=0.75, circle, draw, minimum size=2.2em},
  edge from parent/.style={draw,<-},
  caption/.style = {draw=none, rectangle},
]{
  \node (t) {2}
  child { node {11}
    child { node {10}
      child { node {1} }
      child { node {5} }
    }
    child { node {8}
      child { node {4} }
      child { node {3} }
    }
  }
  child { node {12}
    child { node {9}
      child { node {7} }
      child[missing] {node {}}
    }
    child { node {6} }
  };

  \node[caption, anchor=north] at ([yshift=-11em]t.south)
       {Ultimele trei nivele sunt organizate};
}
\centeredTikzFigure[
  scale=0.75,
  level/.style={sibling distance=20em/#1},
  every node/.style = {scale=0.75, circle, draw, minimum size=2.2em},
  edge from parent/.style={draw,<-},
  caption/.style = {draw=none, rectangle},
]{
  \node (t) {12}
  child { node {11}
    child { node {10}
      child { node {1} }
      child { node {5} }
    }
    child { node {8}
      child { node {4} }
      child { node {3} }
    }
  }
  child { node {9}
    child { node {7}
      child { node {2} }
      child[missing] {node {}}
    }
    child { node {6} }
  };

  \node[caption, anchor=north] at ([yshift=-11em]t.south)
       {Structură de heap};
}

\begin{lstlisting}[language=C]
void BuildHeap(Heap H, int N)
{ int i;

  for (i=N/2; i; Sift(H, N, i--));
}
\end{lstlisting}

S-a apelat căderea începând de la al $N/2$-lea nod, deoarece s-a arătat că
acesta este ultimul nod care mai are fii, restul fiind frunze. Să calculăm
acum complexitatea acestui algoritm. Un calcul sumar ar putea spune: există
$N$ noduri, fiecare din ele se „cerne” pe $O(\log N)$ nivele, deci timpul de
construcție a heap-ului este $O(N \log N)$. Totuși nu este așa. Presupunem că
ultimul nivel al heap-ului este plin. În acest caz, jumătate din noduri vor fi
frunze și nu se vor cerne deloc. Un sfert din noduri se vor afla deasupra lor
și se vor cerne cel mult un nivel. O optime din noduri se vor putea cerne cel
mult două nivele, și așa mai departe, până ajungem la rădăcina care se află
singură pe nivelul ei și va putea cădea maxim $h$ nivele (reamintim că
$h=\lfloor \log N \rfloor$). Rezultă că timpul total de calcul este dat de
formula:

\begin{equation}
  O \biggl( \sum_{k = 0}^{\lfloor \log_2 N \rfloor} k \cdot \frac{N}{2^{k + 1}} \biggr)
\end{equation}

Demonstrarea egalității se poate face făcând substituția $N=2^h$ și continuând
calculele. Se va obține tocmai complexitatea $O(2^h)$; lăsăm această
verificare ca temă cititorului.

\subsection{Eliminarea unui element}

Dacă eliminăm un element din heap, trebuie numai să refacem structura de
heap. În locul nodului eliminat s-a creat un gol, pe care trebuie să îl umplem
cu un alt nod. Care ar putea fi acela? Întrucât trebuie ca toate nivelele să
fie complet ocupate, cu excepția ultimului, care poate fi gol numai în partea
sa dreaptă, rezultă că singurul nod pe care îl putem pune în locul celui
eliminat este ultimul din heap. Odată ce l-am pus în gaura făcută, trebuie să
ne asigurăm că acest nod „de umplutură” nu strică proprietatea de heap. Deci
vom verifica: dacă nodul pus în loc este mai mare ca tatăl său, vom apela
procedura {\tt Percolate}. Altfel vom apela procedura {\tt Sift}, în
eventualitatea că nodul este mai mic decât unul din fiii săi. Iată exemplul de
mai jos:

\centeredTikzFigure[
  scale=0.75,
  level/.style={sibling distance=20em/#1},
  every node/.style = {scale=0.75, circle, draw, minimum size=2.2em},
  edge from parent/.style={draw,<-},
  caption/.style = {draw=none, rectangle},
]{
  \node {20}
  child { node {10}
    child { node {9}
      child { node {9} }
      child { node {6} }
    }
    child { node {8}
      child { node {5} }
      child { node {4} }
    }
  }
  child { node {19}
    child { node {18}
      child { node {X} }
      child[missing] {node {}}
    }
    child { node {17} }
  };
}

Să presupunem că vrem să eliminăm nodul de valoare 9, aducând în locul lui
nodul de valoare $X$. Însă $X$ poate fi orice număr mai mic sau egal cu
18. Spre exemplu, $X$ poate fi 16, caz în care va trebui urcat deasupra
nodului de valoare 10, sau poate fi 1, caz în care va trebui cernut până la
nivelul frunzelor. Deoarece căderea și urcarea se pot face pe cel mult $\log
N$ nivele, rezultă o complexitate a procedeului de $O(\log N)$.

\begin{lstlisting}[language=C]
void Cut(Heap H, int N, int K)
{ H[K] = H[N--];

  if ((K>1) && (H[K] > H[K>>1]))
    Percolate(H, N, K);
    else Sift(H, N, K)
}
\end{lstlisting}

\subsection{Inserarea unui element}

Dacă vrem să inserăm un nou element în heap, lucrurile sunt mult mai
simple. Nu avem decât să-l așezăm pe a $N$+1-a poziție în vector și apoi să-l
„promovăm” până la locul potrivit. Din nou, urcarea se poate face pe maxim
$\log N$ nivele, de unde complexitatea logaritmică.

\begin{lstlisting}[language=C]
void Insert(Heap H, int N, int Key)
{
  H[++N] = Key;
  Percolate(H, N, N);
}
\end{lstlisting}

\subsection{Sortarea unui vector (heapsort)}

Acum, că am stabilit toate aceste lucruri, ideea algoritmului de sortare vine
de la sine. Începem prin a construi un heap. Apoi extragem maximul (adică
vârful heap-ului) și refacem heap-ul. Cele două operații luate la un loc
durează $O(1) + O(\log N) = O(\log N)$. Apoi extragem din nou maximul, (care
va fi al doilea element ca mărime din vector) și refacem din nou heap-ul. Din
nou, complexitatea operației compuse este $O(\log N)$. Dacă facem acest lucru
de $N$ ori, vom obține vectorul sortat într-o complexitate de $O(N \log N)$.

Partea cea mai frumoasă a acestui algoritm, la prima vedere destul de mare
consumator de memorie, este că el nu folosește deloc memorie
suplimentară. Iată explicația: când heap-ul are $N$ elemente, vom extrage
maximul și îl vom ține minte undeva în memorie. Pe de altă parte, în locul
maximului (adică în rădăcina arborelui) trebuie adus ultimul element al
vectorului, adică $H[N]$. După această operație, heap-ul va avea $N-1$ noduri,
al $N$-lea rămânând liber. Ce alt loc mai inspirat decât acest al $N$-lea nod
ne-am putea dori pentru a stoca maximul? Practic, am interschimbat rădăcina,
adică pe $H[1]$ cu $H[N]$. Același lucru se face la fiecare pas, ținând cont
de micșorarea permanentă a heap-ului.

\begin{lstlisting}[language=C]
void HeapSort(Heap H, int N)
{ int i;

  /* Construieste heap-ul */
  for (i=N>>1; i; Sift(H, N, i--));
  /* Sorteaza vectorul */
  for (i=N; i>=2;)
    { G[1]=(G[1]^G[i])^(G[i]=G[1]);
      Sift(H, --i, 1);
    }
}
\end{lstlisting}

\subsection{Căutarea unui element}

Această operație este singura care nu poate fi optimizată (în sensul reducerii
complexității sub $O(N)$). Aceasta deoarece putem fi siguri că un nod mai mic
este descendentul unuia mai mare, dar nu putem ști dacă se află în subarborele
stâng sau drept; din această cauză, nu putem face o căutare binară. Totuși, o
oarecare îmbunătățire se poate aduce față de căutarea secvențială. Dacă
rădăcina unui subarbore este mai mică decât valoarea căutată de noi, cu atât
mai mult putem fi convinși că descendenții rădăcinii vor fi și mai mici, deci
putem să renunțăm la a căuta acea valoare în tot subarborele. În felul acesta,
se poate întâmpla ca bucăți mari din heap să nu mai fie explorate inutil. Pe
cazul cel mai defavorabil, însă, parcurgerea întregului heap este
necesară. Lăsăm scrierea unei proceduri de căutare pe seama cititorului.

\begin{center}
  {\Huge \decofourleft \decofourright}
\end{center}

Sperând că am reușit să explicăm modul de funcționare al unui heap, să
încercăm să rezolvăm și problema propusă. Chiar faptul că ni se cere o
complexitate de ordinul $O(N \log k)$ ne sugerează construirea unui heap cu
$O(k)$ noduri. Să ne închipuim că am construi un heap $H$ format din primele
$k+1$ elemente ale vectorului $V$. Diferența față de ce am spus până acum este
că orice nod va trebui să fie {\bf mai mic} decât fiii săi. Acest heap va
servi deci la extragerea minimului.

Deoarece vectorul este $k$-sortat, rezultă că elementul care s-ar găsi pe
prima poziție în vectorul sortat se poate afla în vectorul nesortat pe oricare
din pozițiile $1, 2, \cdots, k+1$. El se află așadar în heap-ul $H$; în plus,
fiind cel mai mic, știm exact de unde să-l luăm: din vârful heap-ului. Deci
vom elimina acest element din heap și îl vom trece „undeva” separat (vom vedea
mai târziu unde). În loc să punem în locul lui ultimul element din heap, însă,
vom aduce al $k+2$-lea element din vector și îl vom lăsa să se cearnă. Acum
putem fi siguri că al doilea element ca valoare în vectorul sortat se află în
heap, deoarece el se putea afla în vectorul nesortat undeva pe pozițiile $1,
2, \cdots, k+2$, toate aceste elemente figurând în heap (bineînțeles că
minimul deja extras se exclude din discuție). Putem să mergem la sigur, luând
al doilea minim direct din vârful heap-ului.

Vom proceda la fel până când toate elementele vectorului vor fi adăugate în
heap. Din acel moment vom continua să extragem din vârful heap-ului, revenind
la vechea modalitate de a umple locul rămas gol cu ultimul nod
disponibil. Continuăm și cu acest procedeu până când heap-ul se golește. În
acest moment am obținut vectorul sortat „undeva” în memorie. De fapt, dacă ne
gândim puțin, vom constata că, odată ce primele $k+1$ elemente din vector au
fost trecute în heap, ordinea lor în vectorul $V$ nu mai contează, ele putând
servi chiar la stocarea minimelor găsite pe parcurs. Pe măsură ce aceste
locuri se vor umple, altele noi se vor crea prin trecerea altor elemente în
heap. Iată deci cum nici acest algoritm nu necesită memorie suplimentară.

Să urmărim evoluția metodei pe exemplul din enunț:

\begin{equation*}
  V = (6 \quad 2 \quad 7 \quad 4 \quad 10)
\end{equation*}

\newcommand{\threeNodeHeap}[6]{
  \centeredTikzFigure[
    level/.style={sibling distance=8em},
    tnode/.style = {circle, draw, minimum size=2.2em},
    edge from parent/.style={draw,<-},
    caption/.style = {draw=none, rectangle},
  ]{
    \node (out) {#1};

    \node[tnode, fill=gray!50, anchor=west] (t) at ([xshift=8em]out.east) {#1}
    child[#2] { node[tnode] {#3} }
    child[#4] { node[tnode] {#5} };

    \node[anchor=west] (in) at ([xshift=8em]t.east) {#6};

    \draw[->] ([xshift=-5em]t.west) -- (out.east);
    \notblank{#6}{
      \draw[->] (in.west) -- ([xshift=5em]t.east);
    }{}
  }
}
\threeNodeHeap{2}{}{6}{}{7}{4}

\begin{equation*}
  V = (2 \quad | \quad 2 \quad 7 \quad 4 \quad 10)
\end{equation*}

\threeNodeHeap{4}{}{6}{}{7}{10}

\begin{equation*}
  V = (2 \quad 4 \quad | \quad 7 \quad 4 \quad 10)
\end{equation*}

\threeNodeHeap{6}{}{10}{}{7}{}

\begin{equation*}
  V = (2 \quad 4 \quad 6 \quad | \quad 4 \quad 10)
\end{equation*}

\threeNodeHeap{7}{}{10}{missing}{}{}

\begin{equation*}
  V = (2 \quad 4 \quad 6 \quad 7 \quad | \quad 10)
\end{equation*}

\threeNodeHeap{10}{missing}{}{missing}{}{}

\begin{equation*}
  V = (2 \quad 4 \quad 6 \quad 7 \quad 10)
\end{equation*}

\begin{lstlisting}[language=C]
#include <stdio.h>
#include <mem.h>
int V[10001], H[10001], N, K;

void ReadData(void)
{ FILE *F=fopen("input.txt","rt");
  int i;

  fscanf(F,"%d %d\n",&N, &K);
  for (i=1; i<=N; fscanf(F, "%d", &V[i++]));
  fclose(F);
}

void Sift(int X, int N)
/* Cerne al X-lea element dintr-un heap de N elemente */
{ int Son;

  /* Alege un fiu mai mare ca tatal */
  if (X<<1<=N)
    { Son=X<<1;
      if (X<<1<N && H[(X<<1)+1]<H[(X<<1)])
        Son++;
      if (H[Son]>=H[X]) Son=0;
    }
    else Son=0;
  while (Son)
    { /* Schimba H[X] cu H[Son] */
      H[X]=(H[X]^H[Son])^(H[Son]=H[X]);
      X=Son;
      /* Alege un alt fiu */
      if (X<<1<=N)
        { Son=X<<1;
          if (X<<1<N && H[(X<<1)+1]<H[(X<<1)])
            Son++;
          if (H[Son]>=H[X]) Son=0;
        }
        else Son=0;
    }
}

void SortVector(void)
{ int i;

  /* Construieste heap-ul de K+1 elemente */
  for (i=1; i<=K+1; H[i++]=V[i]);
  for (i=(K+1) >> 1; i; Sift(i--, K+1));

  for (i=1; i<=N; i++)
    { V[i] = H[1]; // minimul trece in vector
      /* Se adauga un element din vector sau din heap */
      H[1] = (i<=N-K-1) ? V[i+K+1] : H[K+1];
      /* Daca vectorul s-a terminat, heap-ul incepe
         sa se micsoreze */
      if (i>N-K-1) K--;
      /* Cerne noul element */
      Sift(1, K+1);
    }
}

void WriteSolution(void)
{ FILE *F=fopen("con","wt");
  int i;

  for (i=1; i<=N; fprintf(F, "%d ", V[i++]));
  fprintf(F, "\n");
  fclose(F);
}

void main(void)
{
  ReadData();
  SortVector();
  WriteSolution();
}
\end{lstlisting}

  \section{Tabele HASH}

În multe aplicații lucrăm cu structuri mari de date în care avem nevoie să
facem căutări, inserări, modificări și ștergeri. Aceste structuri pot fi
vectori, matrice, liste etc. În cazurile mai fericite ale vectorilor, aceștia
pot fi sortați, caz în care localizarea unui element se face prin metoda
înjumătățirii intervalului, adică în timp logaritmic. Chiar dacă nu avem voie
să sortăm vectorul, tot se pot face anumite optimizări care reduc foarte mult
timpul de căutare. De exemplu, probabil că mulți dintre cititori au idee
despre ce înseamnă {\it indexarea} unei baze de date. Dacă avem o bază de date
cu patru elemente de tip string, și anume

\begin{equation*}
  B = (bac, zugrav, abac, zarva)  
\end{equation*}

putem construi un vector $Ind$ care să ne indice ordinea în care s-ar cuveni
să fie așezate cuvintele în vectorul sortat. Ordinea alfabetică (din cartea de
telefon) a cuvintelor este: „abac”, „bac”, „zarva”, „zugrav”, deci vectorul
$Ind$ este:

\begin{equation*}
  Ind = (3, 1, 4, 2)
\end{equation*}

semnificând că primul cuvânt din vectorul sortat ar trebui să fie al treilea
din vectorul $B$, respectiv „abac” și așa mai departe. În felul acesta am
obținut un vector sortat, care presupune o indirectare a elementelor. Vectorul
sortat este

\begin{equation*}
B' = (B(Ind(1)), B(Ind(2)), B(Ind(3)), B(Ind(4)).
\end{equation*}

Această operație se numește indexare. Ce-i drept, construcția vectorului $Ind$
nu se poate face într-un timp mai bun decât $O(N \log N)$, dar după ce acest
lucru se face (o singură dată, la începutul programului), căutările se pot
face foarte repede. Dacă pe parcurs se fac adăugări sau ștergeri de elemente
în/din baza de date, se va pierde câtva timp pentru menținerea indexului, dar
în practică timpul acesta este mult mai mic decât timpul care s-ar pierde cu
căutarea unor elemente în cazul în care vectorul ar fi neindexat. Nu vom intra
în detalii despre indexare, deoarece nu acesta este obiectul capitolului de
față.

În unele situații nu se poate face nici indexarea structurii de date. Să
considerăm cazul unui program care joacă șah. În esență, modul de funcționare
al acestui program se reduce la o rutină care primește o poziție pe tablă și o
variabilă care indică dacă la mutare este albul sau negrul, rutina întorcând
cea mai bună mutare care se poate efectua din acea poziție. Majoritatea
programelor de șah încep să {\it expandeze} respectiva poziție, examinând tot
felul de variante ce pot decurge din ea și alegând-o pe cea mai promițătoare,
așa cum fac și jucătorii umani. Pozițiile analizate sunt stocate în memorie
sub forma unei liste simplu sau dublu înlănțuite. Memorarea nu se poate face
sub forma unui vector, deoarece numărul de poziții analizate este de ordinul
sutelor de mii sau chiar al milioanelor, din care câteva zeci de mii sunt
reținute în permanență în memorie.

Să ne închipuim acum următoarea situație. Este posibil ca, prin expandarea
unei configurații inițiale a tablei să se ajungă la aceeași configurație
finală pe două căi diferite. Spre exemplu, dacă albul mută întâi calul la f3,
apoi nebunul la c4, poziția rezultată va fi aceeași ca și când s-ar fi mutat
întâi nebunul și apoi calul (considerând bineînțeles că negrul dă în ambele
situații aceeași replică). Dacă configurația finală a fost deja analizată
pentru prima variantă, este inutil să o mai analizăm și pentru cea de-a doua,
pentru că rezultatul (concluzia la care se va ajunge) va fi exact același. Dar
cum își poate da programul seama dacă poziția pe care are de gând s-o
analizeze a fost analizată deja sau nu?

Cea mai simplă metodă este o scanare a listei de configurații examinate din
memorie. Dacă în această listă se află poziția curentă de analizat, înseamnă
că ea a fost deja analizată și vom renunța la ea. Dacă nu, o vom analiza
acum. Ideea în mare a algoritmului este:

\vspace{\algskip}
\begin{algorithm}
  \floatname{algorithm}{Procedura}
  \caption{Analizează(Poziție $P$)}
  \begin{algorithmic}[1]
    \STATE caută $P$ în lista de poziții deja analizate
    \IF{$P$ nu există în listă}
    \STATE expandează $P$ și află cea mai bună mutare $M$
    \STATE adaugă $P$ la lista de poziții analizate
    \RETURN $M$
    \ELSE
    \RETURN valoarea $M$ atașată poziției $P$ găsite în listă
    \ENDIF
  \end{algorithmic}  
\end{algorithm}

Nu vom insista asupra a cum se expandează o poziție și cum se calculează
efectiv cea mai bună mutare. Noi ne vom interesa de un singur aspect, și anume
căutarea unei poziții în listă. Tehnica cea mai „naturală” este o parcurgere a
listei de la cap la coadă, comparând pe rând poziția căutată cu fiecare
poziție din listă. Dacă lista are memorate $N$ poziții, atunci în cazul unei
căutări cu succes (poziția este găsită), numărul mediu de comparații făcute
este $N/2$, iar numărul cel mai defavorabil ajunge până la $N$. În cazul unei
căutări fără succes (poziția nu există în listă), numărul de comparații este
întotdeauna $N$. De altfel, cazul căutării fără succes este mult mai frecvent
pentru problema jocului de șah, unde numărul de poziții posibile crește
exponențial cu numărul de mutări. Același număr de comparații îl presupun și
ștergerea unei poziții din listă (care presupune întâi găsirea ei) și
adăugarea (care presupune ca poziția de adăugat să nu existe deja în listă).

Pentru îmbunătățirea {\bf practică} a acestui timp sunt folosite {\bf tabelele
  de dispersie} sau {\bf tabelele hash} (engl. {\it hash = a toca,
  tocătură}). Menționăm de la bun început că tabelele hash nu au nici o
utilitate din punct de vedere teoretic. Dacă suntem rău intenționați, este
posibil să găsim exemple pentru care căutarea într-o tabelă hash să dureze la
fel de mult ca într-o listă simplu înlănțuită, respectiv $O(N)$. Dar în
practică timpul căutării și al adăugării de elemente într-o tabelă hash
coboară uneori până la $O(1)$, iar în medie scade foarte mult (de mii de ori).

Iată despre ce este vorba. Să presupunem pentru început că în loc de poziții
pe tabla de șah, lista noastră memorează numere între 0 și 999. În acest caz,
tabela hash ar fi un simplu vector $H$ cu 1000 de elemente booleene. Inițial,
toate elementele lui $H$ au valoarea {\tt False} (sau 0). Dacă numărul 473 a
fost găsit în listă, nu avem decât să setăm valoarea lui $H(473)$ la {\tt
  True} (sau 1). La o nouă apariție a lui 473 în listă, vom examina elementul
$H(473)$ și, deoarece el este {\tt True}, înseamnă că acest număr a mai fost
găsit. Dacă dorim ștergerea unui element din hash, vom reseta poziția
corespunzătoare din $H$. Practic, avem de-a face cu un exemplu rudimentar de
ceea ce se cheamă {\bf funcție de dispersie}, aidcă $h(x) = x$. O proprietate
foarte importantă a acestei funcții este injectivitatea; este imposibil ca la
două numere distincte să corespundă aceeași intrare în tabelă. Să încercăm o
reprezentare grafică a metodei:

\centeredTikzFigure[
  scale=0.65,
  every node/.style = {scale=0.65},
  header/.style={font=\Large},
  mat/.style = {
    matrix of nodes,
    ampersand replacement=\&,
    anchor=north,
  },
]{
  % left-hand list
  \matrix[
    mat,
    nodes={minimum width=2.5em, minimum height=2.5em},
    values/.style={
      row #1/.style={
        nodes={rectangle, draw}
      },
    },
    values/.list={1, 4, 7, 10},
    pointers/.style={
      row #1/.style={
        nodes={rectangle, draw, minimum height=1.5em}
      },
    },
    pointers/.list={2, 5, 8, 11},
  ] (l) {
    2 \\
    \ \\
    \ \\
    4 \\
    \ \\
    \ \\
    1 \\
    \ \\
    \ \\
    2 \\
    \ \\
    \ \\
    $\vdots$ \\
  };

  \matrix[
    mat,
    column 1/.style={nodes={rectangle, draw, font=\Large, minimum width=3.5em, minimum height=3.5em}},
    row 6/.style={nodes={draw=none}},
  ] (a) at (5, 0) {
    0 \& 0\\
    1 \& 1 \\
    1 \& 2 \\
    0 \& 3 \\
    1 \& 4 \\
    $\vdots$ \& \  \\
    0 \& 998 \\
    0 \& 999 \\
  };

  % headers
  \node[header, anchor=south] at ([yshift=1em]l-1-1.north) {L};
  \node[header, anchor=south] at ([yshift=1em]a-1-1.north) {H};

  % pointers
  \draw[->] (l-2-1.center) -- (l-4-1.north);
  \draw[->] (l-5-1.center) -- (l-7-1.north);
  \draw[->] (l-8-1.center) -- (l-10-1.north);
  \draw[->] (l-11-1.center) -- (l-13-1.north);

  % arrows from l to a
  \draw[->,dotted] (l-1-1.east) -- (a-3-1.west);
  \draw[->,dotted] (l-4-1.east) -- (a-5-1.west);
  \draw[->,dotted] (l-7-1.east) -- (a-2-1.west);
  \draw[->,dotted] (l-10-1.east) -- (a-3-1.west);
}

Iată primul set de proceduri de gestionare a unui Hash.

\begin{minted}{c}
#define M 1000 // numarul de "intrari" //
typedef int Hash[M];
typedef int DataType;
Hash H;

void InitHash1(Hash H)
{ int i;

  for (i=0; i<M; H[i++]=0);
}

inline int h(DataType K)
{
  return K;
}

int Search1(Hash H, DataType K)
/* Intoarce -1 daca elementul nu exista in hash
   sau indicele in hash daca el exista */
{
  return H[h(K)] ? h(K) : -1;
}

void Add1(Hash H, DataType K)
{
  H[h(K)]=1;
}

void Delete1(Hash H, DataType K)
{
  H[h(K)]=0;
}
\end{minted}

Prin „număr de intrări” în tabelă se înțelege numărul de elemente ale
vectorului $H$; în general, orice tabelă hash este un vector. Pentru ca
funcțiile să fie cât mai generale, am dat tipului de dată {\tt int} un nou
nume - {\tt DataType}. În principiu, tabelele Hash se aplică oricărui tip de
date. În realitate, fenomenul este tocmai cel invers: orice tip de date
trebuie „convertit” printr-o metodă sau alta la tipul de date {\tt int}, iar
funcția de dispersie primește ca parametru un întreg. Funcțiile hash
prezentate în viitor nu vor mai lucra decât cu variabile de tip întreg. Vom
vorbi mai târziu despre cum se poate face conversia. Acum să generalizăm
exemplul de mai sus.

Într-adevăr, cazul anterior este mult prea simplu. Să ne închipuim de pildă că
în loc de numere mai mici ca 1000 avem numere de până la 2.000.000.000. În
această situație posibilitatea de a reprezenta tabela ca un vector
caracteristic iese din discuție. Numărul de intrări în tabelă este de ordinul
miilor, cel mult al zecilor de mii, deci cu mult mai mic decât numărul total
de chei (numere) posibile. Ce avem de făcut? Am putea încerca să adăugăm un
număr $K$ într-o tabelă cu $M$ intrări (numerotate de la 0 la $M-1$) pe
poziția $K \bmod M$, adică $h(K)=K \bmod M$. Care va fi însă rezultatul?
Funcția $h$ își va pierde proprietatea de injectivitate, deoarece mai multor
chei le poate corespunde aceeași intrare în tabelă, cum ar fi cazul numerelor
1234 și 2001234, ambele dând același rest la împărțirea prin $M=1000$. Nu
putem avea însă speranța de a găsi o funcție injectivă, pentru că atunci
numărul de intrări în tabelă ar trebui să fie cel puțin egal cu numărul de
chei distincte. Vrând-nevrând, trebuie să rezolvăm {\bf coliziunile} (sau {\bf
  conflictele}) care apar, adică situațiile când mai multe chei distincte sunt
repartizate la aceeași intrare.

Vom reveni ulterior la oportunitatea alegerii funcției modul și a numărului de
1000 de intrări în tabelă. Deocamdată vom folosi aceste date pentru a explica
modul de funcționare a tabelei hash pentru funcții neinjective. Să presupunem
că avem două chei $K_1$ și $K_2$ care sunt repartizate de funcția $h$ la
aceeași intrare $X$, adică $h(K_1) = h(K_2) = X$. Soluția cea mai comodă este
ca $H(X)$ să nu mai fie un număr, ci o listă liniară simplu sau dublu
înlănțuită care să conțină toate cheile găsite până acum și repartizate la
aceeași intrare $X$. Prin urmare vectorul $H$ va fi un vector de liste:

\centeredTikzFigure[
  scale=0.65,
  every node/.style = {scale=0.65},
  header/.style={font=\Large},
  mat/.style = {
    matrix of nodes,
    ampersand replacement=\&,
    anchor=north,
  },
]{
  % left-hand list
  \matrix[
    mat,
    nodes={minimum width=3.5em, minimum height=2.5em},
    values/.style={
      row #1/.style={
        nodes={rectangle, draw}
      },
    },
    values/.list={1, 4, 7, 10},
    pointers/.style={
      row #1/.style={
        nodes={rectangle, draw, minimum height=1.5em}
      },
    },
    pointers/.list={2, 5, 8, 11},
  ] (l) {
    1002 \\
    \ \\
    \ \\
    4 \\
    \ \\
    \ \\
    5001 \\
    \ \\
    \ \\
    3002 \\
    \ \\
    \ \\
    $\vdots$ \\
  };

  \matrix[
    mat,
    column 1/.style={nodes={rectangle, draw, font=\Large, minimum width=3.5em, minimum height=3.5em}},
    column 3/.style={nodes={minimum width=4em}},
    row 6/.style={nodes={draw=none}},
    cell/.style={rectangle, draw, minimum height=2.5em, anchor=west},
    value/.style={minimum width=2.5em, xshift=2em},
    ptr/.style={minimum width=1.5em},
  ] (a) at (10, 0) {
    $\bullet$ \& \ \& 0\\
    \  \&
    \node[cell, value] (c5001) {5001};
    \node[cell, ptr] (p5001) at (c5001.east) {};
    \& 1 \\
    \  \&
    \node[cell, value] (c1002) {1002};
    \node[cell, ptr] (p1002) at (c1002.east) {};
    \node[cell, value] (c3002) at (p1002.east) {3002};
    \node[cell, ptr] (p3002) at (c3002.east) {};
    \& 2 \\
    $\bullet$ \& \ \& 3 \\
    \  \&
    \node[cell, value] (c4) {4};
    \node[cell, ptr] (p4) at (c4.east) {};
    \& 4 \\
    $\vdots$ \& \ \& $\vdots$  \\
    $\bullet$ \& \ \& 998 \\
    $\bullet$ \& \ \& 999 \\
  };

  % headers
  \node[header, anchor=south] at ([yshift=1em]l-1-1.north) {L};
  \node[header, anchor=south] at ([yshift=1em]a-1-1.north) {H};

  % l pointers
  \draw[->] (l-2-1.center) -- (l-4-1.north);
  \draw[->] (l-5-1.center) -- (l-7-1.north);
  \draw[->] (l-8-1.center) -- (l-10-1.north);
  \draw[->] (l-11-1.center) -- (l-13-1.north);
  
  % a pointers
  \draw[->] (a-2-1.center) -- (c5001.west);
  \draw[->] (a-3-1.center) -- (c1002.west);
  \draw[->] (p1002.center) -- (c3002.west);
  \draw[->] (a-5-1.center) -- (c4.west);

  % arrows from l to a
  \draw[->,dotted] (l-1-1.east) -- (a-3-1.west);
  \draw[->,dotted] (l-4-1.east) -- (a-5-1.west);
  \draw[->,dotted] (l-7-1.east) -- (a-2-1.west);
  \draw[->,dotted] (l-10-1.east) -- (a-3-1.west);

  \node[anchor=west, font=\Large] at ([yshift=-2em]l.south west) {$h(x) = x \bmod M$};
}

Să analizăm acum complexitatea noilor proceduri de căutare, adăugare și
ștergere. Căutarea nu se va mai face în toată lista, ci numai în lista
corespunzătoare din $H$. Altfel spus, o cheie $K$ se va căuta numai în lista
$H(h(K))$, deoarece dacă cheia $K$ a mai apărut, ea a fost în mod sigur
repartizată la intrarea $H(h(K))$. De aceea, căutarea poate ajunge, în cazul
cel mai defavorabil când toate cheile din listă se repartizează la aceeași
intrare în hash, la o complexitate $O(N)$. Dacă reușim însă să găsim o funcție
care să distrbuie cheile cât mai aleator, timpul de intrare se va reduce de
$M$ ori. Avantajele sunt indiscutabile pentru $M=10000$ de exemplu.

Întrucât operațiile cu liste liniare sunt în general cunoscute, nu vom insista
asupra lor. Prezentăm aici numai adăugarea și căutarea, lăsându-vă ca temă
scrierea funcției de ștergere din tabelă.

\begin{minted}{c}
#include <stdio.h>
#include <stdlib.h>
#define M 1000 // numarul de "intrari"
typedef struct _List {
          long P;
          struct _List * Next;
        } List;
typedef List * Hash[M];
Hash H;

void InitHash2(Hash H)
{ int i;

  for (i=0; i<M; H[i++]=NULL);
}

int h2(int K)
{
  return K%M;
}

int Search2(Hash H, int K)
/* Intoarce 0 daca elementul nu exista in hash
   sau 1 daca el exista */
{ List *L;

  for (L=H[h2(K)]; L && (L->P != K); L = L->Next);
  return L!=NULL;
}

void Add2(Hash H, int K)
{ List *L = malloc(sizeof(List));
  L->P = K;
  L->Next = H[h2(K)];
  H[h2(K)] = L;
}
\end{minted}

Am spus că funcțiile de dispersie sunt concepute să lucreze numai pe date de
tip întreg; celelalte tipuri de date trebuie convertite în prealabil la tipuri
de date întregi. Iată câteva exemple:

\begin{itemize}

\item Variabilele de tip string pot fi transformate în numere în baza 256 prin
  înlocuirea fiecărui caracter cu codul său ASCII. De exemplu, șirul „abac”
  poate fi privit ca un număr de 4 cifre în baza 256, și anume numărul (97 98
  97 99). Conversia lui în baza 10 se face astfel:

  \begin{equation}
    X = ((97 \times  256 + 98) \times 256 + 97) \times 256 + 99 = 1.633.837.411
  \end{equation}


  Pentru stringuri mai lungi, rezultă numere mai mari. Uneori, ele nici nu mai
  pot fi reprezentate cu tipurile de date ordinale. Totuși, acest dezavantaj
  nu este supărător, deoarece majoritatea funcțiilor de dispersie presupun o
  împărțire cu rest, care, indiferent de mărimea numărului de la intrare,
  produce un număr controlabil.

\item Variabilele de tip dată se pot converti la întreg prin formula:

  \begin{equation}
    A \times 366 + L \times 31 + Z
  \end{equation}

  unde $A$, $L$ și $Z$ sunt respectiv anul, luna și ziua datei considerate. De
  fapt, această funcție aproximează numărul de zile scurse de la începutul
  secolului I. Ea nu are pretenții de exactitate (ca dovadă, toți anii sunt
  considerați a fi bisecți și toate lunile a avea 31 de zile), deoarece s-ar
  consuma timp inutil cu calcule mai sofisticate, fără ca dispersia însăși să
  fie îmbunătățită cu ceva. Condiția care trebuie neapărat respectată este ca
  funcția de conversie dată $\leftrightarrow$ întreg să fie injectivă, adică
  să nu se întâmple ca la două date $D_1$ și $D_2$ să li se atașeze același
  întreg $X$; dacă acest lucru se întâmplă, pot apărea erori la căutarea în
  tabelă (de exemplu, se poate raporta găsirea datei $D_1$ când de fapt a fost
  găsită data $D_2$). Pentru a respecta injectivitatea, s-au considerat
  coeficienții 366 și 31 în loc de 365 și 30. Dacă numărul de zile scurse de
  la 1 ianuarie anul 1 d.H. ar fi fost calculat cu exactitate, funcția de
  conversie ar fi fost și surjectivă, dar, după cum am mai spus, acest fapt nu
  prezintă interes.

\item Analog, variabilele de tip oră se pot converti la întreg cu formula:

  \begin{equation}
    X = (H \times 60 + M) \times 60 + S
  \end{equation}

  unde $H$, $M$ și $S$ sunt respectiv ora, minutul și secunda considerate, sau
  cu formula

  \begin{equation}
    X = ((H \times 60 + M) \times 60 + S) \times 100 + C
  \end{equation}

  dacă se ține cont și de sutimile de secundă. De data aceasta, funcția este
  surjectivă (oricărui număr întreg din intervalul 0 - 8.639.999 îi corespunde
  în mod unic o oră).

\item În majoritatea cazurilor, datele sunt structuri care conțin numere și
  stringuri. O bună metodă de conversie constă în alipirea tuturor acestor
  date și în convertirea la baza 256. Caracterele se convertesc prin simpla
  înlocuire cu codul ASCII corespunzător, iar numerele prin convertirea în
  baza 2 și tăierea în „bucăți” de câte opt biți. Rezultă numere cu multe
  cifre (prea multe chiar și pentru tipul {\tt longint}), care sunt supuse
  unei operații de împărțire cu rest. Funcția de conversie trebuie să fie
  injectivă. De exemplu, în cazul tablei de șah despre care am amintit mai
  înainte, ea poate fi transformată într-un vector cu 64 de cifre în baza 16,
  cifra 0 semnificând un pătrat gol, cifrele 1-6 semnificând piesele albe
  (pion, cal, nebun, turn, regină, rege) iar cifrele 7-12 semnificând piesele
  negre. Prin trecerea acestui vector în baza 256, rezultă un număr cu 32 de
  cifre. La acesta se mai pot adăuga alte cifre, respectiv partea la mutare (0
  pentru alb, 1 pentru negru), posibilitatea de a efectua rocada mică/mare de
  către alb/negru, numărul de mutări scurse de la începutul partidei și așa
  mai departe.

\end{itemize}

Vom termina prin a prezenta două funcții de dispersie foarte des folosite.

\subsection{Metoda împărțirii cu rest}

Despre această metodă am mai vorbit. Funcția hash este

\begin{equation}
  h(x)=x \bmod M
\end{equation}

unde $M$ este numărul de intrări în tabelă. Problema care se pune este să-l
alegem pe $M$ cât mai bine, astfel încât numărul de coliziuni pentru oricare
din intrări să fie cât mai mic. De asemenea, trebuie ca $M$ să fie cât mai
mare, pentru ca media numărului de chei repartizate la aceeași intrare să fie
cât mai mică. Totuși, experiența arată că nu orice valoare a lui $M$ este
bună.

De exemplu, la prima vedere s-ar putea spune că o bună valoare pentru $M$ este
o putere a lui 2, cum ar fi 1024, pentru că operația de împărțire cu rest se
poate face foarte ușor în această situație. Totuși, funcția $h(x)=x \bmod
1024$ are un mare defect: ea nu ține cont decât de ultimii 10 biți ai
numărului $x$. Dacă datele de intrare sunt numere în mare majoritate pare, ele
vor fi repartizate în aceeași proporție la intrările cu număr de ordine par,
pentru că funcția h păstrează paritatea. Din aceleași motive, alegerea unei
valori ca 1000 sau 2000 nu este prea inspirată, deoarece ține cont numai de
ultimele 3-4 cifre ale reprezentării zecimale. Multe valori pot da același
rest la împărțirea prin 1000. De exemplu, dacă datele de intrare sunt anii de
naștere ai unor persoane dintr-o agendă telefonică, iar funcția este $h(x)=x
\bmod 1000$, atunci majoritatea cheilor se vor îngrămădi (termenul este
sugestiv) între intrările 920 și 990, restul rămânând nefolosite.

Practic, trebuie ca $M$ să nu fie un număr rotund în nici o bază aritmetică,
sau cel puțin nu în bazele 2 și 10. O bună alegere pentru $M$ sunt numerele
prime care să nu fie apropiate de nici o putere a lui 2. De exemplu, în locul
unei tabele cu $M=10000$ de intrări, care s-ar comporta dezastruos, putem
folosi una cu 9973 de intrări. Chiar și această alegere poate fi îmbunătățită;
între puterile lui 2 vecine, respectiv 8192 și 16384, se poate alege un număr
prim din zona 11000-12000. Risipa de memorie de circa 1000-2000 de intrări în
tabelă va fi pe deplin compensată de îmbunătățirea căutării.

\subsection{Metoda înmulțirii}

Pentru această metodă funcția hash este

\begin{equation}
  h(x) = \lfloor M(x \times A \bmod 1) \rfloor
\end{equation}

Aici $A$ este un număr pozitiv subunitar, $0 < A < 1$, iar prin $x \times A
\bmod 1$ se înțelege partea fracționară a lui $x \times A$, adică $x \times A
- \lfloor x \times A \rfloor$. De exemplu, dacă alegem $M = 1234$ și $A =
0,3$, iar $x = 1997$, atunci avem

\begin{equation}
  h(x) = \lfloor 1234 \times (599,1 \bmod 1) \rfloor = \lfloor 1234 \times 0,1 \rfloor = 123
\end{equation}

Se observă că funcția $h$ produce numere între 0 și $M-1$. Într-adevăr, 

\begin{equation}
  0 \leq x \times A \bmod 1 < 1 \implies 0 \leq M(x \times A \bmod 1) < M
\end{equation}

În acest caz, valoarea lui $M$ nu mai are o mare importanță. O putem deci
alege cât de mare ne convine, eventual o putere a lui 2. În practică, s-a
observat că dispersia este mai bună pentru unele valori ale lui $A$ și mai
proastă pentru altele. Donald Knuth propune valoarea

\begin{equation}
  A = \frac{\sqrt{5} - 1}{2} \approx 0.618034
\end{equation}

Ca o ultimă precizare necesară la acest capitol, menționăm că funcția de
căutare e bine să nu întoarcă pur și simplu 0 sau 1, după cum cheia căutată a
mai apărut sau nu înainte între datele de intrare. E recomandabil ca funcția
să întoarcă un pointer la zona de memorie în care se află prima apariție a
cheii căutate. Vom da acum un exemplu în care această valoare returnată este
utilă. Dacă, în cazul prezentat mai sus al unui program de șah, se ajunge la o
anumită poziție $P$ după ce albul a pierdut dreptul de a face rocada, această
poziție va fi reținută în hash. Reținerea nu se va face nicidecum efectiv
(toată tabla), pentru că s-ar ocupa foarte multă memorie. Se va memora în loc
numai un pointer la poziția respectivă din lista de poziții analizate. Pe
lângă economia de memorie în cazul cheilor de mari dimensiuni, mai există și
alt avantaj. Să ne închipuim că, analizând în continuare tabla, programul va
ajunge la aceeași poziție $P$, dar în care albul are încă dreptul de a face
rocada. E limpede că această variantă este mai promițătoare decât precedenta,
deoarece albul are o libertate mai mare de mișcare. Se impune deci fie
ștergerea vechii poziții $P$ din listă și adăugarea noii poziții, fie
modificarea celei vechi prin setarea unei variabile suplimentare care indică
dreptul albului de a face rocada. Această modificare este ușor de făcut,
întrucât căutarea în hash va returna chiar un pointer la poziția care trebuie
modificată. Bineînțeles, în cazul în care poziția căutată nu se află în hash,
funcția de căutare trebuie să întoarcă {\tt NULL}.

În încheiere, prezentăm un exemplu de funcție de dispersie pentru cazul tablei
de șah.

\begin{minted}{c}
#define M 9973 // numarul de "intrari"
typedef struct {
          char b_T[8][8];
            /* tabla de joc, cu 0<= T[i][j] <=12 */
          char b_CastleW, b_CastleB;
            /*  ultimii doi biti ai lui b_CastleW
                indica daca albul are dreptul de a
                efectua rocada mare, respectiv pe cea
                mica. Analog pentru b_CastleB */
          char b_Side;
            /* 0 sau 1, dupa cum la mutare este albul
               sau negrul */
          char b_EP;
            /* 0..8, indicand coloana (0..7) pe care
               partea la mutare poate efectua o
               captura "en passant". 8 indica ca nu
               exista aceasta posibilitate */
          int b_NMoves;
            /* Numarul de mutari efectuate */
        } Board;
Board B;

int h3(Board *B)
{ int i,j;
  /* Valoarea initiala a lui S este un numar pe 17 biti care
     inglobeaza toate variabilele suplimentare pe langa T.
     S se va lua apoi modulo M */
  long S = (B->b_NMoves           /* 8 biti */
          +(B->b_CastleW << 8)    /* 2 biti */
          +(B->b_CastleB << 10)   /* 2 biti */
          +(B->b_Side << 12)      /* 1 bit */
          +B->b_EP<<13) % M;      /* 4 biti */

  for (i=0; i<=7; i++)
    for (j=0; j<=7; j++)
      S=(16*S+B->b_T[i][j])%M;

  return S;
}
\end{minted}

  \chapter{Despre algoritmi exponențiali și îmbunătățirea lor}

Trebuie să spunem de la bun început că cea mai bună îmbunătățire care i se
poate aduce unui algoritm în timp exponențial care rezolvă o anumită problemă
este evitarea lui, adică găsirea - acolo unde este posibil - a unui algoritm
polinomial care să rezolve aceeași problemă. Există cazuri în care acest lucru
nu este posibil; în această situație, însă, orice îmbunătățire nu este decât o
metodă de a ascunde gunoiul sub covor. Un algoritm exponențial rămâne
exponențial, iar îmbunătățirile aduse îl pot face să meargă de două, de trei,
de zece ori mai repede, dar nu-l pot transforma într-un algoritm
polinomial. Creșterea cu două - trei unități a dimensiunii datelor de intrare
va anihila saltul de la un calculator 486 la un Pentium.

În multe situații, nu se cunoaște nici un algoritm care să funcționeze în timp
util și să furnizeze soluția optimă a unei probleme. În asemenea cazuri, dacă
optimalitatea nu este strict necesară, se renunță la ea și se caută algoritmi
care să producă soluții cât mai apropiate de cea optimă și care să meargă mult
mai repede. Este intuitiv că, dacă impunem limite mai dure în ceea ce privește
timpul de rulare al algoritmului, vom avea șanse mai mici să găsim o soluție
apropiată de optim. Nu vom discuta în această carte despre cum se poate
realiza un echilibru între abaterea soluției de la optim și timpul necesar
pentru a o produce. Genul acesta de dileme apar și în cadrul concursului de
informatică, dar acolo timpul nu permite o analiză laborioasă a problemei. Noi
vom indica numai modul în care se poate „inventa” un algoritm polinomial în
locul unuia exponențial și o metodă prin care acest algoritm poate fi
îmbunătățit pentru ca rezultatele pe care acesta le scoate să nu fie departe
de adevăr. Aceasta este una din tendințele din ultimii ani de la concursurile
de informatică. Deoarece toți elevii cunosc problemele „clasice” care s-ar
putea da la concurs, se încearcă departajarea lor prin urmărirea modului în
care ei se adaptează la probleme fără o soluție eficientă cunoscută.

Să pornim de la o problemă celebră, cea a comis-voiajorului.

{\bf ENUNȚ}: Se dă un graf complet orientat cu $N \leq 30$ noduri. Fiecare
muchie are un cost cuprins între 1 și 100. Se cere să se determine un ciclu
hamiltonian de cost minim. Un ciclu hamiltonian este ciclul care parcurge
fiecare nod exact o dată. Costul unui ciclu este suma costurilor muchiilor
componente.

{\bf Intrarea}: Datele de intrare se găsesc în fișierul {\tt INPUT.TXT}, sub
următoarea formă:

\begin{verbatim}
  N
  A[1,1] A[1,2] ... A[1,N]
  ...
  A[N,1] A[N,2] ... A[N,N]
\end{verbatim}

unde {\tt A[i,j]} este lungimea muchiei care iese din nodul $i$ și intră în
nodul $j$. Se garantează că $A[i,i] = 0, \forall 1 \leq i \leq N$.

{\bf Ieșirea} se va face în fișierul {\tt OUTPUT.TXT} pe două linii. Pe prima
linie se va tipări costul minim găsit, iar pe a doua traseul parcurs ($N$
numere separate prin spații).

{\bf Exemplu}:

\texttt{
  \begin{tabular}{| l | l |}
    \hline
        {\bf INPUT.TXT} & {\bf OUTPUT.TXT} \\ \hline
        \begin{tabular}[t]{l}
          4\\
          0 3 4 1\\
          3 0 2 3\\
          5 2 0 6\\
          8 1 3 0
        \end{tabular}
        &
        \begin{tabular}[t]{l}
          9\\
          1 4 2 3
        \end{tabular}
        \\
    \hline
  \end{tabular}
}

{\bf Timp de execuție}: 30 secunde

{\bf Timp de implementare}: 30 minute

{\bf REZOLVARE}: Problema este arhicunoscută și de asemenea este arhicunoscut
faptul că ea nu admite o rezolvare polinomială. Totuși, dacă această problemă
vă este dată la un concurs, nu poate constitui o scuză în fața comisiei
argumentul că problema nu este polinomială. Ea trebuie făcută să meargă cât
mai bine. Spre deosebire de laboratoarele NASA, unde orice greșeală într-o
linie de cod poate distruge un modul spațial cu echipaj cu tot, la olimpiadă
nu se merge pe principiul „totul sau nimic”. Fiecare punct câștigat este bun
câștigat.

Precizăm că metodele de mai jos se aplică mai degrabă în cazurile în care se
cere o soluție optimă, dar se oferă punctaje parțiale și în cazul în care
concurentul oferă o soluție cât de cât apropiată de cea optimă. Există șanse
ca metodele de mai jos să asigure găsirea chiar a soluției optime pentru mare
parte din teste, dar aceste șanse variază de la o problemă la alta.

În primul rând, care este deosebirea fundamentală dintre un algoritm
backtracking și unul greedy? Algoritmul backtracking analizează pe rând
fiecare soluție posibilă și o alege pe cea mai bună. În felul acesta, el nu
poate scăpa soluția optimă. Din nefericire, în multe cazuri spațiul soluțiilor
crește exponențial cu dimensiunea datelor de intrare; în aceste situații
algoritmii backtracking nu mai sunt practici. În schimb, algoritmii greedy
(engl. {\it greedy = lacom}) fac o parcurgere a datelor de intrare și, la
fiecare pas al acestei parcurgeri, aleg o parte (destul de mică) din soluțiile
posibile, iar pe restul le „aruncă”. La pasul următor se aleg o parte din
soluțiile rămase și așa mai departe, până când în final rămâne o singură
soluție care se tipărește.

Criteriul în funcție de care se face trierea soluțiilor este cheia unui
algoritm greedy. Dacă acest criteriu poate garanta că la fiecare pas al
algoritmului soluția optimă (sau cel puțin una din soluțiile optime, dacă pot
exista mai multe) rămâne între soluțiile care sunt păstrate, atunci algoritmul
greedy funcționează perfect. Demonstrația este ușoară: soluția optimă nu este
„aruncată” niciodată, iar la sfârșitul algoritmului rămâne o singură soluție,
de unde rezultă că soluția rămasă este tocmai cea optimă. Asemenea cazuri de
algoritmi pentru care s-a demonstrat că ei funcționează sunt: algoritmii lui
Kruskal și Prim pentru găsirea arborelui parțial de cost minim al unui graf,
algoritmul lui Dijkstra pentru determinarea drumurilor de cost minim de la un
nod la toate celelalte într-un graf ș.a.m.d.

Putem extinde aceste noțiuni și la domeniul jocurilor logice. De exemplu,
jocul Nim (pe care probabil îl cunoașteți cu toții) are o strategie sigură de
câștig pentru anumite poziții, iar pentru celelalte se poate demonstra că nu
există nici o strategie de câștig. În cazul în care strategia există,
jucătorului i se oferă o mutare care îl duce spre victorie. Ce este în fond
această mutare? Tocmai un criteriu de a tria anumite configurații, favorabile
jucătorului, și de a le ignora pe celelalte.

Există însă probleme pentru care nu s-a găsit (iar uneori s-a și demonstrat că
nu există) nici un algoritm greedy. Continuând paralela cu jocurile logice,
există jocuri care nu au nici o strategie de câștig pentru unul din
jucători. Este cazul jocului de șah. Pentru unele asemenea probleme (cum ar fi
cea de față, a comis-voiajorului), care au aplicații largi în diferite domenii
practice, se investesc sume mari în cercetare pentru a se găsi algoritmi cât
mai buni care să funcționeze într-un timp convenabil.

O situație asemănătoare apare la concursul de informatică, unde se cunoaște de
la început timpul pus la dispoziție (atât cel pentru implementare, cât și cel
pentru execuție) și se urmărește obținerea unui punctaj cât mai mare. Iată
câteva metode destul de eficiente.

\section{„Omorârea” backtracking-ului}

Sintagma aceasta oarecum ilară, care stârnește mila pentru bietul
backtracking, se aude foarte des pe la ieșirea din sălile de concurs, atunci
când problema a fost mai „ciudată”, în sensul că foarte puțini concurenți au
descoperit vreo soluție eficientă la ea. Ce este de fapt „backtracking-ul
omorât” și în ce situații este el preferabil?

Problema comis-voiajorului sub diverse forme sau alte probleme exponențiale au
fost propuse în anii trecuți spre rezolvare la concursuri. Dacă vrem să aflăm
soluția optimă (de cost minim), neavând altă soluție la îndemână, trebuie să
recurgem la backtracking. Backtracking-ul, după cum se știe, examinează pe
rând fiecare posibilă soluție. În cazul nostru, backtracking-ul nu are altceva
de făcut decât să genereze pe rând toate permutările mulțimii ${1, 2, \dots,
  N}$. Considerând fiecare permutare ca fiind un posibil ciclu hamiltonian
(știm sigur că oricărei permutări îi corespunde un ciclu hamiltonian, deoarece
graful este complet), mai trebuie doar să calculăm costul fiecărei permutări
și să o afișăm pe cea de cost minim. Până aici, nimic deosebit. Iată și sursa
Pascal:

\begin{minted}{pascal}
program Hamilton;
{$B-,I-,R-,S-}
const NMax=30;
type Vector=array[1..NMax] of Integer;
     Matrix=array[1..NMax,1..NMax] of Integer;
var A:Matrix;
    Route,BestRoute:Vector;
    Seen:set of 1..NMax;
    N,Cost,MinCost:Integer;

procedure ReadData;
var i,j:Integer;
begin
  Assign(Input,'input.txt');Reset(Input);
  ReadLn(N);
  for i:=1 to N do
    begin
      for j:=1 to N do Read(A[i,j]);
      ReadLn;
    end;
  Close(Input);
end;

procedure Bkt(Level,Cost:Integer);
var i:Integer;
begin
  if Level=N+1
    then begin
           Inc(Cost,A[Route[N],1]);
           if Cost<MinCost
             then begin
                    BestRoute:=Route;
                    MinCost:=Cost;
                  end;
         end
    else if Cost<MinCost
           then for i:=1 to N do
             if not (i in Seen)
               then begin
                      Seen:=Seen+[i];
                      Route[Level]:=i;
                      Bkt(Level+1,Cost+A[Route[Level-1],i]);
                      Seen:=Seen-[i];
                    end;
end;

procedure WriteSolution;
var i:Integer;
begin
  Assign(Output,'output.txt');Rewrite(Output);
  WriteLn(MinCost);
  for i:=1 to N do Write(BestRoute[i],' ');
  WriteLn;
  Close(Output);
end;

begin
  ReadData;
  Route[1]:=1;
  Seen:=[1];
  MinCost:=MaxInt;
  Bkt(2,0);
  WriteSolution;
end.
\end{minted}

Programul sub această formă nu se încadrează în timp nici măcar pentru $N=15$
(cifra depinde și de calculatorul folosit pentru testare, dar nu variază cu
mai mult de două-trei nivele; să spunem cu generozitate că pe un calculator
performant programul ar putea merge până la $N=18$ sau 20). Pe de altă parte,
ideea în sine (algoritmul) de rezolvare nu mai poate fi mult îmbunătățită. Și
cu toate acestea, programul trebuie să meargă până la $N=30$. Ce putem face?

Desigur, nu există o rezolvare elegantă. Putem însă încerca fel de fel de
metode de a trișa. Deoarece nu avem timp să examinăm toate soluțiile, trebuie
să renunțăm la o parte din ele, cu riscul ca printre ele să se afle tocmai
soluția căutată. Una dintre tehnici este omorârea backtracking-ului. După
numărul și tipurile soluțiilor pe care le cer, algoritmii backtracking ar
putea fi împărțiți în mai multe categorii:

\begin{itemize}

\item Cei care furnizează o singură soluție;

\item Cei care, pe baza unei funcții care atașează un cost fiecărei soluții,
  furnizează soluția de cost minim;

\item Cei care furnizează toate soluțiile.
\end{itemize}

Backtracking-ul omorât se aplică celui de-al doilea tip de cerințe. Putem face
în așa fel încât să oprim programul exact la expirarea timpului permis pentru
rulare și să afișăm cea mai bună soluție găsită până la momentul
respectiv. Dacă am fost norocoși (termenul este cel mai potrivit în această
situație), atunci programul nostru a apucat, în timpul pe care i l-am permis,
să găsească soluția optimă. Dacă nu, putem totuși spera că a fost găsită o
soluție cât de cât apropiată de cea optimă, pentru care putem eventual să
primim măcar o parte din punctaj. După cum se vede, omorârea backtracking-ului
nu promite marea cu sarea, dar este un artificiu binevenit, pentru că există
trei variante:

\begin{itemize}

\item Programul se încadrează în timp, caz în care backtracking-ul omorât nu
  aduce nimic în plus;

\item Programul nu se încadrează în timp, dar soluția optimă este găsită în
  timp, caz în care backtracking-ul simplu nu furnizează nici o soluție (și de
  obicei este oprit cu Ctrl-Break), pe când backtracking-ul omorât furnizează
  soluția;

\item Programul nu se încadrează în timp și nici soluția optimă nu este găsită
  în timp, caz în care backtracking-ul simplu nu furnizează nici o soluție, pe
  când backtracking-ul omorât furnizează o soluție eventual apropiată de
  optim.

\end{itemize}

Din experiență se poate spune că membrii comisiei de corectare acordă jumătate
din punctele pentru un test mai degrabă atunci când li se oferă o soluție
neoptimă decât atunci când li se oferă o soluție optimă într-un timp
depășit. Adesea programul este oprit îndată ce timpul de rulare expiră, și
părerea autorului e că e mai bine așa.

„Omorârea” nu se pretează la celelalte două versiuni de backtracking. Atunci
când există o singură soluție, nu se mai pune problema de a găsi una apropiată
de ea. Ori găsim soluția, ori nimic. De exemplu, nu are sens să rezolvăm
problema de mai sus cu un backtracking omorât dacă știm sigur că nu se acordă
punctaje parțiale pentru soluții neoptime (dar în general acest lucru nu se
știe sigur...). Rămâne bineînțeles posibilitatea de a opri programul imediat
ce soluția a a fost găsită. În cazul în care se cer toate soluțiile, de
asemenea programul nu poate fi oprit. Atunci când se poate, merită calculat
numărul de soluții, pentru ca imediat ce am găsit toate soluțiile, să oprim
programul. Putem aplica backtracking-ul omorât numai dacă știm că se acordă
punctaj și pentru afișarea unei părți din soluții.

Mai rămâne de stabilit cum anume se face „omorârea” backtracking-ului (și de
fapt a oricărui program). Există mai multe metode. Prima, asupra căreia nu vom
insista deoarece ea are „efecte secundare” și, în plus, este foarte lentă,
constă din două etape:

\begin{itemize}

\item Se setează ceasul sistem la ora 00:00:00 (miezul nopții) cu procedura
  {\tt SetTime} din unitatea Dos a compilatorului Borland Pascal;

\item Periodic se testează ora cu procedura {\tt GetTime} și se oprește
  programul atunci când se apropie „ora critică” (în cazul nostru 00:00:30).

\end{itemize}

După cum se vede, marele neajuns al acestei metode este că „dă peste cap”
ceasul sistem (lucru dezastruos mai ales pentru cei care vin la concurs fără
ceas...). În afară de aceasta, procedurile {\tt GetTime} și {\tt SetTime}
apelează la rândul lor întreruperile DOS, ceea ce consumă mult din timpul care
și așa este limitat.

A doua metodă, care este mai rapidă și nu lasă urme, constă în captarea {\bf
  întreruperii 8}, adică a {\bf timer}-ului. Timer-ul este o rutină care se
apelează automat la fiecare 55 de milisecunde, deci cam de 18,2 ori pe
secundă. În principiu, ea nu face nimic altceva decât să incrementeze ceasul
sistem cu 55 ms. Pe lângă aceasta, însă, putem adăuga și propriul nostru cod,
folosind procedurile Pascal {\tt GetIntVec} și {\tt SetIntVec}. Trebuie doar
să avem grijă ca timer-ul scris de noi să-l apeleze și pe cel vechi, altfel
ceasul sistem se va opri și cine știe ce altceva se mai poate întâmpla. Vom
declara deci o variabilă {\tt Time} care va fi decrementată la fiecare apel al
întreruperii de ceas. Înainte de a intercepta întreruperea 8, vom inițializa
variabila cu valoarea maximă dorită. Știm că timer-ul se apelează de 18,2 ori
pe secundă, deci dacă limita de timp pentru un test este de 30 de secunde,
valoarea inițială pentru variabila {\tt Time} ar putea fi $18,2 \times 30 =
546$. Este bine să nu calculăm însă timpul la limită, deoarece avem nevoie de
câteva fracțiuni și pentru tipărirea soluției în fișier, și poate pur și
simplu ceasul comisiei de corectare o ia puțin înainte. De aceea, e mai sigură
înmulțirea cu 17 în loc de 18,2.

În momentul în care, prin decrementări succesive, {\tt Time} a ajuns la
valoarea 0 (sau la o valoare negativă), programul trebuie oprit. Acest lucru
presupune ieșirea din procedura de backtracking, afișarea soluției și
restaurarea vechii întreruperi 8, pentru ca programul să nu lase „urme”. Iată
deci o versiune a programului Pascal care va fi extrem de punctuală...

\begin{minted}{pascal}
program Hamilton;
{$B-,I-,R-,S-}
uses Dos;
const NMax=30;
      TimeLimit=30; { secunde }

type Vector=array[1..NMax] of Integer;
     Matrix=array[1..NMax,1..NMax] of Integer;
var A:Matrix;
    Route,BestRoute:Vector;
    Seen:set of 1..NMax;
    N,Cost,MinCost:Integer;
    Time:Integer;  { Contorul }
    OldTimer:procedure;

procedure MyTimer; interrupt;
{ Se executa la fiecare 55 ms }
begin
  Dec(Time);   { Ne facem treaba... }
  Inline($9C); { ...pushf... }
  OldTimer;    { ...si executam si vechiul timer }
end;

procedure ReadData;
var i,j:Integer;
begin
  Assign(Input,'input.txt');Reset(Input);
  ReadLn(N);
  for i:=1 to N do
    begin
      for j:=1 to N do Read(A[i,j]);
      ReadLn;
    end;
  Close(Input);
end;

procedure Bkt(Level,Cost:Integer);
var i:Integer;
begin
  if Level=N+1
    then begin
           Inc(Cost,A[Route[N],1]);
           if Cost<MinCost
             then begin
                    BestRoute:=Route;
                    MinCost:=Cost;
                  end;
         end
    else if (Time>0) and (Cost<MinCost)
           then for i:=1 to N do
             if not (i in Seen)
               then begin
                      Seen:=Seen+[i];
                      Route[Level]:=i;
                      Bkt(Level+1,Cost+A[Route[Level-1],i]);
                      Seen:=Seen-[i];
                    end;
end;

procedure WriteSolution;
var i:Integer;
begin
  Assign(Output,'output.txt');Rewrite(Output);
  WriteLn(MinCost);
  for i:=1 to N do Write(BestRoute[i],' ');
  WriteLn;
  Close(Output);
end;

begin
  Time:=TimeLimit*17;
  { Captam intreruperea 8 (timer-ul) }
  GetIntVec(8,@OldTimer);
  SetIntVec(8,@MyTimer);
  ReadData;
  Route[1]:=1;
  Seen:=[1];
  MinCost:=MaxInt;
  Bkt(2,0);
  WriteSolution;
  { Restauram timer-ul }
  SetIntVec(8,@OldTimer);
end.
\end{minted}

Și această a doua metodă are neajunsurile ei, deoarece presupune scrierea a
destul de multe linii de program în plus. În afară de aceasta, instrucțiunea
de decrementare a variabilei {\tt Time}, precum și apelul suplimentar de
procedură din cadrul întreruperii de ceas mai reduc puțin timpul dedicat
calculelor efective. A treia variantă elimină și aceste deficiențe. Ea se
bazează pe accesarea directă a locației de memorie \$0000:\$046C, unde se
află, reprezentat pe 4 octeți, numărul de apeluri ale timer-ului (numărul de
{\bf tacți}) începând de la miezul nopții. Dacă declarăm o variabilă {\tt
  Time} de tip {\tt Longint} (deoarece acest tip de date ocupă 4 octeți) exact
la această adresă, folosind clauza Pascal {\tt absolute}, variabila se va
incrementa la fiecare 55ms, scutindu-ne pe noi de această grijă. Dacă înmulțim
variabila cu 55/1.000, aflăm exact numărul de secunde scurse de la miezul
nopții. Dacă împărțim acest rezultat la 3.600, putem afla ora exactă
ș.a.m.d. Lucrul care ne interesează pe noi este să setăm o „alarmă” care să
oprească programul peste 30 de secunde. 30 de secunde înseamnă $30 \times
18,2$ incrementări ale variabilei {\tt Time}. Folosind în loc de 18,2 valoarea
17 (pentru a păstra o rezervă), rezultă că trebuie să ne oprim atunci când
{\tt Time} are o valoare cu $30 \times 17$ mai mare decât la intrarea în
program. Primul lucru pe care îl va face programul va fi să dea unei variabile
{\tt Alarm} valoarea {\tt Time} + $30 \times 17$. Periodic (cel mai comod la
intrarea în procedura backtracking) se va testa valoarea variabilei {\tt Time}
și atunci când ea este egală cu {\tt Alarm}, se va ieși din program.

\begin{minted}{pascal}
program Hamilton;
{$B-,I-,R-,S-}
const NMax=30;
      TimeLimit=30; { secunde }

type Vector=array[1..NMax] of Integer;
     Matrix=array[1..NMax,1..NMax] of Integer;
var A:Matrix;
    Route,BestRoute:Vector;
    Seen:set of 1..NMax;
    N,Cost,MinCost:Integer;
    Time:LongInt absolute $0000:$046C;
    Alarm:LongInt;

procedure SetAlarm;
begin
  Alarm:=Time+TimeLimit*17;
  { Cifra corecta era nu 17, ci 18.2;
    am pastrat insa o rezerva de siguranta }
end;

procedure ReadData;
var i,j:Integer;
begin
  Assign(Input,'input.txt');Reset(Input);
  ReadLn(N);
  for i:=1 to N do
    begin
      for j:=1 to N do Read(A[i,j]);
      ReadLn;
    end;
  Close(Input);
end;

procedure Bkt(Level,Cost:Integer);
var i:Integer;
begin
  if Level=N+1
    then begin
           Inc(Cost,A[Route[N],1]);
           if Cost<MinCost
             then begin
                    BestRoute:=Route;
                    MinCost:=Cost;
                  end;
         end
    else if (Time<Alarm) and (Cost<MinCost)
           then for i:=1 to N do
             if not (i in Seen)
               then begin
                      Seen:=Seen+[i];
                      Route[Level]:=i;
                      Bkt(Level+1,Cost+A[Route[Level-1],i]);
                      Seen:=Seen-[i];
                    end;
end;

procedure WriteSolution;
var i:Integer;
begin
  Assign(Output,'output.txt');Rewrite(Output);
  WriteLn(MinCost);
  for i:=1 to N do Write(BestRoute[i],' ');
  WriteLn;
  Close(Output);
end;

begin
  SetAlarm;
  ReadData;
  Route[1]:=1;
  Seen:=[1];
  MinCost:=MaxInt;
  Bkt(2,0);
  WriteSolution;
end.
\end{minted}

Singura problemă pe care o poate ridica această ultimă versiune este
următoarea: dacă programul este lansat în execuție la un moment foarte
apropiat de miezul nopții, atunci variabila {\tt Alarm} va avea o valoare mai
mare decât numărul de tacți dintr-o zi. Variabila {\tt Time} nu va ajunge
niciodată la această valoare, deoarece la miezul nopții ea va lua din nou
valoarea 0. Este totuși puțin probabil să vă fie corectat programul la miezul
nopții...

\section{Greedy euristic}

După cum am spus, la unii algoritmi greedy, criteriul de departajare
garantează că soluția optimă nu este niciodată scăpată din vedere. De și mai
multe ori, totuși, criteriile de departajare nu pot promite acest lucru; în
general elevii, la ieșirea din sălile de concurs, în cazul unei probleme mai
controversate, își expun părerile și ideile față de colegii lor, apoi fiecare
îi demonstrează celuilalt că algoritmul propus de el nu merge, prezentându-i
un contraexemplu. Momentele cele mai picante se produc atunci când algoritmul
pare să nu fie corect, dar nici nu se poate găsi un contraexemplu.

În aceste situații, criteriile de departajare a soluțiilor la algoritmii
greedy se numesc {\bf funcții euristice}, iar algoritmul în sine se numește
{\bf greedy euristic} (în greaca veche, {\it heuriskein} însemna {\it a
  afla}). Dacă nu poate promite optimalitatea, funcția euristică trebuie în
orice caz aleasă cât mai bine, respectiv trebuie să aibă șanse cât mai mari să
rețină soluția optimă, sau măcar să rețină la fiecare pas soluții cât mai
apropiate de cea optimă.

Un singur algoritm greedy euristic are șanse mici să găsească soluția
optimă. Dar algoritmii greedy euristici au unele proprietăți interesante:

\begin{itemize}

\item Se încadrează cu ușurință în timpul de rulare.

\item Sunt ușor de implementat.

\item O modificare cât de mică a funcției euristice poate modifica radical
  algoritmul și soluția furnizată de el. Deoarece nu avem de unde ști care
  dintre funcțiile euristice este mai bună (acest lucru depinde de datele pe
  care este testată problema), ideal este să reținem ambele soluții furnizate
  și să o alegem pe cea mai bună.

\item De multe ori, datele de intrare sunt vectori sau matrice; în unele
  situații, sensul în care sunt ele parcurse pentru determinarea soluției nu
  este important. Schimbând sensul de parcurgere, obținem de asemenea două
  soluții distincte pe care le putem compara.

\item Aproape întotdeauna, funcțiile euristice conțin teste de genul:

\begin{minted}{pascal}
if A<B then Actiune1 else Actiune2;
\end{minted}

Din punct de vedere logic, dacă {\tt A=B} se poate executa oricare din cele
două acțiuni. Condiția {\tt A<B} este echivalentă cu condiția {\tt
  A<=B}. Totuși, din punct de vedere al calculatorului, cele două condiții
sunt absolut diferite și pot produce soluții cu totul diferite. Iată de
exemplu, două rutine care caută poziția $k$ pe care se află elementul minim
într-un vector $V$ cu $N$ elemente:

\begin{minted}{pascal}
k:=1;
for i:=2 to N do
  if V[i]<V[k] then k:=i;
\end{minted}

respectiv

\begin{minted}{pascal}
k:=1;
for i:=2 to N do
  if V[i]<=V[k] then k:=i;
\end{minted}

Cele două versiuni vor găsi într-adevăr un indice $k$ astfel încât $V[k]$ să
fie minim. Totuși, dacă există mai multe elemente de valoare minimă, atunci
prima versiune va întoarce indicele cel mai mic, pe când ultima îl va întoarce
pe cel mai mare.

\end{itemize}

Iată un exemplu de funcție euristică pentru problema comis-voiajorului: pornim
din nodul 1 și, la fiecare pas, ne deplasăm în cel mai apropiat nod care nu a
fost vizitat încă. După ce toate nodurile au fost vizitate, trebuie numai să
ne deplasăm din ultimul nod vizitat în nodul 1. Luată în sine, această
euristică nu este strălucită. Ea poate însă să fie „clonată” într-o
multitudine de variante. În primul rând că nu este obligatoriu să pornim din
nodul 1. Putem aplica același algoritm pornind pe rând din fiecare nod; la
sfârșit tipărim soluția de cost minim. Prima variantă a programului Pascal
este:

\begin{minted}{pascal}
program Hamilton;
{$B-,I-,R-,S-}
const NMax=30;
type Vector=array[1..NMax] of Integer;
     Matrix=array[1..NMax,1..NMax] of Integer;
var A:Matrix;
    Route,BestRoute:Vector;
    N,Cost,MinCost,i:Integer;

procedure ReadData;
var i,j:Integer;
begin
  Assign(Input,'input.txt');Reset(Input);
  ReadLn(N);
  for i:=1 to N do
    begin
      for j:=1 to N do Read(A[i,j]);
      ReadLn;
    end;
  Close(Input);
end;

procedure Greedy1(Start:Integer;var R:Vector;
                  var Cost:Integer);
var i,j,Closest:Integer;
    Seen:set of 1..NMax;
begin
  R[1]:=Start;
  Cost:=0;
  Seen:=[Start];
  for i:=2 to N do
    begin
      { Cauta nodul cel mai apropiat }
      Closest:=MaxInt;
      for j:=1 to N do
        if (not (j in Seen)) and (A[R[i-1],j]<Closest)
          then begin
                 Closest:=A[R[i-1],j];
                 R[i]:=j;
               end;
      Inc(Cost,Closest);
      Seen:=Seen+[R[i]];
    end;
  { Inchide ciclul }
  Inc(Cost,A[R[N],Start]);
end;

procedure Update;
begin
  if Cost<MinCost
    then begin
           MinCost:=Cost;
           BestRoute:=Route;
         end;
end;

procedure WriteSolution;
var i:Integer;
begin
  Assign(Output,'output.txt');Rewrite(Output);
  WriteLn(MinCost);
  for i:=1 to N do Write(BestRoute[i],' ');
  WriteLn;
  Close(Output);
end;

begin
  ReadData;
  MinCost:=MaxInt;
  for i:=1 to N do
    begin
      Greedy1(i,Route,Cost);
      Update;
    end;
  WriteSolution;
end.
\end{minted}

De sine stătătoare, funcția euristică nu este strălucită. Totuși, ea poate fi
lesne modificată. Se observă că, în procedura {\tt Greedy1}, instrucțiunea

\begin{minted}{pascal}
for j:=1 to N do...
\end{minted}

poate fi înlocuită cu

\begin{minted}{pascal}
for j:=N downto 1 do...
\end{minted}

, iar condiția

\begin{minted}{pascal}
(A[R[i-1],j] < Closest)
\end{minted}

cu

\begin{minted}{pascal}
(A[R[i-1],j]<=Closest)
\end{minted}

Făcând toate combinațiile posibile, rezultă alte trei proceduri, {\tt
  Greedy2}, {\tt Greedy3} și {\tt Greedy4}, iar noua formă a programului
principal este:

\begin{minted}{pascal}
begin
  ReadData;
  MinCost:=MaxInt;

  for i:=1 to N do
    begin
      Greedy1(i,Route,Cost);
      Update;
      Greedy2(i,Route,Cost);
      Update;
      Greedy3(i,Route,Cost);
      Update;
      Greedy4(i,Route,Cost);
      Update;
    end;

  WriteSolution;
end.
\end{minted}

După cum se vede, adaosul de proceduri face ca sursa să atingă dimensiuni
impunătoare, dar efortul necesar pentru a o scrie este aproape aceeași ca și
când ar fi existat o singură funcție euristică. Practic, trebuie scrisă una
singură din cele patru funcții, restul rezumându-se la copierea unor blocuri
cu ajutorul editorului Borland Pascal.

\section{Decizia între greedy euristic și backtracking}

Pentru a ne asigura și mai multe puncte din cele puse în joc, putem încerca
următoarea combinație: pentru grafuri mici, care pot fi examinate exhaustiv în
timp de câteva secunde, vom apela la algoritmul backtracking pentru rezolvarea
problemei. Numai pentru valori mari, pentru care știm sigur că backtracking-ul
depășește timpul admis, vom apela la funcțiile euristice. Ce înseamnă valori
„mici” si „mari” se poate aproxima sau se poate determina după câteva
teste. Aceste valori depind de problemă și de mașina folosită.

Deoarece primele teste pentru fiecare problemă (uneori o treime sau chiar
jumătate din ele) sunt de dimensiuni mici, e bine dacă vi le puteți asigura
printr-un backtracking care de regulă se implementează în 15-20 minute.

\section{Combinația greedy euristic + backtracking}

Urmărind prima rezolvare de la punctul (1) - varianta backtracking fără nici
un fel de modificări - se observă că variabila {\tt MinCost} se inițializează
cu valoarea {\tt MaxInt}. În felul acesta, prima soluție găsită este implicit
cea mai bună și durează o vreme până când rezultatele încep să se apropie de
optim. Pe de altă parte, evaluarea unui anumit lanț din graf și prelungirea
lui cu noi noduri până la închiderea ciclului hamiltonian nu se fac decât dacă
costul lanțului nu a depășit deja valoarea {\tt MinCost}. De aici provine
întrebarea firească: ce-ar fi dacă, în loc să inițializăm variabila {\tt
  MinCost} cu valoarea {\tt MaxInt}, am lansa mai întâi unul sau mai multe
greedy-uri euristice (depinde câte apucăm să scriem) pentru a da o valoare mai
apropiată de adevăr variabilei {\tt MinCost} ? Sigur, cu o floare nu se face
primăvară, dar în cazul nostru se pot câștiga secunde prețioase. Se poate de
asemenea ca după aceste greedy-uri să apelăm nu un backtracking simplu, ci
unul omorât prin orice metodă, caz în care șansele se îmbunătățesc
considerabil. Programul principal ar putea fi atunci:

\begin{minted}{pascal}
begin
  SetAlarm;
  ReadData;
  MinCost:=MaxInt;
  for i:=1 to N do
    begin
      Greedy1(i,Route,Cost);
      Update;
      Greedy2(i,Route,Cost);
      Update;
      Greedy3(i,Route,Cost);
      Update;
      Greedy4(i,Route,Cost);
      Update;
    end;
  Route[1]:=1;
  Seen:=[1];
  Bkt(2,0);
  WriteSolution;
end.
\end{minted}

\section{Testarea aleatoare a posibilităților}

Oricât ar părea de ciudat, și aceasta este o cale de a ieși din
încurcătură. Ce-i drept, nu cea mai eficientă, dar atunci când imaginația vă
joacă feste iar backtracking-ul nu vă surâde, puteți încerca chiar și o
rezolvare care se bazează puternic pe funcția {\tt Random}. În acest caz, tot
ce aveți de făcut este să generați aleator cicluri hamiltoniene și să-i
calculați fiecăruia costul. La sfârșit îl tipăriți pe cel de cost minim
găsit. Bineînțeles, oprirea programului se va face printr-o „omorâre” de orice
tip. Timpul de implementare al unei asemenea rezolvări este de ordinul
minutelor. Această versiune găsește uneori soluția optimă, dar alteori este
foarte departe de ea. De asemenea, are marele dezavantaj că la două rulări
consecutive nu generează același rezultat, deoarece procedura {\tt Randomize}
își extrage variabila {\tt RandSeed} (folosită pentru a genera numere
aleatoare) pe baza timer-ului... Personal nu o recomand, dar este destul de
des folosită pe la concursuri. Și este momentul să amintim o urare ce li se
adresează concurenților care intră în sala de corectare, respectiv „Să fie
într-un timer bun!”.

\begin{minted}{pascal}
program Hamilton;
{$B-,I-,R-,S-}
const NMax=30;
      TimeLimit=30; { secunde }

type Vector=array[1..NMax] of Integer;
     Matrix=array[1..NMax,1..NMax] of Integer;
var A:Matrix;
    Route,BestRoute:Vector;
    Seen:set of 1..NMax;
    N,Cost,MinCost:Integer;
    Time:LongInt absolute $0000:$046C;
    Alarm:LongInt;

procedure SetAlarm;
begin
  Alarm:=Time+TimeLimit*17;
  { Cifra corecta era nu 17, ci 18.2;
    am pastrat insa o rezerva de siguranta }
end;

procedure ReadData;
var i,j:Integer;
begin
  Assign(Input,'input.txt');Reset(Input);
  ReadLn(N);
  for i:=1 to N do
    begin
      for j:=1 to N do Read(A[i,j]);
      ReadLn;
    end;
  Close(Input);
end;

procedure RandomCycle;
var i,j:Integer;
begin
  Route[1]:=Random(N)+1;
  Seen:=[Route[1]];
  Cost:=0;
  for i:=2 to N do
    begin
      repeat Route[i]:=Random(N)+1;
      until not (Route[i] in Seen);
      Seen:=Seen+[Route[i]];
      Inc(Cost,A[Route[i-1],Route[i]]);
    end;
  Inc(Cost,A[Route[N],Route[1]]);
end;

procedure Update;
begin
  if Cost<MinCost
    then begin
           MinCost:=Cost;
           BestRoute:=Route;
         end;
end;

procedure WriteSolution;
var i:Integer;
begin
  Assign(Output,'output.txt');Rewrite(Output);
  WriteLn(MinCost);
  for i:=1 to N do Write(BestRoute[i],' ');
  WriteLn;
  Close(Output);
end;

begin
  SetAlarm;
  Randomize;
  ReadData;
  MinCost:=MaxInt;
  while Time<Alarm do
    begin
      RandomCycle;
      Update;
    end;
  WriteSolution;
end.
\end{minted}

  \chapter{Probleme de concurs}

\section{Problema 1}

{\bf ENUNȚ}: Se consideră următorul joc: Pe o tablă liniară cu $2N+1$ căsuțe
sunt dispuse $N$ bile albe (în primele $N$ căsuțe) și $N$ bile negre (în
ultimele $N$ căsuțe), căsuța din mijloc fiind liberă. Bilele albe se pot mișca
numai spre dreapta, iar cele negre numai spre stânga. Mutările posibile sunt:

\begin{enumerate}

\item O bilă albă se poate deplasa o căsuță spre dreapta, numai dacă aceasta
  este liberă;

\item O bilă albă poate sări peste bila aflată imediat în dreapta ei
  (indiferent de culoarea acesteia), așezându-se în căsuța de dincolo de ea,
  numai dacă aceasta este liberă;

\item O bilă neagră se poate deplasa o căsuță spre stânga, numai dacă aceasta
  este liberă;

\item O bilă neagră poate sări peste bila aflată imediat în stânga ei
  (indiferent de culoarea acesteia), așezându-se în căsuța de dincolo de ea,
  numai dacă aceasta este liberă.

\end{enumerate}

Trebuie schimbat locul bilelor albe cu cele negre. Se mai cere în plus ca
prima mutare să fie făcută cu o bilă albă.

{\bf Intrarea}: De la tastatură se citește numărul $N \leq 1.000$.

{\bf Ieșirea}: În fișierul {\tt OUTPUT.TXT} se vor tipări două linii terminate
cu <tt><Enter></tt>. Pe prima se va tipări numărul de mutări efectuate, iar pe
a doua o succesiune de cifre cuprinse între 1 și 4, nedespărțite prin spații,
corespunzătoare mutărilor ce trebuie făcute.

{\bf Exemple}:

\begin{itemize}

\item $N=1 \implies$ Ieșirea 141

\item $N=2 \implies$ Ieșirea 14322341

\end{itemize}

{\bf Complexitate cerută}: $O(N^2)$.

{\bf Timp de implementare}: 1h.

{\bf Timp de rulare}: 10 secunde pentru un test.

{\bf REZOLVARE}: La prima vedere, problema pare să se preteze la o rezolvare
în timp exponențial, prin metoda „Branch and Bound”. Un neajuns al enunțului
pare să fie faptul că nu se specifică dacă numărul de mutări efectuate trebuie
sau nu să fie minim. Pentru a ne lămuri, să privim în detaliu soluțiile pentru
$N=1$ și $N=2$:

\begin{itemize}

\item Pentru $N = 1$, toate mutările sunt forțate ((a) - se mută bila albă,
  (b) - se sare cu cea neagră peste ea, (c) - se mută din nou bila albă);
  trebuie remarcat că după mutările (a) și (b) se obțin două configurații
  simetrice una în raport cu cealaltă (oglindite).

\item Pentru $N = 2$, se poate începe sărind cu bila albă de la margine peste
  cealaltă, dar această mutare ar duce la blocarea jocului. Este deci
  obligatoriu să se înceapă prin a împinge bila albă centrală (a). Următoarea
  mutare este forțată ((b) - se sare cu bila neagră peste cea albă), apoi
  toate mutările sunt obligate (în sensul că dacă la orice pas se face altă
  mutare decât cea care conduce la soluție, jocul se blochează în câteva
  mutări): (c) - se împinge bila neagră, (d), (e) - se sare de două ori cu
  bilele albe, (f) - se împinge bila neagră, (g) - se sare cu bila neagră, (h)
  - se împinge bila albă. Trebuie din nou remarcat că după mutările (c) și (e)
  se obțin două configurații simetrice.

\end{itemize}

Așadar în ambele cazuri, soluția este unică. De fapt, există două soluții
asemănătoare, una dacă se începe cu o mutare a bilei albe și una dacă se
începe cu o mutare a bilei negre. Fiindcă enunțul impune ca prima mutare să se
facă cu o bilă albă, soluția este unică. Se mai observă și că, atât pentru
$N=1$ cât și pentru $N=2$ șirul de mutări este simetric. Pentru a indica
efectiv modul de determinare a soluției (care va sugera și ideea de scriere a
programului) și pentru a explica observațiile de mai sus, să generalizăm
observațiile făcute pentru un $N$ oarecare.

\tikzset{
  board/.style = {
    matrix of nodes,
    ampersand replacement=\&,
    nodes=cell,
    column 5/.style=noborder,
    column 15/.style=noborder,
  },
  cell/.style = {
    rectangle,
    draw,
    minimum height=1em,
    minimum width=1em,
    font=\huge,
  },
  noborder/.style={
    nodes={draw=none, font=\normalsize},
  },
  empty/.style={text=white},
}

\begin{itemize}

\item Configurația inițială este:

  \centeredTikzFigure[]{
    \matrix[board] (m) {
      $\circ$ \& $\circ$ \& $\circ$ \& $\circ$ \&
      $\cdots$ \&
      $\circ$ \& $\circ$ \& $\circ$ \& $\circ$ \&
      \node[empty] {$\circ$}; \&
      $\bullet$ \& $\bullet$ \& $\bullet$ \& $\bullet$ \&
      $\cdots$ \&
      $\bullet$ \& $\bullet$ \& $\bullet$ \& $\bullet$
      \\
    };
  }

\item Se împinge bila albă și se sare cu cea neagră peste ea (șirul de mutări
  14):

  \centeredTikzFigure[]{
    \matrix[board] (m) {
      $\circ$ \& $\circ$ \& $\circ$ \& $\circ$ \&
      $\cdots$ \&
      $\circ$ \& $\circ$ \& $\circ$ \& $\bullet$ \&
      $\circ$ \&
      \node[empty] {$\circ$}; \& $\bullet$ \& $\bullet$ \& $\bullet$ \&
      $\cdots$ \&
      $\bullet$ \& $\bullet$ \& $\bullet$ \& $\bullet$
      \\
    };
  }

\item Se împinge bila neagră și se sare de două ori cu cele albe peste ea
  (șirul de mutări 322):

  \centeredTikzFigure[]{
    \matrix[board] (m) {
      $\circ$ \& $\circ$ \& $\circ$ \& $\circ$ \&
      $\cdots$ \&
      $\circ$ \& $\circ$ \& \node[empty] {$\circ$}; \& $\bullet$ \&
      $\circ$ \&
      $\bullet$ \& $\circ$ \& $\bullet$ \& $\bullet$ \&
      $\cdots$ \&
      $\bullet$ \& $\bullet$ \& $\bullet$ \& $\bullet$
      \\
    };
  }

\item Se împinge bila albă și se sare de trei ori cu cele negre peste ea
  (șirul de mutări 1444):

  \centeredTikzFigure[]{
    \matrix[board] (m) {
      $\circ$ \& $\circ$ \& $\circ$ \& $\circ$ \&
      $\cdots$ \&
      $\circ$ \& $\bullet$ \& $\circ$ \& $\bullet$ \&
      $\circ$ \&
      $\bullet$ \& $\circ$ \& \node[empty] {$\circ$}; \& $\bullet$ \&
      $\cdots$ \&
      $\bullet$ \& $\bullet$ \& $\bullet$ \& $\bullet$
      \\
    };
  }

\item Se împinge bila neagră și se sare de patru ori cu cele albe peste ea
  (șirul de mutări 32222):

  \centeredTikzFigure[]{
    \matrix[board] (m) {
      $\circ$ \& $\circ$ \& $\circ$ \& $\circ$ \&
      $\cdots$ \&
      \node[empty] {$\circ$}; \& $\bullet$ \& $\circ$ \& $\bullet$ \&
      $\circ$ \&
      $\bullet$ \& $\circ$ \& $\bullet$ \& $\circ$ \&
      $\cdots$ \&
      $\bullet$ \& $\bullet$ \& $\bullet$ \& $\bullet$
      \\
    };
  }

  \hfil\hdashrule{6cm}{1pt}{1pt 4pt}\hfil

\item Se împinge bila albă (mutarea 1)

  \centeredTikzFigure[]{
    \matrix[board] (m) {
      \node[empty] {$\circ$}; \& $\circ$ \& $\bullet$ \& $\circ$ \&
      $\cdots$ \&
      $\circ$ \& $\bullet$ \& $\circ$ \& $\bullet$ \&
      $\circ$ \&
      $\bullet$ \& $\circ$ \& $\bullet$ \& $\circ$ \&
      $\cdots$ \&
      $\circ$ \& $\bullet$ \& $\circ$ \& $\bullet$
      \\
    };
  }

\item Se sare de $N$ ori cu cele negre peste ea (șirul de mutări 44..44):

  \centeredTikzFigure[]{
    \matrix[board] (m) {
      $\bullet$ \& $\circ$ \& $\bullet$ \& $\circ$ \&
      $\cdots$ \&
      $\circ$ \& $\bullet$ \& $\circ$ \& $\bullet$ \&
      $\circ$ \&
      $\bullet$ \& $\circ$ \& $\bullet$ \& $\circ$ \&
      $\cdots$ \&
      $\circ$ \& $\bullet$ \& $\circ$ \& \node[empty] {$\circ$};
      \\
    };
  }

\end{itemize}

Ultimele două configurații sunt simetrice. În acest moment șirul de mutări se
inversează:

\begin{itemize}

\item Se împinge bila albă (mutarea 1):

  \centeredTikzFigure[]{
    \matrix[board] (m) {
      $\bullet$ \& $\circ$ \& $\bullet$ \& $\circ$ \&
      $\cdots$ \&
      $\circ$ \& $\bullet$ \& $\circ$ \& $\bullet$ \&
      $\circ$ \&
      $\bullet$ \& $\circ$ \& $\bullet$ \& $\circ$ \&
      $\cdots$ \&
      $\circ$ \& $\bullet$ \& \node[empty] {$\circ$}; \& $\circ$
      \\
    };
  }

  \hfil\hdashrule{6cm}{1pt}{1pt 4pt}\hfil

\item Se sare de patru ori cu bilele albe și se împinge bila neagră (șirul
  de mutări 22223):

  \centeredTikzFigure[]{
    \matrix[board] (m) {
      $\bullet$ \& $\bullet$ \& $\bullet$ \& $\bullet$ \&
      $\cdots$ \&
      $\bullet$ \& \node[empty] {$\circ$}; \& $\circ$ \& $\bullet$ \&
      $\circ$ \&
      $\bullet$ \& $\circ$ \& $\bullet$ \& $\circ$ \&
      $\cdots$ \&
      $\circ$ \& $\circ$ \& $\circ$ \& $\circ$
      \\
    };
  }

\item Se sare de trei ori cu bilele negre și se împinge bila albă (șirul de
  mutări 4441):

  \centeredTikzFigure[]{
    \matrix[board] (m) {
      $\bullet$ \& $\bullet$ \& $\bullet$ \& $\bullet$ \&
      $\cdots$ \&
      $\bullet$ \& $\bullet$ \& $\circ$ \& $\bullet$ \&
      $\circ$ \&
      $\bullet$ \& \node[empty] {$\circ$}; \& $\circ$ \& $\circ$ \&
      $\cdots$ \&
      $\circ$ \& $\circ$ \& $\circ$ \& $\circ$
      \\
    };
  }

\item Se sare de două ori cu bilele albe și se împinge bila neagră (șirul de
  mutări 223):

  \centeredTikzFigure[]{
    \matrix[board] (m) {
      $\bullet$ \& $\bullet$ \& $\bullet$ \& $\bullet$ \&
      $\cdots$ \&
      $\bullet$ \& $\bullet$ \& $\bullet$ \& \node[empty] {$\circ$}; \&
      $\circ$ \&
      $\bullet$ \& $\circ$ \& $\circ$ \& $\circ$ \&
      $\cdots$ \&
      $\circ$ \& $\circ$ \& $\circ$ \& $\circ$
      \\
    };
  }

\item Se sare cu bila neagră și se împinge bila albă (șirul de mutări 41),
  obținându-se configurația finală:

  \centeredTikzFigure[]{
    \matrix[board] (m) {
      $\bullet$ \& $\bullet$ \& $\bullet$ \& $\bullet$ \&
      $\cdots$ \&
      $\bullet$ \& $\bullet$ \& $\bullet$ \& $\bullet$ \&
      \node[empty] {$\circ$}; \&
      $\circ$ \& $\circ$ \& $\circ$ \& $\circ$ \&
      $\cdots$ \&
      $\circ$ \& $\circ$ \& $\circ$ \& $\circ$
      \\
    };
  }

\end{itemize}

În concluzie, șirul de mutări este: o împingere - un salt - o împingere - două
salturi - o împingere - trei salturi - ... - o împingere - $N-1$ salturi - o
împingere - $N$ salturi - o împingere - $N-1$ salturi - ... - o împingere -
trei salturi - o împingere - două salturi - o împingere - un salt - o
împingere, culorile alternând la fiecare pas.

Pentru a calcula numărul de mutări, putem să le numărăm pe măsură ce le
efectuăm, dar deoarece se cere afișarea mai întâi a numărului de mutări și
după aceea a mutărilor în sine, trebuie fie să stocăm toate mutările în
memorie, fie să lucrăm cu fișiere temporare, ambele variante putând duce la
complicații nedorite. Din fericire, numărul de mutări se poate calcula cu
ușurință astfel: fiecare piesă albă trebuie mutată în medie cu $N$ pași către
dreapta și fiecare piesă neagră trebuie mutată cu $N$ pași către stânga. Deci
numărul total de pași este $2N(N+1).$ Din secvența generală de mutări expusă
mai sus se observă că nu se fac decât $2N$ împingeri de piese (mutări de un
singur pas), restul fiind salturi (mutări de câte doi pași). Deci numărul de
mutări este:

\begin{equation}
  2N + \frac{2N(N + 1) - 2N}{2} = 2N + N^2 = N(N + 2)
\end{equation}

De aici deducem că nu există un algoritm mai bun decât $O(N^2)$, deoarece
numărul de mutări este $O(N^2)$. Propunem ca temă cititorului să demonstreze
că nu există decât două succesiuni de mutări care rezolvă problema, din care
una începe cu mutarea unei piese albe, iar cealaltă este oglindirea ei și
începe cu mutarea unei piese negre, deci nu poate constitui o soluție
corectă. Demonstrația începe prin a arăta că sunt necesare cel puțin $2N$
împingeri de piese. Această demonstrație explică de ce nu se cere un număr
minim de mutări în enunț - cerința nu ar avea sens întrucât soluția este
oricum unică.  Acestea fiind zise, programul arată astfel:

\inputminted{c}{src/problem1.c}

  \section{Problema 2}

Problema următoare a fost propusă la a VI-a Olimpiadă Internațională de
Informatică, Stockholm 1994. Este și ea un bun exemplu de situație în care
putem cădea în plasa unei rezolvări „Branch and Bound” atunci când nu este
cazul.

{\bf ENUNȚ}: Se dă o configurație de $3 \times 3$ ceasuri, fiecare având un
singur indicator care poate arăta numai punctele cardinale (adică orele 3, 6,
9 și 12). Asupra acestor ceasuri se poate acționa în nouă moduri distincte,
fiecare acțiune însemnând mișcarea limbilor unui anumit grup de ceasuri în
sens orar cu 90°. În figura de mai jos se dă un exemplu de configurație
inițială a ceasurilor și se arată care sunt cele nouă tipuri de mutări (pentru
fiecare tip de mutare se mișcă numai ceasurile reprezentate hașurat).

\newcommand\threeByThree[7]{
  \fill[affected] (#1,#2) rectangle (#3,#4); 
  \draw[grid] (0,0) grid (2,2);
  \node[number] at (#5, #6) {#7};
}
\centeredTikzFigure[
  mat/.style = {
    matrix of nodes,
    ampersand replacement=\&,
    row sep=1em,
    column sep=1em,
    anchor=north,
  },
  grid/.style = {step=2/3,gray,very thin},
  affected/.style = {gray!40!white},
  number/.style = {font=\bf\large, anchor=center},
]{
  % clocks
  \matrix[mat] (m) {
    \draw (-1,0) -- (0,0) circle (1); \&
    \draw (-1,0) -- (0,0) circle (1); \&
    \draw (0,+1) -- (0,0) circle (1); \\
    \draw (0,-1) -- (0,0) circle (1); \&
    \draw (0,-1) -- (0,0) circle (1); \&
    \draw (0,-1) -- (0,0) circle (1); \\
    \draw (0,-1) -- (0,0) circle (1); \&
    \draw (+1,0) -- (0,0) circle (1); \&
    \draw (0,-1) -- (0,0) circle (1); \\
  };

  % moves
  \matrix[mat] (l) at (8,0) {
    %             SW corner |NE corner  |text pos |text
    \threeByThree {0/3}{2/3} {4/3}{6/3} {1/3}{5/3} {1} \&
    \threeByThree {0/3}{4/3} {6/3}{6/3} {3/3}{5/3} {2} \&
    \threeByThree {2/3}{2/3} {6/3}{6/3} {5/3}{5/3} {3} \\
    \threeByThree {0/3}{0/3} {2/3}{6/3} {1/3}{3/3} {4} \&
    % need one extra rectangle here -- just reuse the whole command
    \threeByThree {0/3}{2/3} {6/3}{4/3} {3/3}{3/3} {5}
    \threeByThree {2/3}{0/3} {4/3}{6/3} {3/3}{3/3} {5} \&
    \threeByThree {4/3}{0/3} {6/3}{6/3} {5/3}{3/3} {6} \\
    \threeByThree {0/3}{0/3} {4/3}{4/3} {1/3}{1/3} {7} \&
    \threeByThree {0/3}{0/3} {6/3}{2/3} {3/3}{1/3} {8} \&
    \threeByThree {2/3}{0/3} {6/3}{4/3} {5/3}{1/3} {9} \\
  };
}

Se cere ca, într-un număr minim de mutări, să aducem toate indicatoarele la
ora 12.

Intrarea se face din fișierul {\tt INPUT.TXT}, care conține configurația
inițială sub forma unei matrice $3 \times 3$. Pentru fiecare ceas se specifică
câte o cifră: 0 = ora 12, 1 = ora 3, 2 = ora 6, 3 = ora 9.

Ieșirea se va face în fișierul {\tt OUTPUT.TXT} sub forma unui șir de numere
între 1 și 9, pe un singur rând, separate prin spațiu, reprezentând șirul de
mutări care aduc ceasurile în starea finală. Se cere o singură soluție.

{\bf Exemplu}: Pentru figura de mai sus, fișierul {\tt INPUT.TXT} este

\begin{verbatim}
  3 3 0
  2 2 2
  2 1 2
\end{verbatim}

iar fișierul {\tt OUTPUT.TXT} ar putea fi:

\begin{verbatim}
  5 8 4 9
\end{verbatim}

{\bf Timp de implementare}: 1h - 1h 15min.

{\bf Timp de rulare}: o secundă.

{\bf Complexitate cerută}: $O(1)$ (timp constant).

{\bf REZOLVARE}: Din câte am văzut pe la concursuri, peste jumătate din elevi
s-ar apuca direct să implementeze o rezolvare Branch and Bound la această
problemă, fără să-și mai bată capul prea mult. Există argumente în favoarea
acestei inițiative:

\begin{itemize}

\item Mulți preferă să nu mai piardă timpul căutând o altă soluție, mai ales
  că problema seamănă mult cu „Lampa lui Dario Uri” (care de fapt este exact
  problema ceasurilor, dar în care ceasurile au doar două stări în loc de
  patru). În plus, se știe că pe cazul general al unei table $N \times N$,
  cele două probleme nu admit rezolvări polinomiale și atunci cea mai sigură
  soluție este prin tehnica Branch and Bound.

\item De asemenea, se observă că numărul total de configurații posibile pentru
  o tablă cu 9 ceasuri este de $4^9$, adică aproximativ un sfert de milion. Un
  algoritm Branch and Bound ar furniza așadar o soluție în timp
  rezonabil. Raționamentul multor elevi este „decât să pierd timpul căutând o
  soluție mai bună, fără să am certitudinea că o voi găsi, mai bine folosesc
  timpul implementând un Branch care măcar știu sigur că merge”.

\item Problema cere o soluție într-un număr minim de pași, lucru care îi cam
  descurajează pe cei care încă ar vrea să caute alte rezolvări. „Alte
  rezolvări” înseamnă de obicei un Greedy comod de implementat, iar asupra
  rezolvărilor Greedy se poartă întotdeauna discuții interminabile pe
  culoarele sălilor de concurs referitor la „cât de bune sunt” (adică în cât
  la sută din cazuri furnizează soluția optimă).

\end{itemize}

Se pierd însă din vedere unele lucruri esențiale. În primul rând, tabla nu
este de $N \times N$, ci are dimensiuni fixate, $3 \times 3$. În al doilea
rând, implementarea unui Branch and Bound în timp de concurs este o aventură
nu tocmai ușor de dus la bun sfârșit (personal mi-a fost frică să o încerc
vreodată). În sfârșit, după cum se va vedea mai jos, problema șirului minim de
transformări este o pseudo-problemă, deoarece soluția simplă este oricum
unică.

Ce se înțelege prin „soluție simplă”? Să remarcăm două lucruri:

\begin{enumerate}

\item Aplicarea de patru ori a aceleiași mutări nu schimbă nimic în
  configurația ceasurilor. Într-adevăr, mutarea va afecta de fiecare dată
  același grup de ceasuri, iar aplicarea de patru ori va roti fiecare
  indicator cu 360°, adică îl va aduce în poziția inițială. Din acest motiv,
  toate afirmațiile făcute în cele ce urmează vor fi valabile în algebra
  modulo 4.

\item Ordinea în care se aplică transformările nu contează.

\end{enumerate}

În consecință, prin „soluție simplă” se înțelege un șir de mutări ordonat
crescător în care nici o mutare nu apare de mai mult de trei ori. Să
demonstrăm acum că soluția simplă este unică.

Fie $A \in \mathbb{M}_3(\mathbb{Z}_4)$ matricea citită de la intrare, unde
$a_{i,j}$ arată de câte ori a fost rotit ceasul $C_{i,j}$ peste ora 12. Fie
matricea $B \in \mathbb{M}_3(\mathbb{Z}_4), b_{i,j}=4 - a_{i,j}$. Matricea $B$
arată de câte ori mai trebuie rotit fiecare ceas până la ora 12. O soluție
înseamnă a efectua fiecare din cele 9 mutări de un număr de ori, $p_1, p_2,
\dots, p_9$. Cum afectează aceste mutări ceasurile? Se poate deduce ușor:

\begin{tabular}{| l | l |}
  \hline
  Ceasul & Tipurile de mutări care îl afectează \\ \hline
  $C_{1,1}$ & 1, 2, 4 \\
  $C_{1,2}$ & 1, 2, 3, 5 \\
  $C_{1,3}$ & 2, 3, 6 \\
  $C_{2,1}$ & 1, 4, 5, 7 \\
  $C_{2,2}$ & 1, 3, 5, 7, 9 \\
  $C_{2,3}$ & 3, 5, 6, 9 \\
  $C_{3,1}$ & 4, 7, 8 \\
  $C_{3,2}$ & 5, 7, 8, 9 \\
  $C_{3,3}$ & 6, 8, 9 \\
  \hline
\end{tabular}

Se obține deci un sistem de 9 ecuații cu 9 necunoscute:

\begin{equation}
  P = 
  \begin{pmatrix}
    p_1 + p_2 + p_4 & p_1 + p_2 + p_3 + p_5 & p_2 + p_3 + p_6 \\
    p_1 + p_4 + p_5 + p_7 & p_1 + p_3 + p_5 + p_7 + p_9 & p_3 + p_5 + p_6 + p_9 \\
    p_4 + p_7 + p_8 & p_5 + p_7 + p_8 + p_9 & p_6 + p_8 + p_9 \\
  \end{pmatrix}
  \equiv B
\end{equation}

Să presupunem că acest sistem admite două soluții $p_1, \dots, p_9$ și $q_1,
\dots, q_9$. Atunci $P \equiv B \pmod{4}$ și $Q \equiv B \pmod{4}$, deci $P
\equiv Q \pmod{4}$ și, prin diferite combinații liniare ale celor 9 ecuații,
se deduce $p_1 \equiv q_1$, $p_2 \equiv q_2$, ..., $p_9 \equiv q_9 \pmod{4}$,
adică cele două soluții sunt echivalente.

Odată ce am demonstrat că soluția este unică, algoritmul de găsire a ei este
foarte simplu: găsim o soluție oarecare, o ordonăm crescător și eliminăm orice
grup de 4 mutări identice. Pentru a găsi o soluție oarecare, avem nevoie de
niște mutări predefinite care să miște un singur ceas cu o singură poziție
înainte, fără a afecta celelalte ceasuri. Aceste mutări vor fi reținute sub
forma unui vector cu 9 componente, fiecare componentă indicând de câte ori se
efectuează fiecare din cele 9 tipuri de mutări. Deoarece avem nevoie de 9
asemenea mutări predefinite, câte una pentru fiecare ceas, rezultatul va fi o
matrice predefinită. De exemplu, pentru a determina secvența de mutări care
rotește ceasul $C_{1,1}$ cu o poziție, trebuie rezolvat sistemul

\begin{equation}
  \begin{pmatrix}
    p_1 + p_2 + p_4 & p_1 + p_2 + p_3 + p_5 & p_2 + p_3 + p_6 \\
    p_1 + p_4 + p_5 + p_7 & p_1 + p_3 + p_5 + p_7 + p_9 & p_3 + p_5 + p_6 + p_9 \\
    p_4 + p_7 + p_8 & p_5 + p_7 + p_8 + p_9 & p_6 + p_8 + p_9 \\
  \end{pmatrix}
  \equiv
  \begin{pmatrix}
    1 & 0 & 0 \\
    0 & 0 & 0 \\
    0 & 0 & 0
  \end{pmatrix}
\end{equation}

lucru care nu este foarte ușor, dar se poate duce la bun sfârșit în timp de
concurs. Soluția este $p_1=3, p_2=3, p_3=3, p_4=3, p_5=3, p_6=2, p_7=3, p_8=2,
p_9=0$, adică mutarea 1 trebuie efectuată de trei ori, mutarea 2 de trei ori
ș.a.m.d. Se obține prima linie din matricea predefinită, $(3, 3, 3, 3, 3, 2,
3, 2, 0)$. Mai trebuie rezolvate propriu-zis sistemele de ecuații pentru
ceasurile $C_{1,2}$ și $C_{2,2}$, soluțiile celorlalte sisteme decurgând ușor
prin simetrie. Soluțiile apar în textul sursă.

Odată determinate aceste șiruri elementare de mutări, vom lua pe rând fiecare
ceas, vom aplica șirul elementar corespunzător de atâtea ori cât e nevoie
pentru a-l aduce la ora 12 și vom aduna modulo 4 toate mutările
făcute. Vectorul sumă care rezultă este tocmai soluția noastră.

Pentru exemplul din enunț, folosind constantele din programul sursă, obținem:

\begin{equation}
  B =
  \begin{pmatrix}
    1 & 1 & 0 \\
    2 & 2 & 2 \\
    2 & 3 & 2
  \end{pmatrix}
\end{equation}

și

\begin{align*}
1 & \times (3,3,3,3,3,2,3,2,0) = & (3,3,3,3,3,2,3,2,0) & + \\
1 & \times (2,3,2,3,2,3,1,0,1) = & (2,3,2,3,2,3,1,0,1) & \\
0 & \times (3,3,3,2,3,3,0,2,3) = & (0,0,0,0,0,0,0,0,0) & \\
2 & \times (2,3,1,3,2,0,2,3,1) = & (0,2,2,2,0,0,0,2,2) & \\
2 & \times (2,3,2,3,1,3,2,3,2) = & (0,2,0,2,2,2,0,2,0) & \\
2 & \times (1,3,2,0,2,3,1,3,2) = & (2,2,0,0,0,2,2,2,0) & \\
2 & \times (3,2,0,3,3,2,3,3,3) = & (2,0,0,2,2,0,2,2,2) & \\
3 & \times (1,0,1,3,2,3,2,3,2) = & (3,0,3,1,2,1,2,1,2) & \\
2 & \times (0,2,3,2,3,3,3,3,3) = & (0,0,2,0,2,2,2,2,2) & \\
\cline{3-3}
  &                              & (0,0,0,1,1,0,0,1,1) &
\end{align*}

Prin urmare soluția simplă a exemplului este: 4 5 8 9.

\begin{lstlisting}[language=C]
#include <stdio.h>
typedef int Matrix[9][9];
typedef int Vector[9];
const Matrix A=
  {{3,3,3,3,3,2,3,2,0},  // Mutarile care misca ceasul C11
   {2,3,2,3,2,3,1,0,1},  // .
   {3,3,3,2,3,3,0,2,3},  // .
   {2,3,1,3,2,0,2,3,1},  // .
   {2,3,2,3,1,3,2,3,2},  // .
   {1,3,2,0,2,3,1,3,2},  // .
   {3,2,0,3,3,2,3,3,3},  // .
   {1,0,1,3,2,3,2,3,2},  // .
   {0,2,3,2,3,3,3,3,3}}; // Mutarile care misca ceasul C33

void main(void)
{ FILE *F=fopen("input.txt","rt");
  Vector V={0,0,0,0,0,0,0,0,0}; // Vectorul suma
  int i,j,k;

  for (i=0;i<=8;i++)
    { fscanf(F,"%d",&k);
      for (j=0;j<=8;j++)
        V[j]=(V[j]+(4-k)*A[i][j])%4;
    }
  fclose(F);

  F=fopen("output.txt","wt");
  for(i=0;i<=8;i++)
    for(j=1;j<=V[i];j++)
      fprintf(F,"%d ",i+1);
  fclose(F);
}
\end{lstlisting}

  \section{Problema 3}

Problema de mai jos este un exemplu de situație în care căutarea exhaustivă a
soluției este cea mai bună alegere. Ea a fost propusă spre rezolvare la a
VIII-a Olimpiadă Internațională de Informatică, Veszprem, Ungaria 1996.

{\bf ENUNȚ}: Văzând succesul cubului său magic, Rubik a inventat versiunea
plană a jocului, numit „pătrate magice”. Se folosește o tablă compusă din 8
pătrate de dimensiuni egale. Cele opt pătrate au culori distincte, codificate
prin numere de la 1 la 8, ca în figura următoare:

\centeredTikzFigure[
  mat/.style = {matrix of nodes, ampersand replacement=\&, nodes=cell},
  cell/.style = {rectangle, draw, minimum width=1.5em, minimum height=1.5em},
]{
  \matrix[mat] {
    1 \& 2 \& 3 \& 4 \\
    8 \& 7 \& 6 \& 5 \\
  };
}

Configurația tablei se poate reprezenta într-un vector cu 8 elemente citind
cele opt pătrate, începând din colțul din stânga sus și mergând în sens
orar. De exemplu, configurația din figură se reprezintă prin vectorul (1, 2,
3, 4, 5, 6, 7, 8). Aceasta este configurația inițială a tablei.

Unei configurații i se pot aplica trei transformări elementare, identificate
prin literele „A”, „B” și „C”:

\begin{itemize}

\item „A”  schimbă între ele cele două linii ale tablei;

\item „B”  rotește circular spre dreapta întregul dreptunghi (cu o poziție);

\item „C”  rotește în sens orar cele patru pătrate centrale (cu o poziție);

\end{itemize}

Efectele transformărilor elementare asupra configurației inițiale sunt
reprezentate în figura de mai jos:

\centeredTikzFigure[
  mat/.style = {matrix of nodes, ampersand replacement=\&, nodes=cell},
  cell/.style = {rectangle, draw, minimum width=1.5em, minimum height=1.5em},
  caption/.style = {font=\large\bf, yshift=-1em},
]{
  \matrix[mat] (m1) {
    8 \& 7 \& 6 \& 5 \\
    1 \& 2 \& 3 \& 4 \\
  };

  \matrix[mat] (m2) at (5, 0) {
    4 \& 1 \& 2 \& 3 \\
    5 \& 8 \& 7 \& 6 \\
  };

  \matrix[mat] (m3) at (10, 0) {
    1 \& 7 \& 2 \& 4 \\
    8 \& 6 \& 3 \& 5 \\
  };

  \node[caption] at (m1.south) {A};
  \node[caption] at (m2.south) {B};
  \node[caption] at (m3.south) {C};
}

Din configurația inițială se poate ajunge în orice configurație folosind doar
combinații de tranformări elementare. Trebuie să scrieți un program care
calculează o secvență de transformări elementare care să aducă tabla de la
configurația inițială la o anumită configurație finală cerută.

{\bf Intrarea}: Fișierul {\tt INPUT.TXT} conține 8 întregi pe aceeași linie,
separați prin spații, descriind configurația finală.

Ieșirea se va face în fișierul {\tt OUTPUT.TXT}. Pe prima linie a acestuia se
va tipări lungimea $L$ a secvenței de transformări, iar pe fiecare din
următoarele $L$ linii se va tipări câte un caracter „A”, „B” sau „C”,
corespunzător mutărilor care trebuie efectuate.

{\bf Exemplu}:

\texttt{
  \begin{tabular}{| l | l |}
    \hline
        {\bf INPUT.TXT} & {\bf OUTPUT.TXT} \\ \hline
        2 6 8 4 5 7 3 1 & 7 \\
        & B \\
        & C \\
        & A \\
        & B \\
        & C \\
        & C \\
        & B \\
    \hline
  \end{tabular}
}

{\bf Timp limită pentru un test}: 20 secunde.

{\bf Timp de implementare}: 1h 30min - 1h 45min

{\bf Note}:

\begin{enumerate}

\item La concurs s-au acordat, pentru fiecare test, două puncte dacă se
  furniza o soluție și încă două dacă lungimea ei nu depășea 300 de mutări.

\item Concurenților li s-a furnizat un program auxiliar, {\tt MTOOL.EXE}, cu
  care se puteau verifica soluțiile furnizate.

\end{enumerate}

{\bf REZOLVARE}: Și la această problemă se întrevăd două abordări, ca și în
problema ceasurilor: una bazată pe mutări predefinite, iar cealaltă pe o
căutare exhaustivă a soluției. De data aceasta însă, prima este
neinspirată. Să le analizăm pe rând pe fiecare, plecând de la următoarele
considerente:

\begin{itemize}

\item Dacă se aplică de două ori la rând mutarea A, tabla rămâne nemodificată;

\item Dacă se aplică de patru ori consecutiv una din mutările B sau C, tabla
  rămâne nemodificată;

\item Ordinea în care se efectuează mutările contează.

\end{itemize}

Soluția pe care autorul a prezentat-o la concurs avea predefinite mai multe
mutări care schimbau între ele oricare două pătrate vecine de pe
tablă. Mergând din aproape în aproape, fiecare pătrat era adus în poziția
corespunzătoare. Spre exemplu, succesiunea de mutări predefinite care duceau
la configurația din exemplu este:

\centeredTikzFigure[
  mat/.style = {matrix of nodes, ampersand replacement=\&, nodes=cell},
  cell/.style = {rectangle, draw, minimum width=1.5em, minimum height=1.5em},
  emph/.style = {fill=gray!50!white},
]{
  % the 8 boards
  \matrix[mat] (m1) {
    \node[emph]{1}; \& \node[emph]{2}; \& 3 \& 4 \\
    8 \& 7 \& 6 \& 5 \\
  };

  \matrix[mat] (m2) at (3.5, 0) {
    2 \& \node[emph]{1}; \& 3 \& 4 \\
    8 \& \node[emph]{7}; \& 6 \& 5 \\
  };

  \matrix[mat] (m3) at (7, 0) {
    2 \& 7 \& 3 \& 4 \\
    \node[emph]{8}; \& \node[emph]{1}; \& 6 \& 5 \\
  };

  \matrix[mat] (m4) at (10.5, 0) {
    2 \& 7 \& \node[emph]{3}; \& 4 \\
    1 \& 8 \& \node[emph]{6}; \& 5 \\
  };

  \matrix[mat] (m5) at (0, -2) {
    2 \& \node[emph]{7}; \& \node[emph]{6}; \& 4 \\
    1 \& 8 \& 3 \& 5 \\
  };

  \matrix[mat] (m6) at (3.5, -2) {
    2 \& 6 \& 7 \& 4 \\
    1 \& \node[emph]{8}; \& \node[emph]{3}; \& 5 \\
  };

  \matrix[mat] (m7) at (7, -2) {
    2 \& 6 \& \node[emph]{7}; \& 4 \\
    1 \& 3 \& \node[emph]{8}; \& 5 \\
  };

  \matrix[mat] (m8) at (10.5, -2) {
    2 \& 6 \& 8 \& 4 \\
    1 \& 3 \& 7 \& 5 \\
  };

  % arrows
  \draw[->] (m1.east) -- (m2.west);
  \draw[->] (m2.east) -- (m3.west);
  \draw[->] (m3.east) -- (m4.west);
  \draw[->] (m4.east) -- ++(0.75,0);
  \draw[->] (m5.east) -- (m6.west);
  \draw[->] (m6.east) -- (m7.west);
  \draw[->] (m7.east) -- (m8.west);
}

Această soluție funcționează instantaneu și este relativ ușor de
implementat. Ea are însă defectul că soluția furnizată este extrem de lungă,
ajungând frecvent la 500 de mutări. Din cele zece teste date, numai trei s-au
încadrat în limita de 300 de mutări. Iată mai jos și sursa Pascal prezentată
la concurs, care a câștigat numai 26 din cele 40 de puncte acordate pentru
problemă:

\begin{minted}{pascal}
program Magic;
{$B-,I-,R-,S-}
{ Tabla este 1 2 3 4
             A B C D }
const r12='BCBBB'; { Roteste in sens orar coloanele 12 }
      r23='C';     { Roteste in sens orar coloanele 23 }
      r34='BBBCB'; { Roteste in sens orar coloanele 34 }
      r14='BBCBB'; { Roteste in sens orar coloanele 41 }

      plBCD=r12+r23+r34+r14; { Permuta patratele BCD }
      PL234=plBCD+'A'+plBCD+'A'; { Permuta coloanele 234 }

      { SXY schimba intre ele coloanele X si Y }
      S14='B'+PL234;
      S12='BBB'+S14+'B';
      S23='BB'+S14+'BB';
      S34='B'+S14+'BBB';
      S13=r12+r12+r23+r23+r12+r12;
      S24='B'+S13+'BBB';

      { RevXY schimba intre ele patratele vecine X si Y }
      RevC3=plBCD+r23+r12+r12+r34+r34+'BBB'+plBCD+'A';
      RevD4='BBB'+RevC3+'B';
      RevB2='B'+RevC3+'BBB';
      RevA1='BB'+RevC3+'BB';

      Rev23='C'+RevC3+'CCC';
      Rev12='B'+Rev23+'BBB';
      Rev34='BBB'+Rev23+'B';
      Rev14='BB'+Rev23+'BB';

      RevAB='A'+Rev12+'A';
      RevBC='A'+Rev23+'A';
      RevCD='A'+Rev34+'A';
      RevAD='A'+Rev14+'A';

type Matrix=array[1..2,1..4] of Integer;
     Vector=array[1..60000] of Char;
var A,B:Matrix;
    V:Vector;
    N:Integer;

procedure MakeAMatrix;
begin
  A[1,1]:=1;
  A[1,2]:=2;
  A[1,3]:=3;
  A[1,4]:=4;
  A[2,1]:=8;
  A[2,2]:=7;
  A[2,3]:=6;
  A[2,4]:=5;
end;

procedure ReadBMatrix;
begin
  Assign(Input,'input.txt');
  Reset(Input);
  Read(B[1,1]);
  Read(B[1,2]);
  Read(B[1,3]);
  Read(B[1,4]);
  Read(B[2,4]);
  Read(B[2,3]);
  Read(B[2,2]);
  Read(B[2,1]);
  Close(Input);
end;

procedure AddString(S:String);
{ Adauga o secventa la sirul-solutie }
var i:Integer;
begin
  for i:=1 to Length(S) do
    begin
      Inc(N);
      V[N]:=S[i];
    end;
end;

procedure FindElement(K:Integer;var X,Y:Integer);
{ Cauta un element intr-o permutare }
var i,j:Integer;
begin
  for i:=1 to 2 do
    for j:=1 to 4 do
      if A[i,j]=K then begin
                         X:=i;
                         Y:=j;
                         Exit;
                       end;
end;

procedure Switch(var X,Y:Integer);
{ Schimba intre ele doua numere }
var IAux:Integer;
begin
  IAux:=X;X:=Y;Y:=IAux;
end;

procedure Process;
{ Transforma pozitia in pozitia B prin schimbari
  repetate ale elementelor vecine }
var i,j,k,l,m:Integer;
begin
  for j:=1 to 4 do
    for i:=1 to 2 do
      begin
        FindElement(B[i,j],k,l);
        { Gaseste elementul care trebuie adus
          pe pozitia (i,j) }
        if k<>i then begin
                       { Il aduce pe linia corecta }
                       case l of
                         1:AddString(RevA1);
                         2:AddString(RevB2);
                         3:AddString(RevC3);
                         4:AddString(RevD4);
                       end; {case}
                       Switch(A[k,l],A[i,l]);
                       k:=i;
                     end;
        for m:=l downto j+1 do
          { Il aduce pe coloana corecta }
          begin
            if k=1
              then case m of
                     2:AddString(Rev12);
                     3:AddString(Rev23);
                     4:AddString(Rev34);
                   end
              else case m of
                     2:AddString(RevAB);
                     3:AddString(RevBC);
                     4:AddString(RevCD);
                   end;
            Switch(A[k,m],A[k,m-1]);
          end;
      end;
end;

procedure Cut(K,D:Integer);
{ Taie din vectorul V D pozitii incepand cu K }
var i:Integer;
begin
  for i:=K to N-D do
    V[i]:=V[i+D];
  Dec(N,D);
end;

procedure Reduce;
{ Reduce secventele de mutari identice }
var i:Integer;
begin
  i:=1;
  repeat
    case V[i] of
      'A':if (i<=N-1) and (V[i+1]='A')
            then Cut(i,2)
            else Inc(i);
      'B':if (i<=N-3) and (V[i+1]='B')
            and (V[i+2]='B') and (V[i+3]='B')
            then Cut(i,4)
            else Inc(i);
      'C':if (i<=N-3) and (V[i+1]='C')
            and (V[i+2]='C') and (V[i+3]='C')
            then Cut(i,4)
            else Inc(i);
    end; {case}
  until i=N;
end;

procedure WriteSolution;
var i:Integer;
begin
  Assign(Output,'output.txt');
  Rewrite(Output);
  WriteLn(N);
  for i:=1 to N do WriteLn(V[i]);
  Close(Output);
end;

begin
  N:=0;
  MakeAMatrix;
  ReadBMatrix;
  Process;
  Reduce;
  WriteSolution;
end.
\end{minted}

Singura soluție pare deci a fi una de tipul Branch and Bound, care nu este
tocmai la îndemână. Cu toate acestea, numărul total de configurații posibile
ale tablei este de numai 8! = 40.320. Într-adevăr, fiecare poziție de pe tablă
se reprezintă printr-o permutare a mulțimii {1,2,3,4,5,6,7,8}. Se poate face
deci cu ușurință o căutare exhaustivă a soluției. Aceasta simplifică mult
structurile de date folosite (implementarea Branch and Bound folosește
structuri destul de încâlcite). În plus, practica arată că se poate ajunge în
orice configurație în mai puțin de 25 de mutări.

Algoritmul de căutare este cunoscut sub numele de algoritmul lui Lee și are la
bază următoarea idee: Se pornește cu configurația inițială, care este depusă
într-o coadă. La fiecare pas se extrage prima configurație disponibilă din
coadă, se efectuează pe rând fiecare din cele trei mutări și se obțin trei
succesori. Aceștia sunt adăugați la sfârșitul cozii, dacă nu există deja în
coadă. Acest pas se numește {\bf expandare}. Expandarea continuă până când
elementul selectat spre expandare este tocmai configurația finală.

Figura următoare indică modul de expandare a cozii, cu mențiunea că printr-o
succesiune de litere ne-am referit la configurația care se obține efectuând
mutările respective:

\centeredTikzFigure[
  scale=0.8,
  mat/.style = {
    matrix of nodes,
    ampersand replacement=\&,
    row sep=1em,
    column sep=3em,
    nodes=cell,
  },
  cell/.style = {scale=0.8, anchor=west},
  conf/.style = {rectangle, draw, anchor=west, xshift=1em},
]{
  \matrix[mat] {
    \& {\bf Configurații expandate} \& {\bf Coada de configurații neexpandate}
    \\
    Pasul 0:
    \&
    \&
    \node[conf] {Inițială};
    \\
    Pasul 1:
    \&
    \node[conf] {Inițială};
    \&
    \node[conf] (a1) {A};
    \node[conf] (a2) at (a1.east) {B};
    \node[conf] (a3) at (a2.east) {C};
    \draw[-latex] (a1) -- (a2);
    \draw[-latex] (a2) -- (a3);
    \\
    Pasul 2:
    \&
    \node[conf] (b1) {Inițială};
    \node[conf] (b2) at (b1.east) {A};
    \&
    \node[conf] (b3) {B};
    \node[conf] (b4) at (b3.east) {C};
    \node[conf] (b5) at (b4.east) {AB};
    \node[conf] (b6) at (b5.east) {AC};
    \draw[-latex] (b3) -- (b4);
    \draw[-latex] (b4) -- (b5);
    \draw[-latex] (b5) -- (b6);
    \\
    Pasul 3:
    \&
    \node[conf] (c1) {Inițială};
    \node[conf] (c2) at (c1.east) {A};
    \node[conf] (c3) at (c2.east) {B};
    \&
    \node[conf] (c4) {C};
    \node[conf] (c5) at (c4.east) {AB};
    \node[conf] (c6) at (c5.east) {AC};
    \node[conf] (c7) at (c6.east) {BB};
    \node[conf] (c8) at (c7.east) {BC};
    \draw[-latex] (c4) -- (c5);
    \draw[-latex] (c5) -- (c6);
    \draw[-latex] (c6) -- (c7);
    \draw[-latex] (c7) -- (c8);
    \\
    Pasul 4:
    \&
    \node[conf] (d1) {Inițială};
    \node[conf] (d2) at (d1.east) {A};
    \node[conf] (d3) at (d2.east) {B};
    \node[conf] (d4) at (d3.east) {C};
    \&
    \node[conf] (d5) {AB};
    \node[conf] (d6) at (d5.east) {AC};
    \node[conf] (d7) at (d6.east) {BB};
    \node[conf] (d8) at (d7.east) {BC};
    \node[conf] (d9) at (d8.east) {CA};
    \node[conf] (d10) at (d9.east) {CB};
    \node[conf] (d11) at (d10.east) {CC};
    \draw[-latex] (d5) -- (d6);
    \draw[-latex] (d6) -- (d7);
    \draw[-latex] (d7) -- (d8);
    \draw[-latex] (d8) -- (d9);
    \draw[-latex] (d9) -- (d10);
    \draw[-latex] (d10) -- (d11);
    \\
  };
}

Se observă că, la pasul 2, în coadă au fost adăugate doar configurațiile „AB”
și „AC”, iar configurația „AA” nu, deoarece prin efectuarea de două ori a
mutării „A” se revine la configurația inițială, care a fost deja expandată. De
asemenea, la pasul 3, după expandarea configurației „B” au fost adăugate în
coadă numai configurațiile „BB” și „BC”, deoarece configurația „BA” este
echivalentă cu configurația „AB”, aflată deja în listă.

Pseudocodul algoritmului este:

\vspace{\algskip}
\begin{algorithmic}[1]
  \STATE citește datele de intrare
  \STATE inițializează coada cu configurația inițială
  \WHILE{primul element al cozii nu este configurația finală}
  \STATE expandează primul element al cozii
  \FOR{$i = A, B, C$}
  \IF{succesorul $i$ nu a fost deja pus în coadă}
  \STATE adaugă succesorul $i$ în coadă
  \ENDIF
  \ENDFOR
  \STATE șterge primul element al cozii
  \ENDWHILE
  \STATE reconstituie șirul de mutări
\end{algorithmic}

Algoritmul de mai sus garantează și găsirea soluției optime (în număr minim de
mutări). Rămân de lămurit două lucruri: (1) Cum ne dăm seama dacă o
configurație există deja în coadă și (2) cum se face reconstituirea soluției.

Pentru a afla dacă o configurație mai există în listă, cea mai simplă metodă
ar fi o căutare secvențială a listei. Totuși, această versiune ar fi extrem de
lentă, deoarece coada atinge rapid dimensiuni respectabile (de ordinul miilor
de elemente). În plus, un element al listei ar reține configurația
propriu-zisă (un vector cu opt elemente), ceea ce ar duce la un consum ridicat
de memorie. Testul de egalitate a doi vectori ar fi și el costisitor din punct
de vedere al timpului.

Există însă o altă metodă mai simplă. Am demonstrat că numărul de configurații
posibile ale tablei este 8! = 40.320.  Dacă am putea găsi o funcție bijectivă
$H: \mathbf{P_8} \to \{0, 1, \dots, 40.319\}$, unde $\mathbf{P_8}$ este
mulțimea permutărilor de 8 elemente, atunci ar fi suficient un vector
caracteristic cu 40.320 elemente. De îndată ce introducem în coadă o nouă
configurație $K$, nu avem decât să bifăm elementul corespunzător din vectorul
caracteristic. Înainte de a adăuga o configurație în coadă, testăm dacă nu
cumva elementul corespunzător ei a fost deja bifat, semn că nodul a mai fost
vizitat.

Cum se construiește funcția $H$? Pentru orice permutare $p \in \mathbf{P_8}$,
$H(p)$ este poziția lui $p$ în ordonarea lexicografică a lui $\mathbf{P_8}$
(începând de la 0):

\begin{align*}
H(1, 2, 3, 4, 5, 6, 7, 8) & = 0 \\
H(1, 2, 3, 4, 5, 6, 8, 7) & = 1 \\
H(1, 2, 3, 4, 5, 7, 6, 8) & = 2 \\
\cdots \\
H(8, 7, 6, 5, 4, 3, 1, 2) & = 40.318 \\
H(8, 7, 6, 5, 4, 3, 2, 1) & = 40.319 \\
\end{align*}

Se observă că primele 7! = 5.040 elemente din ordonare au pe prima poziție un
1, următoarele 5.040 au pe prima poziție un 2 etc. De asemenea, dintre
elementele care au pe prima poziție un 1, primele 6! = 720 au pe a doua
poziție un 2, următoarele 720 au pe a doua poziție un 3 etc.

Să calculăm de exemplu $H(2, 6, 8, 4, 5, 7, 3, 1)$. Prima cifră a permutării
este 2, deci se adaugă 7! = 5.040. Rămân cifrele 1, 3, 4, 5, 6, 7 și 8. A doua
cifră a permutării este 6, a cincea ca valoare dintre cifrele rămase, deci se
adaugă $4 \times 6! = 2.880$. Rămân cifrele 1, 3, 4, 5, 7 și 8 etc. Se aplică
procedeul până la ultima cifră și rezultă:

\begin{tabular}{|l|l|l|}
  \hline
  Cifre rămase           & Permutarea & Valoarea adăugată \\ \hline
  1, 2, 3, 4, 5, 6, 7, 8 & 2          & $1 \times 7! = 5.040$ \\
  1, 3, 4, 5, 6, 7, 8    & 6          & $4 \times 6! = 2.880$ \\
  1, 3, 4, 5, 7, 8       & 8          & $5 \times 5! = 600$ \\
  1, 3, 4, 5, 7          & 4          & $2 \times 4! = 48$ \\
  1, 3, 5, 7             & 5          & $2 \times 3! = 12$ \\
  1, 3, 7                & 7          & $2 \times 2! = 4$ \\
  1, 3                   & 3          & $1 \times 1! = 1$ \\
  1                      & 1          & $0 \times 0! = 0$ \\ \hline
                         &            & $H(p) = 8.585 $\\ \hline
\end{tabular}

Reciproc se construiește permutarea când i se cunoaște valoarea atașată:

\small{
  \begin{tabular}{|l|l|l|l|l|}
    \hline
    Cifre nefolosite       & $H(p)$ &                      & Cifra selectată & \\ \hline
    1, 2, 3, 4, 5, 6, 7, 8 & 8.585  & $8.585 \bdiv 7! = 1$ & 2 & $8.585 \bmod 7! = 3.545$ \\
    1, 3, 4, 5, 6, 7, 8    & 3.545  & $3.545 \bdiv 6! = 4$ & 6 & $3.545 \bmod 6! = 665$ \\
    1, 3, 4, 5, 7, 8       & 665    & $665 \bdiv 5! = 5$   & 8 & $665 \bmod 5! = 65$ \\
    1, 3, 4, 5, 7          & 65     & $65 \bdiv 4! = 2$    & 4 & $65 \bmod 4! = 17$ \\
    1, 3, 5, 7             & 17     & $17 \bdiv 3! = 2$    & 5 & $17 \bmod 3! = 5$ \\
    1, 3, 7                & 5      & $5 \bdiv 2! = 2$     & 7 & $5 \bmod 2! = 1$ \\
    1, 3                   & 1      & $1 \bdiv 1! = 1$     & 3 & $1 \bmod 1! = 0$ \\
    1                      & 0      & $0 \bdiv 0! = 0$     & 1 & \\ \hline
  \end{tabular}
}

Rezultă $p = (2, 6, 8, 4, 5, 7, 3, 1)$.

Această metodă de căutare are și avantajul că în listă se va ține un singur
număr pe doi octeți, făcându-se economie de memorie. Expandarea unui nod
constă din trei pași:

\begin{enumerate}

\item Se extrage primul număr din listă și se reconstituie configurația
  atașată;

\item Se fac cele trei mutări, obținându-se trei succesori;

\item Pentru fiecare succesor se calculează funcția $H$ și dacă configurația
  nu este găsită în listă, este adăugată.

\end{enumerate}

Pentru a face reconstituirea soluției avem nevoie de date
suplimentare. Respectiv, vectorul caracteristic atașat permutărilor nu va mai
reține doar dacă o poziție a fost „văzută” sau nu, ci și poziția din care ea
provine (prin valoarea funcției $H$). Trebuie de asemenea reținut tipul
mutării (A, B sau C) prin care s-a ajuns în acea configurație. Cei doi vectori
se numesc {\tt Father} și {\tt MoveKind}. Inițial, toate elementele vectorului
{\tt Father} au eticheta „Unknown”, semnificând că nodurile nu au fost încă
vizitate, cu excepția elementului atașat configurației inițiale, care poartă
eticheta specială „Root” (rădăcină).

Pseudocodul pentru expandarea unui nod arată cam așa:

\vspace{\algskip}
\begin{algorithmic}[1]
  \STATE $K \leftarrow$ primul număr din coadă
  \STATE $P \leftarrow H^{-1}(K)$
  \STATE află cei trei succesori $Q_A, Q_B, Q_C$
  \FOR{$i = A, B, C$}
  \IF{$Father[H(Q_i)] = Unknown$}
  \STATE $Father[H(Q_i)] \leftarrow K$
  \STATE $MoveKind[H(Q_i)] \leftarrow i$
  \STATE adaugă $H(Q_i)$ în coadă
  \ENDIF
  \ENDFOR
  \STATE șterge $K$ din coadă
\end{algorithmic}

Reconstituirea soluției se face recursiv: se pornește de la configurația
finală și se merge înapoi (folosind informația din vectorul {\tt Father}) până
la configurația inițială, măsurându-se astfel numărul de mutări. La revenire
se tipăresc toate mutările efectuate (folosind informația din vectorul {\tt
  MoveKind}).

\begin{lstlisting}[language=C]
#include <stdio.h>
#include <stdlib.h>
#define Unknown 0xFFFF
#define Root 0xFFFE

typedef huge unsigned Vector[40320];
typedef char CharVector[40320];
typedef int Perm[8];
typedef struct list { unsigned X; struct list * Next; } List;

Perm StartPerm={1,2,3,4,5,6,7,8}, EndPerm;
const Perm Moves[3]=
   {{7,6,5,4,3,2,1,0},
    {3,0,1,2,5,6,7,4},
    {0,6,1,3,4,2,5,7}};
/* Cele trei tipuri de mutari */
Vector Father; /* Legaturile de tip tata */
CharVector MoveKind;
unsigned StartValue,EndValue;
/* Valorile atasate configuratiilor initiala si finala */
List *Head, *Tail; /* Coada de expandat */

/**** Bijectia care asociaza un numar unei permutari ****/

unsigned Perm2Int(Perm P)
{ int i,j,k,Fact=5040;
  unsigned Sum=0;

  for (i=0;i<=6;i++)
    { k=P[i]-1;
      for (j=0;j<i;j++)
        if (P[j]<P[i]) k--;
      Sum+=k*Fact;
      Fact/=(7-i);
    }
  return Sum;
}

void Int2Perm(unsigned Sum, Perm P)
{ int i,j,k,Order,Fact=5040;
  Perm Used={0,0,0,0,0,0,0,0};

  for (i=0;i<=7;i++)
    { Order=Sum/Fact;
      j=-1;
      for (k=0;k<=Order;k++)
        do j++; while (Used[j]);
      Used[j]=1;
      P[i]=j+1;
      Sum%=Fact;
      if (i!=7) Fact/=(7-i);
    }
}

/**** Lucrul cu liste ****/

void InitList(void)
{
  Head=Tail=malloc(sizeof(List));
  Tail->X=StartValue;
  Tail->Next=NULL;
}

void AddToTail(unsigned K)
{
  Tail->Next=malloc(sizeof(List));
  Tail=Tail->Next;
  Tail->X=K;
  Tail->Next=NULL;
}

void Behead(void)
/* Sterge capul listei */
{ List *LCor=Head;

  Head=Head->Next;
  free(LCor);
}

/**** Cautarea propriu-zisa ****/

void MakeMove(Perm P,Perm Q,int Kind)
/* Kind = 0, 1 sau 2 */
{ int i;
  for (i=0;i<=7;i++)
    Q[i]=P[Moves[Kind][i]];
}

void Expand(void)
{ List *LCor;
  Perm P1,P2;
  unsigned i,XSon,Done;

  InitList();
  do {
    Int2Perm(Head->X,P1);
    for(i=0;i<=2;i++)
      { MakeMove(P1,P2,i);
        XSon=Perm2Int(P2);
        if (Father[XSon]==Unknown)
          { Father[XSon]=Head->X;
            MoveKind[XSon]=i+65;
            AddToTail(XSon);
          }
      }
    Done=(Head->X==EndValue);
    Behead(); }
  while (!Done);
}

/* Intrarea si iesirea */

void InitData(void)
{ FILE *F=fopen("input.txt","rt");
  unsigned i;

  for (i=0;i<=7;i++) fscanf(F,"%d",&EndPerm[i]);
  fclose(F);
  StartValue=Perm2Int(StartPerm);
  EndValue=Perm2Int(EndPerm);
  for (i=0;i<40320;) Father[i++]=Unknown;
  Father[StartValue]=Root;
}

void WriteMove(FILE *F,unsigned K,int Len)
{
  if (K!=StartValue)
    { WriteMove(F,Father[K],Len+1);
      fprintf(F,"%c\n",MoveKind[K]);
    }
    else fprintf(F,"%d\n",Len);
}

void Restore(void)
{ FILE *F=fopen("output.txt","wt");

  WriteMove(F,EndValue,0);
  fclose(F);
}

void main(void)
{
  InitData();
  Expand();
  Restore();
}
\end{lstlisting}

  \section{Problema 4}

Continuăm cu o problemă care a fost de asemenea dată spre rezolvare la a
VIII-a Olimpiadă Internațională de Informatică, Veszprem 1996. Problema în
sine nu a fost foarte grea și mulți elevi au luat punctaj maxim. Totuși,
enunțul permite unele modificări interesante care practic schimbă cu totul
problema.

{\bf ENUNȚ}: Să considerăm următorul joc de două persoane. Tabla de joc constă
într-o secvență de întregi pozitivi. Cei doi jucători mută pe rând. Mutarea
fiecărui jucător constă în alegerea unui număr de la unul din cele două capete
ale secvenței. Numărul ales este șters de pe tablă. Jocul se termină când
toate numerele au fost selectate. Primul jucător câștigă dacă suma numerelor
alese de el este mai mare sau egală cu cea a numerelor alese de cel de-al
doilea jucător. În caz contrar, câștigă al doilea jucător.

Dacă tabla conține inițial un număr par de elemente, atunci primul jucător are
o strategie de câștig. Trebuie să scrieți un program care implementează
strategia cu care primul jucător câștigă jocul. Răspunsurile celui de-al
doilea jucător sunt date de un program rezident. Cei doi jucători comunică
prin trei proceduri ale modulului {\tt Play} care v-a fost pus la
dispoziție. Procedurile sunt {\tt StartGame}, {\tt MyMove} și {\tt
  YourMove}. Primul jucător începe jocul apelând procedura fără parametri {\tt
  StartGame}. Dacă alege numărul de la capătul din stânga, el va apela
procedura {\tt MyMove('L')}. Analog, apelul de procedură {\tt MyMove('R')}
trimite un mesaj celui de-al doilea jucător prin care îl informează că a ales
numărul de la capătul din dreapta. Cel de-al doilea jucător, deci computerul,
mută imediat, iar primul jucător poate afla mutarea acestuia executând
procedura {\tt YourMove(C)}, unde {\tt C} este o variabilă de tip {\tt Char}
(în C/C++ apelul este {\tt YourMove(\&C)}). Valoarea lui {\tt C} este {\tt
  'L'} sau {\tt 'R'}, după cum numărul ales este de la capătul din stânga sau
din dreapta.

{\bf Intrarea}: Prima linie din fișierul {\tt INPUT.TXT} conține dimensiunea
inițială $N$ a tablei. $N$ este par și $2 \leq N \leq 100$. Următoarele $N$
linii conțin fiecare câte un număr, reprezentând conținutul tablei de la
stânga la dreapta. Fiecare număr este cel mult 200.

{\bf Ieșirea}: Când jocul se termină, programul trebuie să scrie rezultatul
final în fișierul text {\tt OUTPUT.TXT}. Fișierul conține două numere pe prima
linie, reprezentând suma numerelor alese de primul, respectiv de cel de-al
doilea jucător. Programul trebuie să joace un joc corect și ieșirea trebuie să
corespundă jocului jucat.

{\bf Exemplu}:

\texttt{
  \begin{tabular}{| l | l |}
    \hline
        {\bf INPUT.TXT} & {\bf OUTPUT.TXT} \\ \hline
        6 & 15 14 \\
        4 & \\
        7 & \\
        2 & \\
        9 & \\
        5 & \\
        2 & \\
    \hline
  \end{tabular}
}

{\bf Timp limită de execuție}: 20 secunde pentru un test.

Acesta a fost enunțul original, la care va trebui să facem câteva modificări,
în parte deoarece nu putem folosi modulul {\tt Play}, în parte pentru a face
problema mai restrictivă:

\begin{itemize}

\item Mutările vor fi anunțate pe ecran prin tipărirea unui caracter {\tt 'L'}
  sau {\tt 'R'};

\item Mutările celui de-al doilea jucător vor fi comunicate de un partener
  uman, prin introducerea de la tastatură a unui caracter {\tt 'L'} sau {\tt
    'R'};

\item Rezultatul final se va tipări pe ecran, sub aceeași formă (pereche de
  numere).

\item Timpul de gândire pentru fiecare mutare trebuie să fie cât mai mic
  (practic răspunsul să fie instantaneu);

\item {\bf Complexitatea totală} a calculelor efectuate să fie $O(N)$.

\item {\bf Timpul de implementare} a fost cam de 1h 40 min. Propunem reducerea
  lui la 30 minute.

\end{itemize}

{\bf REZOLVARE}: Este ușor de demonstrat că o rezolvare „greedy” a problemei
(la fiecare mutare jucătorul 1 alege numărul mai mare) nu atrage întotdeauna
câștigul. Iată un contraexemplu:

\centeredTikzFigure[
  mat/.style = {
    matrix of nodes,
    ampersand replacement=\&,
    column sep = 0.5em,
    nodes=cell,
  },
  cell/.style = {rectangle, draw, minimum width=1.5em, minimum height=1.5em},
]{
  \matrix[mat] {
    7 \& 10 \& 1 \& 2 \\
  };
}

La prima mutare, jucătorul 1 poate să aleagă fie numărul 7, fie numărul
2. Dacă se va „lăcomi” la 7, jucătorul 2 va lua numărul 10 și inevitabil va
câștiga. Soluția pentru primul jucător este să ia numărul 2, apoi, indiferent
de ce va juca partenerul său, va putea lua numărul 10 și va câștiga.

Iată o soluție izbitor de simplă de complexitate $O(N)$: La citirea datelor se
face suma elementelor aflate pe poziții pare și a celor aflate pe poziții
impare. Să presupunem că suma elementelor de ordin par este mai mare sau egală
cu cea a elementelor de ordin impar (cazul invers se tratează analog). Atunci,
dacă primul jucător ar putea să aleagă toate elementele de ordin par (care
sunt într-adevăr $N/2$, adică atâtea câte are el dreptul să aleagă), ar
câștiga jocul. Jucătorul 1 poate începe jocul prin a lua primul sau ultimul
element din secvență, deci îl va alege pe ultimul, care are număr de ordine
par. Al doilea jucător are de ales între primul și al $N-1$-lea element,
ambele având număr de ordine impar. Indiferent ce variantă o va adopta, primul
jucător va avea din nou acces la un element de pe o poziție pară. Dacă
jucătorul 2 alege elementul din stânga (primul), atunci jucătorul 1 va putea
lua elementul de după el (al doilea), iar dacă jucătorul 2 alege elementul din
dreapta (al $N-1$-lea), atunci jucătorul 1 va putea lua elementul dinaintea el
(al $N-2$-lea). Deci primul jucător nu are altceva de făcut decât să repete
mutările făcute de cel de-al doilea. Să privim de exemplu desfășurarea jocului
pe tabla dată în enunț:

\centeredTikzFigure[
  mat/.style = {
    matrix of nodes,
    ampersand replacement=\&,
    row sep=0.5em,
    column sep=0.5em,
    nodes=cell,
  },
  cell/.style = {draw, minimum width=1.5em, minimum height=1.5em},
  row 1/.style = {draw=white, nodes={minimum height=2em}},
  noborder/.style={column #1/.style={nodes={draw=none}}},
  noborder/.list={1, 8},
]{
  \matrix[mat] (m) {
    Jucătorul 1 \& 1 \& 2 \& 3 \& 4 \& 5 \& 6 \& Jucătorul 2 \\
    0           \& 4 \& 7 \& 2 \& 9 \& 5 \& 2 \& 0  \\
    2           \& 4 \& 7 \& 2 \& 9 \& 5 \&   \& 0  \\
    2           \& 4 \& 7 \& 2 \& 9 \&   \&   \& 5  \\
    11          \& 4 \& 7 \& 2 \&   \&   \&   \& 5  \\
    11          \&   \& 7 \& 2 \&   \&   \&   \& 9  \\
    18          \&   \&   \& 2 \&   \&   \&   \& 9  \\
    18          \&   \&   \&   \&   \&   \&   \& 11 \\
  };

  \node at ([yshift=-1em]m.south) {$7 + 9 + 2 > 4 + 2 + 5$};
}

Programul în sine nici nu are nevoie să mai rețină vectorul de numere în
memorie, din moment ce primul jucător nu are altceva de făcut decât să imite
mutările celui de-al doilea. Un calcul al sumelor la citirea datelor este
suficient. Complexitatea $O(N)$ este optimă, deoarece vectorul trebuie parcurs
cel puțin o dată pentru citirea configurației inițiale a tablei.

\begin{minted}{c}
#include <stdio.h>

void main(void)
{ FILE *F=fopen("input.txt","rt");
  int SEven,SOdd,N,i,K;

  fscanf(F,"%d\n",&N);
  for (i=1, SEven=SOdd=0; i<=N; i++)
    { fscanf(F, "%d\n", &K);
      if (i&1) SOdd+=K;
        else SEven+=K;
    }
  fclose(F);

  printf("Mutarea mea: %c\n", SEven>=SOdd ? 'R' : 'L');
  for (i=1; i<N/2; i++)
    { printf("Mutarea dvs. (L/R) ? ");
      printf("Mutarea mea: %c\n", getchar());
      getchar(); /* Caracterul newline */
    }
  printf("Mutarea dvs. (L/R) ? ");
  getchar();

  if (SEven>=SOdd)
    printf("%d %d\n", SEven, SOdd);
    else printf("%d %d\n", SOdd, SEven);
}
\end{minted}

O a doua variantă a enunțului aduce unele condiții suplimentare:

\begin{itemize}

\item Se cere să se tipărească numai diferența maximă de scor pe care o poate
  obține primul jucător, considerând că ambii parteneri joacă perfect;

\item {\bf Complexitatea cerută} este $O(N^2)$.

\item {\bf Timpul de implementare} este de 45 minute, maxim 1h.

\end{itemize}

{\bf REZOLVARE}: Trebuie mai întâi să lămurim ce se înțelege prin „joc
perfect”. Jucătorul 1 are întotdeauna victoria la îndemână (metoda este
arătată mai sus), dar nu la orice scor. Jucătorul 2 urmărește să minimizeze
diferența de scor. Fie $D$ diferența de scor cu care se termină un joc. $D$
poate lua diferite valori pentru aceeași configurație inițială a tablei, în
funcție de mutările făcute de cei doi jucători. Fie $D_{MAX}$ diferența maximă
de scor pe care o poate obține primul jucător indiferent de mutările celui
de-al doilea. Exact această valoare trebuie aflată. $D_{MAX}$ nu este
propriu-zis o diferență maximă. Jucătorul 1 poate să câștige și la diferențe
mai mari decât $D_{MAX}$, dar trebuie ca jucătorul 2 să-l „ajute”. Să reluăm
exemplul cu 4 numere:

\centeredTikzFigure[
  mat/.style = {
    matrix of nodes,
    ampersand replacement=\&,
    column sep = 0.5em,
    nodes=cell,
  },
  cell/.style = {rectangle, draw, minimum width=1.5em, minimum height=1.5em},
]{
  \matrix[mat] {
    7 \& 10 \& 1 \& 2 \\
  };
}

În acest caz, primul jucător are asigurat scorul 12-8 (deci diferența
4). Pentru aceasta, el începe prin a lua numărul 2, apoi, orice ar replica
celălalt, va lua numărul 10, jucătorului 2 revenindu-i așadar numerele 1 și
7. El poate obține și scorul 17-3 (jucătorul 1 ia numărul 7, celălalt ia 2,
jucătorul 1 ia 10, iar celălalt ia 1), dar aceasta se întâmplă numai dacă
jucătorul 2 face o greșeală. După cum am arătat mai sus, dacă primul jucător
începe luând numărul 7, el pierde în mod normal partida. Iată deci că în acest
caz $D_{MAX}=4$.

Pentru a putea afla diferența maximă de scor, este bine să privim mereu în
adâncime. Există patru variante în care ambii parteneri pot face câte o
mutare:

\begin{enumerate}

\item Ambii jucători aleg numere din partea stângă;

\item Ambii aleg numere din partea dreaptă;

\item Primul jucător alege numărul din stânga, iar celălalt pe cel din
  dreapta;

\item Primul jucător alege numărul din dreapta, iar celălalt pe cel din
  stânga;

\end{enumerate}

În urma oricărei variante de mutare, secvența se scurtează cu două
elemente. Dacă am putea cunoaște dinainte care este rezultatul jocului pentru
fiecare din secvențele scurte, am putea să decidem care variantă de joc este
cea mai convenabilă pentru secvența inițială, ținând cont și de modul de joc
al jucătorului al doilea. Tocmai de aici vine și ideea de rezolvare. Să notăm
cu $A[1], A[2], ..., A[N]$ secvența citită la intrare. Vom construi o
matrice $D$ cu $N$ linii și $N$ coloane, unde $D[i,j]$ este diferența maximă
pe care o poate obține jucătorul 1 pentru secvența $A[i], A[i+1], \dots,
A[j]$. Bineînțeles, sunt luate în considerare numai secvențele de lungime
pară. Scopul nostru este să-l aflăm pe $D[1,N]$.

Elementele matricei pe care le putem afla fără multă bătaie de cap sunt
$D[1,2]$, $D[2,3]$, $\dots$, $D[N-1,N]$. Într-adevăr, dintr-o secvență de
numai două numere, primului jucător îi revine cel mai mare, iar celui de-al
doilea - cel mai mic. Așadar

\begin{equation}
  D[i,i + 1] = |A[i] - A[i + 1]|
\end{equation}

Cum calculăm $D[i,j]$ dacă cunoaștem valorile matricei $D$ pentru toate
subsecvențele incluse în secvența $A[i], A[i+1], ..., A[j]$? După cum am mai
spus, avem patru variante:

\centeredTikzFigure[
  scale=0.7,
  every node/.style={scale=0.7, anchor=base west},
  mat/.style = {
    matrix of nodes,
    ampersand replacement=\&,
  },
  header/.style={text width=9em, yshift=4em},
]{
  \matrix[mat] (m) {
    \node[header] {Elementul selectat de primul jucător}; \&
    \node[header] {Elementul selectat de al doilea jucător}; \&
    \node[header] {Rezultatul pentru secvența rămasă}; \&
    \node[header] {Rezultatul final}; \\
    $A[i]$ \&
    $A[i + 1]$ \&
    $D[i + 2, j]$ \&
    $R_1 = D[i + 2, j] + A[i] - A[i + 1]$ \\
    \&
    $A[j]$ \&
    $D[i + 1, j - 1]$ \&
    $R_2 = D[i + 1, j - 1] + A[i] - A[j]$ \\
    $A[j]$ \&
    $A[i]$ \&
    $D[i + 1, j - 1]$ \&
    $R_3 = D[i + 1, j - 1] + A[j] - A[i]$ \\
    \&
    $A[j - 1]$ \&
    $D[i, j - 2]$ \&
    $R_4 = D[i, j - 2] + A[j] - A[j - 1]$ \\
  };

  \draw[->] (m-2-1.east) -- (m-2-2.west);
  \draw[->] (m-2-2.east) -- (m-2-3.west);
  \draw[->] (m-2-3.east) -- (m-2-4.west);
  \draw[->] (m-2-1.east) -- (m-3-2.west);
  \draw[->] (m-3-2.east) -- (m-3-3.west);
  \draw[->] (m-3-3.east) -- (m-3-4.west);

  \draw[->] (m-4-1.east) -- (m-4-2.west);
  \draw[->] (m-4-2.east) -- (m-4-3.west);
  \draw[->] (m-4-3.east) -- (m-4-4.west);
  \draw[->] (m-4-1.east) -- (m-5-2.west);
  \draw[->] (m-5-2.east) -- (m-5-3.west);
  \draw[->] (m-5-3.east) -- (m-5-4.west);
}

Trebuie să ținem minte că, dacă primul jucător optează să-l aleagă pe $A[i]$
(una din primele două variante), atunci jucătorul 2 va juca în așa fel încât
pierderea să fie minimă, iar scorul final va fi $\min(R_1,R_2)$. Dacă
jucătorul 1 alege varianta 3 sau 4, scorul final va fi $\min(R_3,R_4)$. Dar
jucătorul 1 este primul la mutare, deci va alege varianta care îi maximizează
profitul. Rezultatul este

\begin{equation}
  D[i,j] = \max(\min(R_1, R_2), \min(R_3, R_4))
\end{equation}

adică

\begin{align}
  \begin{split}
    D[i,j] = \max( & A[i] + \min(D[i + 2, j] - A[i + 1], D[i + 1, j - 1] - A[j]), \\
    & A[j] + \min(D[i + 1, j - 1] - A[i], D[i, j - 2] - A[j - 1]))
  \end{split}
\end{align}

Matricea $D$ se completează pe diagonală, pornind de la diagonala principală
și mergând până în colțul de N-E. Iată cum arată matricea atașată datelor de
intrare din enunț:

\begin{equation}
  D =
  \begin{pmatrix}
    X & 3 & X & 10 & X & 7 \\
    X & X & 5 & X & 9 & X \\
    X & X & X & 7 & X & 4 \\
    X & X & X & X & 4 & X \\
    X & X & X & X & X & 3 \\
    X & X & X & X & X & X
  \end{pmatrix}
\end{equation}

Pentru exemplul din enunț, răspunsul este deci $D_{MAX}=7$. Cum elementele
matricei sunt parcurse cel mult o dată, rezultă o complexitate de $O(N^2)$.

\begin{minted}{c}
#include <stdio.h>
#include <stdlib.h>
#define NMax 101

int D[NMax][NMax], A[NMax], N;

void ReadData(void)
{ FILE *F=fopen("input.txt","rt");
  int i;

  fscanf(F,"%d\n",&N);
  for (i=1; i<=N;)
    fscanf(F, "%d\n", &A[i++]);
  fclose(F);
}

int Min(int A, int B)
{
  return A<B ? A : B;
}

int Max(int A, int B)
{
  return A>B ? A : B;
}

void FindMax(void)
{ int i,j,k;

  for (i=1;i<N;i++)
    D[i][i+1]=abs(A[i]-A[i+1]);
  for (k=3;k<=N-1;k++)
    for (i=1;i+k<=N;i++)
      { j=i+k;
        D[i][j]=Max(A[i]+Min(D[i+2][j]-A[i+1],
                             D[i+1][j-1]-A[j]),
                    A[j]+Min(D[i+1][j-1]-A[i],
                             D[i][j-2]-A[j-1]));
      }
  printf("Diferenta maxima este %d\n",D[1][N]);
}

void main(void)
{
  ReadData();
  FindMax();
}
\end{minted}

Programul prezentat mai sus poate fi optimizat, dacă timpul o permite și dacă
acest lucru este necesar. Lăsăm cititorul să încerce să rezolve aceeași
problemă folosind o cantitatate de memorie direct proporțională cu $N$.

  \section{Problema 5}

Această problemă a fost propusă la Olimpiada Națională de Informatică, Slatina
1995, la clasa a XI-a. Pe atunci programarea dinamică era o tehnică de
programare destul de puțin cunoscută de către majoritatea elevilor.

{\bf ENUNȚ}: O regiune deșertică este reprezentată printr-un tablou de
dimensiuni $M \times N$ ($1 \leq M \leq 100$, $1 \leq N \leq 100$). Elementele
tabloului sunt numere naturale mai mici ca 255, reprezentând diferențele de
altitudine față de nivelul mării (cota 0). Să se stabilească:

a) Un traseu pentru a traversa deșertul de la nord la sud (de la linia 1 la
linia $M$), astfel:

\begin{itemize}

\item Se pornește dintr-un punct al liniei 1;

\item Deplasarea se poate face în una din direcțiile: E, SE, S, SV, V;

\item Suma diferențelor de nivel (la urcare și la coborâre) trebuie să fie
  minimă.

\end{itemize}

b) Un traseu pentru a traversa deșertul de la nord la sud în condițiile
punctului (a), la care se adaugă condiția:

\begin{itemize}

\item Lungimea traseului să fie minimă.

\end{itemize}

{\bf Intrarea}: Fișierul de intrare {\tt INPUT.TXT} conține un singur set de
date cu următoarea structură:

\begin{tabular}{ll}
  linia 1: & $M$ $N$ \\
  liniile $2 \dots M + 1$: & elementele tabloului (pe linii) separate prin spații \\
\end{tabular}

{\bf Ieșirea}: Fișerul de ieșire {\tt OUTPUT.TXT} va conține rezultatele în
următorul format:

\begin{verbatim}
  (a)
  <suma diferențelor de nivel>
  TRASEU: (i_1, j_1)->(i_2, j_2)->...->(i_k, j_k)
  (b)
  <suma diferențelor de nivel> <lungime traseu>
  TRASEU: (i_1, j_1)->(i_2, j_2)->...->(i_p, j_p)
\end{verbatim}

unde $i_x$ și $j_x$ sunt linia și coloana fiecărei celule vizitate.

{\bf Exemplu}:

\texttt{
  \begin{tabular}{|l|l|}
    \hline
        {\bf INPUT.TXT} & {\bf OUTPUT.TXT} \\ \hline
        4 4 &             (a) \\
        10 7 2 5 &        5 \\
        13 20 25 3 &      TRASEU: (1,3)->(2,4)->(3,3)->(3,2)->(4,1) \\
        2 4 2 20 &        (b) \\
        5 10 9 11 &       9 3 \\
        &                 TRASEU: (1,3)->(2,4)->(3,3)->(4,2) \\
    \hline
  \end{tabular}
}

Acesta a fost enunțul original. Iată acum completările propuse și o precizare
importantă:

\begin{itemize}

\item {\bf Timpul de implementare}: 45 minute - 1h (la concurs a fost cam 1h
  30 min);

\item {\bf Timpul de rulare}: 2-3 secunde;

\item {\bf Complexitatea cerută}: $O(N^2)$;

\item La punctul (b), condiția nou adăugată este mai puternică decât cea de la
  punctul (a). Cu alte cuvinte, în primul rând contează lungimea drumului și
  abia apoi, dintre toate drumurile de lungime minimă, trebuie ales cel pentru
  care suma denivelărilor este minimă. Pentru a vă convinge că ordinea în care
  sunt impuse condițiile este importantă, să privim exemplul de mai sus. Dacă
  este mai importantă minimizarea sumei denivelărilor, atunci minimul este 5,
  iar drumul este soluția de la punctul (a). Dacă este mai importantă
  minimizarea lungimii drumului, atunci lungimea minimă este 3, iar din toate
  drumurile de lungime 3, cel mai puțin costisitor este cel indicat la punctul
  (b).

\end{itemize}

{\bf REZOLVARE}: Vom lăsa punctul (b) al acestei probleme în seama
cititorului, întrucât el nu este altceva decât o simplificare a punctului
(a). Să ne ocupăm acum de punctul (a). Vom numi efort diferența de altitudine
(în modul) la deplasarea cu un pas. Scopul este deci găsirea unor drumuri de
efort total minim. Matricea de altitudini o vom nota cu $Alt$.

O primă posibilitate de abordare a problemei este „greedy”, dar aceasta nu e
cea mai fericită alegere, chiar dacă este una comodă. Ideea de bază este
următoarea: Încercăm să pornim din colțul de NV și să ne deplasăm la fiecare
pas pe acea direcție pentru care efortul este minim, până ajungem la ultima
linie. Apoi pornim din a doua coloană a primei linii și aplicăm aceeași
tactică, apoi din a treia coloană și așa mai departe până la colțul de NE. În
final tipărim soluția cea mai bună găsită. Iată însă un exemplu pe care
această metodă dă greș:

\begin{equation}
  Alt =
  \begin{pmatrix}
    2 & 1 & 2 \\
    10 & 1 & 10 \\
    10 & 10 & 10
  \end{pmatrix}
\end{equation}

Pe această matrice, algoritmul greedy va găsi traseele (1,1) $\to$ (2,2) $\to$
(3,2) de efort total 10, (1,2) $\to$ (2,2) $\to$ (3,2) de efort total 9 și
(3,1) $\to$ (2,2) $\to$ (3,2) de efort total 10. Așadar, rezultatul optim ar
fi 9, ceea ce este fals deoarece alegerea traseului (1,1) $\to$ (2,1) $\to$
(3,1) ar duce la un efort total de 8, deci mai mic.

Motivul pentru care acest algoritm nu funcționează cum trebuie este că el nu
privește în perspectivă. În cazul de mai sus, coborârea în „văile” de
altitudine 1 era o primă mutare tentantă, dar fără nici un rezultat, deoarece
până la urmă tot era necesară suirea la altitudinea 10. Soluția corectă este
ca, pentru a afla efortul minim cu care se poate ajunge la o locație oarecare,
să analizăm toate drumurile care duc la acea locație. Dacă am cunoaște efortul
minim cu care se poate ajunge la fiecare din vecinii din E, NE, N, NV, și V ai
unei celule, atunci putem cu ușurință, pe baza unor comparații, să deducem din
ce parte este cel mai avantajos să venim în respectiva celulă și cu ce efort
minim.

\newcommand\EFF{\mathit{Eff}}

Mai concret, vom construi o matrice cu aceleași dimensiuni ca și matricea
$Alt$, pe care o vom denumi $\EFF$. În această matrice, $\EFF[i,j]$ reprezintă
efortul minim necesar pentru a ajunge de pe un punct oarecare de pe linia 1 în
celula $(i, j)$. Deducem că $\EFF[1,j] = 0, \forall 1 \leq j \leq N$. Noi
trebuie să completăm matricea $\EFF$, apoi să căutăm minimul dintre toate
elementele de pe linia $M$ (care este chiar efortul minim căutat) și să
reconstituim traseul de urmat.

Ca să vedem cum anume se face completarea matricei, facem mai întâi observația
că, odată ce am ajuns pe o linie, putem fie să coborâm direct pe linia imediat
inferioară, fie să ne deplasăm câțiva pași numai spre stânga sau numai spre
dreapta, apoi să coborâm pe linia următoare. În orice locație $(X,Y)$ a
matricei putem veni dinspre E, NE, N, NV, sau V. Pentru acești cinci vecini
presupunem deja calculate eforturile minime necesare, respectiv $\EFF[X,Y+1],
\EFF[X-1,Y+1], \EFF[X-1,Y], \EFF[X-1,Y-1], \EFF[X,Y-1]$. Atunci, în funcție de
direcția din care venim, efortul depus până la punctul $(X,Y)$ va fi:

\begin{tabular}{lll}
  dinspre est:       & $\EFF[X,Y+1]+ |Alt[X,Y+1] - Alt[X,Y]|$     & (1) \\
  dinspre nord-est:  & $\EFF[X-1,Y+1]+ |Alt[X-1,Y+1] - Alt[X,Y]|$ & (2) \\
  dinspre nord:      & $\EFF[X-1,Y]+ |Alt[X-1,Y] - Alt[X,Y]|$     & (3) \\
  dinspre nord-vest: & $\EFF[X-1,Y-1]+ |Alt[X-1,Y-1] - Alt[X,Y]|$ & (4) \\
  dinspre vest:      & $\EFF[X,Y-1]+ |Alt[X,Y-1] - Alt[X,Y]|$     & (5) \\
\end{tabular}

În principiu, nu avem decât să calculăm minimul dintre aceste expresii ca să
aflăm valoarea lui $\EFF[X,Y]$. În felul acesta, matricea $\EFF$ se va completa
pe linie, de sus în jos. Totuși, apare o problemă: pentru a-l afla pe
$\EFF[X,Y]$ avem nevoie de $\EFF[X,Y-1]$ (dacă ne deplasăm spre est), iar pentru
a-l afla pe $\EFF[X,Y-1]$ avem nevoie de $\EFF[X,Y]$ (dacă ne deplasăm spre
vest)! Bineînțeles, avem sentimentul că ne învârtim după propria
coadă. Totuși, dezlegarea nu e complicată, ținând cont de observația făcută
mai sus, că pe aceeași linie deplasarea se face într-o singură direcție. Este
suficient să parcurgem fiecare linie de două ori: prima oară o parcurgem de la
stânga la dreapta, în ipoteza că deplasarea pe linia respectivă se face spre
est, apoi încă o dată de la dreapta la stânga, în ipoteza că deplasarea pe
linia respectivă se face spre vest. La prima parcurgere, vom minimiza efortul
pentru fiecare căsuță cu expresia (5), iar la a doua - cu expresia
(1). Minimizarea cu expresiile (2), (3) și (4) se poate face la oricare din
parcurgeri, deoarece elementele liniei superioare nu se mai modifică.

După cum am spus, efortul minim se obține căutând minimul de pe ultima linie a
matricei $\EFF$ (aceasta deoarece nu contează în ce punct de pe ultima linie
este sosirea). Punctul în care se atinge acest minim este tocmai punctul de
sosire. Reconstituirea efectivă a drumului se face în sens invers: se pleacă
din punctul de sosire $(i_k,j_k)$ și se caută un punct vecin lui pe una din
cele cinci direcții permise, $(i_{k-1},j_{k-1})$ astfel încât

\begin{equation}
  \EFF[i_k,j_k] = \EFF[i_{k-1},j_{k-1}] + |Alt[i_k,j_k] - Alt[i_{k-1},j_{k-1}]|
\end{equation}

Cu alte cuvinte, se testează pentru care din expresiile (1) - (5) se verifică
egalitatea. Se reia, recursiv, același procedeu pentru locația
$(i_{k-1},j_{k-1})$.

Iată cum se completează matricea $\EFF$ pentru exemplul dat și cum se
reconstituie drumul:

\begin{equation}
  Alt =
  \begin{pmatrix}
    10 &  7 &  2 &  5 \\
    13 & 20 & 25 &  3 \\
    2 &  4 &  2 & 20 \\
    5 & 10 &  9 & 11
  \end{pmatrix}
\end{equation}

\begin{equation}
  \EFF =
  \begin{pmatrix}
    0 &  0 &  0 &  0 \\
    3 & 10 & 15 &  1 \\
    6 &  4 &  2 & 18 \\
    5 & 10 &  9 & 11
  \end{pmatrix}
\end{equation}

Minimul de pe linia a 4-a a matricei $\EFF$ este $\EFF[4,1]=5$, deci sosirea se
face în colțul de SV. Din ce parte am ajuns aici? Se testează toți vecinii și
se constată că $\EFF[4,1] = \EFF[3,2] + |Alt[4,1] - Alt[3,2]|$, deci s-a venit
de la locația (3,2). Apoi se constată că:

\begin{equation}
  \begin{split}
    \EFF[3,2] & = \EFF[3,3] + |Alt[3,2] - Alt[3,3]| \\
    \EFF[3,3] & = \EFF[2,4] + |Alt[3,3] - Alt[2,4]| \\
    \EFF[2,4] & = \EFF[1,3] + |Alt[2,4] - Alt[1,3]| \\
  \end{split}
\end{equation}

Din aceste relații rezultă că traseul urmat este $(1,3) \to (2,4) \to (3,3)
\to (3,2) \to (4,1)$.

Pentru o mai mare ușurință a implementării, se vor adăuga două coloane fictive
la matricea $Alt$: coloanele 0 și $N + 1$. Facem acest lucru pentru a ne putea
referi la celula $(X,Y-1)$ atunci când $(X, Y)$ este o celulă din prima
coloană (respectiv la celula $(X,Y+1)$ atunci când $(X,Y)$ este o celulă de pe
ultima coloană) fără a primi un mesaj de eroare. Trebuie însă să fim atenți ca
nu cumva noile coloane adăugate să perturbe datele de ieșire și să rezulte că
traseul optim trece prin coloana 0 sau $N+1$. Pentru a scăpa de grija
celulelor de pe aceste două coloane și a ne asigura că ele nu vor putea fi
selectate pentru traseul optim, le vom atribui altitudini foarte
mari. Deoarece diferența maximă de nivel la fiecare pas este 255, rezultă că
efortul total maxim ce se poate obține este $255 \times 99 = 25.245$. Așadar,
o altitudine a coloanelor laterale de 30.000 este suficientă.

\begin{lstlisting}[language=C]
#include <stdio.h>
#include <math.h>
#define NMax 101
#define Infinity 30000
typedef int Matrix[NMax][NMax+1];

Matrix Alt, Eff;
int M, N;
FILE *OutF;

void ReadData(void)
{ FILE *F=fopen("input.txt", "rt");
  int i,j;

  fscanf(F, "%d %d\n", &M, &N);
  for (i=1; i<=M; i++)
    for (j=1; j<=N; j++)
      fscanf(F, "%d", &Alt[i][j]);
  fclose(F);
}

void Optimize(int X1, int Y1, int X2, int Y2)
/* Testeaza daca in (X1,Y1) se poate ajunge
   cu efort mai mic dinspre (X2,Y2) */
{
  if (Eff[X2][Y2]+abs(Alt[X1][Y1]-Alt[X2][Y2])<Eff[X1][Y1])
    Eff[X1][Y1]=Eff[X2][Y2]+abs(Alt[X1][Y1]-Alt[X2][Y2]);
}

void Traverse(void)
{ int i,j;

  for (j=1; j<=N;) Eff[1][j++]=0;
  for (i=1; i<=M; i++)
    Eff[i][0]=Eff[i][N+1]=Infinity; /* Bordeaza matricea */
  for (i=2; i<=M; i++)
    {
      for (j=1; j<=N; j++)
        {
          Eff[i][j]=Infinity;
          Optimize(i, j, i-1, j);       /* De la N  */
          Optimize(i, j, i-1, j-1);     /* De la NV */
          Optimize(i, j, i-1, j+1);     /* De la NE */
          Optimize(i, j, i, j-1);       /* De la V  */
        }
      for (j=N; j; j--)
        Optimize(i, j, i, j+1);         /* De la E  */
    }
}

void GoBack(int X, int Y)
/* Reconstituie drumul */
{
  if (X>1)
    if (Eff[X][Y]==Eff[X][Y-1]
                   +abs(Alt[X][Y-1]-Alt[X][Y]))
    GoBack(X, Y-1);
    else if (Eff[X][Y]==Eff[X-1][Y-1]
                        +abs(Alt[X-1][Y-1]-Alt[X][Y]))
    GoBack(X-1, Y-1);
    else if (Eff[X][Y]==Eff[X-1][Y]
                        +abs(Alt[X-1][Y]-Alt[X][Y]))
    GoBack(X-1, Y);
    else if (Eff[X][Y]==Eff[X-1][Y+1]
                        +abs(Alt[X-1][Y+1]-Alt[X][Y]))
    GoBack(X-1, Y+1);
    else if (Eff[X][Y]==Eff[X][Y+1]
                        +abs(Alt[X][Y+1]-Alt[X][Y]))
    GoBack(X, Y+1);
  if (X>1) fprintf(OutF, "->");
  fprintf(OutF,"(%d,%d)", X, Y);
}

void WriteSolution(void)
{ int j,k;

  OutF=fopen("output.txt", "wt");
  /* Cauta punctul de sosire */
  fputs("(a)\n",OutF);
  for (j=2, k=1; j<=N; j++)
    if (Eff[M][j]<Eff[M][k]) k=j;
  fprintf(OutF, "%d\n", Eff[M][k]);
  fputs("TRASEU: ",OutF);
  GoBack(M, k);
  fprintf(OutF,"\n");
  fclose(OutF);
}

void main(void)
{
  ReadData();
  Traverse();
  WriteSolution();
}
\end{lstlisting}

  
\section{Problema 6}

Propunem în continuare o problemă care s-a dat la Olimpiada Națională de
Informatică, Suceava 1996, la clasa a XII-a. Menționăm că un singur concurent
a reușit să o ducă la bun sfârșit în timpul concursului. Problema se numește
„Cartierul Enicbo”.

{\bf ENUNȚ}: În orașul Acopan s-a construit un nou cartier. Noul cartier are
patru bulevarde paralele și un număr de $N$ străzi perpendiculare pe
ele. Există deci în total $4N$ intersecții. Furgoneta oficiului poștal trebuie
să distribuie poșta în fiecare zi; în acest scop, furgoneta pleacă de la
oficiul poștal aflat la intersecția bulevardului 1 cu strada 1 și, urmând
rețeaua stradală, trece exact o dată prin fiecare intersecție astfel încât să
încheie traseul în punctul de plecare.

Conducerea oficiului poștal roagă participanții la olimpiadă să o ajute să
afle în câte moduri distincte se poate stabili traseul furgonetei.

{\bf Intrarea}: Programul va citi de la tastatură valoarea lui $N$ ($2 \leq N
\leq 200$).

{\bf Ieșirea}: Pe ecran se va afișa soluția (numărul de trasee distincte
pentru valoarea respectivă a lui $N$).

{\bf Exemplu}: Pentru $N=3$ există 4 soluții (se citește de la tastatură
numărul 3 și se afișează pe ecran numărul 4). Iată soluțiile efective:

\newcommand\enicboContour[5]{
  \draw[grid] (0,0) grid (#1,#2);

  \enicboContourNoGrid{#3}{#4}{#5}
}

\newcommand\enicboContourNoGrid[3]{
  \def\lx{#1} \def\ly{#2}
  \foreach \x[remember=\x as \lx]/\y[remember=\y as \ly] in {#3} {
    \draw[->-] (\lx,\ly) -- (\x,\y);
  }
}

\newcommand\enicboSquares[1]{
  \foreach \x/\y in {#1} {
    \node[times] at ($(\x+0.5,\y+0.5)$) {$\times$};
  }
}

\newcommand\enicboDots[2]{
  \foreach \x in {#1} {
    \foreach \y in {#2} {
      \node at (\x,\y) {$\cdots$};
    }
  }
}

\tikzset{
  grid/.style = {gray!50,very thin},
  ->-/.style={
    thick,
    decoration={
      markings,
      mark=at position .65 with {\arrow{>},
      },
    },
    postaction={decorate},
  },
  times/.style = {
    font=\Large,
  }
}
\centeredTikzFigure[
  mat/.style = {
    matrix of nodes,
    ampersand replacement=\&,
    column sep=2em,
    row sep=2em,
  },
]{
  \matrix[mat] {
    \enicboContour{2}{3}{0}{0}{
      0/1, 0/2, 0/3, 1/3, 2/3, 2/2, 1/2, 1/1, 2/1, 2/0, 1/0, 0/0
    }
    \&
    \enicboContour{2}{3}{0}{0}{
      1/0, 2/0, 2/1, 1/1, 1/2, 2/2, 2/3, 1/3, 0/3, 0/2, 0/1, 0/0
    }
    \\
    \enicboContour{2}{3}{0}{0}{
      0/1, 1/1, 1/2, 0/2, 0/3, 1/3, 2/3, 2/2, 2/1, 2/0, 1/0, 0/0
    }
    \&
    \enicboContour{2}{3}{0}{0}{
      1/0, 2/0, 2/1, 2/2, 2/3, 1/3, 0/3, 0/2, 1/2, 1/1, 0/1, 0/0
    }
    \\
  };
}

{\bf Timp de execuție}: 30 secunde pentru un text

{\bf Timp de implementare}: 1h 30 min.

{\bf Complexitate cerută}: $O(N^2)$

{\bf REZOLVARE}: Primul lucru care ne vine în gând este „se cere numărul de
cicluri hamiltoniene într-un graf, deci problema e exponențială”. Rezolvarea
backtracking nu e deloc greu de implementat, dar nu are nici o șansă să meargă
pentru valori mari ale lui $N$. Afirmația de mai sus este corectă, dar
incompletă; din această cauză concluzia este falsă. Se scapă din vedere faptul
că graful nu este oarecare, ci are un aspect foarte particular.

Și în această problemă vom încerca să utilizăm soluțiile locale (pentru valori
mici ale lui $N$) pentru aflarea soluției globale. Respectiv, vom rezolva
problema pentru $N=2$, apoi o vom extinde pentru $N=3, 4$ și așa mai
departe. Pentru început, însă, încercăm să simplificăm enunțul, reducând
problema la una echivalentă, dar mai simplă.

Să considerăm o posibilă soluție pentru $N=5$:

\centeredTikzFigure[]{
  \enicboContour{4}{3}{0}{0}{
    0/1, 0/2, 0/3, 1/3, 1/2, 1/1, 2/1, 2/2, 2/3, 3/3, 4/3, 4/2, 3/2, 3/1, 4/1,
    4/0, 3/0, 2/0, 1/0, 0/0
  }
  \enicboSquares{0/0, 0/1, 0/2, 1/0, 2/0, 2/1, 2/2, 3/0, 3/2}
}

În loc să lucrăm cu segmente în această rețea, vom lucra cu ochiuri. Furgoneta
parcurge un ciclu, deci închide în circuitul ei un număr de ochiuri. Am marcat
aceste ochiuri cu un „$\times$” în figura de mai sus. Așadar, oricărui drum al
furgonetei i se poate atașa o matrice cu 3 linii și $N-1$ coloane, în care
unele celule sunt bifate cu „$\times$”, iar altele nu. Să vedem în primul rând
care este corespondența între numărul de circuite hamiltoniene și numărul de
matrice de acest tip.

Se observă că pentru orice circuit există un altul căruia îi este atașată
aceeași matrice. Circuitul pereche este tocmai circuitul parcurs în sens
invers, care închide în interior aceleași ochiuri de rețea:

\centeredTikzFigure[]{
  \enicboContour{4}{3}{0}{0}{
    1/0, 2/0, 3/0, 4/0, 4/1, 3/1, 3/2, 4/2, 4/3, 3/3, 2/3, 2/2, 2/1, 1/1, 1/2,
    1/3, 0/3, 0/2, 0/1, 0/0
  }
  \enicboSquares{0/0, 0/1, 0/2, 1/0, 2/0, 2/1, 2/2, 3/0, 3/2}
}

Acest lucru se întâmplă deoarece transformarea graf-matrice ignoră sensul de
parcurgere a circuitului hamiltonian. De aici rezultă că pentru a calcula
numărul de circuite hamiltoniene trebuie să calculăm numărul de matrice și
să-l înmulțim cu 2.

În continuare, să analizăm câteva proprietăți ale matricelor în discuție.

\begin{property}
  Elementele bifate cu „$\times$” în matrice formează o singură figură conexă.
\end{property}

\begin{proof}
  Dacă figura nu ar fi conexă, adică dacă ar exista mai multe figuri, ele nu
  ar putea fi înconjurate de furgonetă într-un singur drum. De remarcat că
  {\bf toate} pătratele înconjurate de furgonetă trebuie bifate cu $\times$,
  deci furgoneta nu poate înconjura pătrate nebifate.
\end{proof}

De aceea, traseul de mai jos (care prezintă o porțiune oarecare de circuit)
este imposibil.

\centeredTikzFigure[]{
  \enicboContour{3}{3}{}{}{}
  \enicboSquares{0/1, 0/2, 2/1}
  \enicboDots{-1, 4}{0.5, 1.5, 2.5}
}

Conexitatea se referă numai la vecinătatea pe latură, nu și pe colț. Spre
exemplu, figura de mai jos este incorectă, deoarece, pentru a o înconjura,
furgoneta trebuie să treacă de două ori prin punctul încercuit:

\centeredTikzFigure[]{
  \enicboContour{3}{3}{3}{1}{2/1, 1/1, 1/2, 2/2, 3/2}
  \enicboContourNoGrid{0}{3}{1/3, 1/2, 0/2}
  \enicboSquares{0/2, 1/1, 2/1}
  \enicboDots{-1, 4}{0.5, 1.5, 2.5}

  \draw (1,2) circle (0.2);
}

\begin{property}
  Elementele bifate cu „$\times$” formează o structură aciclică.
\end{property}

\begin{proof}
  Dacă structura ar fi ciclică, ar rezulta că elementele bifate cu „$\times$”
  închid între ele elemente nebifate, pe care furgoneta însă nu poate să le
  ocolească.
\end{proof}

Iată un exemplu de ciclicitate:

\centeredTikzFigure[]{
  \enicboContour{3}{3}{}{}{}
  \enicboSquares{0/0, 0/1, 0/2, 1/0, 1/2, 2/0, 2/1, 2/2}
  \enicboDots{-1, 4}{0.5, 1.5, 2.5}
  \node[times] at (1.5,1.5) {?};
}

Practic, situația de mai sus obligă furgoneta să facă două drumuri: unul pe
exterior și unul în jurul ochiului marcat cu „?”.

\begin{property}
  Nici un nod interior al rețelei nu poate avea toate cele patru ochiuri
  vecine marcate cu „$\times$”.
\end{property}

\begin{proof}
  Dacă ar exista un asemenea nod, el nu ar putea fi parcurs de furgonetă, deci
  ciclul nu ar mai fi hamiltonian. Este cazul nodului încercuit în figura
  următoare:
\end{proof}

\centeredTikzFigure[]{
  \enicboContour{3}{3}{0}{3}{1/3, 1/2, 2/2, 3/2, 3/1, 2/1, 2/0, 1/0, 0/0}
  \enicboSquares{0/0, 0/1, 0/2, 1/0, 1/1, 2/1}
  \enicboDots{-1, 4}{0.5, 1.5, 2.5}

  \draw (1,1) circle (0.2);
}

\begin{property}
  Structura elementelor bifate cu „$\times$” în cadrul matricei este
  arborescentă.
\end{property}

\begin{proof}
  Rezultă imediat din punctele anterioare: figura este conexă și aciclică.
\end{proof}

\begin{property}
  Numărul de celule bifate este $P = 2N - 1$.
\end{property}

\begin{proof}
  Folosim inducția matematică. Să presupunem că structura noastră ar avea un
  singur pătrat bifat. Atunci structura ar avea patru laturi „la
  vedere”. Traseul furgonetei care ocolește structura ar avea patru laturi. O
  structură de două pătrate (desigur lipite) va avea șase laturi la vedere:

  \centeredTikzFigure[]{
    \enicboContour{2}{1}{0}{0}{0/1, 1/1, 2/1, 2/0, 1/0, 0/0}
    \enicboSquares{0/0, 1/0}
    \node at (0.5, 1.5) {1};
    \node at (1.5, 1.5) {2};
    \node at (2.5, 0.5) {3};
    \node at (1.5, -0.5) {4};
    \node at (0.5, -0.5) {5};
    \node at (-0.5, 0.5) {6};
  }

  Să ne imaginăm acum că orice structură cu $k$ pătrate are $S_k$ laturi la
  vedere. Trebuie să demonstrăm că toate structurile cu $k+1$ pătrate au
  același număr de laturi la vedere și să aflăm efectiv acest număr,
  $S_{k+1}$. Cel de-al $k+1$-lea pătrat trebuie alipit la structura deja
  existentă în așa fel încât să nu se închidă nici un ciclu. El se va lipi
  deci de o latură la vedere a unui pătrat din structură. În acest fel, va
  dispărea o latură la vedere, dar vor apărea trei în loc. Numărul de laturi
  la vedere va crește prin urmare cu 2. Această cifră nu depinde de locul în
  care este alipit al $k+1$-lea pătrat, nici de forma structurii deja
  existente, deci am demonstrat că toate structurile arborescente cu $k$
  pătrate au același număr de laturi la vedere. Pentru a afla efectiv acest
  număr, pornim de la relațiile recurente stabilite prin inducție și eliminăm
  recurența:

  \begin{equation}
    \begin{split}
      S_{k + 1} & = S_k + 2 \\
      S_1 & = 4
    \end{split}
    \Biggr\}
    \implies S_k = 2k + 2
  \end{equation}

  Deoarece numărul total de noduri al rețelei este de $4N$, rezultă că
  structura noastră trebuie să aibă $4N$ laturi la vedere. Notând cu $P$
  numărul de pătrate bifate din matrice și rezolvând ecuația de mai jos,
  rezultă valoarea lui $P$:

  \begin{equation}
    2P + 2 = 4N \implies P = 2N - 1 = 2(N - 1) + 1
  \end{equation}

  Cum numărul de coloane al matricei este $N-1$, deducem că în medie pe
  fiecare coloană se vor afla două pătrate bifate, cu excepția uneia pe care
  se vor afla trei pătrate bifate.
\end{proof}

La nivel local, proprietatea este de asemenea respectată: numărul de pătrate
bifate din primele $k$ coloane ale matricei este $2k$, existând posibilitatea
să mai fie un pătrat suplimentar. De exemplu, în figura dată mai sus pentru
$N=5$, în primele două coloane se află patru elemente „$\times$”, deci o medie
de două pătrate pe fiecare coloană. În primele trei coloane există șapte
elemente „$\times$”, adică o medie de două pătrate pe coloană și un surplus de
un pătrat. Lăsăm ca temă cititorului să demonstreze că în primele $k$ coloane
există întotdeauna fie $2k$, fie $2k+1$ pătrate. Orice număr mai mare duce la
ciclicitatea figurii, orice număr mai mic duce la neconexitatea ei.

Pe fiecare coloană există opt combinații posibile de elemente bifate și
nebifate, pe care le vom codifica cu numere de la 0 la 7, conform unei
numărători binare:

\centeredTikzFigure[
  mat/.style = {
    matrix of nodes,
    ampersand replacement=\&,
    nodes in empty cells,
    column sep=1em,
    row 1/.style=times,
    row 2/.style=times,
    row 3/.style=times,
    row 4/.style = {
      font=\normalsize,
      nodes = {draw=none, yshift=-1.5em},
    }
  },
  times/.style = {
    nodes={draw},
    font=\Large,
    anchor=center,
    minimum height=2em,
    minimum width=2em,
  }
]{
  \matrix[mat] {
    \& \& \& \& $\times$ \& $\times$ \& $\times$ \& $\times$ \\
    \& \& $\times$ \& $\times$ \& \& \& $\times$ \& $\times$ \\
    \& $\times$ \& \& $\times$ \& \& $\times$ \& \& $\times$ \\
    0 \& 1 \& 2 \& 3 \& 4 \& 5 \& 6 \& 7 \\
  };
}

Să vedem acum care dintre aceste combinații rămân valabile. O coloană de tipul
0 nu poate exista, deoarece ea ar „rupe” matricea în două bucăți separate,
deci proprietatea de conexitate nu ar fi respectată.

Dacă pe coloana $k$ se află o combinație de tipul 3, ce s-ar putea afla pe
coloana $k+1$?

\centeredTikzFigure[]{
  \enicboContour{2}{3}{}{}{}
  \enicboSquares{0/0, 0/1, 1/1, 1/2}
  \enicboDots{-1, 3}{0.5, 1.5, 2.5}

  \node at (0.5, -0.5) {$k$};
  \node at (1.5, -0.5) {$k + 1$};

  \node at (2, 4) (a) {$A$};
  \node at (2.8, 0) (b) {$B$};

  \draw (a.west) -- (1,3);
  \draw (b.west) -- (1,1);
}

Pentru ca punctul $A$ să se afle pe traseul furgonetei, este obligatoriu să
bifăm pătratul de sub el. Pentru a menține conexitatea figurii apărute,
trebuie bifat și pătratul din centrul coloanei $k+1$. Cea de-a treia celulă a
coloanei $k+1$ nu poate fi bifată, deoarece punctul $B$ ar fi înconjurat din
patru părți de celule bifate, lucru care s-a stabilit că este imposibil. Se
vede că singura combinație posibilă pentru coloana $k+1$ este 6. Ce combinație
putem pune pe coloana $k+2$? Printr-un raționament analog, deducem că numai
combinația 3:

\centeredTikzFigure[
  label/.style={font=\small},
]{
  \enicboContour{4}{3}{}{}{}
  \enicboSquares{0/0, 0/1, 1/1, 1/2, 2/0, 2/1, 3/1, 3/2}
  \enicboDots{-1, 5}{0.5, 1.5, 2.5}

  \node[label] at (0.5, -0.5) {$k$};
  \node[label] at (1.5, -0.5) {$k + 1$};
  \node[label] at (2.5, -0.5) {$k + 2$};
  \node[label] at (3.5, -0.5) {$k + 3$};
}

Iată că, pentru a putea respecta condițiile de corectitudine a matricei, am fi
nevoiți să continuăm la nesfârșit cu coloane cu combinațiile 3-6-3-6 etc. Deci
niciuna din aceste combinații nu poate apărea în matrice.

În continuare, vom defini mai multe șiruri de forma $S_k(i,t)$, unde:

\begin{itemize}

\item $k$ este numărul unei coloane;

\item $i$ este un număr de combinație (respectiv 1, 2, 4, 5 sau 7);

\item $t$ este un număr care poate avea valoarea 0 sau 1.

\end{itemize}

$S_k(i,t)$ semnifică „numărul de matrice (corecte) cu $k$ coloane astfel încât
pe coloana cu numărul $k$ să se afle combinația $i$, iar surplusul de pătrate
bifate peste media de două pătrate pe fiecare coloană să fie $t$”. De exemplu,
$S_7(5,1)$ reprezintă numărul de matrice corecte (care respectă regulile de
construcție) cu 7 coloane, astfel încât pe ultima coloană să se afle
combinația 5 și să existe un surplus de 1 pătrat (adică numărul total de
pătrate să fie $2 \times 7 + 1 = 15$).

Facem observația că pe a $N-1$-a coloană se pot afla doar combinațiile 5 sau 7
(pentru a acoperi colțurile de NE și SE ale grafului), iar surplusul de
pătrate trebuie să fie 1 (deoarece în $N-1$ coloane trebuie să se afle
$P=2(N-1)+1$ pătrate bifate). Deci scopul nostru este să calculăm suma
$S_{N-1}(5,1) + S_{N-1}(7,1)$ și să o înmulțim cu 2 ca să aflăm numărul de
cicluri hamiltoniene.

De asemenea, remarcăm că șirurile $S_k(1,1)$, $S_k(2,1)$ și $S_k(4,1)$ nu sunt
definite. Aceasta deoarece combinațiile 1, 2 și 4 au un singur pătrat bifat pe
coloană, adică mai puțin decât media de două pătrate. Este imposibil ca după
adăugarea unei asemenea coloane să mai existe un surplus. (deoarece ar rezulta
că în primele $k$-1 coloane exista un surplus de două pătrate). La polul opus,
șirul $S_k(7,0)$ nu este definit, deoarece combinația 7 are toată coloana
bifată, adică peste medie, deci nu se poate să nu apară un surplus de pătrate
bifate.

Mai trebuie stabilite formulele de recurență între șirurile $S_k(1,0)$,
$S_k(2,0)$, $S_k(4,0)$, $S_k(5,0)$, $S_k(5,1)$ și $S_k(7,1)$. Termenii
inițiali ai recurenței sunt:

\begin{itemize}

\item $S_1(1,0)=0$ deoarece matricea nu poate începe cu combinația 1

\item $S_1(2,0)=0$ deoarece matricea nu poate începe cu combinația 2

\item $S_1(4,0)=0$ deoarece matricea nu poate începe cu combinația 4

\item $S_1(5,0)=1$ deoarece există o singură matrice de o coloană cu
  combinația 5

\item $S_1(5,1)=0$ deoarece combinația 5 are două pătrate, deci nu există
  surplus

\item $S_1(7,1)=1$ deoarece există o singură matrice de o coloană cu
  combinația 7

\end{itemize}

Pentru a stabili relația de recurență pentru șirul $S_k(1,0)$, ne întrebăm:
cărei coloane îi poate urma coloana $k$ de tip 1 astfel încât să nu mai existe
surplus? Dacă observăm că pe coloana $k$ avem un singur element bifat (deci
sub medie), rezultă că pe coloana $k-1$ exista un surplus de un pătrat. Deci
coloana $k-1$ putea fi de tipul 5 sau 7, acestea fiind singurele tipuri de
coloană după care poate exista un surplus. Rezultă formula:

\begin{equation}
  S_k(1,0)=S_{k-1}(5,1)+S_{k-1}(7,1)
\end{equation}

Printr-o simetrie perfectă se calculează aceeași formulă și pentru șirul
$S_k(4,0)$:

\begin{equation}
  S_k(4,0)=S_{k-1}(5,1)+S_{k-1}(7,1)
\end{equation}

La șirul $S_k(2,0)$, mai trebuie făcută observația că o coloană de tip 2 nu
poate urma unei coloane de tip 5, deoarece se strică conexitatea
figurii. Rezultă:

\begin{equation}
  S_k(2,0)=S_{k-1}(7,1)
\end{equation}

Șirul $S_k(5,0)$ provine din adăugarea unei coloane de tipul 5 după o coloană
de tipul 1, 4 sau 5. Coloana $k-1$ nu poate fi de tipul 2 deoarece figura
rezultată nu este conexă, nici de tipul 7 deoarece ar rezulta că în primele
$k-2$ coloane media de celule bifate este mai mică decât 2.

\begin{equation}
  S_k(5,0)=S_{k-1}(1,0)+S_{k-1}(4,0)+S_{k-1}(5,0)
\end{equation}

Șirul $S_k(5,1)$ provine din adăugarea unei coloane de tipul 5 după o coloană
de tipul 5 sau 7, deoarece tipul de coloană 5 are două pătrate bifate, deci
conservă surplusul:

\begin{equation}
  S_k(5,1)=S_{k-1}(5,1)+S_{k-1}(7,1)
\end{equation}

În sfârșit, o coloană de tip 7 poate urma oricărui tip de coloană pentru care
surplusul este 0, adică:

\begin{equation}
  S_k(7,1)=S_{k-1}(1,0)+S_{k-1}(2,0)+S_{k-1}(4,0)+S_{k-1}(5,0)
\end{equation}

Acestea sunt formulele de recurență. Rezultatul care trebuie afișat pe ecran
este $2[S_{N-1}(5,1)+S_{N-1}(7,1)]$, deoarece după $N-1$ coloane surplusul
trebuie să fie 1, iar colțurile matricei trebuie să fie bifate. Se observă că
$S_k(1,0)=S_k(4,0)=S_k(5,1)$. Practic, problema se reduce la trei
șiruri. Notăm:

\begin{equation}
  \begin{split}
    A_k & = S_k(5,0) \\
    B_k & = S_k(1,0) = S_k(4,0) = S_k(5,1) \\
    C_k & = S_k(7,1) \implies S_k(2,0)=C_{k-1}
  \end{split}
\end{equation}

De aici rezultă grupul de relații:

\begin{equation}
  \begin{cases}
    A_k & = A_{k - 1} + 2B_{k - 1} \\
    B_k & = B_{k - 1} + C_{k - 1} \\
    C_k & = A_{k - 1} + 2B_{k - 1} + C_{k - 2} = A_k + C_{k - 2}
  \end{cases}
\end{equation}

și

\begin{equation}
  \begin{cases}
    A_1 & = 1 \\
    B_1 & = 0 \\
    C_1 & = 1
  \end{cases}
\end{equation}

Noi avem nevoie de valoarea 

\begin{equation}
  2(B_{N-1} + C_{N-1}) = 2B_N
\end{equation}

Programul de mai jos nu face decât să implementeze calculul acestor șiruri
recurente. Trebuie avut grijă însă cu reprezentarea internă a numerelor,
deoarece pentru $N=200$ valorile ajung la 81 de cifre. Este deci necesară
reprezentarea numerelor ca șiruri de cifre.

\begin{minted}{c}
#include <stdio.h>
#include <mem.h>

typedef int Huge[85];
Huge A,B,C,C2,HTemp;
int N,k;

void Atrib(Huge H, int V)
/* H <- V */
{
  memset(H,0,sizeof(Huge));
  H[0]=1;
  H[1]=V;
}

void Add(Huge A, Huge B)
/* A <- A+B */
{ int i,T=0;

  if (B[0]>A[0])
    { for (i=A[0]+1;i<=B[0];) A[i++]=0;
      A[0]=B[0];
    }
    else for (i=B[0]+1;i<=A[0];) B[i++]=0;
  for (i=1;i<=A[0];i++)
    { A[i]+=B[i]+T;
      T=A[i]/10;
      A[i]%=10;
    }
  if (T) A[++A[0]]=T;
}

void WriteHuge(Huge H)
{ int i;

  for (i=H[0];i;printf("%d",H[i--]));
  printf("\n");
}

void main(void)
{
  printf("N=");scanf("%d",&N);
  Atrib(A,1);
  Atrib(B,0);
  Atrib(C,1);
  Atrib(C2,0);
  for (k=2;k<=N;k++)
    { memmove(HTemp,C,sizeof(Huge));
      Add(A,B);Add(A,B);  /* A(k) = A(k-1) + 2*B(k-1) */
      Add(B,C);           /* B(k) = B(k-1) + C(k-1)   */
      memmove(C,A,sizeof(Huge));
      Add(C,C2);          /* C(k) = A(k) + C(k-2)     */
      memmove(C2,HTemp,sizeof(Huge)); /* noul C(K-2) */
    }
  Add(B,B);               /* Rezultatul este 2*B(n)   */
  WriteHuge(B);
}
\end{minted}

  \section{Problema 7}

Problema programării unui turneu de fotbal a fost dată spre rezolvare la a
III-a Balcaniadă de Informatică, Varna 1995. Vom prezenta mai întâi enunțul
nemodificat al problemei, după care vom adăuga câteva detalii care o vor face
mai „provocatoare”.

{\bf ENUNȚ}: Una din sarcinile Ministerului Sporturilor dintr-o țară balcanică
este de a organiza un campionat de fotbal cu $N$ echipe (numerotate de la 1 la
$N$). Campionatul constă din $N$ etape (dacă $N$ este impar) sau $N-1$ etape
(dacă $N$ este par); orice două echipe dispută între ele un joc și numai
unul. Scrieți un program care realizează o programare a campionatului.

{\bf Intrarea}: Numărul de echipe $N$ ($2 \leq N \leq 24$) va fi dat la
intrarea standard (tastatură).

{\bf Ieșirea} se va face în fișierul text {\tt OUTPUT.TXT} sub forma unui
tabel cu numere întregi avînd $N$ linii. Al $j$-lea element din linia $i$ este
numărul echipei care joacă cu echipa $i$ în etapa $j$ (evident, dacă $i$ joacă
cu $k$ în etapa $j$, atunci în tabel $k$ joacă cu $i$ în aceeași etapă). Dacă
echipa $i$ este liberă în etapa $j$, atunci al $j$-lea element al liniei $i$
este zero.

{\bf Exemple}:

\texttt{
  \begin{tabular}{|l|l|}
    \hline
        {\bf INPUT.TXT} & {\bf OUTPUT.TXT} \\ \hline
        3 &           2 3 0 \\
          &           1 0 3 \\
          &           0 1 2 \\ \hline
        4 &           2 3 4 \\
          &           1 4 3 \\
          &           4 1 2 \\
          &           3 2 1 \\
    \hline
  \end{tabular}
}

{\bf Timp de implementare}: circa 1h 45’

{\bf Timp de execuție}: 33 secunde

Iată și modificările pe care le propunem pentru a aduce cu adevărat problema
la nivel de concurs:

\begin{itemize}

\item Limita pentru numărul de echipe este $N \leq 200$;

\item {\bf Timpul de implementare} este de 45 minute;

\item {\bf Timpul de execuție} este de 2-3 secunde;

\item {\bf Complexitatea cerută} este $O(N^2)$.

\end{itemize}

{\bf REZOLVARE}: Vom lăsa pe seama cititorului conceperea, implementarea și
testarea unei rezolvări backtracking (adoptată de majoritatea în timpul
concursului). De altfel, la Balcaniadă concurenții au avut la dispoziție
calculatoare 486 / 50 Mhz, iar un backtracking îngrijit funcționa cam până la
$N = 18$ în timpul impus, ceea ce asigura cam 75\% din punctajul maxim. În
afară de aceasta, se mai puteau face și alte lucruri nu tocmai elegante, având
în vedere că datele de ieșire nu erau foarte mari. Pentru $N > 18$, mulți
concurenți lăsau programul să meargă până când termina (câteva minute bune sau
chiar mai mult), apoi scriau rezultatele într-un fișier temporar. Matricea
rezultată era inclusă ca o constantă în codul sursă. Programul care era dat
comisiei de corectare se prefăcea că „se gândește” timp de câteva secunde,
apoi scria pur și simplu matricea în fișierul de ieșire. Cu aceasta s-au luat
punctaje foarte apropiate de maxim. Totuși, pentru $N = 24$ un backtracking ar
fi stat foarte mult pentru a găsi soluția. Tot timpul acesta, calculatorul era
blocat, neputând fi folosit. De aceea, mulți concurenți nu au luat punctajul
maxim, preferând să renunțe la ultimele teste și să rezolve celelalte
probleme. Oricum, backtracking-ul este în acest caz o soluție pentru care
raportul punctaj obținut / timp consumat este foarte convenabil.

Există însă și o soluție care poate asigura un punctaj maxim fără bătăi de cap
și fără să folosească „date preprocesate”. Ea are complexitatea $O(N^2)$ și nu
face decât o singură parcurgere a matricei de ieșire. Ce-i drept, autorul a
pierdut cam trei ore pentru a o găsi și implementa în timp de concurs,
compromițând aproape rezolvarea celorlalte două probleme din ziua respectivă,
dar comisia de corectare a fost plăcut impresionată, programul fiind singurul
care mergea instantaneu. Rămâne ca voi să alegeți între eleganță și
eficiență...

Dacă ținem cont și de restricțiile suplimentare propuse, rezolvarea
backtracking nu mai este valabilă. În această situație, iată care este metoda
care stă la baza rezolvării în timp pătratic. În primul rând se reduce cazul
când $N$ este impar la un caz când $N$ este par, prin mărirea cu 1 a lui $N$
și introducerea unei echipe fictive. În fiecare etapă se consideră că echipa
care trebuia să joace cu echipa fictivă stă de fapt „pe bară”. Iată de exemplu
cum se rezolvă cazul $N = 3$:

\begin{enumerate}[label=(\alph*)]

\item Se programează un campionat cu 4 echipe, echipa 4 fiind echipa
  fictivă:

  \begin{table}[H]
    \rowcolors{1}{white}{gray!5}
    \centering
    \begin{tabular}{c|ccc}
      \hline
          {\bf Echipa} & {\bf Etapa 1} & {\bf Etapa 2} & {\bf Etapa 3}\\ \hline
          {\bf 1}      & 2             & 3             & 4 \\
          {\bf 2}      & 1             & 4             & 3 \\
          {\bf 3}      & 4             & 1             & 2 \\
          {\bf 4}      & 3             & 2             & 1 \\
          \hline
    \end{tabular}
  \end{table}

\item Se neglijează ultima linie din tabel, meciurile echipei fictive nefiind
  importante:

  \begin{table}[H]
    \rowcolors{1}{white}{gray!5}
    \centering
    \begin{tabular}{c|ccc}
      \hline
          {\bf Echipa} & {\bf Etapa 1} & {\bf Etapa 2} & {\bf Etapa 3}\\ \hline
          {\bf 1}      & 2             & 3             & 4 \\
          {\bf 2}      & 1             & 4             & 3 \\
          {\bf 3}      & 4             & 1             & 2 \\
          \hline
    \end{tabular}
  \end{table}

\item Peste tot unde apare cifra 4, ea este înlocuită cu 0:

  \begin{table}[H]
    \rowcolors{1}{white}{gray!5}
    \centering
    \begin{tabular}{c|ccc}
      \hline
          {\bf Echipa} & {\bf Etapa 1} & {\bf Etapa 2} & {\bf Etapa 3}\\ \hline
          {\bf 1}      & 2             & 3             & 0 \\
          {\bf 2}      & 1             & 0             & 3 \\
          {\bf 3}      & 0             & 1             & 2 \\
          \hline
    \end{tabular}
  \end{table}

\end{enumerate}

Mai rămâne să vedem cum se tratează cazul când $N$ este par. E destul de greu
de dat o demonstrație matematică metodei care urmează; de fapt, nici nu există
una, problema fiind rezolvată în timp de concurs prin inducție incompletă
(adică s-a constatat cu creionul pe hârtie că metoda merge pentru $N = 4, 6,
8$ și 10, apoi s-a scris programul care să facă același lucru și s-a constatat
că merge și pentru valori mai mari). Să tratăm și aici cazul $N = 8$, apoi să
generalizăm procedeul.

Vor fi șapte etape și, fără a reduce cu nimic generalitatea problemei, putem
presupune că echipa 1 joacă pe rând cu echipele 2, 3, 4, 5, 6, 7 și
8. Deocamdată tabelul arată astfel:

\begin{table}[H]
  \setlength{\tabcolsep}{5pt}
  \rowcolors{1}{white}{gray!5}
  \centering
  \begin{tabular}{c|ccccccc}
    \hline
        {\bf Echipa} & {\bf Etapa 1} & {\bf Etapa 2} & {\bf Etapa 3} &
        {\bf Etapa 4} & {\bf Etapa 5} & {\bf Etapa 6} & {\bf Etapa 7} \\ \hline
        {\bf 1} & 2 & 3 & 4 & 5 & 6 & 7 & 8 \\
        {\bf 2} & 1 &   &   &   &   &   &   \\
        {\bf 3} &   & 1 &   &   &   &   &   \\
        {\bf 4} &   &   & 1 &   &   &   &   \\
        {\bf 5} &   &   &   & 1 &   &   &   \\
        {\bf 6} &   &   &   &   & 1 &   &   \\
        {\bf 7} &   &   &   &   &   & 1 &   \\
        {\bf 8} &   &   &   &   &   &   & 1 \\
        \hline
  \end{tabular}
\end{table}

Echipa 2 are deja programat un meci cu echipa 1 în prima etapă și mai are de
programat meciurile cu echipele 3, 4, 5, 6, 7 și 8 în celelalte
etape. Așezarea se poate face oricum, cu condiția ca echipele 1 și 2 să nu își
aleagă același partener în aceeași etapă. De exemplu, putem programa meciul
2-3 în etapa a 3-a, meciul 2-4 în etapa a 4-a, ..., iar meciul 2-8 în etapa a
2-a. Astfel am completat linia a doua a tabloului:

\begin{table}[H]
  \setlength{\tabcolsep}{5pt}
  \rowcolors{1}{white}{gray!5}
  \centering
  \begin{tabular}{c|ccccccc}
    \hline
        {\bf Echipa} & {\bf Etapa 1} & {\bf Etapa 2} & {\bf Etapa 3} &
        {\bf Etapa 4} & {\bf Etapa 5} & {\bf Etapa 6} & {\bf Etapa 7} \\ \hline
        {\bf 1} & 2 & 3 & 4 & 5 & 6 & 7 & 8 \\
        {\bf 2} & 1 & 8 & 3 & 4 & 5 & 6 & 7 \\
        {\bf 3} &   & 1 & 2 &   &   &   &   \\
        {\bf 4} &   &   & 1 & 2 &   &   &   \\
        {\bf 5} &   &   &   & 1 & 2 &   &   \\
        {\bf 6} &   &   &   &   & 1 & 2 &   \\
        {\bf 7} &   &   &   &   &   & 1 & 2 \\
        {\bf 8} &   & 2 &   &   &   &   & 1 \\
        \hline
  \end{tabular}
\end{table}

Echipa 3 are programate meciurile cu 1 și 2 și mai are de programat meciurile
cu echipele 4, 5, 6, 7 și 8. Pentru a nu crea conflicte (adică două echipe să
nu-și aleagă același adversar), putem aplica același procedeu: pornim de la
etapa a 5-a și completăm toate celulele goale ale liniei a 3-a cu numerele de
la 4 la 8, mergând circular spre dreapta:

\begin{table}[H]
  \setlength{\tabcolsep}{5pt}
  \rowcolors{1}{white}{gray!5}
  \centering
  \begin{tabular}{c|ccccccc}
    \hline
        {\bf Echipa} & {\bf Etapa 1} & {\bf Etapa 2} & {\bf Etapa 3} &
        {\bf Etapa 4} & {\bf Etapa 5} & {\bf Etapa 6} & {\bf Etapa 7} \\ \hline
        {\bf 1} & 2 & 3 & 4 & 5 & 6 & 7 & 8 \\
        {\bf 2} & 1 & 8 & 3 & 4 & 5 & 6 & 7 \\
        {\bf 3} & 7 & 1 & 2 & 8 & 4 & 5 & 6 \\
        {\bf 4} &   &   & 1 & 2 & 3 &   &   \\
        {\bf 5} &   &   &   & 1 & 2 & 3 &   \\
        {\bf 6} &   &   &   &   & 1 & 2 & 3 \\
        {\bf 7} & 3 &   &   &   &   & 1 & 2 \\
        {\bf 8} &   & 2 &   & 3 &   &   & 1 \\
        \hline
  \end{tabular}
\end{table}

Se folosește aceeași metodă pentru a completa și celelalte linii ale
matricei. Pentru fiecare linie $i$ ($i \leq N-1$):

\begin{itemize}

\item Se caută pe linia $i$ apariția valorii $i$-1;

\item Se caută primul spațiu liber (mergând circular spre dreapta). Primul
  spațiu liber înseamnă prima etapă în care se poate programa meciul dintre
  echipele $i$ și $i+1$ (trebuie ca ambele să fie libere în acea etapă). Se
  constată experimental că trebuie început de la a doua poziție liberă de după
  apariția valorii $i-1$;

\item Se merge circular spre dreapta și, în căsuțele libere întâlnite se trec
  valorile $i+1, i+2, \dots, N$. Concomitent, pe aceleași coloane ale liniilor
  $i+1, i+2, \dots, N$ se trece valoarea $i$.

\end{itemize}

\begin{table}[H]
  \setlength{\tabcolsep}{5pt}
  \rowcolors{1}{white}{gray!5}
  \centering
  \begin{tabular}{c|ccccccc}
    \hline
        {\bf Echipa} & {\bf Etapa 1} & {\bf Etapa 2} & {\bf Etapa 3} &
        {\bf Etapa 4} & {\bf Etapa 5} & {\bf Etapa 6} & {\bf Etapa 7} \\ \hline
        {\bf 1} & 2 & 3 & 4 & 5 & 6 & 7 & 8 \\
        {\bf 2} & 1 & 8 & 3 & 4 & 5 & 6 & 7 \\
        {\bf 3} & 7 & 1 & 2 & 8 & 4 & 5 & 6 \\
        {\bf 4} & 6 & 7 & 1 & 2 & 3 & 8 & 5 \\
        {\bf 5} & 8 & 6 & 7 & 1 & 2 & 3 & 4 \\
        {\bf 6} & 4 & 5 & 8 & 7 & 1 & 2 & 3 \\
        {\bf 7} & 3 & 4 & 5 & 6 & 8 & 1 & 2 \\
        {\bf 8} & 5 & 2 & 6 & 3 & 7 & 4 & 1 \\
        \hline
  \end{tabular}
\end{table}

Fiecare linie a matricei poate fi parcursă in acest fel de maxim 2 ori (o dată
pentru găsirea coloanei de start și o dată pentru completarea liniei), deci
numărul total de celule vizitate este cel mult $2 \times N^2$. Complexitatea
pătratică este cea optimă, deoarece programul trebuie în orice caz să
tipărească la ieșire circa $N^2$ numere.

\inputminted{c}{src/problem7.c}

  \section{Problema 8}

Problema {\bf codului lui Prüfer} pentru arborii generali, mai exact cea a
decodificării acestui cod, este genul de problemă care nu este excesiv de
grea, dar care necesită un artificiu fără de care nu se poate ajunge la
complexitatea optimă.

{\bf ENUNȚ}: Un arbore general neorientat conex cu $N+2$ vârfuri se poate
codifica eficient printr-un vector cu $N$ numere, astfel:

\begin{itemize}

\item Numerotăm nodurile de la 1 la $N+2$ într-o ordine oarecare;

\item Eliminăm cea mai mică frunză (nod de grad 1) și adăugăm în vector
  numărul nodului de care ea aparținea;

\item Reluăm procedeul pentru arborele rămas: tăiem cea mai mică frunză și
  adăugăm în vector numărul nodului de care ea aparținea;

\item Repetăm procedeul până mai rămân doar două noduri.

\end{itemize}

Vectorul rezultat se numește codificare Prüfer a arborelui dat. Iată un
exemplu de construcție a codului Prüfer atașat arborelui din figura de mai
jos:

\centeredTikzFigure[
  mat/.style = {
    matrix of nodes,
    ampersand replacement=\&,
    nodes=cell,
    column sep=2em,
    row sep=2em,
  },
  cell/.style={
    draw,
    circle,
    minimum height=2.5em,
  }
]{
  \matrix[mat] (m) {
    3 \& 2 \& 6 \\
    1 \& 5 \&   \\
    7 \& 4 \&   \\
  };

  \draw (m-1-1) edge (m-2-1);
  \draw (m-2-1) edge (m-3-1);
  \draw (m-2-1) edge (m-2-2);
  \draw (m-2-2) edge (m-1-2);
  \draw (m-2-2) edge (m-3-2);
  \draw (m-1-2) edge (m-1-3);
}

Codul lui Prüfer se obține astfel:

\begin{itemize}

\item Îl tăiem pe 3 și scriem 1;

\item Îl tăiem pe 4 și scriem 5;

\item Îl tăiem pe 6 și scriem 2;

\item Îl tăiem pe 2 și scriem 5;

\item Îl tăiem pe 5 și scriem 1;

\end{itemize}

Deci codificarea este (1, 5, 2, 5, 1) (și mai rămâne muchia 1-7, lucru care
este evident, deoarece 1 și 7 sunt singurele noduri care nu au fost tăiate).

Așadar fiecărui arbore oarecare cu $N+2$ noduri i se poate atașa un vector cu
$N$ componente numere naturale cuprinse între 1 și $N+2$. Se poate demonstra
că funcția definită între cele două mulțimi (mulțimea arborilor și mulțimea
vectorilor) este bijectivă. De aici rezultă două lucruri:

\begin{enumerate}

\item Există $(N+2)^N$ arbori generali cu $N+2$ noduri.

\item Codificarea Prüfer admite și decodificare (deoarece funcția de
  codificare este bijectivă). Tocmai aceasta este problema de rezolvat. Se dă
  un vector de $N$ numere întregi, fiecare cuprins între 1 și $N+2$. Se cere
  să se tipărească cele $N+1$ muchii ale arborelui decodificat.

\end{enumerate}

{\bf Intrarea} se face din fișierul text {\tt INPUT.TXT} care conține două
linii. Pe prima linie se dă $N$ ($1 \leq N \leq 10000$), pe a doua cele $N$
numere separate prin spații.

{\bf Ieșirea} se va face în fișierul text {\tt OUTPUT.TXT}. Acesta va conține
muchiile arborelui, câte una pe linie, o muchie fiind indicată prin vârfurile
adiacente separate printr-un spațiu.

{\bf Exemplu}:

\texttt{
  \begin{tabular}{|l|l|}
    \hline
        {\bf INPUT.TXT} & {\bf OUTPUT.TXT} \\ \hline
        5         & 5 2 \\
        1 5 2 5 1 & 1 3 \\
        & 4 5 \\
        & 7 1 \\
        & 6 2 \\
        & 1 5 \\
    \hline
  \end{tabular}
}

{\bf Timp de implementare}: 45 minute.

{\bf Timp de rulare}: 2-3 secunde.

{\bf Complexitate cerută}: $O(N)$.

{\bf REZOLVARE}: Să pornim de la exemplul particular prezentat mai sus, urmând
ca apoi să generalizăm algoritmul.

Primim la intrare $N=5$ (deducem că arborele are 7 noduri) și codificarea (1,
5, 2, 5, 1). Primul element din vector este 1. Știm deci că, la primul pas, a
fost eliminată o frunză al cărei părinte este nodul 1. Întrebarea este: ce
număr purta respectiva frunză? În nici un caz 1, deoarece numărul 1 îl avea
tatăl ei. Nici 2, nici 5 nu ar putea fi, deoarece aceste numere apar mai
târziu în codificare, deci „mai este nevoie” de ele și nu pot fi încă
eliminate. Rămân 3, 4, 6 și 7. Dintre acestea noi știm că a fost tăiată cea
mai mică {\bf frunză}. Este însă ușor de văzut că toate nodurile enumerate
sunt frunze în arborele inițial (deoarece nu mai apar în vector, adică nici un
alt nod nu mai este legat de ele), deci îl vom alege pe cel mai mic cu
putință, adică pe 3. Rezultă că prima muchie tăiată a fost (1,3).

Pe poziția a doua în codificare apare numărul 5. Ce număr ar putea avea frunza
tăiată? 1 și 2 mai apar ulterior în codificare deci nu pot fi eliminate încă,
5 este chiar tatăl frunzei necunoscute, iar 3 a fost deja eliminat. Dintre 4,
6 și 7, frunza cu numărul cel mai mic este 4, deci următoarea muchie tăiată
este (4,5).

În continuare apare un 2. Cel mai mic nod care nu mai apare în vector și nici
nu a fost deja eliminat este 6, deci muchia este (6,2). Se observă că mai
departe nu mai apare nici un 2 în codificare, de unde deducem că după tăierea
nodului 6, nodul 2 a devenit frunză și va putea fi la rândul său eliminat. Cu
un raționament analog, următoarele muchii eliminate sunt (2,5) și (5,1), iar
singurele noduri rămase sunt 1 și 7. Arborele a fost reconstituit corect.

Deci procedeul general este: pentru fiecare număr dintre cele $N$ din vector,
nodul care aparținea de el poartă cel mai mic număr care nu intervine ulterior
în codificare și nu a fost tăiat deja. Astfel se generează $N$ muchii. A
$N+1$-a muchie are drept capete ultimele noduri rămase netăiate.

Acesta este algoritmul. Trebuie să ne ocupăm acum de partea de implementare.

Pentru a testa la orice moment dacă un nod mai apare sau nu în vector, este
bine să se creeze la citirea datelor un vector care să rețină numărul de
apariții în codificare al fiecărui nod. Pentru exemplul dat, vectorul de
apariții va fi $Apar=(2,1,0,0,2,0,0)$, semnificând că 1 și 5 apar de câte două
ori, 2 apare o singură dată, iar 3, 4, 6 și 7 nu apar deloc. La fiecare pas,
când se „restaurează” o muchie, se decrementează poziția corespunzătoare
tatălui în vectorul de apariții. Un număr nu mai apare ulterior în codificare
dacă pe poziția sa din vectorul de apariții se află un 0.

Astfel, o primă versiune de program ar putea fi:

\vspace{\algskip}
\begin{algorithmic}[1]
  \REQUIRE $N$, $V[1..N]$
  \STATE construiește vectorul $Apar[1..N+2]$
  \FOR{$i = 1$ la $N$}
  \STATE caută primul $j$ pentru care $Apar[j]=0$ și $j$ nu a fost tăiat
  \PRINT muchia $(j,V[i])$
  \STATE $Apar[V[i]] \leftarrow Apar[V[i]]-1$
  \STATE marchează nodul $j$ ca fiind tăiat
  \ENDFOR
\end{algorithmic}

După cum se observă, căutarea celui mai mic nod $j$ se face în timp liniar,
ceea ce înseamnă că algoritmul complet va necesita un timp pătratic. Pentru
a-l reduce la o complexitate liniară, începem prin a observa că la fiecare pas
alegem frunza cu numărul cel mai mic. Aceasta înseamnă că, în general, numărul
frunzei alese spre a fi tăiată va crește la fiecare pas. Astfel, prima oară am
tăiat nodul 4, apoi nodul 7. Singurul caz când va fi tăiată o frunză mai mică
decât cea dinaintea ei este atunci când unui nod cu număr mic i se taie toate
frunzele și devine el însuși o frunză, putănd fi tăiat. În exemplul nostru,
nodul 2 nu putea fi eliminat de la început, deși avea un număr mic, deoarece
mai avea atașată frunza 6. După eliminarea muchiei (6,2), nodul 2 a devenit
frunză și a fost eliminat imediat.

Putem deci păstra într-o variabilă (care în program se numește $Next$)
următoarea frunză care care trebuie eliminată. Dacă prin decrementarea
numărului de apariții la pasul curent nu s-a creat nici un zero în vectorul
$Apar$, sau s-a creat un zero, dar pe o poziție $K>Next$, atunci totul este
bine și la pasul următor se va elimina nodul $Next$. Dacă s-a creat un zero pe
o poziție $K$ mai mică decât $Next$, rezultă că în arbore există acum două
frunze, $K$ și $Next$, $K$ fiind mai mică, deci prioritară. Atunci la pasul
următor se va elimina frunza de pe poziția $K$, urmând ca peste doi pași să se
revină la frunza $Next$. Dacă prin eliminarea acestei frunze $K$ s-a creat un
alt zero în vectorul de apariții, tot pe o poziție $K'$ mai mică decât Next,
se va trece mai întâi la acea poziție, urmând ca după aceea să se revină la
poziția $Next$ etc.

La prima vedere, pare necesară menținerea unei stive în care să depunem
numerele $Next, K, K'$... și să le scoatem din stivă pe măsură ce nodurile
respective sunt eliminate. Acest lucru ar presupune în continuare un algoritm
pătratic. Totuși nu este așa deoarece, odată ce am tăiat un nod și l-am marcat
ca atare, putem fi siguri că nu ne vom mai intâlni cu el până la sfârșitul
decodificării, deci nu mai este necesară stocarea lui. Există o pseudo-stivă
care are înălțimea 2: frunza asupra căreia se operează curent și nodul care
urmează, $Next$.

În acest fel, numărul total de incrementări al variabilei $Next$ nu depășește
$N$ (iar ea nu este niciodată decrementată), deci programul este rezolvat în
timp liniar. Facem observația că algoritmul $O(N)$ este optim; nu poate exista
unul mai bun deoarece există $O(N)$ muchii care trebuie tipărite.

\begin{minted}{c}
#include <stdio.h>
#define NMax 10002
typedef int Vector[NMax];

Vector V,Apar;
int N;

void ReadData(void)
{ int i;
  FILE *InF=fopen("input.txt","rt");

  fscanf(InF,"%d",&N);
  for (i=1;i<=N+2;Apar[i++]=0);
  for (i=1;i<=N;i++)
    { fscanf(InF,"%d",&V[i]);
      Apar[V[i]]++;
    }
  fclose(InF);
}

void Decode(void)
{ int Current=0,Next,i;
  FILE *OutF=fopen("output.txt","wt");

  do; while (Apar[++Current]);    /* Se cauta prima frunza */
  Next=Current;
  for (i=1;i<=N;i++)
    { fprintf(OutF,"%d %d\n",Current,V[i]);
      if (Current==Next) do; while (Apar[++Next]);
      /* Daca am ajuns la ultimul 0, mai caut unul */
      Apar[V[i]]--;
      Current=(V[i]<Next) && (Apar[V[i]]==0) ? V[i] : Next;
      /* Daca exista o frunza mai mica decat Next, */
      /* ea este prioritara, altfel revin la Next */
    }
  fprintf(OutF,"%d %d\n",Current,Next);
  fclose(OutF);
}

void main(void)
{
  ReadData();
  Decode();
}
\end{minted}

  \section{Problema 9}

Iată o problemă care necesită pentru reducerea complexității un artificiu
asemănător celui din problema codului Prüfer. Ea a fost propusă în 1995 la
concursul de selecție a echipelor României pentru IOI și CEOI. Deși cerința
este puțin modificată pentru a nu ne izbi de dificultăți secundare, ideea
generală de abordare este aceeași.

{\bf ENUNȚ}: Președintele companiei X dorește să organizeze o petrecere cu
angajații. Această companie are o structură ierarhică în formă de
arbore. Fiecare angajat are asociat un număr întreg reprezentând măsura
sociabilității sale. Pentru ca petrecerea să fie agreabilă pentru toți
participanții, președintele dorește să facă invitațiile astfel încât:

\begin{itemize}

\item el însuși să participe la petrecere;

\item pentru nici un participant la petrecere să nu fie invitat și șeful lui
  direct;

\item suma măsurilor sociabilităților invitaților să fie maximă.

\end{itemize}

Se cere să se spună care este suma maximă a sociabilităților care se poate obține.

{\bf Intrarea} se face din fișierul {\tt INPUT.TXT} care conține trei linii de
forma:

\begin{verbatim}
N
T(1) T(2) ... T(N)
S(1) S(2) ... S(N)
\end{verbatim}

unde $N$ este numărul de angajați ai companiei, inclusiv președintele ($N \leq
1000$), $T(k)$ este numărul de ordine al șefului direct al lui $k$ (dacă $T(k)
= 0$, atunci $k$ este președintele), iar $S(k)$ este măsura sociabilității lui
$k$. Valorile vectorului $S$ sunt de tipul întreg.

{\bf Ieșirea}: Pe ecran se va tipări suma maximă a sociabilităților ce se
poate obține.

{\bf Exemplu}: Pentru intrarea:

\begin{verbatim}
7
2 5 2 5 0 4 2
2 5 3 13 8 4 3
\end{verbatim}

pe ecran se va afișa numărul 20 (fiindcă la petrecere participă 1, 3, 5, 6 și
7).

{\bf Timp de implementare}: 1h - 1h 15min.

{\bf Timp de rulare}: 2-3 secunde.

{\bf Complexitate cerută}: $O(N)$. De asemenea, menționăm că s-a specificat $N
\leq 1000$ numai pentru ca programul sursă de mai jos să nu se complice și să
distragă atenția asupra unor lucruri neimportante. În mod normal, limita
trebuia să fie $N \leq 10.000$.

{\bf REZOLVARE}: Vom expune mai întâi principiul de rezolvare al problemei,
apoi vom aborda detaliile de implementare.

Algoritmul are la bază programarea dinamică și necesită o parcurgere de jos în
sus a arborelui, decizia pentru fiecare nod depinzând de valorile tuturor
fiilor lui. Ne propunem să aflăm două caracteristici pentru fiecare nod $k$:

\begin{itemize}

\item $P(k)$ - suma maximă a sociabilităților care se poate obține în
  subarborele de rădăcină $k$ în cazul în care $k$ participă la petrecere;

\item $Q(k)$ - suma maximă a sociabilităților care se poate obține în
  subarborele de rădăcină $k$ în cazul în care $k$ nu participă la petrecere;

\end{itemize}

Dacă reușim să determinăm aceste caracteristici, nu ne rămâne decât să-l
tipărim pe $P(R)$ ($R$ fiind rădăcina arborelui). Într-adevăr, problema cere
să se determine suma maximă a sociabilităților din întregul arbore, adică din
subarborele de rădăcină $R$. În plus, se mai cere ca $R$ să participe la
petrecere. Rămâne de aflat cum se stabilește relația între caracteristicile
unui nod și cele ale fiilor săi.

\begin{itemize}

\item Dacă angajatul $k$ participă la petrecere, atunci automat nici unul din
  subordonații săi direcți nu participă, și obținem relația:

  \begin{equation}
    P(k) = S(k) + \sum_j Q(j), \quad j \text{ fiu al lui } k
  \end{equation}

\item Dacă angajatul $k$ nu participă la petrecere, atunci subordonații săi
  direcți pot să participe sau nu la petrecere, după cum este mai avantajos,
  și obținem relația:

  \begin{equation}
    Q(k) = \sum_j \max(P(j), Q(j)), \quad j \text{ fiu al lui } k
  \end{equation}

\end{itemize}

Problema care se pune acum este cum să facem parcurgerea arborelui într-un mod
cât mai avantajos. Pseudocodul (recursiv) sub forma sa cea mai generală este:

\vspace{\algskip}
\begin{algorithm}
  \floatname{algorithm}{Procedura}
  \caption{Calcul($k$, $P$, $Q$)}
  \begin{algorithmic}[1]
    \STATE $P \leftarrow S(k)$
    \STATE $Q \leftarrow 0$
    \FORALL{$j$ fiu al lui $k$}
    \STATE Calcul($j$, $P_1$, $Q_1$)
    \STATE $P \leftarrow P + Q_1$
    \STATE $Q \leftarrow Q + \max(P_1, Q_1)$
    \ENDFOR
    \RETURN $P$, $Q$
  \end{algorithmic}
\end{algorithm}

Soluția „de bun simț” este de a construi arborele alocat dinamic. Ar apărea
însă o sumedenie de dificultăți. În primul rând, arborele este general, deci
nu se cunoaște numărul maxim de fii pe care îi poate avea un nod (pe cazul cel
mai defavorabil, rădăcina poate avea $N-1$ fii). Aceasta înseamnă că legătura
trebuie să fie de tip tata (fiecare nod pointează la tatăl său), ceea ce
complică procedura de parcurgere: nu se poate apela procedura recursiv,
dintr-un nod pentru toți fiii săi, deoarece nu se cunosc fiii, ci numai tatăl!
O altă modalitate de alocare a arborelui ar fi cu doi pointeri pentru fiecare
nod: unul către primul său fiu și unul către fratele său din dreapta (exemplul
din enunț are reprezentate grafic mai jos cele două metode de construcție).

\centeredTikzFigure[
  mat/.style = {
    matrix of nodes,
    ampersand replacement=\&,
    anchor=north,
    column sep=4em,
    row sep=1em,
  },
  level/.style={sibling distance=10em/#1},
  row 1/.style={
    nodes=tree,
  },
  tree/.style={circle, draw, minimum size=2.2em},
  edge from parent/.style={draw, <-},
]{
  \matrix[mat] {
    \node {5}
    child { node {2}
      child { node {1} }
      child { node {3} }
      child { node {7} }
    }
    child { node {4}
      child { node {6} }
    };
    %
    \&
    %
    \node (t2) {5}
    child { node (n2) {2} edge from parent[draw, ->]
      child { node (n1){1} edge from parent[draw, ->] }
    };
    \node[anchor=center] (n4) at ([xshift=12em]n2.center) {4};
    \node[anchor=center] (n3) at ([xshift=4em]n1.center) {3};
    \node[anchor=center] (n7) at ([xshift=8em]n1.center) {7};
    \node[anchor=center] (n6) at ([xshift=12em]n1.center) {6};
    %
    \draw[->] (n2) -- (n4);
    \draw[->] (n4) -- (n6);
    \draw[->] (n1) -- (n3);
    \draw[->] (n3) -- (n7);
    \\
    Legătură de tip tată \&
    \node[xshift=6em] {Legătură de tip fiu + frate}; \\
  };
}

Probabil că veți fi de acord cu mine că e riscant să te aventurezi la o
asemenea implementare în timp de concurs, deoarece lucrul cu pointeri
presupune o atenție deosebită. Greșelile sunt mai greu de observat și de multe
ori duc la blocarea calculatorului, care trebuie resetat mereu, pierzându-se
astfel o mulțime de timp. Trebuie deci căutată o metodă de parcurgere a
arborelui care să nu necesite o alocare dinamică a memoriei. Putem încerca
astfel: inițial $P(i)=S(i)$ și $Q(i)=0$ pentru orice nod. Urmează acum să
tratăm pe rând fiecare nod. Cum? Știm că pentru fiecare nod $k$, numerele
$P(k)$ și $Q(k)$ intervin în expresia lui $P(T(k))$ și $Q(T(k))$. Uitându-ne
la formulele de mai sus, observăm că tot ce avem de făcut este să incrementăm
$P(T(k))$ cu $Q(k)$ și $Q(T(k))$ cu $\max(P(k), Q(k))$.

Există o singură problemă: Pentru a putea folosi numerele $P(k)$ și $Q(k)$
trebuie să ne asigurăm că ele au fost deja calculate corect, în funcție de
caracteristicile tuturor fiilor lui $k$. În cazul frunzelor, problema este
rezolvată, deoarece ele nu au fii. Pentru nodurile interne, este necesar să
știm câți fii au ele și câți din aceștia au fost tratați. În momentul în care
toți fiii unui nod au fost tratați, poate fi tratat și nodul în sine. În
program, vectorul $F$ reține numărul de fiii netratați ai fiecărui nod. La
tratarea unui nod $k$ se face decrementarea lui $F(T(k))$. Un nod $k$ poate fi
ales spre tratare dacă $F(k)=0$. Se observă că formatul datelor de intrare
permite cu ușurință construcția vectorului $F$ (numărul de fii al lui $k$ este
egal cu numărul de apariții al lui $k$ în vectorul $T$).

Un algoritm mai ușor de implementat ar fi:

\vspace{\algskip}
\begin{algorithmic}[1]
  \STATE numără fiii fiecărui nod
  \FOR{$i = 1$ la $N-1$}
  \STATE caută un nod $k$ cu $F(k)=0$
  \STATE $P(T(k)) \leftarrow P(T(k))+Q(k)$
  \STATE $Q(T(K)) \leftarrow Q(T(K))+\max(P(k),Q(k))$
  \STATE $F(T(K)) \leftarrow F(T(K))-1$
  \ENDFOR
  \PRINT $P(R)$
\end{algorithmic}

Partea delicată a acestui algoritm este căutarea unui nod $k$ cu
$F(k)=0$. Dacă ea se face secvențial, pornind de fiecare dată de la primul nod
și cercetând fiecare element, programul care rezultă are complexitatea
$O(N^2)$. Această căutare trebuie deci optimizată, și iată cum: În principiu,
putem căuta nodurile cu $F(k)=0$ mergând numai în sensul crescător al
indicilor în vector. Vom folosi, ca și la problema codului lui Prüfer, două
variabile: $K$ și $Next$. $K$ este nodul tratat în prezent, iar $Next$ este
nodul care urmează a fi tratat la pasul următor, așadar $Next > K$. După
tratarea nodului $K$ și decrementarea lui $F(T(k))$, pot surveni trei
situații:

\begin{enumerate}

\item $F(T(k))>0$ și în vectorul $F$ nu a apărut nici un alt element zero, caz
  în care nimic nu se schimbă, algoritmul continuând cu tratarea nodului
  $Next$;

\item $F(T(k))=0$ și $T(k)>Next$, caz în care de asemenea se poate trata nodul
  $Next$ (deoarece selectarea nodurilor cu $F(k)=0$ se face în ordine
  crescătoare a indicilor);

\item $F(T(k))=0$ și $T(k)<Next$, caz în care se va trata mai întâi nodul
  $T(k)$, urmând a se reveni apoi la nodul $Next$.

\end{enumerate}

Pentru motivele explicate la codul lui Prüfer, acest algoritm nu necesită
menținerea unei stive, iar timpul total de căutare este $O(N)$. Facem
observația că nu se poate găsi un algoritm mai bun, deoarece trebuie parcurse
toate cele $N$ noduri ale arborelui.

Iată în încheiere cum arată arborele cu valorile $P$ și $Q$ atașate fiecărui
nod:

\centeredTikzFigure[
  level/.style={sibling distance=15em/#1},
  every node/.style = {circle, draw, minimum size=3em, font=\Large},
  every label/.style = {draw=none, font=\scriptsize, minimum size=1em},
]{
  \node [label=north west:{\begin{tabular}{l}S=8\\P=20\\Q=21\end{tabular}}] {5}
  child { node [label=north west:{\begin{tabular}{l}S=5\\P=5\\Q=8\end{tabular}}] {2}
    child { node [label=south:{\begin{tabular}{l}S=2\\P=2\\Q=0\end{tabular}}] {1} }
    child { node [label=south:{\begin{tabular}{l}S=3\\P=3\\Q=0\end{tabular}}] {3} }
    child { node [label=south:{\begin{tabular}{l}S=3\\P=3\\Q=0\end{tabular}}] {7} }
  }
  child { node [label=north east:{\begin{tabular}{l}S=13\\P=13\\Q=4\end{tabular}}] {4}
    child { node [label=south:{\begin{tabular}{l}S=4\\P=4\\Q=0\end{tabular}}] {6} }
  };
}

\begin{lstlisting}[language=C]
#include <stdio.h>
#define NMax 1000
typedef long Vector[NMax+1];

Vector T,S,P,Q,F; /* T = Vectorul de tati,
                     S = sociabilitatile,
                     P,Q = caracteristicile,
                     F = numarul de fii */
int N,Root;

void InitData(void)
{ int i;
  FILE *InF=fopen("input.txt","rt");

  fscanf(InF,"%d",&N);
  for (i=1;i<=N+1;F[i++]=0);
  for (i=1;i<=N;i++)
    { fscanf(InF,"%d",&T[i]);
      if (T[i]) F[T[i]]++; else Root=i;
    }
  for (i=1;i<=N;i++) fscanf(InF,"%d",&S[i]);
  fclose(InF);

  for (i=1;i<=N;P[i++]=S[i]);
  for (i=1;i<=N;Q[i++]=0);
}

void FindNext(int *K,int *Next)
{ F[T[*K]]--;
  F[*K]=-1;
  if ((F[T[*K]]>0) || (T[*K]>*Next))
    { *K=*Next;
      while (F[*Next]) (*Next)++;
    }
    else *K=T[*K];
}

void TraverseTree(void)
{ int K=0,Next,i;

  do; while (F[++K]);
  Next=K; do; while (F[++Next]);
  for (i=1;i<N;i++)
    { P[T[K]]+=Q[K];
      Q[T[K]]+=(P[K]>Q[K]) ? P[K] : Q[K];
      FindNext(&K,&Next);
    }
}

void main(void)
{
  InitData();
  TraverseTree();
  printf("%ld\n",P[Root]);
}
\end{lstlisting}

  \section{Problema 10}

Această problemă a fost dată la unul din barajele pentru selecționarea lotului
restrâns al României pentru Olimpiada Internațională din 1996.

{\bf ENUNȚ}: Fie un număr prim $P$. Pe mulțimea $\{0, 1, \dots, P-1\}$ se
definesc operațiile binare $+$, $-$, $\times$, $/$ modulo $P$, în felul
următor:

\begin{enumerate}

\item $a + b \bmod P$ este restul împărțirii lui $a+b$ la $P$. Analog pentru
  „$\times$”.

\item Expresia $a-b$ este definită ca fiind soluția ecuației $b+x \equiv a
  \pmod{P}$. Analog pentru „/”.

\end{enumerate}

Se știe că ecuația $b + x = a$ are întotdeauna soluție unică, iar $b \times x
= a$ are soluție unică pentru orice $b \neq 0$. Pentru $b = 0$, operația $a/b$
nu e definită. De exemplu, dacă $P=11$, atunci $6+7=2$, $6-7=10$, $6 \times
7=9$, $6/7=4$. Se știe că adunarea și înmulțirea modulo $P$ sunt comutative,
asociative, posedă elemente neutre (pe 0, respectiv pe 1), iar adunarea este
distributivă față de înmulțire. În plus, pentru orice $a$ există $b$ astfel
încât $a+b=0$; notând $b$ cu ($-a$) avem $c-a=c+(-a)=c+b$ pentru orice $c$. De
asemenea, pentru orice $a \neq 0$ există $b$ astfel încât $a \times b = 1$;
notând $b$ cu $(1/a)$ avem că $c/a=c \times (1/a)=c \times b$.

Dându-se un număr prim $P$, un întreg $D$ între 0 și $P-1$ și un șir de $N$
numere, cuprinse fiecare între 0 și $P-1$, se cere să se introducă între
elementele șirului operatorii $+$, $-$, $\times$, $/$ și parantezele
corespunzătoare, astfel încât să se obțină o expresie corectă a cărei valoare
să fie $D$ (lucrând în aritmetica modulo $P$). În caz că acest lucru nu este
posibil, se va afișa un mesaj corespunzător.

{\bf Intrarea}: Fișierul {\tt INPUT.TXT} conține două linii:

\begin{itemize}

\item pe prima linie se găsesc trei numere întregi: $P$, $N$ și $D$, separate
  prin spații ($2 \leq P \leq 23$, $P$ prim, $1 \leq N \leq 30$, $0 \leq D
  \leq P-1$);

\item pe următoarea linie se găsește șirul de numere ce formează expresia ($N$
  numere întregi cuprinse între 0 și $P-1$).

\end{itemize}

{\bf Ieșirea}: Fișierul text {\tt OUTPUT.TXT} va conține o singură linie pe
care se va găsi expresia generată sau mesajul „Nu există soluție.”. Expresia
va fi parantezată complet (fiecare operator va avea o pereche de paranteze
atașate).

{\bf Exemple}:

\texttt{
  \begin{tabular}{|l|l|}
    \hline
        {\bf INPUT.TXT} & {\bf OUTPUT.TXT} \\ \hline
        11 3 6 & (4+(7/9)) \\
        4 7 9  &           \\ \hline
        11 3 7 & Nu exista solutie. \\
        1 1 1  &           \\
        \hline
  \end{tabular}
}

{\bf Timp de execuție} pentru un test: 1 minut.

Modificările și completările propuse de autor sunt:

\begin{itemize}

\item {\bf Timp de execuție}: 10 secunde

\item {\bf Timp de implementare}: 1h 30 minute, maxim 1h 45 minute.

\item {\bf Complexitate cerută}: $O(N^3 \times P^2)$.

\end{itemize}

{\bf REZOLVARE}: Problema în sine nu este foarte complicată. Ea face apel la
programarea dinamică, dar este foarte asemănătoare cu una din problemele bine
cunoscute de elevi - înmulțirea optimă a unui șir de matrice. Ceea ce o face
mai „provocatoare” este timpul alocat implementării. De fapt, la respectiva
probă au fost propuse trei probleme, cam de același nivel de dificultate, iar
timpul total permis a fost de 4 ore. De asemenea, structurile de date impuse
și modul de utilizare a lor sunt mai rar întâlnite.

Ideea de la care se pornește în rezolvarea acestei probleme este următoarea:
Parantezarea și completarea cu operatori a unui șir de $N$ numere, $V=(V(1),
V(2), \dots, V(N))$, astfel încât să se obțină rezultatul $D$ este posibilă
dacă și numai dacă există două valori $D_1$ și $D_2$, un număr întreg $K$
cuprins între 1 și $N-1$ și un operator $\oplus \in \{ +, -, \times, / \}$
astfel încât următoarele condiții să fie îndeplinite simultan:

\begin{enumerate}

\item Este posibilă parantezarea și completarea cu operatori a șirului $V' =
  (V(1),$ $V(2),$ $\dots, V(K))$ astfel încât să se obțină rezultatul $D_1$;

\item Este posibilă parantezarea și completarea cu operatori a șirului
  $V''=(V(K+1), V(K+2), \dots, V(N))$ astfel încât să se obțină rezultatul
  $D_2$;

\item $D_1 \oplus D_2 = D$

\end{enumerate}

Cu alte cuvinte, trebuie să găsim un loc în care să „spargem” expresia noastră
în așa fel încât, luând două valori care se pot obține pentru partea din
stânga, respectiv din dreapta, și inserând între ele operatorul potrivit, să
obținem valoarea $D$.

Pentru aflarea operatorului, a locului de împărțire a expresiei în două și a
celor două valori necesare, ar fi de ajuns patru instrucțiuni repetitive {\tt
  for}. Totuși, rămâne o singură întrebare: cum se poate verifica dacă se
poate sau nu obține valoarea $D_1$ pentru subexpresia din stânga, respectiv
valoarea $D_2$ pentru subexpresia din dreapta? Aceasta este o subproblemă
similară cu problema în sine, dar redusă la dimensiuni mai mici. O abordare
directă ar fi comodă: se scrie o procedură care efectuează cele patru
instrucțiuni for și, pentru fiecare combinație posibilă de operatori și
operanzi, se reapelează recursiv ca să afle dacă valorile operanzilor se pot
obține. Totuși, este intuitiv că această variantă va avea o complexitate
uriașă, care o face inutilizabilă.

Motivul principal al nerentabilității acestei implementări este că ea reface
de nenumărate ori exact aceleași calcule. Spre exemplu, dacă $N=4$, programul
va încerca să spargă vectorul $(V(1), V(2), V(3), V(4))$ în două părți. Există
trei moduri posibile:

\begin{enumerate}[label=(\alph*)]

\item $(V(1))$ și $(V(2), V(3), V(4))$;

\item $(V(1), V(2))$ și $(V(3), V(4))$;

\item $(V(1), V(2), V(3))$ și $(V(4))$;

\end{enumerate}

Pentru a studia cazul ({\bf c}), expresia stângă va trebui la rândul ei ruptă
în două bucăți, lucru care poate fi făcut în două moduri:

\begin{enumerate}[label=(\alph*)]
  \setcounter{enumi}{3}

\item $(V(1))$ și $(V(2), V(3))$;

\item $(V(1), V(2))$ și $(V(3))$;

\end{enumerate}

Se observă că deja secvența $(V(1), V(2))$ a fost studiată de două ori (în
cazurile ({\bf b}) și ({\bf e})), iar secvența $(V(1))$ tot de două ori (în
cazurile ({\bf a}) și ({\bf d})). Exemplele pot continua. Dacă însă am reuși
să nu mai evaluăm de două ori aceeași secvență, complexitatea programului s-ar
reduce foarte mult. Pentru aceasta, trebuie să pornim cu secvențe foarte
scurte (întâi cele de un singur număr, apoi cele de două numere), și să trecem
la secvențe mai lungi, bazându-ne pe faptul că secvențele mai lungi se
descompun în secvențe mai scurte care au fost deja analizate. Facem mențiunea
că o secvență cu un singur număr poate fi parantezată într-un singur fel
(practic nu este nevoie de paranteze și operatori) și poate produce un singur
rezultat, egal cu valoarea numărului. O secvență de două numere poate produce
maximum patru rezultate distincte, prin folosirea pe rând a celor patru
operatori disponibili.

Pentru a stoca rezultatele obținute, vom folosi o matrice tridimensională $A$
de dimensiuni $N \times N \times P$. $A[i,j,r]$ indică dacă există vreo
parantezare corespunzătoare a secvenței $(V(i), V(i+1), \dots, V(j))$ astfel
încât să se obțină rezultatul $r$. Dacă o asemenea parantezare există,
$A[i,j,r]$ va indica punctul în care trebuie spartă în două expresia
(printr-un număr între $i$ și $j-1$). Dacă nu există o asemenea parantezare,
$A[i,j,r]$ va lua o valoare specială (0 de exemplu).

De aici decurge modul de inițializare al matricei:

\begin{equation}
  \begin{cases}
    A[i,i,V(i)] = i, & \quad \forall 1 \leq i \leq N \\

    A[i,j,r] = 0, & \quad \text{pentru orice alte valori ale lui } i, j \text{
      și } r
  \end{cases}
\end{equation}

Matricea se va completa pe diagonală, începând de la diagonala principală și
terminând în colțul de NE. Pentru a afla toate valorile care se pot obține
prin parantezarea secvenței $(V(i), V(i+1), \dots, V(j))$, se va împărți
această expresie în două părți disjuncte, în toate modurile posibile: $(V(i))$
și $(V(i+1), \dots, V(j))$, apoi $(V(i), V(i+1))$ și $(V(i+2), \dots, V(j))$
și așa mai departe până la $(V(i), V(i+1), \dots, V(j-1))$ și
$(V(j))$. Fiecăreia din părțile obținute îi va corespunde în matrice un
element de forma $A[i,k]$ sau $A[k+1,j]$, cu $i \leq k < j$. În orice caz,
toate elementele de această formă se vor afla în matrice dedesubtul diagonalei
din care face parte elementul $A[i,j]$, deci pentru secvențele respective se
cunosc deja toate valorile pe care le pot lua prin parantezare și introducerea
operatorilor. Tot ce avem de făcut este să combinăm în toate modurile aceste
valori prin inserarea fiecăruia din cei patru operatori pentru a obține toate
valorile posibile ale expresiei $(V(i), V(i+1), \dots, V(j))$.

Dacă în final $A[1,N,D] \neq 0$, atunci problema are soluție. Vom vedea
imediat și cum se reconstituie ea. Iată modul de compunere a matricei pentru
exemplul din enunț (cu deosebirea că, în figurile de mai jos, $A[i,j]$ nu mai
este un vector cu $P$ elemente, ci o mulțime de valori între 0 și $P-1$,
această reprezentare fiind mai comodă):

\begin{equation}
  A[1,1] = \{4\} \quad A[2,2] = \{7\} \quad A[3,3] = \{9\}
\end{equation}

\begin{equation}
  A =
  \begin{pmatrix}
    \{4\} & ? & ? \\
    ? & \{7\} & ? \\
    ? & ? & \{9\}
  \end{pmatrix}
\end{equation}

Între numerele 4 și 7 plasăm cei patru operatori și obținem:

\begin{align*}
  4 + 7 & \equiv 0 \pmod{11} \\
  4 - 7 & \equiv 4 + 4 \equiv 8 \pmod{11} \\
  4 \times 7 & \equiv 6 \pmod{11} \\
  4\ /\ 7 & \equiv 4 \times 8 \equiv 10 \pmod{11} \\
  A[1,2] & = \{0, 6, 8, 10\}
\end{align*}

Analog se procedează pentru numerele 7 și 9:

\begin{align*}
  7 + 9 & \equiv 5 \pmod{11} \\
  7 - 9 & \equiv 7 + 2 \equiv 9 \pmod{11} \\
  7 \times 9 & \equiv 8 \pmod{11} \\
  7\ /\ 9 & \equiv 7 \times 5 \equiv 2 \pmod{11} \\
  A[2,3] & = \{2, 5, 8, 9\}
\end{align*}

\begin{equation}
  A =
  \begin{pmatrix}
    \{4\} & \{0, 6, 8, 10\} & ? \\
    ? & \{7\} & \{2, 5, 8, 9\} \\
    ? & ? & \{9\}
  \end{pmatrix}
\end{equation}

Pentru a calcula $A[1,3]$, putem grupa termenii în două moduri: Fie primul
separat și ultimii doi separat, fie primii doi separat și ultimul separat. În
primul caz, ultimii doi termeni - după cum s-a văzut - pot produce patru
rezultate distincte (2, 5, 8 și 9). Combinând oricare din aceste rezultate cu
primul termen (4) și adăugând orice operator, vor rezulta 16 valori posibile,
din care evident unele vor coincide. Analog se procedează și pentru celălalt
caz:

\begin{table}[H]
  \centering
  \begin{tabular}{l@{\hspace{1in}}l}
    \hline
    Cazul I & Cazul II \\ \hline
    {$\begin{aligned}
        4 + 2 & \equiv 6 \pmod{11} \\
        4 - 2 & \equiv 2 \pmod{11} \\
        4 \times 2 & \equiv 8 \pmod{11} \\
        4\ /\ 2 & \equiv 2 \pmod{11} \\
      \end{aligned}$}
    &
    {$\begin{aligned}
        0 + 9 & \equiv 9 \pmod{11} \\
        0 - 9 & \equiv 2 \pmod{11} \\
        0 \times 9 & \equiv 0 \pmod{11} \\
        0\ /\ 9 & \equiv 0 \pmod{11} \\
      \end{aligned}$}
    \\ \hline
    {$\begin{aligned}
        4 + 5 & \equiv 9 \pmod{11} \\
        4 - 5 & \equiv 10 \pmod{11} \\
        4 \times 5 & \equiv 9 \pmod{11} \\
        4\ /\ 5 & \equiv 3 \pmod{11} \\
      \end{aligned}$}
    &
    {$\begin{aligned}
        6 + 9 & \equiv 4 \pmod{11} \\
        6 - 9 & \equiv 8 \pmod{11} \\
        6 \times 9 & \equiv 10 \pmod{11} \\
        6\ /\ 9 & \equiv 8 \pmod{11} \\
      \end{aligned}$}
    \\ \hline
    {$\begin{aligned}
        4 + 8 & \equiv 1 \pmod{11} \\
        4 - 8 & \equiv 7 \pmod{11} \\
        4 \times 8 & \equiv 10 \pmod{11} \\
        4\ /\ 8 & \equiv 6 \pmod{11} \\
      \end{aligned}$}
    &
    {$\begin{aligned}
        8 + 9 & \equiv 6 \pmod{11} \\
        8 - 9 & \equiv 10 \pmod{11} \\
        8 \times 9 & \equiv 6 \pmod{11} \\
        8\ /\ 9 & \equiv 7 \pmod{11} \\
      \end{aligned}$}
    \\ \hline
    {$\begin{aligned}
        4 + 9 & \equiv 2 \pmod{11} \\
        4 - 9 & \equiv 6 \pmod{11} \\
        4 \times 9 & \equiv 3 \pmod{11} \\
        4\ /\ 9 & \equiv 9 \pmod{11} \\
      \end{aligned}$}
    &
    {$\begin{aligned}
        10 + 9 & \equiv 8 \pmod{11} \\
        10 - 9 & \equiv 1 \pmod{11} \\
        10 \times 9 & \equiv 2 \pmod{11} \\
        10\ /\ 9 & \equiv 6 \pmod{11} \\
      \end{aligned}$}
    \\ \hline
  \end{tabular}
\end{table}

\begin{align*}
  A[1,3] = \{0, 1, 2, 3, 4, 6, 7, 8, 9, 10\}
\end{align*}

\begin{equation}
  A =
  \begin{pmatrix}
    \{4\} & \{0, 6, 8, 10\} & \{0 \dots 4, 6 \dots 10\} \\
    ? & \{7\} & \{2, 5, 8, 9\} \\
    ? & ? & \{9\}
  \end{pmatrix}
\end{equation}

Se observă că singura valoare care nu se poate obține prin parantezarea
șirului (4, 7, 9) este 5. Să vedem acum și care este metoda de reconstituire a
soluției. Fie $A[1,N,D]=X$. Dacă $X=0$, atunci nu există soluție. Dacă $X \neq
0$, atunci $X$ indică poziția din vector după care trebuie inserat semnul. Nu
se indică însă ce semn trebuie inserat, nici care sunt valorile care trebuie
obținute pentru partea stângă, respectiv dreaptă. De aceea, vom căuta o
combinație oarecare de valori $D_1$ și $D_2$ care se pot obține și un operator
oarecare astfel încât $D_1 \oplus D_2 = D$. Odată ce le găsim, vom căuta, prin
aceeași metodă, o modalitate de a obține valoarea $D_1$ în partea stângă a
vectorului (știm sigur că această modalitate există) și o modalitate de a
obține valoarea $D_2$ în partea dreaptă a vectorului.

Iată în continuare câteva detalii de implementare. Pentru efectuarea
operațiilor matematice modulo $P$ s-a scris o funcție separată, $Expr$, care
primește două numere $X$ și $Y$ între 0 și $P-1$ și un număr $Op$ între 1 și 4
reprezentând codificarea operatorului și întoarce un număr între 0 și $P$,
reprezentând valoarea operației ($X\,Op\,Y$). De fapt, ar trebui ca această
funcție să întoarcă un număr între 0 și $P-1$ dacă operația este posibilă și
să nu întoarcă nimic dacă operația este imposibilă, respectiv dacă se încearcă
o împărțire la 0. Cum acest lucru nu este posibil în Pascal, am asimilat
valoarea $P$ cu un cod de eroare, iar funcția va întoarce această valoare
(care nu poate fi atinsă prin operații obișnuite) în cazul unei împărțiri prin
0. Mai departe, pentru a nu face un test separat dacă rezultatul funcției
reprezintă o adresă valabilă în cea de-a treia dimensiune a tabloului, am
preferat să supradimensionăm tabloul cu o unitate și să ignorăm tot ceea ce se
scrie în coloana $P$.

Să ne ocupăm acum de operațiile aritmetice. Adunarea, scăderea și înmulțirea
se fac în timp constant. O problemă apare în cazul împărțirii, deoarece ea nu
mai seamănă deloc cu cea învățată pe mulțimea $\mathbb{R}$. Pentru a calcula
$X/Y$, trebuie găsit acel număr $Z$ care, înmulțit cu $Y$, să dea $X$. Acest
lucru se poate face într-o primă fază prin căutare secvențială (se încearcă
valoarea 0, apoi 1, apoi 2 și așa mai departe; trebuie să existe un cât
deoarece $P$ este prim). Tehnica are însă influențe neplăcute asupra
complexității, supărătoare asupra timpului de rulare și dezastruoase asupra
punctajului obținut. De aceea, este bine ca, în măsura în care timpul o
permite, să se construiască o tabelă predefinită de calculare a
inverșilor. Atunci, în loc să se efectueze împărțirea $X/Y$, se efectuează
înmulțirea $X \times Y^{-1}$. Inversul unui element depinde și de modulul
ales. Programul care urmează construiește un tabel care a fost importat în
programul sursă ca o constantă, matricea {\tt Invers} ({\tt Invers[A,B]} este
inversul lui $B$ modulo $A$). Liniile corespunzătoare unor numere neprime au
fost totuși inserate, pentru ușurința implementării.

\begin{lstlisting}[language=Pascal]
program Invert;
const NMax=30;
      PMax=23;
var i,P:Integer;

function Invers(P,K:Integer):Integer;
var i:Integer;
begin
  i:=0;
  repeat Inc(i) until (K*i) mod P=1;
  Invers:=i;
end;

begin
  Assign(Output,'invers.txt');Rewrite(Output);
  for P:=2 to PMax do
    begin
      Write('{',P:2,'} (');
      for i:=1 to NMax do
        begin
          if (P in [2,3,5,7,11,13,17,19,23]) and (i<P)
            then Write(Invers(P,i):2)
            else Write(99);
          if i<>NMax then Write(',');
          if i=NMax div 2 then Write(#13#10'      ');
        end;
      WriteLn('),');
    end;
  Close(Output);
end.
\end{lstlisting}

Să analizăm în sfârșit complexitatea: Trebuie completată o matrice, deci
$O(N^2)$ elemente. Pentru fiecare element din matrice, secvența
corespunzătoare din vector trebuie spartă în două în toate modurile posibile,
deci încă $O(N)$. Pentru fiecare descompunere, trebuie combinate în toate
felurile toate valorile disponibile pentru partea stângă, respectiv
dreaptă. Cum numărul de valori este $O(P)$, rezultă o complexitate de
$O(P^2)$. Înmulțind toți acești factori rezultă o complexitate totală de
$O(N^3 \times P^2)$. Dacă nu am fi făcut împărțirea a două numere modulo P în
timp constant, ci în $O(P)$, atunci complexitatea totală ar fi fost $O(N^3
\times P^3)$, deci timpul de rulare putea fi și de 20 de ori mai mare.

Programul de mai jos pare îngrozitor de lung, dar, dacă avem în vedere faptul
că o bună bucată o reprezintă constanta {\tt Invers}, care este generată,
putem avea speranțe să-l scriem în timpul alocat.

\begin{lstlisting}[language=Pascal]
program ParaNT;
{$B-,I-,R-,S-}
const NMax=30;
      PMax=23;
      NoWay=0;
      OpNames:String[4]='+-*/';
      Invers:array[2..PMax,1..NMax] of Integer=
{ 2}(( 1,99,99,99,99,99,99,99,99,99,99,99,99,99,99,
      99,99,99,99,99,99,99,99,99,99,99,99,99,99,99),
{ 3} ( 1, 2,99,99,99,99,99,99,99,99,99,99,99,99,99,
      99,99,99,99,99,99,99,99,99,99,99,99,99,99,99),
{ 4} (99,99,99,99,99,99,99,99,99,99,99,99,99,99,99,
      99,99,99,99,99,99,99,99,99,99,99,99,99,99,99),
{ 5} ( 1, 3, 2, 4,99,99,99,99,99,99,99,99,99,99,99,
      99,99,99,99,99,99,99,99,99,99,99,99,99,99,99),
{ 6} (99,99,99,99,99,99,99,99,99,99,99,99,99,99,99,
      99,99,99,99,99,99,99,99,99,99,99,99,99,99,99),
{ 7} ( 1, 4, 5, 2, 3, 6,99,99,99,99,99,99,99,99,99,
      99,99,99,99,99,99,99,99,99,99,99,99,99,99,99),
{ 8} (99,99,99,99,99,99,99,99,99,99,99,99,99,99,99,
      99,99,99,99,99,99,99,99,99,99,99,99,99,99,99),
{ 9} (99,99,99,99,99,99,99,99,99,99,99,99,99,99,99,
      99,99,99,99,99,99,99,99,99,99,99,99,99,99,99),
{10} (99,99,99,99,99,99,99,99,99,99,99,99,99,99,99,
      99,99,99,99,99,99,99,99,99,99,99,99,99,99,99),
{11} ( 1, 6, 4, 3, 9, 2, 8, 7, 5,10,99,99,99,99,99,
      99,99,99,99,99,99,99,99,99,99,99,99,99,99,99),
{12} (99,99,99,99,99,99,99,99,99,99,99,99,99,99,99,
      99,99,99,99,99,99,99,99,99,99,99,99,99,99,99),
{13} ( 1, 7, 9,10, 8,11, 2, 5, 3, 4, 6,12,99,99,99,
      99,99,99,99,99,99,99,99,99,99,99,99,99,99,99),
{14} (99,99,99,99,99,99,99,99,99,99,99,99,99,99,99,
      99,99,99,99,99,99,99,99,99,99,99,99,99,99,99),
{15} (99,99,99,99,99,99,99,99,99,99,99,99,99,99,99,
      99,99,99,99,99,99,99,99,99,99,99,99,99,99,99),
{16} (99,99,99,99,99,99,99,99,99,99,99,99,99,99,99,
      99,99,99,99,99,99,99,99,99,99,99,99,99,99,99),
{17} ( 1, 9, 6,13, 7, 3, 5,15, 2,12,14,10, 4,11, 8,
      16,99,99,99,99,99,99,99,99,99,99,99,99,99,99),
{18} (99,99,99,99,99,99,99,99,99,99,99,99,99,99,99,
      99,99,99,99,99,99,99,99,99,99,99,99,99,99,99),
{19} ( 1,10,13, 5, 4,16,11,12,17, 2, 7, 8, 3,15,14,
       6, 9,18,99,99,99,99,99,99,99,99,99,99,99,99),
{20} (99,99,99,99,99,99,99,99,99,99,99,99,99,99,99,
      99,99,99,99,99,99,99,99,99,99,99,99,99,99,99),
{21} (99,99,99,99,99,99,99,99,99,99,99,99,99,99,99,
      99,99,99,99,99,99,99,99,99,99,99,99,99,99,99),
{22} (99,99,99,99,99,99,99,99,99,99,99,99,99,99,99,
      99,99,99,99,99,99,99,99,99,99,99,99,99,99,99),
{23} ( 1,12, 8, 6,14, 4,10, 3,18, 7,21, 2,16, 5,20,
      13,19, 9,17,15,11,22,99,99,99,99,99,99,99,99));

type Matrix=array[1..NMax, 1..NMax, 0..PMax] of Integer;
       { S-a inclus si valoarea PMax, care nu poate fi
         atinsa, pentru a se depozita "deseurile" }
     Vector=array[1..NMax] of Integer;
var A:Matrix;       { Matricea de calcul }
    V:Vector;       { Numerele }
    N,P,D:Integer;

procedure ReadData;
var i:Integer;
begin
  Assign(Input,'input.txt');Reset(Input);
  ReadLn(P,N,D);
  for i:=1 to N do Read(V[i]);
  Close(Input);
end;

function Expr(X,Y,Op:Integer):Integer;
{ Calculeaza expresia (X Op Y) unde Op=1 ('+'),
  Op=2 ('-'), Op=3 ('*'), Op=4 ('/'). Daca Op=4
  si Y=0 se returneaza valoarea P (care nu poate
  fi atinsa prin alte operatii corecte). }
begin
  case Op of
    1:Expr:=(X+Y) mod P;
    2:Expr:=(X+P-Y) mod P;
    3:Expr:=(X*Y) mod P;
    4:if Y=0 then Expr:=P  { = imposibil }
             else Expr:=(X*Invers[P,Y]) mod P
      { S-a creat o tabela predefinita de inversi,
        deoarece altfel impartirea se efectua numai
        in O(P) }
  end; {case}
end;

procedure Combine(i,j,k:Integer);
{ Urmeaza a se combina toate valorile posibile
  pentru A[i,k] si A[k+1,j] pentru a se afla
  toate valorile posibile pentru A[i,j] }
var p1,p2,Op:Integer;
begin
  for p1:=0 to P-1 do
    if A[i,k,p1]<>NoWay
      then for p2:=0 to P-1 do
             if A[k+1,j,p2]<>NoWay
               then { Am gasit doua valori posibile
                      si aplicam cei patru operatori }
                    for Op:=1 to 4 do
                      A[i,j,Expr(p1,p2,Op)]:=k;
end;

procedure ComposeMatrix;
var i,j,k,l:Integer;
begin
  { Initializarea matricei }
  for i:=1 to N do
    for j:=1 to N do
      for k:=0 to P-1 do A[i,j,P]:=NoWay;

  for i:=1 to N do
    A[i,i,V[i]]:=1; { sau orice <> NoWay }

  for l:=2 to N do { Lungimea intervalelor }
    for i:=1 to N-l+1 do
      begin
        j:=i+l-1; { S-au fixat [i,j] capetele intervalului }
        for k:=i to j-1 do { Se alege locul de impartire }
          Combine(i,j,k);
      end;
end;

procedure SeekValues(Lo,Hi,Mid,Value:Integer;
                     var v1,v2:Integer; var Op:Char);
{ Se stie unde e "sparta" expresia in doua; se cauta
  valorile care trebuie obtinute pentru partea stanga,
  respectiv dreapta, si pentru operator }
var i,j,k:Integer;
begin
  for i:=0 to P-1 do
    if A[Lo,Mid,i]<>NoWay
      then for j:=0 to P-1 do
             if A[Mid+1,Hi,j]<>NoWay
               then for k:=1 to 4 do
                      if Expr(i,j,k)=Value
                        then begin
                               v1:=i;
                               v2:=j;
                               Op:=OpNames[k];
                               Exit;
                             end;
end;

procedure WriteExpression(Lo,Hi,Value:Integer);
var v1,v2,Place:Integer;
    Op:Char;
begin
  if Lo=Hi
    then Write(V[Lo])
    else begin
           Place:=A[Lo,Hi,Value];
           SeekValues(Lo,Hi,Place,Value,v1,v2,Op);
           Write('(');
           WriteExpression(Lo,Place,v1);
           Write(Op);
           WriteExpression(Place+1,Hi,v2);
           Write(')');
         end;
end;

procedure WriteSolution;
begin
  Assign(Output,'output.txt');Rewrite(Output);
  if A[1,N,D]=NoWay
    then Write('Nu exista solutie')
    else WriteExpression(1,N,D);
  WriteLn;
  Close(Output);
end;

begin
  ReadData;
  ComposeMatrix;
  WriteSolution;
end.
\end{lstlisting}

O posibilă îmbunătățire a programului de sus ar fi să reținem în $A[i,j,r]$ nu
numai locul unde se face secționarea expresiei, ci și operatorul introdus și
eventual și valorile care trebuie obținute pe partea stângă, respectiv
dreaptă. Totuși, volumul de date ar fi crescut corespunzător și ar fi devenit
greu de manipulat. În schimb, în versiunea prezentă, programul merge puțin mai
lent, dar nesesizabil. Să vedem de ce. Complexitatea reconstituirii expresiei
în sine se află astfel: avem de reconstituit $O(N)$ operatori. Pentru fiecare
din ei, trebuie să căutăm valorile stângă și dreaptă și operatorul în sine,
deci $O(4 \times P^2)$. Complexitatea totală a reconstituirii datelor este
$O(N \times P^2)$, adică oricum mult mai mică față de cea a compunerii
matricei. Este preferabil să nu complicăm structurile de date și codul scris,
mai ales că diferența ca timp de rulare este infimă.

Propunem ca temă cititorului o versiune a acestei probleme, în care nu se va
mai lucra în inelul $\mathbb{Z}_P$, ci într-un grup cu elementele $a, b, c, d,
\dots,$ pentru care se cunoaște tabela de compoziție. În acest caz avem un
singur operator $\oplus$, iar cerința este să se parantezeze expresia $x_1
\oplus x_2 \oplus \cdots \oplus x_k$ astfel încât rezultatul să fie $y$.

  \section{Problema 11}

Problema celui mai lung prefix a fost dată la a VIII-a Olimpiadă
Internațională de Informatică, Veszprem 1996. Iată enunțul nemodificat al
problemei:

{\bf ENUNȚ}: Structura unor compuși biologici este reprezentată prin
succesiunea con\-sti\-tu\-en\-ți\-lor lor. Acești constituenți sunt notați cu
litere mari. Biologii sunt interesați să descompună o secvență lungă în altele
mai scurte, numite primitive. Spunem că o secvență $S$ poate fi compusă
dintr-un set de primitive $P$ dacă există $N$ primitive $p_1, \dots, p_N$ în
$P$ astfel încât concatenarea $p_1 p_2 \dots p_N$ a primitivelor să fie egală
cu $S$. Aceeași primitivă poate interveni de mai multe ori în concatenare și
nu trebuie neapărat ca toate primitivele să fie prezente.

Primele $M$ caractere din $S$ se numesc prefixul lui $S$ de lungime
$M$. Scrieți un program care primește la intrare un set de primitive $P$ și o
secvență de constituenți $T$. Programul trebuie sa afle lungimea celui mai
lung prefix al lui $T$ care se poate compune din primitive din $P$.

{\bf Datele de intrare} apar în două fișiere. Fișierul {\tt INPUT.TXT} descrie
setul de primitive $P$, iar fișierul {\tt DATA.TXT} conține secvența de
examinat. Pe prima linie din {\tt INPUT.TXT} se află $N$, numărul de primitive
din $P$ ($1 \leq N \leq 100$). Fiecare primitivă se dă pe două linii
consecutive: pe prima lungimea $L$ a primitivei ($1 \leq L \leq 20$), iar pe a
doua un șir de litere mari de lungime $L$. Toate cele $N$ primitive sunt
distincte.

Fiecare linie din fișierul {\tt DATA.TXT} conține o literă mare pe prima
poziție. El se termină cu o linie conținând un punct („.”). Lungimea secvenței
este cuprinsă între 1 si 500.000.

{\bf Ieșirea}: Pe ecran se va tipări lungimea celui mai lung prefix din $T$
care poate fi compus din primitive din $P$.

{\bf Exemplu}:

\texttt{
  \begin{tabular}{|l|l|}
    \hline
        {\bf INPUT.TXT} & {\bf OUTPUT.TXT} \\ \hline
        5   & A \\
        1   & B \\
        A   & A \\
        2   & B \\
        AB  & A \\
        3   & C \\
        BBC & A \\
        2   & B \\
        CA  & A \\
        2   & A \\
        BA  & B \\
        \   & C \\
        \   & B \\
        \   & . \\
        \hline
  \end{tabular}
}

Pe ecran se va tipări numărul 11.

{\bf Timp de rulare}: 30 secunde pentru un test.

{\bf Timp de implementare}: 1h.

{\bf Complexitate cerută}: $O(S \times L \times N)$, unde $S$ este lungimea
secvenței.

{\bf REZOLVARE}: Menționăm de la început că datele problemei sunt
supradimensionate. Nici unul din cele zece teste cu care au fost verificate
programele nu a depășit în realitate 12 primitive. În schimb, fișierul {\tt
  DATA.TXT} a fost unic pentru toate testele și a conținut o secvență de
lungime 500.000.

Problema se rezolvă și în acest caz prin reducerea ei la una similară, dar cu
date de intrare mai mici. Respectiv, un prefix $S$ al lui $T$ se poate
descompune în primitive dacă există o primitivă $p_i$ astfel încât $S=S'+p_i$
și $S'$ se poate descompune în primitive. Am redus împărțirea în primitive a
lui $S$ la despărțirea în primitive a lui $S'$, care are o lungime mai mică
decât $S$. O primă modalitate, pur teoretică, de a rezolva problema, este să
reținem toate prefixele lui $T$ care se pot descompune în primitive; în felul
acesta, putem studia prefixe din ce în ce mai lungi, bazându-ne pe prefixe mai
scurte deja studiate. Totuși, este imposibil să ținem minte toate prefixele
lui $T$, deoarece lungimea medie a unui prefix poate atinge 250.000 caractere.

O îmbunătățire care poate fi adusă acestui algoritm este următoarea: deoarece
toate prefixele aparțin aceleiași secvențe de constituenți, $T$, este
suficient să reținem în întregime secvența $T$ și, pentru fiecare prefix ce se
poate descompune în primitive, păstrăm doar lungimea sa. Făcând abstracție de
limitările de memorie, putem crea un vector $V$ cu $S$ variabile booleene
(unde $S \leq 500.000$), iar $V[i]$ va indica dacă subșirul de lungime $i$ din
$T$ se poate descompune în primitive. $V[i]$ va primi valoarea {\bf True} dacă
și numai dacă există o primitivă $p$ în $P$ de lungime $L$ astfel încât să fie
îndeplinite simultan condițiile:

\begin{itemize}

\item $V[i-L] = \mathbf{True}$;

\item secvența de caractere {\tt T[i-L+1]T[i-L+2]$\dots$T[i]} este egală cu
  $p$.

\end{itemize}

Iată o primă variantă (în pseudocod) a algoritmului:

\vspace{\algskip}
\begin{algorithmic}[1]
  \REQUIRE $T$, primitivele
  \FOR{$i = 1$ la $S$}
  \IF{există o primitivă $p$ astfel încât $V[i-L]$ și $T[i-L+1]T[i-L+2] \dots T[i] =p$}
  \STATE $V[i] \leftarrow \mathbf{True}$
  \ELSE
  \STATE $V[i] \leftarrow \mathbf{False}$
  \ENDIF
  \ENDFOR
  \STATE caută cel mai mare $i$ pentru care $V[i] = \mathbf{True}$
  \PRINT $i$
\end{algorithmic}

Chiar și în acest caz, apare o problemă, deoarece avem nevoie de doi vectori,
unul de caractere și altul de variabile booleene, ambii de lungime maxim
500.000. Necesarul de memorie este deci cam de 1MB. Sigur, pentru
calculatoarele de astăzi această sumă este ușor de alocat, dar problema admite
oricum o soluție la fel de rapidă și mult mai economică. Iată care este
principiul:

Pentru a vedea dacă un prefix $S$ de lungime $L$ se poate descompune în
primitive, noi avem nevoie să cunoaștem dacă prefixele de lungime mai mică se
pot descompune. Dar avem oare nevoie de toate prefixele? Nu, deoarece noi vom
concatena unul din prefixele de lungime mai mică cu o primitivă pentru a
obține noul prefix $S$. Însă primitivele au lungime de maxim 20
caractere. Așadar, noi nu trebuie să cunoaștem decât dacă prefixele de lungime
$L-1, L-2, \dots, L-20$ se pot descompune; restul nu ne interesează. În felul
acesta am eliminat vectorul $V$ și l-am redus la un vector de numai 20 de
elemente (care în program se numește {\tt CanGet}). La fiecare moment, când se
prelucrează un nou caracter din secvența de constituenți $T$, primul element
din {\tt CanGet} se pierde (deoarece informația pe care el o stochează este
învechită), iar următoarele 19 elemente se deplasează spre stânga cu câte o
poziție. Al 20-lea element, care acum a rămas disponibil, va fi calculat la
pasul curent.

O altă modificare pornește de la observația că, datorită aceleiași limitări a
lungimii primitivelor la 20 de caractere, nu avem nevoie nici măcar să reținem
întregul vector $T$, ci numai ultimele 20 de litere ale lui. La fiecare pas,
litera cea mai „veche” din $T$ (adică de indice minim) se va pierde, iar la
celelalte 19 litere se va adăuga litera nou citită. Avem așadar nevoie doar de
un string de 20 de caractere, pe care în program l-am numit {\tt Last}. Pentru
a deplasa spre stânga vectorii {\tt CanGet} și {\tt Last}, se pot folosi fie
atribuirile succesive, fie rutinele de acces direct la memorie.

Deoarece prefixele care se pot descompune sunt identificate în ordinea
crescătoare a lungimii, ultimul asemenea prefix găsit este tocmai cel de
lungime maximă. Dar pentru că, în momentul în care găsim un prefix, nu putem
ști dinainte că el este ultimul, trebuie să reținem într-o variabilă lungimea
celui mai lung prefix găsit până la momentul respectiv, variabilă pe care o
actualizăm de fiecare dată când găsim un nou prefix.

În felul acesta, am reușit să reducem memoria folosită aproape la strictul
necesar, adică numai la dicționarul de primitive și la doi vectori de câte
douăzeci de caractere. Se recomandă totuși să se aloce un buffer cât mai mare
pentru citirea datelor din fișierul {\tt DATA.TXT} pentru mărirea vitezei de
citire. Repartizarea buffer-ului se face cu procedura Pascal {\tt SetTextBuf}.

O optimizare care nu a fost inclusă în program, fiind lăsată ca temă
cititorului, este următoarea: dacă la un moment dat, în timpul examinării
secvenței de constituenți, este întâlnit un șir de cel puțin 20 de prefixe
consecutive, din care nici unul nu se poate descompune în primitive, atunci
nici mai departe nu vom mai întâlni vreun prefix care să se poată
descompune. Explicați de ce. Această optimizare poate să nu aducă uneori nimic
nou în evoluția programului, dar alteori poate să reducă la zero timpul de
rulare.

Prezentăm mai jos codul sursă al programului. A fost preferat limbajul Pascal,
deoarece pune la dispoziție rutine mai comode de manevrare a șirurilor de
caractere.

\inputminted{pascal}{src/problem11.pas}

  \section{Problema 12}

Această problemă a fost propusă la a IV-a Balcaniadă de Informatică, Nicosia
1996.

{\bf ENUNȚ}: Grupul de rock U2 va da un concert în Nicosia. Un grup de $N \leq
200$ fani U2 așteaptă la coadă în scopul de a cumpăra bilete de la singura
caserie deschisă. Fiecare persoană vrea să cumpere numai un bilet, iar
casierul poate vinde unei persoane cel mult două bilete.

Casierul folosește $T[i]$ unități de timp pentru a servi al $i$-lea fan ($1
\leq i \leq N$). Este posibil totuși ca doi fani așezați la coadă unul după
altul (de exemplu, al $j$-lea și al $j+1$-lea) să convină ca numai unul din ei
să rămână la coadă, iar celălalt să plece, dacă timpul $R[j]$ ($1 \leq j \leq
N-1$) în care casierul servește al $j$-lea și al $j+1$-lea fan este mai mic
decât $T[j]+T[j+1]$. Deci, pentru a minimiza timpul de lucru al casierului,
fiecare persoană din coadă încearcă să negocieze cu predecesorul și cu
succesorul său, ceea ce va duce în final la o servire mai rapidă.

Fiind date numerele întregi pozitive $N$, $T[i]$ ($1 \leq i \leq N$) și $R[j]$
($1 \leq j \leq N-1$), se cere să se minimizeze timpul total al
casierului. Acest lucru va fi realizat grupând într-un mod optim perechi de
persoane consecutive. Atenție! Nu este necesar ca un anumit fan să se cupleze
neapărat cu predecesorul sau cu succesorul său.

{\bf Intrarea}: În fișierul {\tt INPUT.TXT}, datele de intrare sunt date pe
trei linii:

\begin{itemize}

\item prima linie conține numărul întreg $N$;

\item a doua linie conține $N$ întregi: valorile $T[i]$, separate prin câte un
  spațiu;

\item a treia linie conține $N$-1 întregi: valorile $R[j]$, separate de
  asemenea prin câte un spațiu;

\end{itemize}

Ieșirea se va face în fișierul {\tt OUTPUT.TXT}, astfel:

\begin{itemize}

\item prima linie conține un întreg care reprezintă timpul total (minim) al
  casierului;

\item pe fiecare din următoarele linii se află un singur număr sau două numere
  separate prin caracterul '+'. Mai exact, fiecare linie conține numărul $i$
  dacă al $i$-lea fan este servit singur, sau $i+(i+1)$ dacă cei doi fani sunt
  serviți ca o pereche.

\end{itemize}

{\bf Exemplu}:

\texttt{
  \begin{tabular}{|l|l|}
    \hline
        {\bf INPUT.TXT} & {\bf OUTPUT.TXT} \\ \hline
        7              & 14 \\
        5 4 3 2 1 4 4  & 1 \\
        7 3 4 2 2 4    & 2+3 \\
        \              & 4+5 \\
        \              & 6+7 \\
        \hline
  \end{tabular}
}

{\bf Timp de rulare}: 15 secunde pentru un test.

Acesta este enunțul în forma lui de la Nicosia. Iată acum și completările „din
studio”:

\begin{itemize}

\item Numărul de fani este $N \leq 5.000$;

\item {\bf Complexitatea cerută} este $O(N)$;

\item {\bf Timpul de rulare} este de o secundă.

\end{itemize}

{\bf REZOLVARE}: O primă metodă de rezolvare o vom lăsa în seama cititorului,
întrucât ea este foarte asemănătoare cu rezolvarea problemei înmulțirii optime
a unui șir de matrice. Ideea de pornire este de a defini o matrice $D$ cu $N$
linii și $N$ coloane, în care $D[X,Y]$ reprezintă timpul minim în care pot fi
serviți fanii $X, X+1, \dots, Y-1, Y$. Scopul este de a afla valoarea
$D[1,N]$.

După cum se știe, însă, înmulțirea optimă a unui șir de matrice se poate
determina în timp $O(N^3)$. Pentru condițiile inițiale ale problemei ($N \leq
200$), metoda se încadrează în timp. Ea putea fi deci folosită la concurs, pe
câtă vreme dacă se impun condițiile suplimentare, rezolvarea în timp cubic nu
mai dă rezultate. Iată ideea de rezolvare a problemei în timp liniar.

Vom denumi o {\bf cuplare de ordin $K$} modul în care primii $K$ fani sunt
serviți câte unul sau câte doi. Fiecare cuplare are atașat un cost, respectiv
timpul consumat de casier pentru a servi toți fanii. Dintre toate cuplările de
ordin $K$, cele pentru care costul este minim (în cazul general pot fi mai
multe) se vor numi {\bf cuplări optime de ordinul $K$}. Cerința problemei
exprimată cu noua terminologie este: să se găsească o cuplare optimă de
ordinul $N$.

Să presupunem acum că am găsit cumva această cuplare optimă de ordinul $N$, pe
care o vom nota cu $C_N$. Dacă în această cuplare al $K$-lea fan este servit
de unul singur, atunci modul de servire al primilor $K-1$ fani reprezintă o
cuplare optimă de ordinul $K-1$, pe care o vom nota cu $C_{K-1}$. Demonstrația
nu este grea: dacă fanii $1, 2, \dots, K-1$ nu ar fi cuplați în mod optim în
cadrul lui $C_N$, atunci ar exista o cuplare a lor $C'_{K-1}$ de cost mai mic
decât $C_{K-1}$. Dar această cuplare mai bună ar putea fi folosită pentru a
obține o cuplare mai bună a tuturor celor $N$ fani (servind primii $K-1$ fani
conform cuplării $C'_{K-1}$, iar pe ceilalți conform cuplării $C_N$). S-ar
obține astfel o cuplare $C'_N$ de cost mai mic decât $C_N$, ceea ce este
absurd, deoarece am presupus $C_N$ ca fiind optimă.

În mod absolut identic se poate demonstra că dacă $C_N$ este o cuplare optimă
de ordinul $N$ în care fanii $K$ și $K+1$ sunt serviți împreună, atunci modul
de servire al primilor $K-1$ fani reprezintă o cuplare optimă de ordinul
$K-1$. Recunoaștem în aceste afirmații principiul programării dinamice:
optimul global presupune optime locale. De aici deducem că, pentru a realiza o
cuplare optimă a primilor $K$ fani avem nevoie de câte o cuplare optimă pentru
primii $K-1$ fani, respectiv pentru primii $K-2$ fani, urmând ca apoi să aflăm
care este costul minim al unei cuplări de ordin $K$, atât în ipoteza că fanul
al $K$-lea este servit singur, cât și în ipoteza că el este servit împreună cu
al $K-1$-lea fan.

Vom crea așadar doi vectori $T_1$ și $T_2$, ambii cu câte $N$ elemente, în
care:

\begin{itemize}

\item $T_1[K]$ este timpul minim în care pot fi serviți fanii $1, 2, \dots,
  K$, astfel încât fanul al $K$-lea să rămână singur;

\item $T_2[K]$ este timpul minim în care pot fi serviți fanii $1, 2, \dots,
  K$, astfel încât fanii $K$ și $K-1$ să fie cuplați.

\end{itemize}

Datele inițiale pe care le putem trece în cei doi vectori sunt:

\begin{itemize}

\item $T_1[1] = T[1]$;

\item $T_2[1] = \infty$ (un singur fan nu poate fi cuplat cu nimeni);

\item $T_1[2] = T[1]+T[2]$ (dacă al doilea fan rămâne singur, atunci și primul
  rămâne singur);

\item $T_2[2] = R[1]$.

\end{itemize}

Relațiile de recurență se deduc fără prea multă bătaie de cap:

\begin{itemize}

\item $T_1[K] = T[K] + \min(T_1[K-1], T_2[K-1])$ (dacă fanul $K$ rămâne
  singur, atunci el este servit în timpul $T[K]$, iar fanul $K-1$ se va cupla
  cu $K-2$ sau va rămâne singur, după cum este mai convenabil);

\item $T_2[K] = R[K-1] + \min(T_1[K-2], T_2[K-2])$ (dacă fanul $K$ se cuplează
  cu $K-1$, atunci ei sunt serviți în timpul $R[K-1]$, iar fanul $K-2$ se va
  cupla cu $K-3$ sau va rămâne singur, după cum este mai convenabil);

\end{itemize}

Putem deci să completăm vectorii de la stânga la dreapta. Odată ce am făcut
aceasta, costul minim al cuplării de ordinul $N$ este $\min(T_1[N],
T_2[N])$. Pentru a reconstitui și așezarea fanilor, vom proceda recurent,
astfel: dacă $T_2[N]<T_1[N]$, înseamnă că este mai avantajos ca fanii $N$ și
$N-1$ să fie cuplați și reluăm reconstituirea pentru fanii $1, 2, \dots,
N-2$. În caz contrar, înseamnă că este mai avantajos ca fanul $N$ să rămână
singur și reluăm reconstituirea pentru fanii $1, 2, \dots, N-1$. De exemplu,
iată modul în care se completează vectorii $T_1$ și $T_2$ pentru exemplul din
enunț:

\centeredTikzFigure[
  scale = 0.75,
  mat/.style = {
    matrix of nodes,
    ampersand replacement=\&,
    nodes = {
      scale = 0.75,
      draw,
      rectangle,
      anchor=center,
      font=\Large,
      minimum width=3.2em,
      minimum height=3.2em,
    },
  },
  double/.style = {
    nodes={
      draw = none,
      font = \tiny,
    },
  },
]{
  \matrix[mat] at (0,0) {
    5 \& 4 \& 3 \& 2 \& 1 \& 4 \& 4 \\
  };

  \matrix[mat] at (0,-2) {
    7 \& 3 \& 4 \& 2 \& 2 \& 4 \\
  };

  \matrix[mat,double] at  (0, -3.6) {
    \ \& 4 + 5 = \& 3 + 7 = \& 2 + 8 = \& 1 + 10 = \& 4 + 10 = \& 4 + 12 = \\
  };
  \matrix[mat] at (0,-4) {
    5 \& 9 \& 10 \& 10 \& 11 \& 14 \& 16 \\
  };

  \matrix[mat,double] at  (0, -5.6) {
    \ \& \ \& 3 + 5 = \& 4 + 7 = \& 2 + 8 = \& 2 + 10 = \& 4 + 10 = \\
  };
  \matrix[mat] at (0,-6) {
    $\infty$ \& 7 \& 8 \& 11 \& 10 \& 12 \& 14 \\
  };

  \node at (-6,0) {\Large $T$};
  \node at (-6,-2) {\Large $R$};
  \node at (-6,-4) {\Large $T_1$};
  \node at (-6,-6) {\Large $T_2$};
}

\begin{itemize}

\item $T_2[7]<T_1[7] \implies$ Fanii 6 și 7 sunt cuplați. Reconstituim
  așezarea primilor 5 fani.

\item $T_2[5]<T_1[5] \implies$ Fanii 4 și 5 sunt cuplați. Reconstituim
  așezarea primilor 3 fani.

\item $T_2[3]<T_1[3] \implies$ Fanii 2 și 3 sunt cuplați, iar primul fan este
  singur.

\end{itemize}

\inputminted{c}{src/problem12.c}

  \section{Problema 13}

Probabil că orice elev care are cât de cât experiență în programare a auzit
despre {\bf problema celui mai lung subșir crescător}. Când este prezentată la
concurs, problema e „învelită” sub diverse forme. Aici o vom formaliza din
punct de vedere matematic.

{\bf ENUNȚ}: Se dă un vector cu $N$ elemente numere întregi. Se cere să se
determine cel mai lung subșir crescător.

{\bf Intrarea}: Fișierul de tip text {\tt INPUT.TXT} conține pe prima linie
numărul $N$ de elemente din vector ($N \leq 10.000$), iar pe fiecare din
următoarele $N$ linii se află câte un element al vectorului.

{\bf Ieșirea}: În fișierul de tip text {\tt OUTPUT.TXT} se vor lista pe prima
linie lungimea $L$ a celui mai lung subșir crescător, iar pe următoarele $L$
linii subșirul în sine. Dacă există mai multe soluții, se va tipări una
singură.

{\bf Exemplu}:

\texttt{
  \begin{tabular}{|l|l|}
    \hline
        {\bf INPUT.TXT} & {\bf OUTPUT.TXT} \\ \hline
        6 & 3 \\
        2 & 2 \\
        5 & 3 \\
        7 & 4 \\
        3 &   \\
        4 &   \\
        1 &   \\
        \hline
  \end{tabular}
}

{\bf Timp de implementare}: 45 minute.

{\bf Timp de rulare}: 5 secunde.

{\bf Complexitate cerută}: $O(N \log N)$.

{\bf REZOLVARE}: Începem prin a lămuri diferența dintre noțiunile de „subșir”
și „sub\-sec\-ven\-ță”. Fie $V[1], V[2], \dots, V[N]$ vectorul citit. Prin
{\bf subșir de lungime $L$} al vectorului $V$ se înțelege o succesiune nu
neapărat continuă de elemente $V[K_1], V[K_2], \dots, V[K_L]$, unde $K_1 < K_2
< \dots < K_L$. Prin {\bf subsecvență de lungime $L$} a vectorului, începând
de la poziția $K$, se înțelege succesiunea continuă de elemente $V[K], V[K+1],
\dots, V[K+L-1]$.

Rezolvarea prin metoda programării dinamice este în general cunoscută și nu
vom insista asupra ei. Probabil că aflarea celui mai lung subșir crescător
este punctul de plecare al oricărui elev în învățarea programării
dinamice. Totuși, această metodă de rezolvare are complexitatea $O(N^2)$. În
continuare prezentăm o metodă mai puțin cunoscută și ceva mai dificil de
implementat, dar mult mai eficientă. Ea are complexitatea $O(N \log N)$. Vom
enunța principiul de rezolvare, urmat de o schiță a demonstrației de
corectitudine.

Fie $V$ vectorul citit. Se parcurge vectorul de la stânga la dreapta și se
construiesc în paralel doi vectori $P$ și $Q$, astfel: inițial vectorul $Q$
este vid. Se ia fiecare element din $V$ și se suprascrie peste cel mai mic
element din $Q$ care este strict mai mare ca el. Dacă nu există un asemenea
element în $Q$, cu alte cuvinte dacă elementul analizat din $V$ este mai mare
ca toate elementele din $Q$, atunci el este adăugat la sfârșitul vectorului
$Q$. Concomitent, se notează în vectorul $P$ poziția pe care a fost adăugat în
vectorul $Q$ elementul din $V$. Iată cum se construiesc vectorii $P$ și $Q$
pentru exemplul din enunț:

\centeredTikzFigure[
  mat/.style = {
    matrix of nodes,
    ampersand replacement=\&,
    nodes = {
      draw,
      rectangle,
      minimum width=2em,
      minimum height=2em,
    },
    row 1/.style = {
      nodes = {
        draw = none,
        font=\bf,
      },
    }
  },
  emph/.style = {
    fill=gray!40,
  },
]{
  \matrix[mat] {
    V \&[2em] Q \&   \&   \&[2em] P \&   \&   \&   \&   \&   \\[1em]
    2 \& \node[emph] {2}; \&   \&   \& 1 \&   \&   \&   \&   \&   \\[1em]
    5 \& 2 \& \node[emph] {5}; \&   \& 1 \& 2 \&   \&   \&   \&   \\[1em]
    7 \& 2 \& 5 \& \node[emph] {7}; \& 1 \& 2 \& 3 \&   \&   \&   \\[1em]
    3 \& 2 \& \node[emph] {3}; \& 7 \& 1 \& 2 \& 3 \& 2 \&   \&   \\[1em]
    4 \& 2 \& 3 \& \node[emph] {4}; \& 1 \& 2 \& 3 \& 2 \& 3 \&   \\[1em]

    1 \& \node[emph] {1}; \& 3 \& 4 \&
    \node[emph] {1}; \& 2 \& 3 \& \node[emph] {2}; \& \node[emph] {3}; \& 1 \\
  };
}

Lungimea $L$ la care ajunge vectorul $Q$ la sfârșitul acestei prelucrări este
tocmai lungimea celui mai lung subșir crescător al vectorului $V$. Pentru a
afla exact și care sunt elementele subșirului crescător se procedează astfel:
se caută ultima apariție în vectorul $P$ a valorii $L$. Să spunem că ea este
găsită pe poziția $K_L$. Se caută apoi ultima apariție în vectorul $P$ a
valorii $L-1$, anterior poziției $K_L$. Ea va fi pe poziția $K_{L-1} <
K_{L}$. Analog se caută în vectorul $P$ valorile $L-2, L-3, \dots, 2,
1$. Subșirul crescător este $S=(V[K_1], V[K2], \dots, V[K_L])$.

În figura de mai sus au fost hașurate elementele respective găsite în vectorul
$P$. Vectorul $Q$ are la sfârșit lungimea $L=3$, deci cel mai lung subșir
crescător are trei elemente. Se caută în vectorul $P$ cifrele 3, 2 și 1 și se
găsesc pe pozițiile $K_1=1$, $K_2=4$ și $K_3=5$, ceea ce înseamnă că cel mai
lung subșir crescător este (2, 3, 4).

Demonstrația (care, fără a pierde din corectitudine, face apel la intuiție...)
folosește aceleași notații de mai sus și are următoarele etape:

\begin{proposition}
  Vectorul $Q$ este în permanență sortat crescător.
\end{proposition}

\begin{proof}
  Folosim inducția matematică. După primul pas (inserarea elementului $V[1]$),
  vectorul $Q$ are un singur element și este bineînțeles ordonat. Trebuie acum
  arătat că dacă vectorul Q este sortat după inserarea elementului $V[i-1]$,
  el rămâne sortat și după inserarea lui $V[i]$. Într-adevăr, pentru elementul
  $V[i]$ există două variante:

  \begin{enumerate}[label=\alph*)]

  \item $V[i]$ este mai mare decât toate elementele lui $Q$, caz în care este
    adăugat la sfârșitul lui $Q$, care rămâne sortat;

  \item Există $t>1$ astfel încât $Q[t-1] \leq V[i] <Q[t]$, caz în care $V[i]$
    se suprascrie peste $Q[t]$. Dar $Q[t] \leq Q[t+1] \implies Q[t-1] \leq
    V[i] < Q[t+1]$ și vectorul $Q$ rămâne sortat.

  \end{enumerate}
\end{proof}

Această deducție ne va fi utilă în calculul complexității algoritmului.

\begin{proposition}
  Odată ce au fost scrise pentru prima oară, elementele vectorului $Q$ nu mai
  pot decât să scadă sau să rămână constante. Ele nu pot crește niciodată.
\end{proposition}

\begin{proof}
  Afirmația este evidentă, decurgând din modul de construcție a lui $Q$.
\end{proof}

\begin{proposition}
  Elementele din vectorul $V$ de la poziția $K_{i-1}+1$ la poziția $K_{i}-1$
  sunt fie mai mici decât $V[K_{i-1}]$, fie mai mari decât $V[K_i]$.
\end{proposition}

\begin{proof}
  Acest lucru este natural, deoarece dacă ar exista $K_{i-1} < X <K_i$ astfel
  încât $V[K_{i-1}] \leq V[X] \leq V[K_i]$, atunci șirul $S$ ar putea fi
  extins cu elementul $V[X]$, deci nu ar fi subșir maximal, contradicție.
\end{proof}

\begin{proposition}
  Toate elementele care urmează în $V$ după $V[K_L]$ sunt mai mici decât
  $V[K_L]$.
\end{proposition}

\begin{proof}
  Dacă ar exista $X>K_L$ astfel încât $V[X] \geq V[K_L]$, atunci $S$ ar putea fi
  extins la dreapta cu $V[X]$, contradicție.
\end{proof}

\begin{proposition}
  Elementul $V[K_1]$ este suprascris peste poziția $Q[1]$.
\end{proposition}

\begin{proof}
  Dacă elementul $V[K_1]$ este inserat în $Q$ pe o poziție $t>1$, rezultă că,
  în momentul tratării lui $V[K_1]$, pe poziția $t-1$ în vectorul $Q$ exista
  deja un număr $X < V[K_1]$. Aceasta înseamnă că $S$ poate fi prelungit la
  stânga cu $X$, deci $S$ nu este un subșir crescător maxim, contradicție.
\end{proof}

\begin{proposition}
  Orice element $V[K_i]$ va fi suprascris în $Q$ pe poziția următoare celei pe
  care a fost scris $V[K_{i-1}]$
\end{proposition}

\begin{proof}
  Să presupunem că $V[K_{i-1}]$ a fost scris peste $Q[t]$. Înainte de a ajunge
  să tratăm elementul $V[K_i]$, a trebuit să tratăm elementele $V[K_{i-1}+1],
  V[K_{i-1}+2], \dots, V[K_{i}-1]$, care, după cum s-a stabilit la punctul
  (3), pot fi:

  \begin{enumerate}[label=\alph*)]

  \item mai mici decât $V[K_{i-1}]$, caz în care ele vor fi scrise în vector
    pe poziții mai mici sau egale cu $t$, iar $Q[t]$ va scădea, deci $Q[t]
    \leq V[K_{i-1}]$ (conform punctului 2);

  \item mai mari decât $V[K_i]$, caz în care vor fi scrise în $Q$ pe poziții
    mai mari decât $t$, așadar $Q[t+1]>V[K_i]$.
  \end{enumerate}

  Se obține lanțul de inegalități $Q[t] \leq V[K_{i-1}] \leq V[K_i] \leq
  Q[t+1]$, de unde rezultă conform modului de construcție că $V[K_i]$ va fi
  scris peste $Q[t+1]$. Cu aceasta, folosind și punctul (5), am demonstrat că
  $V[K_i]$ este scris pe poziția $Q[i], \forall i=1, 2, \dots, L$.

\end{proof}

După tratarea primelor $K_L$ elemente din $V$, vectorul $Q$ are lungimea
$L$. Mai rămâne de văzut ce se întâmplă cu elementele $V[K_L+1], \dots, V[N]$.

\begin{proposition}
  Elementele care îi urmează lui $V[K_L]$ se scriu în $Q$ pe poziții mai mici
  sau egale cu $L$.
\end{proposition}

\begin{proof}
  Acest fapt este intuitiv dacă se ține cont de punctul (4). Deducem că la
  final vectorul $Q$ are lungime $L$, adică aceeași cu a lui $S$.
\end{proof}

Modul de reconstituire a lui $S$ din vectorul $P$ este corect. Trebuie să avem
grijă ca, atunci când căutăm în vectorul $P$ o apariție a valorii $X$, să o
alegem pe ultima disponibilă, altfel pot apărea erori. Spre exemplu, pentru
vectorii dați în exemplu, $V=(2, 5, 7, 3, 4, 1)$ și $P=(1, 2, 3, 2, 3, 1)$,
dacă se alege prima apariție a lui 2 (pe poziția a doua), avem subșirul $S=(2,
5, 4)$ care nu este crescător. Dacă însă alegem ultima apariție a lui 2 (pe
poziția a patra), avem subșirul crescător maximal $S=(2, 3, 4)$.

Calculul complexității este ușor de făcut:

\begin{itemize}

\item Pentru construcția vectorilor $P$ și $Q$ sunt necesare $N$ inserții în
  vectorul $Q$, care este sortat; o inserție binară cere $O(\log N)$, așadar
  complexitatea primei părți este $O(N \log N)$.

\item Pentru reconstituirea lui $S$ se face o singură parcurgere a vectorului
  $P$, deci $O(N)$.

\end{itemize}

Complexitatea totală a algoritmului este $O(N \log N)$.

\begin{minted}{c}
#include <stdio.h>
#include <stdlib.h>
#define Infinity 30000
typedef int Vector[10001];

Vector V,P,Q,*S;
int N,Len; /* Len = Lungimea vectorului Q */

void ReadData(void)
{ FILE *F=fopen("input.txt","rt");
  int i;

  fscanf(F,"%d",&N);
  for(i=1;i<=N;i++) fscanf(F,"%d",&V[i]);
  fclose(F);
}

int Insert(int K,int Lo,int Hi)
{ int Mid=(Lo+Hi)/2;

  if (Lo==Hi)
    { if (Hi>Len) Q[++Len+1]=Infinity;
      Q[Lo]=K;
      return Lo;
    }
    else if (K<Q[Mid]) return Insert(K,Lo,Mid);
                     else return Insert(K,Mid+1,Hi);
}

void BuildPQ(void)
{ int i,Place;

  Len=0;Q[1]=Infinity;
  for (i=1;i<=N;i++)
    P[i]=Insert(V[i],1,Len+1);
}

void BuildS(void)
{ int i,K=N;

  S=malloc(sizeof(*S));
  for (i=Len;i;i--)
    { while (P[K]!=i) K--;
      (*S)[i]=V[K];
    }
}

void WriteSolution(void)
{ FILE *F=fopen("output.txt","wt");
  int i;

  fprintf(F,"%d\n",Len);
  for(i=1;i<=Len;i++) fprintf(F,"%d\n",(*S)[i]);

  fclose(F);
}

void main(void)
{
  ReadData();
  BuildPQ();
  BuildS();
  WriteSolution();
}
\end{minted}

Menționăm că în sursa de mai sus se putea construi vectorul $S$ peste vectorul
$Q$, deoarece pentru construirea lui $S$ nu avem nevoie decât de elementele
vectorului $P$. În acest fel, programul ar fi avut nevoie numai de trei
vectori, iar volumul total de date nu ar fi depășit un segment. Am preferat
totuși varianta în care vectorul $S$ este alocat dinamic pentru a evita
confuziile.

  \section{Problema 14}

Continuăm cu două probleme foarte asemănătoare. Atât de asemănătoare, încât
diferența dintre ele pare - la o privire superficială - neglijabilă. Totuși,
algoritmul de rezolvare se schimbă fundamental.

{\bf ENUNȚ}: $N$ grămezi de mere trebuie împărțite la $N$ copii. Deoarece
copiii sunt buni prieteni, trebuie ca împărțirea să se facă în mod echitabil,
fiecare primind același număr de mere. Spiritul de dreptate al copiilor este
atât de puternic, încât ei preferă ca unele mere să nu fie date nici unui
copil, decât ca unii să primească mai multe mere ca alții. O condiție
suplimentară este ca fie toate merele dintr-o grămadă să fie împărțite
copiilor, fie grămada să nu mai fie împărțită deloc. Desigur, interesul este
ca fiecare copil să primească un număr cât mai mare de mere.

Să se selecteze un număr de grămezi din cele $N$ astfel încât numărul total de
mere să se dividă cu $N$, iar suma selectată să fie maximă. Dacă există mai
multe soluții, se cere una singură.

{\bf Intrarea}: Fișierul de intrare {\tt INPUT.TXT} conține pe prima linie
numărul $N$ de grămezi ($1 \leq N \leq 100$). Pe a doua linie se dau
cantitățile de mere din cele $N$ grămezi ($N$ numere naturale pozitive, toate
mai mici ca 200, separate prin spații).

{\bf Ieșirea}: În fișierul {\tt OUTPUT.TXT} se va scrie pe prima linie numărul
maxim de mere găsit. Pe a doua linie se vor tipări indicii grămezilor
selectate, în ordine crescătoare.

{\bf Exemplu}:

\texttt{
  \begin{tabular}{|l|l|}
    \hline
        {\bf INPUT.TXT} & {\bf OUTPUT.TXT} \\ \hline
        4 & 12 \\
        3 2 5 7 & 1 2 4 \\
        \hline
  \end{tabular}
}

{\bf Timp de implementare}: 45 minute - maxim o oră.

{\bf Timp de rulare}: 1 secundă.

{\bf Complexitate cerută}: $O(N^2)$.

{\bf REZOLVARE}: O primă modalitate, de altfel foarte comodă, este să
verificăm toate posibilitățile de a selecta grămezi. Aceasta presupune să
generăm toate submulțimile mulțimii de grămezi, iar pentru fiecare submulțime
să calculăm suma merelor. În felul acesta putem afla submulțimea pentru care
suma merelor se divide cu $N$ și este maximă. Din nefericire, această soluție,
banal de implementat, are o complexitate exponențială, mai precis $O(N \times
2^N)$, deoarece există $2^N$ submulțimi ale mulțimii grămezilor și pentru
fiecare submulțime putem calcula suma merelor în timp liniar. Prin urmare,
suntem departe de complexitatea cerută în enunț.

Punctul de plecare pentru rezolvarea corectă a problemei este din nou
principiul de optimalitate al programării dinamice. Să notăm cu $M[1], M[2],
\dots, M[N]$ cantitățile de mere din fiecare grămadă. Să considerăm că
submulțimea optimă conține în total $S$ mere, iar grămada cu numărul $N$ face
parte din ea. Fie $K$ restul împărțirii lui $M[N]$ la $N$, deci

\begin{equation}
  M[N] \equiv K \pmod{N}
\end{equation}

Atunci suma $S - M[N]$ dă restul $(N-K) \bmod N$ la împărțirea prin $N$, de
unde deducem că

\begin{equation}
  S - M[N] \equiv N - K \pmod{N}
\end{equation}

De asemenea, putem afirma că $S - M[N]$ este cea mai mare dintre toate sumele
care se pot obține folosind grămezile $1, 2, \dots, N-1$ și care dau același
rest la împărțirea prin $N$. Demonstrația nu este grea: dacă ar exista o sumă
mai mare decât $S - M[N]$ congruentă cu $N - K$ modulo $N$, am putea folosi
această sumă și grămada $M[N]$ pentru a obține o sumă $S'>S$ astfel încât

\begin{equation}
  S' \equiv 0 \pmod{N}
\end{equation}

Această observație ne sugerează și metoda de rezolvare a problemei. Pentru a
afla care este submulțimea de sumă maximă, avem două variante:

\begin{itemize}

\item Grămada $N$ face parte din această submulțime, caz în care trebuie să
  descoperim cea mai mare sumă care se poate obține adunând grămezi dintre
  primele $N-1$ și care dă restul $N - K$ la împărțirea prin $N$;

\item Grămada $N$ nu face parte din această submulțime, caz în care trebuie să
  descoperim cea mai mare sumă care se poate obține adunând grămezi dintre
  primele $N-1$ și care se împarte exact la $N$.

\end{itemize}

Pentru că nu putem ști de la început dacă grămada a $N$-a face sau nu parte
din submulțimea maximală, trebuie să avem răspunsul pregătit pentru ambele
situații. Mai mult, pentru a putea afla care sunt sumele maxime formate cu
primele $N-1$ grămezi care dau diferite resturi (în cazul nostru 0 sau $N -
K$) la împărțirea prin $N$, trebuie să reluăm exact aceeași problemă: grămada
cu numărul $N-1$ poate face sau nu parte din submulțime. În concluzie, putem
formaliza problema astfel: avem nevoie să putem răspunde la toate întrebările
de forma „Care este suma maximă care se poate forma cu grămezi din primele $P$
astfel încât restul la împărțirea prin $N$ să fie $Q$?”. Vom nota răspunsul la
această întrebare cu $R[P,Q]$ (unde $1 \leq P \leq N$ și $0 \leq Q <
N$). Răspunsurile tuturor întrebărilor se pot deci dispune într-o matrice $R$,
iar scopul nostru este să-l aflăm pe $R[N,0]$. Am observat că pentru a-l putea
afla pe $R[P,Q]$ avem nevoie de două valori din linia $P$-1 a matricei (după
cum grămada $P$ face sau nu parte din submulțimea optimă). Pentru a afla toate
elementele liniei $P$ a matricei, este deci foarte probabil să avem nevoie de
întreaga linie $P-1$.

Astfel, problema se reduce la a compune o linie a matricei din cea
precedentă. Dacă presupunem că grămada $P$ nu intră în componența submulțimii
optime, atunci linia $P$ este identică cu linia $P-1$:

\begin{equation}
  R[P,Q] = R[P - 1, Q], \quad \forall 0 \leq Q < N
\end{equation}

Dacă presupunem că grămada $P$ face parte din submulțime, atunci avem
egalitatea:

\begin{equation}
  R[P,Q] = R[P - 1, (Q - M[P]) \bmod N] + M[P], \quad \forall 0 \leq Q < N
\end{equation}

Alegând dintre aceste variante pe cea care ne convine mai mult, obținem:

\begin{equation}
  R[P,Q] =  \max
  \begin{Bmatrix}
    R[P - 1, Q] \\
    R[P - 1, (Q - M[P]) \bmod N] + M[P]
  \end{Bmatrix}
  \quad \forall 0 \leq Q < N
\end{equation}

În felul acesta se completează fără nici un fel de probleme matricea $R$. După
cum am spus, $R[N,0]$ indică suma maximă divizibilă cu $N$. Mai rămâne de
văzut cum se face reconstituirea soluției. De fapt, nu avem decât să parcurgem
aceleași etape ale raționamentului, dar în sens invers. Respectiv: dacă
$R[N,0] = R[N-1,0]$, atunci deducem că grămada a $N$-a nu a fost folosită
pentru a se obține suma maximă și avem nevoie să obținem suma maximă
divizibilă cu $N$ din primele $N-1$ grămezi (adică $R[N-1,0]$). Dacă $R[N,0]
\neq R[N-1,0]$, atunci grămada a $N$-a a fost folosită și avem nevoie să
obținem suma maximă de rest $N-M[N]$ modulo $P$ folosind primele $N-1$
grămezi. Pe cazul general, pentru a obține suma maximă de rest $Q$ folosind
primele $P$ grămezi, avem două posibilități:

\begin{itemize}

\item Dacă $R[P,Q] = R[P-1,Q]$, atunci grămada $P$ nu este folosită și trebuie
  să obținem același rest $Q$ folosind doar primele $P-1$ grămezi;

\item Dacă $R[P,Q] \neq R[P-1,Q]$, atunci grămada $P$ este folosită și trebuie
  să obținem restul $Q- M[P]$ modulo $N$ folosind primele $P-1$ grămezi.

\end{itemize}

Menționăm că programul este puțin diferit de ceea ce s-a explicat până acum,
în sensul că prima linie din matricea $R$ corespunde ultimei grămezi de mere,
a doua linie corespunde penultimei grămezi de mere ș.a.m.d. Cu alte cuvinte,
grămezile de mere sunt procesate în ordine inversă. Am făcut acest lucru
pentru a ușura procedura de aflare a soluției; se observă că prima oară se
decide dacă ultima grămadă face parte din submulțime, apoi penultima etc. Deci
și la găsirea soluției se generează grămezile în ordine inversă. Prin această
dublă inversiune, indicii grămezilor de mere selectate vor fi listați în
ordine crescătoare. Dacă am fi prelucrat grămezile de mere în ordinea lor din
fișier, s-ar fi impus scrierea unei proceduri recursive pentru afișarea
soluției.

Să facem și calculul complexității acestui program. Citirea datelor se face în
$O(N)$, la fel și reconstituirea soluției. Pentru a completa o linie din
matrice, avem nevoie să parcurgem linia precedentă, adică $O(N)$. Pentru
compunerea întregii matrice, timpul necesar este pătratic.

Mai trebuie făcută o singură observație referitoare la modul de inițializare a
matricei. Vom adăuga în matricea $R$ linia cu numărul 0, care va conține
sumele maxime ce se pot obține fără a folosi nici o grămadă. Desigur, se poate
obține numai suma 0, care dă restul 0 la împărțirea prin $N$, iar alte sume nu
se pot obține. Vom pune deci $R[0,0]=0$ și $R[0,i]=\mathbf{Impossible}$ pentru
orice $1 \leq i < N$, unde $\mathbf{Impossible}$ este o constantă specială
(preferabil negativă, pentru a nu se confunda cu valorile obișnuite din
matrice).

\begin{lstlisting}[language=C]
#include <stdio.h>
#define NMax 101
#define Impossible -30000

int M[NMax], R[NMax][NMax], N;
void ReadData(void)
{ FILE *F = fopen("input.txt", "rt");
  int i;

  fscanf(F, "%d\n", &N);
  for (i=1; i<=N; fscanf(F, "%d", &M[i++]));
  fclose(F);
}

void Share(void)
{ int i,j;

  R[0][0]=0;
  for (i=1; i<N; R[0][i++]=Impossible);

  for (i=1; i<=N; i++)
    {
      for (j=0; j<N; j++)
        R[i][j] = R[i-1][j];
      for (j=0; j<N; j++)
        if (R[i-1][j] != Impossible
           && R[i-1][j] + M[N+1-i] > 
              R[i][ (R[i-1][j]+M[N+1-i]) % N ])
         R[i][ (R[i-1][j]+M[N+1-i]) % N ] =
         R[i-1][j] + M[N+1-i];
    }
}

void WriteSolution(void)
{ FILE *F = fopen("output.txt", "wt");
  int i,j;

  fprintf(F, "%d\n", R[N][0]);
  j=0;
  for (i=N; i; i--)
    if (R[i][j] != R[i-1][j])
      {
        fprintf(F, "%d ", N+1-i);
        j = (j + N - M[N+1-i]%N) % N;
      }
  fprintf(F, "\n");
  fclose(F);
}

void main(void)
{
  ReadData();
  Share();
  WriteSolution();
}
\end{lstlisting}

  \section{Problema 15}

Problema următoare a fost propusă la proba de baraj de la Olimpiada Națională
de Informatică, Slatina 1995 și este un exemplu tipic de aplicare a
principiului lui Dirichlet.

{\bf ENUNȚ}: La un SHOP din Slatina se găsesc spre vînzare $P-1$ ($P$ este un
număr prim) produse unicat de costuri $X(1), X(2), \dots, X(P-1)$. Nici unul
din produse nu poate fi cumpărat prin plata exactă cu bancnote de $P$\$. În
SHOP intră un olimpic care are un număr nelimitat de bancnote de $P$\$ și o
singură bancnotă de $Q$\$ ($1 \leq Q \leq P-1$). Ce produse trebuie să cumpere
olimpicul pentru a putea plăti exact produsele cumpărate?

{\bf Intrarea}: Datele de intrare se dau în fișierul de intrare {\tt
  INPUT.TXT} ce conține două linii:

\begin{itemize}

\item pe prima linie valorile lui $P$ și $Q$;

\item pe a doua linie valorile costurilor produselor.

\end{itemize}

Ieșirea se va face în fișierul {\tt OUTPUT.TXT} unde se vor lista în ordine
crescătoare indicii produselor cumpărate de olimpic.

\texttt{
  \begin{tabular}{|l|l|}
    \hline
        {\bf INPUT.TXT} & {\bf OUTPUT.TXT} \\ \hline
        5 4 & 1 2 \\
        1 3 6 7 & \\
        \hline
  \end{tabular}
}

{\bf Timp de implementare}: la Slatina s-au acordat cam 90 de minute, dar 45
ar trebui să fie suficiente.

{\bf Timp de rulare pentru fiecare test}: 1 sec.

{\bf Complexitate cerută}: $O(P)$.

Enunțul original nu specifica nici o limită maximă pentru valoarea lui P. Vom
adăuga noi această limită, respectiv $P < 10.000$.

{\bf REZOLVARE}: Problema se reduce la a găsi un grup de obiecte pentru care
suma costurilor să fie divizibilă cu $P$ sau să fie congruentă cu $Q$ modulo
$P$. Am văzut deja în problema precedentă că dispunem de o soluție $O(N^2)$
pentru a găsi un număr de elemente care să se dividă cu $P$. În cazul nostru,
trebuie să observăm însă că nu avem $P$ obiecte, ci numai $P-1$; în schimb,
dispunem de o bancnotă suplimentară de valoare $Q$. Aceste diferențe vor fi
explicate mai târziu și se va vedea că ele nu schimbă cu nimic natura
problemei. Diferența esențială provine din faptul că nu se mai cere ca suma
numerelor să fie maximă, ca în problema precedentă. Orice combinație de numere
care dau o sumă potrivită este suficientă.

Să începem prin a explica principiul lui Dirichlet, care de altfel face apel
numai la intuiție și nu necesită cunoștințe speciale de matematică. Acest
principiu spune că dacă distribuim $N$ obiecte în $K$ cutii, atunci cel puțin
într-o cutie se vor afla minim $\lceil N/K \rceil$ obiecte (aici prin $\lceil
N/K \rceil$ se înțelege „cel mai mic întreg mai mare sau egal cu
$N/K$”). Demonstrația se face prin reducere la absurd: dacă în fiecare cutie
s-ar afla mai puțin decât $\lceil N/K \rceil$ obiecte, atunci numărul total de
obiecte ar fi mai mic decât $K \times \lceil N/K \rceil$, adică mai mic decât
$N$.

Spre exemplu, oricum am distribui 7 obiecte în 4 cutii, putem fi siguri că cel
puțin într-o cutie se vor afla minim $\lceil 7/4 \rceil = 2$
obiecte. Într-adevăr, dacă toate cutiile ar conține cel mult câte un obiect,
atunci numărul total de obiecte nu ar putea fi mai mare ca 4, ceea ce este
absurd.

Să vedem acum cum s-ar putea aplica acest principiu la problema de față. Să
facem notația

\begin{equation}
  S(k) = \sum_{i = 1}^{k} X(i)
\end{equation}

și convenția $S(0) = 0$. Prin urmare, putem scrie egalitatea

\begin{equation}
  S(k_2) - S(k_1) = \sum_{i = k_1 + 1}^{k_2} X(i)
\end{equation}

Dacă găsim în vectorul $S$ două valori $S(k_1)$ și $S(k_2)$ care dau același
rest la împărțirea prin $P$, înseamnă că diferența lor se divide cu $P$, deci
șirul de obiecte $k_1+1, k_1+2, \dots, k_2$ poate constitui o soluție.

Să presupunem pentru început că dispunem de $P$ obiecte. Se pune întrebarea:
ce valori poate lua restul împărțirii lui $S(k)$ prin $P$? Desigur, orice
valoare între 0 și $P-1$. Există deci în total $P$ resturi distincte. Pe de
altă parte, există $P+1$ elemente în vectorul $S$ (se consideră și elementul
$S(0)$). Începem să recunoaștem aici principiul lui Dirichlet, în care
„obiectele” sunt resturile $S(0) \bmod P, S(1) \bmod P, \dots, S(P) \bmod P$,
iar cutiile sunt clasele de resturi modulo $P$. Avem de distribuit P+1 obiecte
în P cutii, așadar cel puțin într-o cutie se vor afla $\lceil (P+1)/P \rceil =
2$ obiecte. Prin urmare, vor exista cu siguranță doi indici $k_1 < k_2$ astfel
încât $S(k_2) - S(k_1) \equiv 0 \pmod{P}$. Nu avem decât să tipărim secvența
$k_1+1, k_1+2, \dots, k_2$.

Să vedem acum ce se întâmplă dacă avem numai $P-1$ obiecte, așa cum este cazul
problemei. Atunci avem numai $P$ resturi posibile, deci se poate ca toate
elementele din $S$ să dea resturi distincte la împărțirea prin $P$. Dar în
acest caz, există un indice $k$ astfel încât $S(k) \equiv Q \pmod{N}$, deci
trebuie doar să tipărim secvența de indici $1, 2, \dots, k$.

Pentru a reuni aceste două cazuri într-unul singur, putem considera expresia
$S(k)$ drept un alt mod de a scrie expresia $S(k) - S(0)$. Problema se reduce
la a căuta doi indici $k_1, k_2 \in \{0, 1, 2, \dots, P\}$ astfel încât
$(S(k_2) - S(k_1)) \bmod N \in [0,Q]$. Să nu uităm că trebuie să efectuăm
această operație într-un timp liniar, deci nu avem voie să comparăm pur și
simplu două câte două elementele vectorului $S$. Vom prezenta modul în care se
pot găsi două elemente congruente modulo $P$, cazul celălalt tratându-se
analog. Metoda constă în crearea unui alt vector, $L$, în care $L(i) = j$
înseamnă că suma $S(j)$ dă restul $i$ la împărțirea prin $P$. Inițial, toate
elementele vectorului $L$ vor avea o valoare specială, eventual negativă. Apoi
se parcurge vectorul $S$ și pentru fiecare $S(j)$ se efectuează operația
$L(S(j) \bmod P) \leftarrow j$. În momentul în care se încearcă reatribuirea
unui element din $L$ care are deja o valoare dată, înseamnă că am găsit cei
doi indici pe care îi căutam.

Iată un exemplu. Dacă $P=7$ și $X=(8, 8, 2, 6, 13, 3)$, rezultă vectorul
$S=(8, 16, 18, 24, 37, 40)$. Resturile la împărțirea prin 7 sunt respectiv 1,
2, 4, 3, 2 și 5.

\centeredTikzFigure[
  mat/.style = {
    matrix of nodes,
    ampersand replacement=\&,
    nodes = {
      draw,
      rectangle,
      anchor=center,
      minimum width=2em,
      minimum height=2em,
    },
    row 1/.style = noborder,
    row 2/.style = noborder,
    column 1/.style = noborder,
  },
  noborder/.style = {
    nodes = {
      draw = none,
      font=\bf,
    },
  },
]{
  \matrix[mat] (m) {
    R \&[2em] \&   \&   \& L \&   \&   \&   \\[1em]
    \ \& 0 \& 1 \& 2 \& 3 \& 4 \& 5 \& 6 \\
    1 \& 0 \& 1 \& -1 \& -1 \& -1 \& -1 \& -1 \\[1em]
    2 \& 0 \& 1 \&  2 \& -1 \& -1 \& -1 \& -1 \\[1em]
    4 \& 0 \& 1 \&  2 \& -1 \&  3 \& -1 \& -1 \\[1em]
    3 \& 0 \& 1 \&  2 \&  4 \&  3 \& -1 \& -1 \\[1em]
    2 \\
  };

  \draw[<->] (m-7-1.east) -| (m-6-4.south);
}

După cum se vede, restul 2 poate fi obținut atât cu primele două obiecte, cât
și cu primele 5, deci suma prețurilor obiectelor 3, 4 și 5 este divizibilă cu
7.

Un ultim detaliu de implementare constă în aceea că nu este necesară memorarea
vectorului $S$, ci numai a elementului curent; orice alte informații care ne
trebuie la un moment dat le putem afla din vectorii $X$ și $L$. Pentru
memorarea elementului curent din $S$, se pornește cu valoarea 0 și la fiecare
pas se adaugă valoarea elementului corespunzător din $X$.

\begin{lstlisting}[language=C]
#include <stdio.h>
#define NMax 10000
#define None -1

int X[NMax], L[NMax], P, Q;

void ReadData(void)
{ FILE *F = fopen("input.txt", "rt");
  int i;

  fscanf(F, "%d %d\n", &P, &Q);
  for (i=1; i<P; fscanf(F, "%d", &X[i++]));
  fclose(F);
}

void FindSum(void)
{ long S=0;
  FILE *F = fopen("output.txt", "wt");
  int i,j;

  for (i=1, L[0]=0; i<P; L[i++] = None);
  i=0;
  while ( L[ (S+=X[++i]) % P ]==None && // Restul 0
          L[ (S%P+P-Q) % P ]==None )    // Restul Q
    L[S%P]=i;
  for (j = 1 + ((L[S%P]!=None)? L[S%P]: L[ (S%P+P-Q) % P ]);
       j <= i; fprintf(F, "%d ", j++));
  fclose(F);
}

void main(void)
{
  ReadData();
  FindSum();
}
\end{lstlisting}

  \section{Problema 16}

Următoarele probleme aparțin categoriei de probleme pe care, dacă ne grăbim,
le putem clasifica drept „ușoare”. Într-adevăr, ele au soluții vizibile și
foarte la îndemână, dar și soluții mai subtile și mult mai performante. Pentru
a obliga cititorul să se gândească și la aceste soluții, am ales limite pentru
datele de intrare suficient de mari încât să facă nepractice rezolvările „la
minut”.

{\bf ENUNȚ}: (Generarea unui arbore oarecare când i se cunosc gradele) Se dă
un vector cu $N$ numere întregi. Se cere să se construiască un arbore cu $N$
noduri numerotate de la 1 la $N$ astfel încât gradele celor $N$ noduri să fie
exact numerele din vector. Dacă acest lucru nu este posibil, se va da un mesaj
de eroare corespunzător.

{\bf Intrarea}: Datele de intrare se află în fișierul {\tt INPUT.TXT}. Pe
prima linie se află numărul de noduri $N$ ($N \leq 10.000$), iar pe a doua
linie se află cele $N$ numere separate prin spații. Toate numerele sunt strict
pozitive și mai mici ca 10.000.

Ieșirea se va face în fișierul {\tt OUTPUT.TXT}. Dacă problema are soluție, se
va tipări arborele prin muchiile lui. Fiecare muchie se va lista pe câte o
linie, prin nodurile adiacente separate printr-un spațiu. Dacă problema nu are
soluție, se va afișa un mesaj corespunzător.

\texttt{
  \begin{tabular}{|l|l|}
    \hline
        {\bf INPUT.TXT} & {\bf OUTPUT.TXT} \\ \hline
        6 & 1 4 \\
        1 2 3 2 1 1 & 2 5 \\
        & 3 6 \\
        & 4 6 \\
        & 5 6 \\ \hline
        3 & Problema nu are solutie! \\
        2 2 1 & \\
        \hline
  \end{tabular}
}

{\bf Timp de implementare}: 30 minute - 45 minute.

{\bf Timp de rulare}: 2-3 secunde.

{\bf Complexitate cerută}: $O(N \log N)$; dacă vectorul citit la intrare se
presupune sortat, se cere o complexitate $O(N)$.

{\bf REZOLVARE}: Să începem prin a ne pune întrebarea: când are problema
soluție și când nu?

Se știe că un arbore oarecare cu $N$ noduri are $N-1$ muchii. Fiecare din
aceste muchii va contribui cu o unitate la gradele nodurilor
adiacente. Deducem de aici că suma gradelor tuturor nodurilor este egală cu
dublul numărului de muchii, adică, notând cu $G[1], G[2], \dots, G[N]$ gradele
nodurilor,

\begin{equation}
  \sum_{i = 1}^{N} G[i] = 2 \cdot (N - 1)
\end{equation}

Am aflat deci o condiție necesară pentru ca problema să aibă soluție. O a doua
condiție este ca toate nodurile să aibă grade cuprinse între 1 și
$N-1$. Totuși, ținând cont de afirmația enunțului că toate numerele din vector
sunt strict pozitive, rezultă că a doua condiție nu mai trebuie
verificată. Iată de ce: să presupunem că am verificat prima condiție și am
constatat că suma celor $N$ numere este $2(N-1)$, iar unul dintre numere este
cel puțin $N$. Atunci ar rezulta că suma celorlalte $N-1$ numere este cel mult
$N-2$, de unde rezultă că există cel puțin un nod de grad 0, ceea ce
contrazice informația din enunț. Prin urmare, numai prima condiție este
importantă, cea de-a doua fiind redundantă.

Vom demonstra că această condiție este și suficientă indicând efectiv modul de
construcție a arborelui în cazul în care ea este satisfăcută. Începem prin a
sorta vectorul de numere. Acest lucru era oricum de așteptat, deoarece
complexitatea $N \log N$ ne-o permite. Trebuie numai să avem grijă să alegem
un algoritm de sortare de complexitate $N \log N$. Programul care urmează
folosește heapsort-ul. Odată ce am sortat vectorul, trebuie să reconstituim
muchiile în timp liniar, și iată cum:

\begin{itemize}

\item Se poate demonstra că primele două elemente din vectorul sortat au
  valoarea 1. Într-adevăr, dacă toate elementele ar fi mai mari sau egale cu
  2, atunci suma lor ar fi mai mare sau egală cu $2N$, ori noi știm că suma
  trebuie să fie $2N-2$, adică există cel puțin două elemente egale cu 1 în
  vector. Acest lucru rezultă imediat dacă ne gândim că orice arbore are cel
  puțin două frunze.

\item Vom căuta în vector primul număr mai mare sau egal cu 2. Se pune
  întrebarea: există întotdeauna acest număr? Nu cumva există un arbore în
  care toate nodurile au grad 1? Să aplicăm condiția precedentă și să vedem ce
  se întâmplă. Dacă toate nodurile au grad 1, atunci suma gradelor este $N$,
  ceea ce conduce la ecuația:

  \begin{equation}
    N = 2 \cdot (N - 1) \implies N = 2
  \end{equation}

\item Iată deci că există un singur arbore în care toate nodurile sunt frunze,
  anume cel cu 2 noduri unite printr-o muchie. Vom reveni mai târziu la acest
  caz particular. Deocamdată presupunem că există în vector un număr mai mare
  ca 1, pe poziția $K$ în vector. Atunci vom uni nodul 1 din arbore (care știm
  că are gradul 1) cu nodul $K$. În felul acesta, nodul 1 și-a completat
  numărul necesar de vecini și poate fi neglijat pe viitor, iar $G[K]$ va fi
  decrementat cu o unitate, întrucât nodul $K$ și-a completat unul din
  vecini. Astfel, problema s-a redus la un arbore cu $N-1$ noduri numerotate
  de la 2 la $N$.

\item Vectorul $G$ este în continuare sortat, deoarece $G[K-1] = 1 < G[K] \leq
  G[K+1]$ înainte de decrementarea lui $G[K]$, deci după decrementare vom avea
  $G[K-1] = 1 \leq G[K] < G[K+1]$, adică dubla relație de ordonare se
  păstrează.

\item Întrucât secvența $G[2], G[3], \dots, G[N]$ reprezintă gradele unui
  arbore, putem aplica același raționament ca mai înainte pentru a deduce că
  $G[2]=1$. Cu ce nod vom uni nodul 2? Dacă $G[K] > 1$, îl vom uni cu nodul
  $K$. Dacă prin decrementare, $G[K]$ a ajuns la valoarea 1, vom trece la
  nodul $K+1$ (despre care știm că are gradul mai mare ca 1) și vom trasa
  muchia $2 \leftrightarrow (K+1)$.

\item Procedeul acesta se repetă până când au fost trasate $N-2$
  muchii. Aceasta înseamnă că a mai rămas o singură muchie de trasat. Iată
  deci că, mai devreme sau mai târziu, este oricum inevitabil să ajungem la
  cazul particular de arbore de care am amintit mai devreme. Deoarece la
  primul pas am unit nodul 1 cu nodul $K$, la al doilea pas am unit nodul 2 cu
  un alt nod ($K$ sau $K+1$) ș.a.m.d., rezultă că în $N-2$ iterații, toate
  nodurile de la 1 la $N-2$ și-au completat numărul de vecini. De aici rezultă
  că ultima muchie pe care o vom trasa este $(N-1) \leftrightarrow N$; putem
  să tipărim această muchie „cu ochii închiși”, fără nici un fel de teste
  suplimentare. Ultima muchie trasată este diferită de celelalte și necesită o
  operație separată de trasare din cauză că, în timp ce primele $N-2$ iterații
  uneau o frunză cu un nod intern, această ultimă iterație are de unit două
  frunze, deci nu are sens să mai căutăm un nod de grad mai mare ca 1.
\end{itemize}

Aceasta este metoda de lucru. Calculul complexității este simplu: Avem nevoie
doar de doi indici: Unul care marchează frunza curentă (în program el se
numește pur și simplu {\tt i}) și care avansează la fiecare pas, și unul care
marchează primul număr mai mare ca 1 din vector (în program se numește {\tt
  First}) și care se incrementează cu cel mult 1 la fiecare pas (deci de mai
puțin de $N$ ori în total). De aici rezultă complexitatea liniară a
algoritmului.

Să vedem cum se comportă această metodă pe cazul particular al exemplului 1:

\centeredTikzFigure[
  mat/.style = {
    matrix of nodes,
    ampersand replacement=\&,
    anchor=north,
    nodes = {
      draw,
      rectangle,
      anchor=center,
      minimum width=2em,
      minimum height=2em,
    },
  },
  pad/.style = {
    ->,
    shorten >= 3pt,
    shorten <= 3pt,
  },
  index/.style = {
    <-,
    shorten >= 3pt,
    font=\footnotesize,
  },
  edge/.style = {
    xshift=80,
    yshift=-10.5,
  },
]{
  \node (g) at (0,1) {$G$};

  \matrix[mat] (m) at (0,0) {
    1 \& 2 \& 3 \& 2 \& 1 \& 1 \\[2em]
    1 \& 1 \& 1 \& 2 \& 2 \& 3 \\[2em]
    1 \& 1 \& 1 \& 1 \& 2 \& 3 \\[2em]
    1 \& 1 \& 1 \& 1 \& 1 \& 3 \\[2em]
    1 \& 1 \& 1 \& 1 \& 1 \& 2 \\[2em]
    1 \& 1 \& 1 \& 1 \& 1 \& 1 \\[2em]
  };

  \draw[pad] (m-1-3.south) -- node[right] {sortare} (m-2-3.north);

  \draw[index] (m-2-1.south) -- node[right] {i} ++(0,-0.7);
  \draw[index] (m-2-4.south) -- node[right] {First} ++(0,-0.7);

  \draw[index] (m-3-2.south) -- node[right] {i} ++(0,-0.7);
  \draw[index] (m-3-5.south) -- node[right] {First} ++(0,-0.7);

  \draw[index] (m-4-3.south) -- node[right] {i} ++(0,-0.7);
  \draw[index] (m-4-6.south) -- node[right] {First} ++(0,-0.7);

  \draw[index] (m-5-4.south) -- node[right] {i} ++(0,-0.7);
  \draw[index] (m-5-6.south) -- node[right] {First} ++(0,-0.7);

  \node[edge] at (m-2-6.south east) { $\implies$ muchia (1,4)};
  \node[edge] at (m-3-6.south east) { $\implies$ muchia (2,5)};
  \node[edge] at (m-4-6.south east) { $\implies$ muchia (3,6)};
  \node[edge] at (m-5-6.south east) { $\implies$ muchia (4,6)};
  \node[edge] at (m-6-6.north east) { $\implies$ muchia (5,6)};
}

Mai trebuie remarcat că soluția nu este unică. Propunem ca temă cititorului să
scrie un program care să verifice în timp $O(N \log N)$ dacă soluția furnizată
de un alt program este corectă.

\inputminted{c}{src/problem16.c}

  \section{Problema 17}

Iată un nou exemplu de problemă care admite două rezolvări: una evidentă, dar
neeficientă și una mai puțin evidentă, dar cu mult mai eficientă.

{\bf ENUNȚ}: Fie $V$ un vector. Arborele cartezian atașat vectorului $V$ este
un arbore binar care se obține astfel: Dacă vectorul $V$ este vid (are 0
elemente), atunci arborele cartezian atașat lui este de asemenea vid. Altfel,
se selectează elementul de valoare minimă din vector și se pune în rădăcina
arborelui, iar arborii cartezieni atașați fragmentelor de vector din stânga
(respectiv din dreapta) elementului minim se pun în subarborele stâng,
respectiv drept al rădăcinii. Iată, de exemplu, care este arborele cartezian
al următorului vector cu 8 elemente:

\tikzset{
  level/.style={sibling distance=10em/#1},
  circ/.style = {circle, draw, minimum size=2em},
  mat/.style = {
    matrix of nodes,
    ampersand replacement=\&,
    nodes = {
      draw,
      anchor=center,
      minimum size=2em,
    },
  },
}
\newcommand\cartesianStack[3] {
  \matrix[mat] (m) at (0,2) {
    #1 \& #2 \& #3 \& \ \\
  };

  \node at ([xshift=-1em]m.west) {$S$};
}

\centeredTikzFigure[]{
  \node[circ] (n1) at (-3,-2) {8};
  \node[circ] (n2) at (-2,-1) {2};
  \node[circ] (n3) at (-1,-2) {4};
  \node[circ] (n4) at (0,0) {1};
  \node[circ] (n5) at (1,-2) {5};
  \node[circ] (n6) at (2,-1) {3};
  \node[circ] (n7) at (3,-3) {6};
  \node[circ] (n8) at (4,-2) {4};
  \draw (n1) -- (n2) -- (n3);
  \draw (n2) -- (n4) -- (n6) -- (n5);
  \draw (n6) -- (n8) -- (n7);

  \matrix[mat, column sep=0.38em] (v) at (0.5, 2) {
    8 \& 2 \& 4 \& 1 \& 5 \& 3 \& 6 \& 4 \\
  };
  \node at ([xshift=-1em] v.west) {$V$};

  % no easy anchors; just use hardcoded coordinates
  \draw[densely dotted, semithick] (-3.52,-2.53) rectangle (-0.49,2.55);
  \draw[densely dotted, semithick] (0.49,-3.53) rectangle (4.51,2.55);
}

În figură au fost încadrate prin dreptunghiuri punctate porțiunile din stânga,
respectiv din dreapta elementului minim, împreună cu subarborii
atașați. Trebuie observat că arborele cartezian atașat unui vector poate să nu
fie unic, în cazul în care există mai multe elemente de valoare minimă. Vom
impune ca o condiție suplimentară ca elementul care va fi trecut în rădăcină
să fie cel mai din stânga dintre minime (cel cu indicele cel mai mic). Astfel,
arborele cartezian este unic.

Cerința problemei este ca, dându-se un vector, să i se construiască arborele
cartezian.

{\bf Intrarea}: Fișierul de intrare {\tt INPUT.TXT} conține pe prima linie
valoarea lui $N$ ($N \leq 10.000$), iar pe a doua $N$ numere naturale mai mici
ca 30.000, separate prin spații.

Ieșirea se va face în fișierul text {\tt OUTPUT.TXT} sub următoarea formă:

$T_1 \quad T_2 \quad T_3 \quad \dots \quad T_N$

unde $T_i$ este indicele în vector al elementului care este părintele lui
$V[i]$ în arborele cartezian. Dacă $V[i]$ este rădăcina arborelui, atunci $T_i
= 0$.

{\bf Exemplu}: Pentru exemplul dat mai sus, fișierul {\tt INPUT.TXT} este:

\begin{verbatim}
  8
  8 2 4 1 5 3 6 4
\end{verbatim}

După cum reiese din figură, tatăl elementului 8 este elementul 2, adică al
doilea în vector; tatăl elementului 2 este elementul 1, adică al 4-lea în
vector; tatăl elementului 5 este elementul 3, adică al 6-lea în vector
ș.a.m.d. Fișierul de ieșire este deci:

\begin{verbatim}
  2 4 2 0 6 4 8 6
\end{verbatim}

{\bf Complexitate cerută}: $O(N)$.

{\bf Timp de implementare}: 45 minute - 1h.

{\bf Timp de rulare}: 2 secunde.

{\bf REZOLVARE}: Nu întâmplător s-a impus o complexitate liniară pentru
rezolvarea acestei probleme. Altfel, ea ar fi trivială în $O(N^2)$, prin
următoarea metodă: scriem o procedură care parcurge vectorul și caută minimul,
apoi se reapelează pentru bucățile de vector aflate în stânga, respectiv în
dreapta minimului. Pentru a demonstra că această variantă de rezolvare are
complexitate pătratică, să ne imaginăm cum s-ar comporta ea pe cazul:

\begin{equation}
  V = (N \quad N-1 \quad N-2 \quad \dots \quad 2 \quad 1)
\end{equation}

La primul apel, procedura ar face $N$ comparații pentru a parcurge vectorul
(deoarece elementul minim este ultimul în vector) și s-ar reapela pentru
porțiunea din vector care cuprinde primele $N-1$ elemente. La al doilea apel,
ar face $N-1$ comparații și s-ar reapela pentru primele $N-2$ elemente etc. În
concluzie, numărul total de comparații făcute este

\begin{equation}
  N + (N-1) + (N-2) + \cdots + 1 = \frac{N(N + 1)}{2}
\end{equation}

de unde rezultă complexitatea. O problemă interesantă, pe care îi vom lăsa
plăcerea cititorului să o rezolve, este de a demonstra că această versiune
{\bf nu poate atinge o complexitate mai bună decât $O(N \log N)$} și de a
arăta care sunt cazurile cele mai favorabile pe care se obține această
complexitate.

A doua metodă este și ea destul de ușor de înțeles și de implementat. Ceea
este mai greu de acceptat este că ea are complexitate liniară, așa cum vom
încerca să explicăm la sfârșit. Iată mai întâi principiul de rezolvare: vom
porni cu un arbore cartezian vid și, la fiecare pas, vom adăuga câte un
element al vectorului $V$ la acest arbore, astfel încât structura obținută să
rămână un arbore cartezian. La al $k$-lea pas, vom adăuga elementul $V[k]$ în
arbore și vom restructura arborele în așa fel încât să obținem arborele
cartezian atașat primelor $k$ elemente din $V$. Trebuie să ne concentrăm
atenția asupra a două lucruri:

\begin{enumerate}

\item Cunoscând arborele cartezian atașat primelor $k$-1 vârfuri și elementul
  $V[k]$, cum se obține arborele cartezian atașat primelor $k$ vârfuri?

\item Cum reușim să actualizăm de $N$ ori arborele astfel încât timpul total
  consumat să fie liniar?

\end{enumerate}

Pentru a răspunde la prima întrebare, pe lângă vectorii $V$ și $T$, mai este
necesară o stivă $S$, în care vom stoca elemente ale vectorului $V$. Inițial,
stiva este vidă. Atunci când un nou element $X$ sosește, el va fi introdus în
stivă imediat după ultimul număr din stivă care are o valoare mai mică sau
egală cu $X$. Toate elementele care se aflau înainte în stivă pe poziții mai
mari sau egale cu poziția pe care a fost inserat $X$ vor fi eliminate din
stivă, iar elementul care se afla exact pe poziția lui $X$ va deveni fiul
stâng al lui $X$. $X$ însuși va deveni fiul drept al predecesorului său în
stivă. La fiecare moment, primul element din stivă este rădăcina arborelui
cartezian.

Pentru a înțelege mai bine principiul de funcționare a stivei, să analizăm mai
de aproape exemplul din enunț.

La început stiva este vidă. Primul element din V are valoarea 8, drept care îl
vom pune în stivă, iar arborele cartezian va avea un singur nod:

\centeredTikzFigure[]{
  \node[circ] {8};

  \cartesianStack{8}{\ }{\ }
}

Următorul element sosit este 2. Acesta este mai mic decât 8, deci trebuie
introdus înaintea lui în stivă. El va fi deci primul element din stivă și
rădăcina arborelui cartezian la acest moment. Concomitent, 8 va fi eliminat
din stivă și va deveni fiul stâng al lui 2:

\centeredTikzFigure[]{
  \node[circ] {2}
  child { node[circ] {8} }
  child[missing] {node[circ] {}};

  \cartesianStack{2}{\ }{\ }
}

Se observă că arborele obținut este tocmai arborele cartezian atașat secvenței
$(V[1], V[2])$. Următorul element este 4, care este mai mare decât 2, deci
trebuie adăugat în vârful stivei. Nici un element nu este eliminat din stivă,
iar 4 devine fiul drept al lui 2:

\centeredTikzFigure[]{
  \node[circ] {2}
  child { node[circ] {8} }
  child { node[circ] {4} };

  \cartesianStack{2}{4}{\ }
}

Următorul element sosit este 1, care este mai mic decât toate numerele din
stivă. Stiva se va goli, iar numărul 2 (cel peste care se va scrie 1) va
deveni fiul stâng al lui 1:

\centeredTikzFigure[]{
  \node[circ] {1}
  child { node[circ] {2}
    child { node[circ] {8} }
    child { node[circ] {4} }
  }
  child[missing] {node[circ] {}};

  \cartesianStack{1}{\ }{\ }
}

Deja arborele începe să semene cu forma sa finală. Urmează elementul 5, care
va fi adăugat în stivă și „atârnat” în dreapta lui 1:

\centeredTikzFigure[]{
  \node[circ] {1}
  child { node[circ] {2}
    child { node[circ] {8} }
    child { node[circ] {4} }
  }
  child { node[circ] {5} };

  \cartesianStack{1}{5}{\ }
}

Elementul 3 este mai mare ca 1, căruia îi va deveni fiu drept, dar mai mic ca
5, pe care îl va elimina din stivă:

\centeredTikzFigure[]{
  \node[circ] {1}
  child { node[circ] {2}
    child { node[circ] {8} }
    child { node[circ] {4} }
  }
  child { node[circ] {3}
    child { node[circ] {5} }
    child[missing] {node[circ] {}}
  };

  \cartesianStack{1}{3}{\ }
}

Următorul număr, 6, va fi adăugat la extremitatea dreaptă a arborelui și în
vârful stivei:

\centeredTikzFigure[]{
  \node[circ] {1}
  child { node[circ] {2}
    child { node[circ] {8} }
    child { node[circ] {4} }
  }
  child { node[circ] {3}
    child { node[circ] {5} }
    child { node[circ] {6} }
  };

  \cartesianStack{1}{3}{6}
}

În sfârșit, elementul 4 va fi fiul drept al lui 3 și îl va elimina din stivă
pe 6, care îi va deveni fiu stâng:

\centeredTikzFigure[]{
  \node[circ] {1}
  child { node[circ] {2}
    child { node[circ] {8} }
    child { node[circ] {4} }
  }
  child { node[circ] {3}
    child { node[circ] {5} }
    child { node[circ] {4}
      child { node[circ] {6} }
      child[missing] {node[circ] {}}
    }
  };

  \cartesianStack{1}{3}{4}
}

Se observă că arborele a ajuns tocmai la forma sa corectă. Trebuie acum să ne
ocupăm de un detaliu de implementare. Pentru a afla poziția pe care trebuie
inserat un element în stivă avem două metode:

\begin{enumerate}

\item Putem să căutăm în stivă de la dreapta la stânga (ar fi mai corect spus
  „de la vârf spre bază”) până dăm de un element mai mic decât cel de inserat;
  programul folosește această metodă și îi vom discuta în final eficiența.

\item Putem face o căutare binară în stivă, întrucât elementele din stivă au
  valori crescătoare de la bază spre vârf (lăsăm demonstrația acestei
  afirmații în seama cititorului). O căutare binară într-un vector de $k$
  elemente poate necesita, în cazul cel mai nefavorabil, $\log k$
  comparații. În cazul cel mai nefavorabil, când vectorul $V$ este sortat
  crescător, elementele vor fi introduse pe rând în stivă și nu vor mai fi
  scoase, deci la fiecare pas se vor face $\log k$ comparații, unde $k$ ia
  valori de la 1 la $N$. Complexitatea care rezultă este mai slabă decât cea
  cerută:

  \begin{equation}
    \begin{split}
      O(\log 1 + \log 2 + \cdots + \log N) = O(\sum_{k = 1}^{N} \log k) = \\
      = O(\log \prod_{k = 1}^{N} k) = O(\log N!) = O(N \cdot \log N)
    \end{split}
  \end{equation}

\end{enumerate}

Acesta este unul din puținele cazuri în care căutarea binară este mai
ineficientă decât cea secvențială.

Pentru ușurința programării, sursa C de mai jos reține în stiva $S$ nu
valorile elementelor, ci indicii lor în vectorul $V$ (deoarece aceștia sunt
ceruți pentru construcția vectorului $T$).

\begin{minted}{c}
#include <stdio.h>
#define NMax 10001

int V[NMax], /* Vectorul */
    T[NMax], /* Vectorul de tati */
    S[NMax], /* Stiva */
    N;       /* Numarul de elemente */

void ReadData(void)
{ FILE *F=fopen("input.txt","rt");
  int i;

  fscanf(F,"%d\n",&N);
  for (i=1; i<=N;)
    fscanf(F, "%d", &V[i++]);
}

void BuildTree(void)
{ int i,k,LenS=0;

  S[0]=0; /* Pentru ca initial T[1] sa fie 0 */
  for (i=1; i<=N; i++)
    { /* Cauta pozitia pe care va fi inserat V[i] */
      k=LenS+1;
      while (V[S[k-1]]>V[i]) k--;
      /* Face corecturile in S si T */
      T[i]=S[k-1];
      if (k<=LenS) T[S[k]]=i;
      /* i este ultimul element din stiva, deci... */
      S[LenS=k]=i;
    }
}

void WriteSolution(void)
{ FILE *F=fopen("output.txt","wt");
  int i;
  for (i=1; i<=N;)
    fprintf(F,"%d ",T[i++]);
  fprintf(F,"\n");
}

void main(void)
{
  ReadData();
  BuildTree();
  WriteSolution();
}
\end{minted}

Acum să analizăm și complexitatea acestui algoritm. În primul rând, ea nu
poate fi mai bună decât $O(N)$, pentru că aceasta este complexitatea
funcțiilor de intrare și ieșire. Procedura {\tt BuildTree} se compune dintr-un
ciclu {\tt for} în care se execută patru operații în timp constant și o
instrucțiune repetitivă {\tt while}. Numărul total de operații în timp
constant care se execută în procedură este prin urmare $O(N)$. Problema este:
care este numărul total maxim de evaluări ale condiției logice din bucla {\tt
  while}? Aparent, bucla while se execută de $O(N)$ ori, deci numărul total de
evaluări ar fi $O(N^2)$. Să aruncăm totuși o privire mai atentă.

Fiecare evaluare a condiției din bucla {\tt while} are ca efect decrementarea
lui $k$ și, implicit, eliminarea unui element deja existent în stivă. Pe de
altă parte, fiecare element este introdus în stivă o singură dată și deci nu
poate fi eliminat din stivă decât cel mult o dată. Așadar numărul maxim de
elemente ce pot fi eliminate din stivă pe parcursul executării procedurii {\tt
  BuildTree} este $N-1$, deci numărul total de evaluări ale condiției este
$O(N)$. De aici rezultă că programul are complexitate liniară.

  \section{Problema 18}

{\bf ENUNȚ}: Se dă un vector nesortat cu elemente numere reale
oarecare. Considerând că vectorul ar fi sortat, se cere să se găsească
distanța maximă între două elemente consecutive ale sale, fără însă a sorta
efectiv vectorul.

{\bf Intrarea}: Fișierul {\tt INPUT.TXT} conține pe prima linie numărul $N$ de
elemente din vector ($N \leq 5.000$). Pe următoare linie se dau numerele
separate prin spații.

{\bf Ieșirea}: Pe ecran se va tipări un mesaj de forma:

Distanța maximă este $D$

{\bf Exemplu}: Pentru fișierul de intrare cu conținutul

\begin{verbatim}
  4
  5 3.2 2 3.7
\end{verbatim}

răspunsul trebuie să fie

\begin{verbatim}
  Distanta maxima este 1.3
\end{verbatim}

{\bf Timp de implementare}: 30 minute.

{\bf Timp de rulare}: 2-3 secunde.

{\bf Complexitate cerută}: $O(N)$.

{\bf REZOLVARE}: Desigur că primul lucru la care ne gândim este să sortăm
vectorul și să îl parcurgem apoi de la stânga la dreapta, căutând distanța
maximă între două elemente consecutive. Complexitatea unui asemenea algoritm
este $O(N \log N)$. Nici această soluție nu este rea, iar la un concurs,
comisiei de corectare i-ar veni destul de greu să găsească teste care să
departajeze un algoritm în $O(N \log N)$ de unul liniar, chiar și pentru $N =
5.000$. Totuși, vom arăta care este algoritmul liniar; în primul rând de
dragul „artei”, iar în al doilea rând pentru că nu este cu mult mai greu de
implementat decât o sortare.

Primul lucru care trebuie făcut este găsirea maximului și a minimului din
vector; să notăm aceste valori cu $V_{max}$ și $V_{min}$. Aceste operații se
fac în timp liniar, eventual chiar la citirea datelor din fișier. Apoi se
împarte intervalul $[V_{min}, V_{max}]$ de pe axa reală în $N-1$ intervale
egale. Iată cazul exemplului din enunț, unde $V_{min} = 2$ și $V_{max} = 5$:

\tikzset{
  axisLabel/.style = {
    font=\bf,
    text depth=0,
  },
}
\newcommand\axisNumber[3] {
  \draw[ultra thick] (#1,0.2) -- (#1,-0.2);
  \node[axisLabel] at (#3,0.6) {#2};
}
\centeredTikzFigure[
]{
  %axis
  \draw[->] (-2.5,0) -- (8,0);

  % ticks
  \draw (0,0.1) -- (0,-0.1);
  \draw (3,0.1) -- (3,-0.1);
  \draw (4,0.1) -- (4,-0.1);

  % numbers
  \node at (0,0.6) {0};
  \axisNumber{2}{2}{2};
  \axisNumber{3.2}{3,2}{3.0};
  \axisNumber{3.7}{3,7}{3.9};
  \axisNumber{5}{5}{5};

  % variables
  \node at (2, -0.8) {$V_{min}$};
  \node at (5, -0.8) {$V_{max}$};
}

Lungimea fiecărui interval va fi deci de

\begin{equation}
  D = \frac{V_{max} - V_{min}}{N - 1} \qquad
  (\text{în cazul nostru } D = 1)
\end{equation}

De ce s-a făcut această împărțire? Dacă notăm cu $D_{max}$ distanța maximă pe
axă între două numere vecine, adică tocmai valoarea pe care o căutăm, se poate
demonstra că $D_{max} \geq D$. Într-adevăr, între cele $N$ numere de pe axă se
formează $N-1$ intervale. Dacă presupunem că $D_{max} < D$, rezultă că
distanța între oricare două numere consecutive de pe axă este mai mică decât
$D$. De aici deducem că distanța dintre primul și ultimul număr, adică
$V_{max} - V_{min}$, este mai mică decât $(N-1) \times D$. Dar aceasta duce
la relația:

\begin{equation}
  V_{max} - V_{min} < (N - 1) \cdot \frac{V_{max} - V_{min}}{N - 1}
  \implies
  V_{max} - V_{min} < V_{max} - V_{min}
\end{equation}

relație care este absurdă; demonstrația afirmației $D_{max} \geq D$ este
completă.

Următorul pas pe care îl avem de făcut este să parcurgem încă o dată vectorul
de numere și să aflăm pentru fiecare element căruia dintre intervalele de
lungime îi aparține. Și această operație se poate face în $O(N)$. Convenim ca
dacă un număr $X$ se află exact la limita dintre două intervale, adică

\begin{equation}
  X = V_{min} + k \cdot D, \quad 0 \leq k < N
\end{equation}

el să fie considerat ca aparținând intervalului din dreapta. Aceasta înseamnă
că elementul de valoare $V_{max}$ nu aparține nici unuia din cele $N-1$
intervale, ci celui de-al $N$-lea interval, $[V_{max}, V_{max} + D]$. Să vedem
la ce ne ajută acest lucru. Din moment ce $D_{max} \geq D$, rezultă că este
imposibil ca distanța maximă să se producă între două numere din același
interval, deoarece distanța în cadrul aceluiași interval nu poate atinge
valoarea $D$. Este deci obligatoriu ca distanța maximă să apară între două
elemente din intervale distincte. Să urmărim în figura următoare ce alte
proprietăți mai au aceste numere:

\newcommand\axisPair[3] {
  \draw[ultra thick] (#1,0.2) -- (#1,-0.2);
  \node[axisLabel] at (#1,0.6) {#2};
  \node[axisLabel] at (#1,-0.6) {#3};
}
\centeredTikzFigure[
]{
  %axis
  \draw[->] (-5,0) -- (9,0);

  % ticks
  \foreach \x in {0,...,4} {
    \draw (\x,0.1) -- (\x,-0.1);
  };

  % values
  \axisPair{-3}{}{$V_{min}$}
  \axisPair{0.25}{}{$X$}
  \axisPair{0.75}{?}{$X'$}
  \axisPair{3.25}{?}{$Y'$}
  \axisPair{3.75}{}{$Y$}
  \axisPair{7}{}{$V_{max}$}

  \node[axisLabel] at (-1.5, 0.3) {$\cdots$};
  \node[axisLabel] at (5.5, 0.3) {$\cdots$};
}

Dacă $X$ și $Y$ sunt valorile între care diferența este maximă, este de la
sine înțeles că între ele nu mai există nici un număr, deoarece se prespune că
$X$ și $Y$ sunt consecutive în vectorul sortat. Aceasta înseamnă însă că $X$
este cel mai mare număr din intervalul său, iar numărul $X'$ nu poate exista
acolo unde a fost el figurat. Analog, $Y$ este cel mai mic număr din
intervalul său, iar numărul $Y'$ nu poate exista. De fapt, în nici unul din
intervalele dintre cele care le cuprind pe $X$ și $Y$ nu poate exista nici un
număr.

Prin urmare, diferența maximă se poate produce numai între maximul unui
interval și minimul imediat următor. Următorul pas în găsirea soluției
presupune aflarea pentru fiecare din cele $N-1$ intervale (sau $N$ dacă îl
considerăm și pe ultimul, cel care nu îl conține decât pe $V_{max}$) a
minimului și a maximului. Și acest pas se execută în $O(N)$, deoarece
procesarea fiecărui element din vector se reduce la numai două comparări, cu
minimul și cu maximul intervalului în care se încadrează el. Vor rezulta doi
vectori care în program se vor numi $Lo$ și $Hi$. Iată care sunt valorile lor
pentru exemplul nostru:

\begin{table}[H]
  \centering
  \begin{tabular}{ccccc}
    {\bf Intervalul}: & [2, 3) & [3, 4) & [4, 5) & [5, 6) \\
            $Lo$: & 2 & 3,2 & $-$ & 5 \\
            $Hi$: & 2 & 3,7 & $-$ & 5 \\
  \end{tabular}
\end{table}

Deoarece în intervalul [4,5) nu se află elemente, rezultă că elementele
  corespunzătoare din vectorii $Lo$ și $Hi$ trebuie să aibă o valoare specială
  care să informeze programul asupra acestui lucru. De exemplu, sursa oferită
  mai jos folosește următorul artificiu: inițializează vectorul $Lo$ cu valori
  foarte mari ($V_{max} + 1$), astfel încât orice număr „repartizat” într-un
  interval să modifice această valoare. Similar, vectorul $Hi$ este
  inițializat cu $V_{min} - 1$. Dacă pentru un interval aceste valori se
  păstrează până la sfârșit, putem trage concluzia că în respectivul interval
  nu se află nici un număr.

În continuare, elementele vectorilor $Lo$ și $Hi$ se amestecă formând un nou
vector $W$ care de data aceasta este sortat. Sortarea este foarte ușoară,
pentru că nu avem decât să așezăm numerele în ordinea $Lo[1]$, $Hi[1]$,
$Lo[2]$, $Hi[2]$, $\dots$, $Lo[N]$, $Hi[N]$. Deși la prima vedere pare că noul
vector rezultat are $2N$ elemente, de fapt el are numai $N$ elemente, pentru
că:

\begin{itemize}

\item Dacă într-un interval $K$ există un singur număr, (cazul intervalelor
  [2,3) și [5,6)) sau există numai numere egale, atunci $Lo[K] = Hi[K]$ și
      este suficient să copiem în $W$ una singură dintre aceste două valori;

\item Dacă într-un interval nu există nici un număr, putem să nu copiem nici o
  valoare în vectorul $W$.

\end{itemize}

Astfel, construcția vectorului $W$ se poate face în timp liniar, mai exact în
$O(2N)$. Se observă că la această construcție se poate întâmpla ca unele
numere să „dispară”, adică să nu fie trecute în vectorul $W$. De exemplu, dacă
între numerele 3,2 și 3,7 ar mai fi existat un număr, 3,5, el nu ar fi fost
nici minim, nici maxim pentru intervalul său, deci nu ar fi fost
copiat. Totuși, trierea în acest fel a elementelor nu afectează în nici un fel
soluția. În cazul nostru, nu se întâmplă să dispară nici un element, deci $W =
(2, 3,2, 3,7, 5)$.

După ce am construit vectorul $W$, nu mai avem decât să-l parcurgem de la
stânga la dreapta și să tipărim diferența maximă întâlnită între două numere
consecutive (repetăm, vectorul $W$ este sortat), această ultimă etapă
necesitând și ea un timp liniar. Nu se poate obține o complexitate inferioară
celei liniare, întrucât citirea datelor presupune ea însăși $N$ operații.

Ca un detaliu de implementare, odată ce au fost construiți vectorii $Lo$ și
$Hi$, vectorul $V$ nu mai este necesar, deci putem construi chiar în el
vectorul $W$, pentru a economisi memorie.

\inputminted{c}{src/problem18.c}

  \addcontentsline{toc}{chapter}{Bibliografie} 

\begin{thebibliography}{9}

\bibitem{knuth}
  Donald E. Knuth,
  {\it The Art of Computer Programming},
  Addison Wesley, Massachusetts,
  1973.
  Tradusă în limba română de către Editura Tehnică cu denumirea {\it Tratat de
    programarea calculatoarelor}.

\bibitem{cormen}
  T.H. Cormen, C.E. Leiserson., R.L. Rivest,
  {\it Introduction to Algorithms},
  The MIT Press, Cambridge, Massachusetts,
  1992.

\bibitem{andonie}
  Răzvan Andonie, Ilie Gârbacea,
  {\it Algoritmi fundamentali - o perspectivă C++},
  Editura Libris, Cluj-Napoca,
  1995.

\end{thebibliography}


  \tableofcontents

\end{document}
