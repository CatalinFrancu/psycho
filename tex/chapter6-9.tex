\section{Problema 9}

Iată o problemă care necesită pentru reducerea complexității un artificiu
asemănător celui din problema codului Prüfer. Ea a fost propusă în 1995 la
concursul de selecție a echipelor României pentru IOI și CEOI. Deși cerința
este puțin modificată pentru a nu ne izbi de dificultăți secundare, ideea
generală de abordare este aceeași.

{\bf ENUNȚ}: Președintele companiei X dorește să organizeze o petrecere cu
angajații. Această companie are o structură ierarhică în formă de
arbore. Fiecare angajat are asociat un număr întreg reprezentând măsura
sociabilității sale. Pentru ca petrecerea să fie agreabilă pentru toți
participanții, președintele dorește să facă invitațiile astfel încât:

\begin{itemize}

\item el însuși să participe la petrecere;

\item pentru nici un participant la petrecere să nu fie invitat și șeful lui
  direct;

\item suma măsurilor sociabilităților invitaților să fie maximă.

\end{itemize}

Se cere să se spună care este suma maximă a sociabilităților care se poate obține.

{\bf Intrarea} se face din fișierul {\tt INPUT.TXT} care conține trei linii de
forma:

\begin{verbatim}
  N
  T(1) T(2) ... T(N)
  S(1) S(2) ... S(N)
\end{verbatim}

unde $N$ este numărul de angajați ai companiei, inclusiv președintele ($N \leq
1000$), $T(k)$ este numărul de ordine al șefului direct al lui $k$ (dacă $T(k)
= 0$, atunci $k$ este președintele), iar $S(k)$ este măsura sociabilității lui
$k$. Valorile vectorului $S$ sunt de tipul întreg.

{\bf Ieșirea}: Pe ecran se va tipări suma maximă a sociabilităților ce se
poate obține.

{\bf Exemplu}: Pentru intrarea:

\begin{verbatim}
  7
  2 5 2 5 0 4 2
  2 5 3 13 8 4 3
\end{verbatim}

pe ecran se va afișa numărul 20 (fiindcă la petrecere participă 1, 3, 5, 6 și
7).

{\bf Timp de implementare}: 1h - 1h 15min.

{\bf Timp de rulare}: 2-3 secunde.

{\bf Complexitate cerută}: $O(N)$. De asemenea, menționăm că s-a specificat $N
\leq 1000$ numai pentru ca programul sursă de mai jos să nu se complice și să
distragă atenția asupra unor lucruri neimportante. În mod normal, limita
trebuia să fie $N \leq 10.000$.

{\bf REZOLVARE}: Vom expune mai întâi principiul de rezolvare al problemei,
apoi vom aborda detaliile de implementare.

Algoritmul are la bază programarea dinamică și necesită o parcurgere de jos în
sus a arborelui, decizia pentru fiecare nod depinzând de valorile tuturor
fiilor lui. Ne propunem să aflăm două caracteristici pentru fiecare nod $k$:

\begin{itemize}

\item $P(k)$ - suma maximă a sociabilităților care se poate obține în
  subarborele de rădăcină $k$ în cazul în care $k$ participă la petrecere;

\item $Q(k)$ - suma maximă a sociabilităților care se poate obține în
  subarborele de rădăcină $k$ în cazul în care $k$ nu participă la petrecere;

\end{itemize}

Dacă reușim să determinăm aceste caracteristici, nu ne rămâne decât să-l
tipărim pe $P(R)$ ($R$ fiind rădăcina arborelui). Într-adevăr, problema cere
să se determine suma maximă a sociabilităților din întregul arbore, adică din
subarborele de rădăcină $R$. În plus, se mai cere ca $R$ să participe la
petrecere. Rămâne de aflat cum se stabilește relația între caracteristicile
unui nod și cele ale fiilor săi.

\begin{itemize}

\item Dacă angajatul $k$ participă la petrecere, atunci automat nici unul din
  subordonații săi direcți nu participă, și obținem relația:

  \begin{equation}
    P(k) = S(k) + \sum_j Q(j), \quad j \text{ fiu al lui } k
  \end{equation}

\item Dacă angajatul $k$ nu participă la petrecere, atunci subordonații săi
  direcți pot să participe sau nu la petrecere, după cum este mai avantajos,
  și obținem relația:

  \begin{equation}
    Q(k) = \sum_j \max(P(j), Q(j)), \quad j \text{ fiu al lui } k
  \end{equation}

\end{itemize}

Problema care se pune acum este cum să facem parcurgerea arborelui într-un mod
cât mai avantajos. Pseudocodul (recursiv) sub forma sa cea mai generală este:

\vspace{\algskip}
\begin{algorithm}
  \floatname{algorithm}{Procedura}
  \caption{Calcul($k$, $P$, $Q$)}
  \begin{algorithmic}[1]
    \STATE $P \leftarrow S(k)$
    \STATE $Q \leftarrow 0$
    \FORALL{$j$ fiu al lui $k$}
    \STATE Calcul($j$, $P_1$, $Q_1$)
    \STATE $P \leftarrow P + Q_1$
    \STATE $Q \leftarrow Q + \max(P_1, Q_1)$
    \ENDFOR
    \RETURN $P$, $Q$
  \end{algorithmic}
\end{algorithm}

Soluția „de bun simț” este de a construi arborele alocat dinamic. Ar apărea
însă o sumedenie de dificultăți. În primul rând, arborele este general, deci
nu se cunoaște numărul maxim de fii pe care îi poate avea un nod (pe cazul cel
mai defavorabil, rădăcina poate avea $N-1$ fii). Aceasta înseamnă că legătura
trebuie să fie de tip tata (fiecare nod pointează la tatăl său), ceea ce
complică procedura de parcurgere: nu se poate apela procedura recursiv,
dintr-un nod pentru toți fiii săi, deoarece nu se cunosc fiii, ci numai tatăl!
O altă modalitate de alocare a arborelui ar fi cu doi pointeri pentru fiecare
nod: unul către primul său fiu și unul către fratele său din dreapta (exemplul
din enunț are reprezentate grafic mai jos cele două metode de construcție).

\centeredTikzFigure[
  mat/.style = {
    matrix of nodes,
    ampersand replacement=\&,
    anchor=north,
    column sep=4em,
    row sep=1em,
  },
  level/.style={sibling distance=10em/#1},
  row 1/.style={
    nodes=tree,
  },
  tree/.style={circle, draw, minimum size=2.2em},
  edge from parent/.style={draw, <-},
]{
  \matrix[mat] {
    \node {5}
    child { node {2}
      child { node {1} }
      child { node {3} }
      child { node {7} }
    }
    child { node {4}
      child { node {6} }
    };
    %
    \&
    %
    \node (t2) {5}
    child { node (n2) {2} edge from parent[draw, ->]
      child { node (n1){1} edge from parent[draw, ->] }
    };
    \node[anchor=center] (n4) at ([xshift=12em]n2.center) {4};
    \node[anchor=center] (n3) at ([xshift=4em]n1.center) {3};
    \node[anchor=center] (n7) at ([xshift=8em]n1.center) {7};
    \node[anchor=center] (n6) at ([xshift=12em]n1.center) {6};
    %
    \draw[->] (n2) -- (n4);
    \draw[->] (n4) -- (n6);
    \draw[->] (n1) -- (n3);
    \draw[->] (n3) -- (n7);
    \\
    Legătură de tip tată \&
    \node[xshift=6em] {Legătură de tip fiu + frate}; \\
  };
}

Probabil că veți fi de acord cu mine că e riscant să te aventurezi la o
asemenea implementare în timp de concurs, deoarece lucrul cu pointeri
presupune o atenție deosebită. Greșelile sunt mai greu de observat și de multe
ori duc la blocarea calculatorului, care trebuie resetat mereu, pierzându-se
astfel o mulțime de timp. Trebuie deci căutată o metodă de parcurgere a
arborelui care să nu necesite o alocare dinamică a memoriei. Putem încerca
astfel: inițial $P(i)=S(i)$ și $Q(i)=0$ pentru orice nod. Urmează acum să
tratăm pe rând fiecare nod. Cum? Știm că pentru fiecare nod $k$, numerele
$P(k)$ și $Q(k)$ intervin în expresia lui $P(T(k))$ și $Q(T(k))$. Uitându-ne
la formulele de mai sus, observăm că tot ce avem de făcut este să incrementăm
$P(T(k))$ cu $Q(k)$ și $Q(T(k))$ cu $\max(P(k), Q(k))$.

Există o singură problemă: Pentru a putea folosi numerele $P(k)$ și $Q(k)$
trebuie să ne asigurăm că ele au fost deja calculate corect, în funcție de
caracteristicile tuturor fiilor lui $k$. În cazul frunzelor, problema este
rezolvată, deoarece ele nu au fii. Pentru nodurile interne, este necesar să
știm câți fii au ele și câți din aceștia au fost tratați. În momentul în care
toți fiii unui nod au fost tratați, poate fi tratat și nodul în sine. În
program, vectorul $F$ reține numărul de fiii netratați ai fiecărui nod. La
tratarea unui nod $k$ se face decrementarea lui $F(T(k))$. Un nod $k$ poate fi
ales spre tratare dacă $F(k)=0$. Se observă că formatul datelor de intrare
permite cu ușurință construcția vectorului $F$ (numărul de fii al lui $k$ este
egal cu numărul de apariții al lui $k$ în vectorul $T$).

Un algoritm mai ușor de implementat ar fi:

\vspace{\algskip}
\begin{algorithmic}[1]
  \STATE numără fiii fiecărui nod
  \FOR{$i = 1$ la $N-1$}
  \STATE caută un nod $k$ cu $F(k)=0$
  \STATE $P(T(k)) \leftarrow P(T(k))+Q(k)$
  \STATE $Q(T(K)) \leftarrow Q(T(K))+\max(P(k),Q(k))$
  \STATE $F(T(K)) \leftarrow F(T(K))-1$
  \ENDFOR
  \PRINT $P(R)$
\end{algorithmic}

Partea delicată a acestui algoritm este căutarea unui nod $k$ cu
$F(k)=0$. Dacă ea se face secvențial, pornind de fiecare dată de la primul nod
și cercetând fiecare element, programul care rezultă are complexitatea
$O(N^2)$. Această căutare trebuie deci optimizată, și iată cum: În principiu,
putem căuta nodurile cu $F(k)=0$ mergând numai în sensul crescător al
indicilor în vector. Vom folosi, ca și la problema codului lui Prüfer, două
variabile: $K$ și $Next$. $K$ este nodul tratat în prezent, iar $Next$ este
nodul care urmează a fi tratat la pasul următor, așadar $Next > K$. După
tratarea nodului $K$ și decrementarea lui $F(T(k))$, pot surveni trei
situații:

\begin{enumerate}

\item $F(T(k))>0$ și în vectorul $F$ nu a apărut nici un alt element zero, caz
  în care nimic nu se schimbă, algoritmul continuând cu tratarea nodului
  $Next$;

\item $F(T(k))=0$ și $T(k)>Next$, caz în care de asemenea se poate trata nodul
  $Next$ (deoarece selectarea nodurilor cu $F(k)=0$ se face în ordine
  crescătoare a indicilor);

\item $F(T(k))=0$ și $T(k)<Next$, caz în care se va trata mai întâi nodul
  $T(k)$, urmând a se reveni apoi la nodul $Next$.

\end{enumerate}

Pentru motivele explicate la codul lui Prüfer, acest algoritm nu necesită
menținerea unei stive, iar timpul total de căutare este $O(N)$. Facem
observația că nu se poate găsi un algoritm mai bun, deoarece trebuie parcurse
toate cele $N$ noduri ale arborelui.

Iată în încheiere cum arată arborele cu valorile $P$ și $Q$ atașate fiecărui
nod:

\centeredTikzFigure[
  level/.style={sibling distance=15em/#1},
  every node/.style = {circle, draw, minimum size=3em, font=\Large},
  every label/.style = {draw=none, font=\scriptsize, minimum size=1em},
]{
  \node [label=north west:{\begin{tabular}{l}S=8\\P=20\\Q=21\end{tabular}}] {5}
  child { node [label=north west:{\begin{tabular}{l}S=5\\P=5\\Q=8\end{tabular}}] {2}
    child { node [label=south:{\begin{tabular}{l}S=2\\P=2\\Q=0\end{tabular}}] {1} }
    child { node [label=south:{\begin{tabular}{l}S=3\\P=3\\Q=0\end{tabular}}] {3} }
    child { node [label=south:{\begin{tabular}{l}S=3\\P=3\\Q=0\end{tabular}}] {7} }
  }
  child { node [label=north east:{\begin{tabular}{l}S=13\\P=13\\Q=4\end{tabular}}] {4}
    child { node [label=south:{\begin{tabular}{l}S=4\\P=4\\Q=0\end{tabular}}] {6} }
  };
}

\begin{lstlisting}[language=C]
#include <stdio.h>
#define NMax 1000
typedef long Vector[NMax+1];

Vector T,S,P,Q,F; /* T = Vectorul de tati,
                     S = sociabilitatile,
                     P,Q = caracteristicile,
                     F = numarul de fii */
int N,Root;

void InitData(void)
{ int i;
  FILE *InF=fopen("input.txt","rt");

  fscanf(InF,"%d",&N);
  for (i=1;i<=N+1;F[i++]=0);
  for (i=1;i<=N;i++)
    { fscanf(InF,"%d",&T[i]);
      if (T[i]) F[T[i]]++; else Root=i;
    }
  for (i=1;i<=N;i++) fscanf(InF,"%d",&S[i]);
  fclose(InF);

  for (i=1;i<=N;P[i++]=S[i]);
  for (i=1;i<=N;Q[i++]=0);
}

void FindNext(int *K,int *Next)
{ F[T[*K]]--;
  F[*K]=-1;
  if ((F[T[*K]]>0) || (T[*K]>*Next))
    { *K=*Next;
      while (F[*Next]) (*Next)++;
    }
    else *K=T[*K];
}

void TraverseTree(void)
{ int K=0,Next,i;

  do; while (F[++K]);
  Next=K; do; while (F[++Next]);
  for (i=1;i<N;i++)
    { P[T[K]]+=Q[K];
      Q[T[K]]+=(P[K]>Q[K]) ? P[K] : Q[K];
      FindNext(&K,&Next);
    }
}

void main(void)
{
  InitData();
  TraverseTree();
  printf("%ld\n",P[Root]);
}
\end{lstlisting}
