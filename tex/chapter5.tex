\chapter{Despre algoritmi exponențiali și îmbunătățirea lor}

Trebuie să spunem de la bun început că cea mai bună îmbunătățire care i se
poate aduce unui algoritm în timp exponențial care rezolvă o anumită problemă
este evitarea lui, adică găsirea - acolo unde este posibil - a unui algoritm
polinomial care să rezolve aceeași problemă. Există cazuri în care acest lucru
nu este posibil; în această situație, însă, orice îmbunătățire nu este decât o
metodă de a ascunde gunoiul sub covor. Un algoritm exponențial rămâne
exponențial, iar îmbunătățirile aduse îl pot face să meargă de două, de trei,
de zece ori mai repede, dar nu-l pot transforma într-un algoritm
polinomial. Creșterea cu două - trei unități a dimensiunii datelor de intrare
va anihila saltul de la un calculator 486 la un Pentium.

În multe situații, nu se cunoaște nici un algoritm care să funcționeze în timp
util și să furnizeze soluția optimă a unei probleme. În asemenea cazuri, dacă
optimalitatea nu este strict necesară, se renunță la ea și se caută algoritmi
care să producă soluții cât mai apropiate de cea optimă și care să meargă mult
mai repede. Este intuitiv că, dacă impunem limite mai dure în ceea ce privește
timpul de rulare al algoritmului, vom avea șanse mai mici să găsim o soluție
apropiată de optim. Nu vom discuta în această carte despre cum se poate
realiza un echilibru între abaterea soluției de la optim și timpul necesar
pentru a o produce. Genul acesta de dileme apar și în cadrul concursului de
informatică, dar acolo timpul nu permite o analiză laborioasă a problemei. Noi
vom indica numai modul în care se poate „inventa” un algoritm polinomial în
locul unuia exponențial și o metodă prin care acest algoritm poate fi
îmbunătățit pentru ca rezultatele pe care acesta le scoate să nu fie departe
de adevăr. Aceasta este una din tendințele din ultimii ani de la concursurile
de informatică. Deoarece toți elevii cunosc problemele „clasice” care s-ar
putea da la concurs, se încearcă departajarea lor prin urmărirea modului în
care ei se adaptează la probleme fără o soluție eficientă cunoscută.

Să pornim de la o problemă celebră, cea a comis-voiajorului.

{\bf ENUNȚ}: Se dă un graf complet orientat cu $N \leq 30$ noduri. Fiecare
muchie are un cost cuprins între 1 și 100. Se cere să se determine un ciclu
hamiltonian de cost minim. Un ciclu hamiltonian este ciclul care parcurge
fiecare nod exact o dată. Costul unui ciclu este suma costurilor muchiilor
componente.

{\bf Intrarea}: Datele de intrare se găsesc în fișierul {\tt INPUT.TXT}, sub
următoarea formă:

\begin{verbatim}
  N
  A[1,1] A[1,2] ... A[1,N]
  ...
  A[N,1] A[N,2] ... A[N,N]
\end{verbatim}

unde {\tt A[i,j]} este lungimea muchiei care iese din nodul $i$ și intră în
nodul $j$. Se garantează că $A[i,i] = 0, \forall 1 \leq i \leq N$.

{\bf Ieșirea} se va face în fișierul {\tt OUTPUT.TXT} pe două linii. Pe prima
linie se va tipări costul minim găsit, iar pe a doua traseul parcurs ($N$
numere separate prin spații).

{\bf Exemplu}:

\texttt{
  \begin{tabular}{| l | l |}
    \hline
        {\bf INPUT.TXT} & {\bf OUTPUT.TXT} \\ \hline
        \begin{tabular}[t]{l}
          4\\
          0 3 4 1\\
          3 0 2 3\\
          5 2 0 6\\
          8 1 3 0
        \end{tabular}
        &
        \begin{tabular}[t]{l}
          9\\
          1 4 2 3
        \end{tabular}
        \\
    \hline
  \end{tabular}
}

{\bf Timp de execuție}: 30 secunde

{\bf Timp de implementare}: 30 minute

{\bf REZOLVARE}: Problema este arhicunoscută și de asemenea este arhicunoscut
faptul că ea nu admite o rezolvare polinomială. Totuși, dacă această problemă
vă este dată la un concurs, nu poate constitui o scuză în fața comisiei
argumentul că problema nu este polinomială. Ea trebuie făcută să meargă cât
mai bine. Spre deosebire de laboratoarele NASA, unde orice greșeală într-o
linie de cod poate distruge un modul spațial cu echipaj cu tot, la olimpiadă
nu se merge pe principiul „totul sau nimic”. Fiecare punct câștigat este bun
câștigat.

Precizăm că metodele de mai jos se aplică mai degrabă în cazurile în care se
cere o soluție optimă, dar se oferă punctaje parțiale și în cazul în care
concurentul oferă o soluție cât de cât apropiată de cea optimă. Există șanse
ca metodele de mai jos să asigure găsirea chiar a soluției optime pentru mare
parte din teste, dar aceste șanse variază de la o problemă la alta.

În primul rând, care este deosebirea fundamentală dintre un algoritm
backtracking și unul greedy? Algoritmul backtracking analizează pe rând
fiecare soluție posibilă și o alege pe cea mai bună. În felul acesta, el nu
poate scăpa soluția optimă. Din nefericire, în multe cazuri spațiul soluțiilor
crește exponențial cu dimensiunea datelor de intrare; în aceste situații
algoritmii backtracking nu mai sunt practici. În schimb, algoritmii greedy
(engl. {\it greedy = lacom}) fac o parcurgere a datelor de intrare și, la
fiecare pas al acestei parcurgeri, aleg o parte (destul de mică) din soluțiile
posibile, iar pe restul le „aruncă”. La pasul următor se aleg o parte din
soluțiile rămase și așa mai departe, până când în final rămâne o singură
soluție care se tipărește.

Criteriul în funcție de care se face trierea soluțiilor este cheia unui
algoritm greedy. Dacă acest criteriu poate garanta că la fiecare pas al
algoritmului soluția optimă (sau cel puțin una din soluțiile optime, dacă pot
exista mai multe) rămâne între soluțiile care sunt păstrate, atunci algoritmul
greedy funcționează perfect. Demonstrația este ușoară: soluția optimă nu este
„aruncată” niciodată, iar la sfârșitul algoritmului rămâne o singură soluție,
de unde rezultă că soluția rămasă este tocmai cea optimă. Asemenea cazuri de
algoritmi pentru care s-a demonstrat că ei funcționează sunt: algoritmii lui
Kruskal și Prim pentru găsirea arborelui parțial de cost minim al unui graf,
algoritmul lui Dijkstra pentru determinarea drumurilor de cost minim de la un
nod la toate celelalte într-un graf ș.a.m.d.

Putem extinde aceste noțiuni și la domeniul jocurilor logice. De exemplu,
jocul Nim (pe care probabil îl cunoașteți cu toții) are o strategie sigură de
câștig pentru anumite poziții, iar pentru celelalte se poate demonstra că nu
există nici o strategie de câștig. În cazul în care strategia există,
jucătorului i se oferă o mutare care îl duce spre victorie. Ce este în fond
această mutare? Tocmai un criteriu de a tria anumite configurații, favorabile
jucătorului, și de a le ignora pe celelalte.

Există însă probleme pentru care nu s-a găsit (iar uneori s-a și demonstrat că
nu există) nici un algoritm greedy. Continuând paralela cu jocurile logice,
există jocuri care nu au nici o strategie de câștig pentru unul din
jucători. Este cazul jocului de șah. Pentru unele asemenea probleme (cum ar fi
cea de față, a comis-voiajorului), care au aplicații largi în diferite domenii
practice, se investesc sume mari în cercetare pentru a se găsi algoritmi cât
mai buni care să funcționeze într-un timp convenabil.

O situație asemănătoare apare la concursul de informatică, unde se cunoaște de
la început timpul pus la dispoziție (atât cel pentru implementare, cât și cel
pentru execuție) și se urmărește obținerea unui punctaj cât mai mare. Iată
câteva metode destul de eficiente.

\section{„Omorârea” backtracking-ului}

Sintagma aceasta oarecum ilară, care stârnește mila pentru bietul
backtracking, se aude foarte des pe la ieșirea din sălile de concurs, atunci
când problema a fost mai „ciudată”, în sensul că foarte puțini concurenți au
descoperit vreo soluție eficientă la ea. Ce este de fapt „backtracking-ul
omorât” și în ce situații este el preferabil?

Problema comis-voiajorului sub diverse forme sau alte probleme exponențiale au
fost propuse în anii trecuți spre rezolvare la concursuri. Dacă vrem să aflăm
soluția optimă (de cost minim), neavând altă soluție la îndemână, trebuie să
recurgem la backtracking. Backtracking-ul, după cum se știe, examinează pe
rând fiecare posibilă soluție. În cazul nostru, backtracking-ul nu are altceva
de făcut decât să genereze pe rând toate permutările mulțimii ${1, 2, \dots,
  N}$. Considerând fiecare permutare ca fiind un posibil ciclu hamiltonian
(știm sigur că oricărei permutări îi corespunde un ciclu hamiltonian, deoarece
graful este complet), mai trebuie doar să calculăm costul fiecărei permutări
și să o afișăm pe cea de cost minim. Până aici, nimic deosebit. Iată și sursa
Pascal:

\begin{lstlisting}[language=Pascal]
program Hamilton;
{$B-,I-,R-,S-}
const NMax=30;
type Vector=array[1..NMax] of Integer;
     Matrix=array[1..NMax,1..NMax] of Integer;
var A:Matrix;
    Route,BestRoute:Vector;
    Seen:set of 1..NMax;
    N,Cost,MinCost:Integer;

procedure ReadData;
var i,j:Integer;
begin
  Assign(Input,'input.txt');Reset(Input);
  ReadLn(N);
  for i:=1 to N do
    begin
      for j:=1 to N do Read(A[i,j]);
      ReadLn;
    end;
  Close(Input);
end;

procedure Bkt(Level,Cost:Integer);
var i:Integer;
begin
  if Level=N+1
    then begin
           Inc(Cost,A[Route[N],1]);
           if Cost<MinCost
             then begin
                    BestRoute:=Route;
                    MinCost:=Cost;
                  end;
         end
    else if Cost<MinCost
           then for i:=1 to N do
             if not (i in Seen)
               then begin
                      Seen:=Seen+[i];
                      Route[Level]:=i;
                      Bkt(Level+1,Cost+A[Route[Level-1],i]);
                      Seen:=Seen-[i];
                    end;
end;

procedure WriteSolution;
var i:Integer;
begin
  Assign(Output,'output.txt');Rewrite(Output);
  WriteLn(MinCost);
  for i:=1 to N do Write(BestRoute[i],' ');
  WriteLn;
  Close(Output);
end;

begin
  ReadData;
  Route[1]:=1;
  Seen:=[1];
  MinCost:=MaxInt;
  Bkt(2,0);
  WriteSolution;
end.
\end{lstlisting}

Programul sub această formă nu se încadrează în timp nici măcar pentru $N=15$
(cifra depinde și de calculatorul folosit pentru testare, dar nu variază cu
mai mult de două-trei nivele; să spunem cu generozitate că pe un calculator
performant programul ar putea merge până la $N=18$ sau 20). Pe de altă parte,
ideea în sine (algoritmul) de rezolvare nu mai poate fi mult îmbunătățită. Și
cu toate acestea, programul trebuie să meargă până la $N=30$. Ce putem face?

Desigur, nu există o rezolvare elegantă. Putem însă încerca fel de fel de
metode de a trișa. Deoarece nu avem timp să examinăm toate soluțiile, trebuie
să renunțăm la o parte din ele, cu riscul ca printre ele să se afle tocmai
soluția căutată. Una dintre tehnici este omorârea backtracking-ului. După
numărul și tipurile soluțiilor pe care le cer, algoritmii backtracking ar
putea fi împărțiți în mai multe categorii:

\begin{itemize}

\item Cei care furnizează o singură soluție;

\item Cei care, pe baza unei funcții care atașează un cost fiecărei soluții,
  furnizează soluția de cost minim;

\item Cei care furnizează toate soluțiile.
\end{itemize}

Backtracking-ul omorât se aplică celui de-al doilea tip de cerințe. Putem face
în așa fel încât să oprim programul exact la expirarea timpului permis pentru
rulare și să afișăm cea mai bună soluție găsită până la momentul
respectiv. Dacă am fost norocoși (termenul este cel mai potrivit în această
situație), atunci programul nostru a apucat, în timpul pe care i l-am permis,
să găsească soluția optimă. Dacă nu, putem totuși spera că a fost găsită o
soluție cât de cât apropiată de cea optimă, pentru care putem eventual să
primim măcar o parte din punctaj. După cum se vede, omorârea backtracking-ului
nu promite marea cu sarea, dar este un artificiu binevenit, pentru că există
trei variante:

\begin{itemize}

\item Programul se încadrează în timp, caz în care backtracking-ul omorât nu
  aduce nimic în plus;

\item Programul nu se încadrează în timp, dar soluția optimă este găsită în
  timp, caz în care backtracking-ul simplu nu furnizează nici o soluție (și de
  obicei este oprit cu Ctrl-Break), pe când backtracking-ul omorât furnizează
  soluția;

\item Programul nu se încadrează în timp și nici soluția optimă nu este găsită
  în timp, caz în care backtracking-ul simplu nu furnizează nici o soluție, pe
  când backtracking-ul omorât furnizează o soluție eventual apropiată de
  optim.

\end{itemize}

Din experiență se poate spune că membrii comisiei de corectare acordă jumătate
din punctele pentru un test mai degrabă atunci când li se oferă o soluție
neoptimă decât atunci când li se oferă o soluție optimă într-un timp
depășit. Adesea programul este oprit îndată ce timpul de rulare expiră, și
părerea autorului e că e mai bine așa.

„Omorârea” nu se pretează la celelalte două versiuni de backtracking. Atunci
când există o singură soluție, nu se mai pune problema de a găsi una apropiată
de ea. Ori găsim soluția, ori nimic. De exemplu, nu are sens să rezolvăm
problema de mai sus cu un backtracking omorât dacă știm sigur că nu se acordă
punctaje parțiale pentru soluții neoptime (dar în general acest lucru nu se
știe sigur...). Rămâne bineînțeles posibilitatea de a opri programul imediat
ce soluția a a fost găsită. În cazul în care se cer toate soluțiile, de
asemenea programul nu poate fi oprit. Atunci când se poate, merită calculat
numărul de soluții, pentru ca imediat ce am găsit toate soluțiile, să oprim
programul. Putem aplica backtracking-ul omorât numai dacă știm că se acordă
punctaj și pentru afișarea unei părți din soluții.

Mai rămâne de stabilit cum anume se face „omorârea” backtracking-ului (și de
fapt a oricărui program). Există mai multe metode. Prima, asupra căreia nu vom
insista deoarece ea are „efecte secundare” și, în plus, este foarte lentă,
constă din două etape:

\begin{itemize}

\item Se setează ceasul sistem la ora 00:00:00 (miezul nopții) cu procedura
  {\tt SetTime} din unitatea Dos a compilatorului Borland Pascal;

\item Periodic se testează ora cu procedura {\tt GetTime} și se oprește
  programul atunci când se apropie „ora critică” (în cazul nostru 00:00:30).

\end{itemize}

După cum se vede, marele neajuns al acestei metode este că „dă peste cap”
ceasul sistem (lucru dezastruos mai ales pentru cei care vin la concurs fără
ceas...). În afară de aceasta, procedurile {\tt GetTime} și {\tt SetTime}
apelează la rândul lor întreruperile DOS, ceea ce consumă mult din timpul care
și așa este limitat.

A doua metodă, care este mai rapidă și nu lasă urme, constă în captarea {\bf
  întreruperii 8}, adică a {\bf timer}-ului. Timer-ul este o rutină care se
apelează automat la fiecare 55 de milisecunde, deci cam de 18,2 ori pe
secundă. În principiu, ea nu face nimic altceva decât să incrementeze ceasul
sistem cu 55 ms. Pe lângă aceasta, însă, putem adăuga și propriul nostru cod,
folosind procedurile Pascal {\tt GetIntVec} și {\tt SetIntVec}. Trebuie doar
să avem grijă ca timer-ul scris de noi să-l apeleze și pe cel vechi, altfel
ceasul sistem se va opri și cine știe ce altceva se mai poate întâmpla. Vom
declara deci o variabilă {\tt Time} care va fi decrementată la fiecare apel al
întreruperii de ceas. Înainte de a intercepta întreruperea 8, vom inițializa
variabila cu valoarea maximă dorită. Știm că timer-ul se apelează de 18,2 ori
pe secundă, deci dacă limita de timp pentru un test este de 30 de secunde,
valoarea inițială pentru variabila {\tt Time} ar putea fi $18,2 \times 30 =
546$. Este bine să nu calculăm însă timpul la limită, deoarece avem nevoie de
câteva fracțiuni și pentru tipărirea soluției în fișier, și poate pur și
simplu ceasul comisiei de corectare o ia puțin înainte. De aceea, e mai sigură
înmulțirea cu 17 în loc de 18,2.

În momentul în care, prin decrementări succesive, {\tt Time} a ajuns la
valoarea 0 (sau la o valoare negativă), programul trebuie oprit. Acest lucru
presupune ieșirea din procedura de backtracking, afișarea soluției și
restaurarea vechii întreruperi 8, pentru ca programul să nu lase „urme”. Iată
deci o versiune a programului Pascal care va fi extrem de punctuală...

\begin{lstlisting}[language=Pascal]
program Hamilton;
{$B-,I-,R-,S-}
uses Dos;
const NMax=30;
      TimeLimit=30; { secunde }

type Vector=array[1..NMax] of Integer;
     Matrix=array[1..NMax,1..NMax] of Integer;
var A:Matrix;
    Route,BestRoute:Vector;
    Seen:set of 1..NMax;
    N,Cost,MinCost:Integer;
    Time:Integer;  { Contorul }
    OldTimer:procedure;

procedure MyTimer; interrupt;
{ Se executa la fiecare 55 ms }
begin
  Dec(Time);   { Ne facem treaba... }
  Inline($9C); { ...pushf... }
  OldTimer;    { ...si executam si vechiul timer }
end;

procedure ReadData;
var i,j:Integer;
begin
  Assign(Input,'input.txt');Reset(Input);
  ReadLn(N);
  for i:=1 to N do
    begin
      for j:=1 to N do Read(A[i,j]);
      ReadLn;
    end;
  Close(Input);
end;

procedure Bkt(Level,Cost:Integer);
var i:Integer;
begin
  if Level=N+1
    then begin
           Inc(Cost,A[Route[N],1]);
           if Cost<MinCost
             then begin
                    BestRoute:=Route;
                    MinCost:=Cost;
                  end;
         end
    else if (Time>0) and (Cost<MinCost)
           then for i:=1 to N do
             if not (i in Seen)
               then begin
                      Seen:=Seen+[i];
                      Route[Level]:=i;
                      Bkt(Level+1,Cost+A[Route[Level-1],i]);
                      Seen:=Seen-[i];
                    end;
end;

procedure WriteSolution;
var i:Integer;
begin
  Assign(Output,'output.txt');Rewrite(Output);
  WriteLn(MinCost);
  for i:=1 to N do Write(BestRoute[i],' ');
  WriteLn;
  Close(Output);
end;

begin
  Time:=TimeLimit*17;
  { Captam intreruperea 8 (timer-ul) }
  GetIntVec(8,@OldTimer);
  SetIntVec(8,@MyTimer);
  ReadData;
  Route[1]:=1;
  Seen:=[1];
  MinCost:=MaxInt;
  Bkt(2,0);
  WriteSolution;
  { Restauram timer-ul }
  SetIntVec(8,@OldTimer);
end.
\end{lstlisting}

Și această a doua metodă are neajunsurile ei, deoarece presupune scrierea a
destul de multe linii de program în plus. În afară de aceasta, instrucțiunea
de decrementare a variabilei {\tt Time}, precum și apelul suplimentar de
procedură din cadrul întreruperii de ceas mai reduc puțin timpul dedicat
calculelor efective. A treia variantă elimină și aceste deficiențe. Ea se
bazează pe accesarea directă a locației de memorie \$0000:\$046C, unde se
află, reprezentat pe 4 octeți, numărul de apeluri ale timer-ului (numărul de
{\bf tacți}) începând de la miezul nopții. Dacă declarăm o variabilă {\tt
  Time} de tip {\tt Longint} (deoarece acest tip de date ocupă 4 octeți) exact
la această adresă, folosind clauza Pascal {\tt absolute}, variabila se va
incrementa la fiecare 55ms, scutindu-ne pe noi de această grijă. Dacă înmulțim
variabila cu 55/1000, aflăm exact numărul de secunde scurse de la miezul
nopții. Dacă împărțim acest rezultat la 3600, putem afla ora exactă
ș.a.m.d. Lucrul care ne interesează pe noi este să setăm o „alarmă” care să
oprească programul peste 30 de secunde. 30 de secunde înseamnă $30 \times
18,2$ incrementări ale variabilei {\tt Time}. Folosind în loc de 18,2 valoarea
17 (pentru a păstra o rezervă), rezultă că trebuie să ne oprim atunci când
{\tt Time} are o valoare cu $30 \times 17$ mai mare decât la intrarea în
program. Primul lucru pe care îl va face programul va fi să dea unei variabile
{\tt Alarm} valoarea {\tt Time} + $30 \times 17$. Periodic (cel mai comod la
intrarea în procedura backtracking) se va testa valoarea variabilei {\tt Time}
și atunci când ea este egală cu {\tt Alarm}, se va ieși din program.

\begin{lstlisting}[language=Pascal]
program Hamilton;
{$B-,I-,R-,S-}
const NMax=30;
      TimeLimit=30; { secunde }

type Vector=array[1..NMax] of Integer;
     Matrix=array[1..NMax,1..NMax] of Integer;
var A:Matrix;
    Route,BestRoute:Vector;
    Seen:set of 1..NMax;
    N,Cost,MinCost:Integer;
    Time:LongInt absolute $0000:$046C;
    Alarm:LongInt;

procedure SetAlarm;
begin
  Alarm:=Time+TimeLimit*17;
  { Cifra corecta era nu 17, ci 18.2;
    am pastrat insa o rezerva de siguranta }
end;

procedure ReadData;
var i,j:Integer;
begin
  Assign(Input,'input.txt');Reset(Input);
  ReadLn(N);
  for i:=1 to N do
    begin
      for j:=1 to N do Read(A[i,j]);
      ReadLn;
    end;
  Close(Input);
end;

procedure Bkt(Level,Cost:Integer);
var i:Integer;
begin
  if Level=N+1
    then begin
           Inc(Cost,A[Route[N],1]);
           if Cost<MinCost
             then begin
                    BestRoute:=Route;
                    MinCost:=Cost;
                  end;
         end
    else if (Time<Alarm) and (Cost<MinCost)
           then for i:=1 to N do
             if not (i in Seen)
               then begin
                      Seen:=Seen+[i];
                      Route[Level]:=i;
                      Bkt(Level+1,Cost+A[Route[Level-1],i]);
                      Seen:=Seen-[i];
                    end;
end;

procedure WriteSolution;
var i:Integer;
begin
  Assign(Output,'output.txt');Rewrite(Output);
  WriteLn(MinCost);
  for i:=1 to N do Write(BestRoute[i],' ');
  WriteLn;
  Close(Output);
end;

begin
  SetAlarm;
  ReadData;
  Route[1]:=1;
  Seen:=[1];
  MinCost:=MaxInt;
  Bkt(2,0);
  WriteSolution;
end.
\end{lstlisting}

Singura problemă pe care o poate ridica această ultimă versiune este
următoarea: dacă programul este lansat în execuție la un moment foarte
apropiat de miezul nopții, atunci variabila {\tt Alarm} va avea o valoare mai
mare decât numărul de tacți dintr-o zi. Variabila {\tt Time} nu va ajunge
niciodată la această valoare, deoarece la miezul nopții ea va lua din nou
valoarea 0. Este totuși puțin probabil să vă fie corectat programul la miezul
nopții...

\section{Greedy euristic}

După cum am spus, la unii algoritmi greedy, criteriul de departajare
garantează că soluția optimă nu este niciodată scăpată din vedere. De și mai
multe ori, totuși, criteriile de departajare nu pot promite acest lucru; în
general elevii, la ieșirea din sălile de concurs, în cazul unei probleme mai
controversate, își expun părerile și ideile față de colegii lor, apoi fiecare
îi demonstrează celuilalt că algoritmul propus de el nu merge, prezentându-i
un contraexemplu. Momentele cele mai picante se produc atunci când algoritmul
pare să nu fie corect, dar nici nu se poate găsi un contraexemplu.

În aceste situații, criteriile de departajare a soluțiilor la algoritmii
greedy se numesc {\bf funcții euristice}, iar algoritmul în sine se numește
{\bf greedy euristic} (în greaca veche, {\it heuriskein} însemna {\it a
  afla}). Dacă nu poate promite optimalitatea, funcția euristică trebuie în
orice caz aleasă cât mai bine, respectiv trebuie să aibă șanse cât mai mari să
rețină soluția optimă, sau măcar să rețină la fiecare pas soluții cât mai
apropiate de cea optimă.

Un singur algoritm greedy euristic are șanse mici să găsească soluția
optimă. Dar algoritmii greedy euristici au unele proprietăți interesante:

\begin{itemize}

\item Se încadrează cu ușurință în timpul de rulare.

\item Sunt ușor de implementat.

\item O modificare cât de mică a funcției euristice poate modifica radical
  algoritmul și soluția furnizată de el. Deoarece nu avem de unde ști care
  dintre funcțiile euristice este mai bună (acest lucru depinde de datele pe
  care este testată problema), ideal este să reținem ambele soluții furnizate
  și să o alegem pe cea mai bună.

\item De multe ori, datele de intrare sunt vectori sau matrice; în unele
  situații, sensul în care sunt ele parcurse pentru determinarea soluției nu
  este important. Schimbând sensul de parcurgere, obținem de asemenea două
  soluții distincte pe care le putem compara.

\item Aproape întotdeauna, funcțiile euristice conțin teste de genul:

\begin{lstlisting}[language=Pascal]
if A<B then Actiune1 else Actiune2;
\end{lstlisting}

Din punct de vedere logic, dacă {\tt A=B} se poate executa oricare din cele
două acțiuni. Condiția {\tt A<B} este echivalentă cu condiția {\tt
  A<=B}. Totuși, din punct de vedere al calculatorului, cele două condiții
sunt absolut diferite și pot produce soluții cu totul diferite. Iată de
exemplu, două rutine care caută poziția $k$ pe care se află elementul minim
într-un vector $V$ cu $N$ elemente:

\begin{lstlisting}[language=Pascal]
k:=1;
for i:=2 to N do
  if V[i]<V[k] then k:=i;
\end{lstlisting}

respectiv

\begin{lstlisting}[language=Pascal]
k:=1;
for i:=2 to N do
  if V[i]<=V[k] then k:=i;
\end{lstlisting}

Cele două versiuni vor găsi într-adevăr un indice $k$ astfel încât $V[k]$ să
fie minim. Totuși, dacă există mai multe elemente de valoare minimă, atunci
prima versiune va întoarce indicele cel mai mic, pe când ultima îl va întoarce
pe cel mai mare.

\end{itemize}

Iată un exemplu de funcție euristică pentru problema comis-voiajorului: pornim
din nodul 1 și, la fiecare pas, ne deplasăm în cel mai apropiat nod care nu a
fost vizitat încă. După ce toate nodurile au fost vizitate, trebuie numai să
ne deplasăm din ultimul nod vizitat în nodul 1. Luată în sine, această
euristică nu este strălucită. Ea poate însă să fie „clonată” într-o
multitudine de variante. În primul rând că nu este obligatoriu să pornim din
nodul 1. Putem aplica același algoritm pornind pe rând din fiecare nod; la
sfârșit tipărim soluția de cost minim. Prima variantă a programului Pascal
este:

\begin{lstlisting}[language=Pascal]
program Hamilton;
{$B-,I-,R-,S-}
const NMax=30;
type Vector=array[1..NMax] of Integer;
     Matrix=array[1..NMax,1..NMax] of Integer;
var A:Matrix;
    Route,BestRoute:Vector;
    N,Cost,MinCost,i:Integer;

procedure ReadData;
var i,j:Integer;
begin
  Assign(Input,'input.txt');Reset(Input);
  ReadLn(N);
  for i:=1 to N do
    begin
      for j:=1 to N do Read(A[i,j]);
      ReadLn;
    end;
  Close(Input);
end;

procedure Greedy1(Start:Integer;var R:Vector;
                  var Cost:Integer);
var i,j,Closest:Integer;
    Seen:set of 1..NMax;
begin
  R[1]:=Start;
  Cost:=0;
  Seen:=[Start];
  for i:=2 to N do
    begin
      { Cauta nodul cel mai apropiat }
      Closest:=MaxInt;
      for j:=1 to N do
        if (not (j in Seen)) and (A[R[i-1],j]<Closest)
          then begin
                 Closest:=A[R[i-1],j];
                 R[i]:=j;
               end;
      Inc(Cost,Closest);
      Seen:=Seen+[R[i]];
    end;
  { Inchide ciclul }
  Inc(Cost,A[R[N],Start]);
end;

procedure Update;
begin
  if Cost<MinCost
    then begin
           MinCost:=Cost;
           BestRoute:=Route;
         end;
end;

procedure WriteSolution;
var i:Integer;
begin
  Assign(Output,'output.txt');Rewrite(Output);
  WriteLn(MinCost);
  for i:=1 to N do Write(BestRoute[i],' ');
  WriteLn;
  Close(Output);
end;

begin
  ReadData;
  MinCost:=MaxInt;
  for i:=1 to N do
    begin
      Greedy1(i,Route,Cost);
      Update;
    end;
  WriteSolution;
end.
\end{lstlisting}

De sine stătătoare, funcția euristică nu este strălucită. Totuși, ea poate fi
lesne modificată. Se observă că, în procedura {\tt Greedy1}, instrucțiunea

\begin{lstlisting}[language=Pascal]
for j:=1 to N do...
\end{lstlisting}

poate fi înlocuită cu

\begin{lstlisting}[language=Pascal]
for j:=N downto 1 do...
\end{lstlisting}

, iar condiția

\begin{lstlisting}[language=Pascal]
(A[R[i-1],j] < Closest)
\end{lstlisting}

cu

\begin{lstlisting}[language=Pascal]
(A[R[i-1],j]<=Closest)
\end{lstlisting}

Făcând toate combinațiile posibile, rezultă alte trei proceduri, {\tt
  Greedy2}, {\tt Greedy3} și {\tt Greedy4}, iar noua formă a programului
principal este:

\begin{lstlisting}[language=Pascal]
begin
  ReadData;
  MinCost:=MaxInt;

  for i:=1 to N do
    begin
      Greedy1(i,Route,Cost);
      Update;
      Greedy2(i,Route,Cost);
      Update;
      Greedy3(i,Route,Cost);
      Update;
      Greedy4(i,Route,Cost);
      Update;
    end;

  WriteSolution;
end.
\end{lstlisting}

După cum se vede, adaosul de proceduri face ca sursa să atingă dimensiuni
impunătoare, dar efortul necesar pentru a o scrie este aproape aceeași ca și
când ar fi existat o singură funcție euristică. Practic, trebuie scrisă una
singură din cele patru funcții, restul rezumându-se la copierea unor blocuri
cu ajutorul editorului Borland Pascal.

\section{Decizia între greedy euristic și backtracking}

Pentru a ne asigura și mai multe puncte din cele puse în joc, putem încerca
următoarea combinație: pentru grafuri mici, care pot fi examinate exhaustiv în
timp de câteva secunde, vom apela la algoritmul backtracking pentru rezolvarea
problemei. Numai pentru valori mari, pentru care știm sigur că backtracking-ul
depășește timpul admis, vom apela la funcțiile euristice. Ce înseamnă valori
„mici” si „mari” se poate aproxima sau se poate determina după câteva
teste. Aceste valori depind de problemă și de mașina folosită.

Deoarece primele teste pentru fiecare problemă (uneori o treime sau chiar
jumătate din ele) sunt de dimensiuni mici, e bine dacă vi le puteți asigura
printr-un backtracking care de regulă se implementează în 15-20 minute.

\section{Combinația greedy euristic + backtracking}

Urmărind prima rezolvare de la punctul (1) - varianta backtracking fără nici
un fel de modificări - se observă că variabila {\tt MinCost} se inițializează
cu valoarea {\tt MaxInt}. În felul acesta, prima soluție găsită este implicit
cea mai bună și durează o vreme până când rezultatele încep să se apropie de
optim. Pe de altă parte, evaluarea unui anumit lanț din graf și prelungirea
lui cu noi noduri până la închiderea ciclului hamiltonian nu se fac decât dacă
costul lanțului nu a depășit deja valoarea {\tt MinCost}. De aici provine
întrebarea firească: ce-ar fi dacă, în loc să inițializăm variabila {\tt
  MinCost} cu valoarea {\tt MaxInt}, am lansa mai întâi unul sau mai multe
greedy-uri euristice (depinde câte apucăm să scriem) pentru a da o valoare mai
apropiată de adevăr variabilei {\tt MinCost} ? Sigur, cu o floare nu se face
primăvară, dar în cazul nostru se pot câștiga secunde prețioase. Se poate de
asemenea ca după aceste greedy-uri să apelăm nu un backtracking simplu, ci
unul omorât prin orice metodă, caz în care șansele se îmbunătățesc
considerabil. Programul principal ar putea fi atunci:

\begin{lstlisting}[language=Pascal]
begin
  SetAlarm;
  ReadData;
  MinCost:=MaxInt;
  for i:=1 to N do
    begin
      Greedy1(i,Route,Cost);
      Update;
      Greedy2(i,Route,Cost);
      Update;
      Greedy3(i,Route,Cost);
      Update;
      Greedy4(i,Route,Cost);
      Update;
    end;
  Route[1]:=1;
  Seen:=[1];
  Bkt(2,0);
  WriteSolution;
end.
\end{lstlisting}

\section{Testarea aleatoare a posibilităților}

Oricât ar părea de ciudat, și aceasta este o cale de a ieși din
încurcătură. Ce-i drept, nu cea mai eficientă, dar atunci când imaginația vă
joacă feste iar backtracking-ul nu vă surâde, puteți încerca chiar și o
rezolvare care se bazează puternic pe funcția {\tt Random}. În acest caz, tot
ce aveți de făcut este să generați aleator cicluri hamiltoniene și să-i
calculați fiecăruia costul. La sfârșit îl tipăriți pe cel de cost minim
găsit. Bineînțeles, oprirea programului se va face printr-o „omorâre” de orice
tip. Timpul de implementare al unei asemenea rezolvări este de ordinul
minutelor. Această versiune găsește uneori soluția optimă, dar alteori este
foarte departe de ea. De asemenea, are marele dezavantaj că la două rulări
consecutive nu generează același rezultat, deoarece procedura {\tt Randomize}
își extrage variabila {\tt RandSeed} (folosită pentru a genera numere
aleatoare) pe baza timer-ului... Personal nu o recomand, dar este destul de
des folosită pe la concursuri. Și este momentul să amintim o urare ce li se
adresează concurenților care intră în sala de corectare, respectiv „Să fie
într-un timer bun!”.

\begin{lstlisting}[language=Pascal]
program Hamilton;
{$B-,I-,R-,S-}
const NMax=30;
      TimeLimit=30; { secunde }

type Vector=array[1..NMax] of Integer;
     Matrix=array[1..NMax,1..NMax] of Integer;
var A:Matrix;
    Route,BestRoute:Vector;
    Seen:set of 1..NMax;
    N,Cost,MinCost:Integer;
    Time:LongInt absolute $0000:$046C;
    Alarm:LongInt;

procedure SetAlarm;
begin
  Alarm:=Time+TimeLimit*17;
  { Cifra corecta era nu 17, ci 18.2;
    am pastrat insa o rezerva de siguranta }
end;

procedure ReadData;
var i,j:Integer;
begin
  Assign(Input,'input.txt');Reset(Input);
  ReadLn(N);
  for i:=1 to N do
    begin
      for j:=1 to N do Read(A[i,j]);
      ReadLn;
    end;
  Close(Input);
end;

procedure RandomCycle;
var i,j:Integer;
begin
  Route[1]:=Random(N)+1;
  Seen:=[Route[1]];
  Cost:=0;
  for i:=2 to N do
    begin
      repeat Route[i]:=Random(N)+1;
      until not (Route[i] in Seen);
      Seen:=Seen+[Route[i]];
      Inc(Cost,A[Route[i-1],Route[i]]);
    end;
  Inc(Cost,A[Route[N],Route[1]]);
end;

procedure Update;
begin
  if Cost<MinCost
    then begin
           MinCost:=Cost;
           BestRoute:=Route;
         end;
end;

procedure WriteSolution;
var i:Integer;
begin
  Assign(Output,'output.txt');Rewrite(Output);
  WriteLn(MinCost);
  for i:=1 to N do Write(BestRoute[i],' ');
  WriteLn;
  Close(Output);
end;

begin
  SetAlarm;
  Randomize;
  ReadData;
  MinCost:=MaxInt;
  while Time<Alarm do
    begin
      RandomCycle;
      Update;
    end;
  WriteSolution;
end.
\end{lstlisting}
