\Chapter{Concursul de informatică}{- de la extaz la agonie -}

 % ommit subtitle from page headers
\markboth{Capitolul 1. Concursul de informatică}{}

\begin{quote}
  Cine dorește să-și rezolve treburile la vremea potrivită, să-și împartă
  cu atenție timpul. \\
  \attrib{Plaut}
\end{quote}

Experiența demonstrează că, oricât de mare ar fi bagajul de cunoștințe
acumulat de un elev, mai e nevoie de {\it ceva} pentru a-i asigura succesul la
olimpiada de informatică. Aceasta deoarece în timp de concurs lucrurile stau
cu totul altfel decât în fața calculatorului de acasă sau de la
școală. Reușita depinde, desigur, în cea mai mare măsură de puterea fiecăruia
de a pune în practică ceea ce a învățat acasă. Numai că în acest proces
intervin o serie de factori care țin de temperament, de experiența
individuală, de numărul de ore dormite în noaptea dinaintea concursului (care
în taberele naționale este îngrijorător de mic) și așa mai departe.

Cu riscul de a cădea în demagogie, trebuie să spunem că un concurs de
informatică presupune mult mai mult decât un simplu act de prezență la locul
desfășurării ostilităților. Capitolul de față încearcă să enunțe câteva
principii ale concursului, pe care autorul și le-a însușit în cei patru ani de
liceu, atât din experiența proprie, cât și învățând de la alții. Cititorul
este liber să respingă aceste sfaturi sau să le accepte, filtrându-le prin
prisma personalității sale și alegând ceea ce i se potrivește.

\section{Înainte de concurs}

Primul și cel mai de seamă lucru pe care trebuie să îl știți este că e
important și să participi, dar e și mai important să participi onorabil, iar
dacă se poate să și câștigi. Nu trebuie să porniți la drum cu îngâmfare;
modestia e bună, dar nu trebuie în nici un caz să ducă la neîncredere în
sine. Fiecare trebuie să știe clar de ce e în stare și, mai presus de toate,
să se gândească că la urma urmei nu dificultatea concursului contează, căci
concursul, greu sau ușor, este același pentru toți. Mult mai importantă este
valoarea individuală și nu în ultimul rând pregătirea psihologică.

Autorul a fost peste măsură de surprins să constate că mulți elevi merg la
concurs fără ceas și fără hârtie de scris. Aceasta este fără îndoială o
greșeală capitală. În timpul concursului trebuie ținută o evidență drastică a
timpului scurs și a celui rămas. E drept că în general supraveghetorii anunță
din când în când timpul care a trecut, dar e bine să nu vă bazați pe nimeni și
nimic altceva decât pe voi înșivă. Unii colegi spuneau „Ei, ce nevoie am de
ceas, oricum am ceasul calculatorului la îndemână”. Așa e, dar e extrem de
incomod să te oprești mereu la jumătatea unei idei, să deschizi o sesiune DOS
din cadrul limbajului de programare și să afli cât e ceasul.

În ceea ce privește hârtia de scris, ea este în mod sigur necesară. De fapt, o
parte importantă a rezolvării unei probleme este proiectarea matematică a
algoritmului, lucru care nu se poate face decât cu creionul pe hârtie. Pe
lângă aceasta, majoritatea problemelor operează cu vectori, matrice, arbori,
grafuri etc., iar exemplele pe care este testat programul realizat trebuie
neapărat verificate „de mână”. E de preferat să aveți hârtie de matematică;
este foarte folositoare pentru problemele de geometrie analitică, precum și
pentru reprezentarea matricelor. Cantitatea depinde de imaginația
fiecăruia. În unele cazuri speciale, autorului i s-a întâmplat să umple 7-8
coli A4.

\section{În timpul concursului}

Din fericire pentru unii și din nefericire pentru alții, majoritatea
examenelor îți cer să dovedești nu că ești bine pregătit, ci că ești mai bine
pregătit decât alții. Aceasta înseamnă că și la olimpiada de informatică se
aplică legea peștelui mai mare sau, cum i se mai spune, a
concurenței. Valoarea absolută a fiecăruia nu contează chiar în totalitate,
ceea ce constituie sarea și piperul concursului. Într-adevăr, ce farmec ar
avea să mergi la un concurs la care se știe încă dinainte cine este cel mai
bun ? Este destul de amuzant să observi cum fiecare speră să prindă „o zi
bună”, iar adversarii săi „o zi proastă”.

Este ușor să fii printre cei mai buni atunci când concursul este ușor. Mai
greu e să fii cel mai bun atunci când concursul este dur, pentru că atunci
intervine - inevitabil - dramul de noroc al fiecăruia. Niciodată însă nu se
poate invoca greutatea concursului drept o scuză pentru un eventual
eșec. Concursul este la fel de greu pentru toți. Se poate întâmpla, mai ales
dacă probele durează mai multe zile (3-4) ca nici unul din concurenți să nu
acumuleze mai mult de 70-80\% din punctajul maxim. Totuși, aceasta nu înseamnă
că ei nu sunt bine pregătiți; mai mult, unul dintre ei trebuie să fie
primul. Așadar, niciodată nu trebuie adoptată o strategie de genul „problema
asta e grea și n-am s-o pot rezolva perfect, așa că nu mă mai apuc deloc de
ea”. Nu trebuie să vă impacientați dacă vi se întâmplă să nu aveți o idee
genială de rezolvare a unei probleme. Nu vă cere nimeni să faceți perfect o
problemă, ci numai să prezentați o soluție care să acumuleze cât mai multe
puncte. Evident, prima variantă este întotdeauna preferabilă, dar nu
obligatorie.

De multe ori se întâmplă ca un elev să găsească o soluție cât de cât bună la o
problemă și, măcar că știe că nu va lua punctajul maxim, ci doar o parte, să
renunțe să caute o soluție mai eficientă, deoarece timpul pierdut astfel ar
aduce un câștig prea mic și ar putea fi folosit la rezolvarea altor
probleme. Desigur, dacă nu faci toate problemele perfect, nu mai poți fi sigur
de premiul I, pentru că altcineva poate să te întreacă. Dar pe de altă parte,
locul pe care te clasezi contează numai la etapa națională a olimpiadei sau la
concursurile internaționale. În rest, important e numai să te califici, adică
să intri în primele câteva locuri.

Feriți-vă ca de foc de criza de timp. E mare păcat să ratezi o problemă
întreagă pentru că n-ai avut timp să scrii procedura de afișare a
soluției. Rezervați-vă întotdeauna timpul pe care îl socotiți necesar pentru
implementare și depanare.

Niciodată, chiar dacă concursul este ușor, nu e bine să ieșiți din sala de
concurs înainte de expirarea timpului. Oricât ați fi de convinși că ați făcut
totul perfect, mai verificați-vă; veți avea de furcă cu remușcările dacă
descoperiți după aceea că ceva, totuși, nu a mers bine. Puteți face o mulțime
de lucruri dacă mai aveți timp (deși acest lucru se întâmplă rar). Iată o
serie de metode de a exploata timpul:

\begin{itemize}

\item Verificați-vă programul cu cât mai multe teste de mici dimensiuni. Să
  presupunem că programul vostru lucrează cu vectori de maxim 10000 de
  elemente. E o idee bună să îl rulați pentru vectori de unul sau două
  elemente. Nu se știe cum pot să apară erori.

\item Treceți la polul opus și creați-vă un test de dimensiune maximă, dar cu
  o structură particulară, pentru care este ușor de calculat rezultatul și de
  mână. De exemplu, vectori de 10000 elemente cu toate elementele egale, sau
  vectori de forma (1, 2, ..., 9999, 10000). Dacă nu puteți să editați un
  asemenea fișier de mână, copiind și multiplicând blocuri, puteți scrie un
  program care să-l genereze.

\item Dacă încă v-a mai rămas timp, creați-vă un program care să genereze
  teste aleatoare. Spre exemplu, un program care să citească un număr $N$ și
  să creeze un fișier {\tt INPUT.TXT} în care să scrie $N$ numere
  aleatoare. Într-o primă fază, puteți folosi aceste teste pentru a verifica
  dacă nu cumva la valori mai mari programul nu dă eroare, nu se blochează (la
  alocarea unor zone mari de memorie) sau nu depășește limita de timp, caz în
  care mai aveți de lucru.
  
\item Dacă tot nu vă dă nimeni afară din sală, puteți scrie un alt program
  auxiliar care, primind fișierul {\tt INPUT.TXT} și fișierul {\tt OUTPUT.TXT}
  produs de programul vostru, verifică dacă ieșirea este corectă. Aceasta
  deoarece, de obicei, este mult mai ușor de verificat o soluție decât de
  produs una (sau, cum spunea Murphy, „cunoașterea soluției unei probleme
  poate ajuta în multe cazuri la rezolvarea ei”). Folosind „generatorul” de
  teste și „verificatorul”, puteți testa programul mult mai bine. De altfel,
  la multe probleme chiar testele rulate de comisia de corectare sunt create
  tot aleator.

\item În caz că ați dat o soluție euristică la o problemă NP-completă, puteți
  implementa și un backtracking ca să vedeți cât de bune sunt rezultatele
  găsite euristic. Apoi, puteți începe să modificați funcția euristică pentru
  a o face cât mai performantă.

\end{itemize}

Și, ca să nu mai lungim vorba, iată o strategie care pare să dea rezultate:

{\bf A)} Imediat ce primiți problemele, citiți toate enunțurile și faceți-vă o
idee aproximativă despre gradul de dificultate al fiecărei probleme. Neapărat
verificați dacă se dau limite pentru datele de intrare (numărul maxim de
elemente ale unui vector și valoarea maximă a acestora, numărul maxim de
noduri dintr-un graf etc.) și pentru timpii de execuție pentru fiecare
test. Dacă nu se dau, întrebați imediat. Dimensiunea input-ului poate schimba
radical dificultatea problemei. Spre exemplu, pentru un vector cu $N=100$
elemente, un algoritm $O(N^{3})$ merge rezonabil, pe când pentru $N=10000$
același algoritm ar depăși cu mult cele câteva secunde care se acordă de
obicei. Fair-play-ul cere să puneți întrebările cu voce tare, pentru ca și
ceilalți să audă; de altfel, nu aveți nici un motiv să vă feriți de ceilalți
concurenți. Cei care sunt interesați de aceste întrebări le-ar pune oricum și
ei, iar cei care nu sunt interesați vor ignora oricum răspunsul.

Dacă există probleme care cer să se găsească un optim (maxim/minim) al unei
valori, întrebați dacă se acordă punctaje parțiale pentru soluții neoptime. Și
acest fapt poate schimba natura problemei. După aceasta,

\begin{minted}{pascal}
  Nr_probleme_nerezolvate := Nr_probleme_primite;

  while (Nr_probleme_nerezolvate>0) 
    and not ('Timpul a expirat, va rugam sa salvati') do
    begin
\end{minted}

{\bf B)} Faceți o împărțire a timpului pentru problemele rămase proporțional
cu punctajul fiecărei probleme. În general problemele au punctaje egale, dar
nu totdeauna. De exemplu, dacă o problemă e cotată cu 100 puncte, iar alta cu
50, veți aloca de două ori mai mult timp primei probleme, chiar dacă nu vi se
pare prea grea. Încercați să nu depășiți niciodată limitele de timp pe care
le-ați fixat. Dacă în schimb reușiți să economisiți timp față de cât v-ați
propus, cu atât mai bine, veți face o realocare a timpului și veți avea mai
mult pentru celelalte probleme.

{\bf C)} Apucați-vă de problema {\bf cea mai simplă}, chiar dacă e punctată
mai slab. Mai bine să duceți la bun sfârșit o problemă ușoară și să luați un
punctaj mai mic, decât să vă apucați de o problemă grea și să nu terminați
niciuna. Dacă toate problemele par grele, alegeți-o pe cea din domeniul care
vă este cel mai familiar, în care ați lucrat cel mai mult. Dacă vă este
indiferent și acest lucru, alegeți o problemă unde simțiți că aveți o idee
simplă de rezolvare. Dacă, în sfârșit, nu aveți nici o idee la nici o
problemă, apucați-vă de cea mai bine punctată.

{\bf D)} Citiți din nou enunțul, de data aceasta cu mare grijă. Întrebați
supraveghetorul pentru orice nelămurire. Dacă anumite lucruri nu sunt
specificate, iar profesorul nu vă dă nici un fel de informații suplimentare,
tratați problema în cazul cel mai general. Iată mai multe exemple frecvente în
care enunțul nu este limpede:

\begin{itemize}

\item Dacă nu se precizează cât de mari pot fi întregii dintr-un vector, nu
  lucrați pe {\tt Integer}, nici pe {\tt Word}, ci pe {\tt LongInt};

\item În problemele de geometrie analitică, e bine să presupuneți că punctele
  nu au coordonate întregi, ci reale;

\item De asemenea, pătratele și dreptunghiurile nu au neapărat laturi paralele
  cu axele, ci sunt așezate oricum în plan (aceasta poate într-adevăr să
  complice extrem de mult problema; nu vă doresc să vă izbiți de o asemenea
  neclaritate...);
  
\item Dacă fișierul de intrare conține string-uri, să nu presupuneți că ele au
  maxim 255 de caractere. Mai bine scrieți propria voastră procedură de citire
  a unui string, care să citească din fișier caracter cu caracter până la {\tt
    Eoln}, decât să aveți surprize. Dacă $S$ este o variabilă de tip {\tt
    String}, {\tt ReadLn(S)} ignoră tot restul rândului care depășește
  lungimea lui $S$.
  
\item Grafurile nu sunt neorientate, ci orientate. În principiu, enunțul nu
  are voie să fie neclar în această privință, dar au existat cazuri de
  neînțelegere.

\end{itemize}

{\bf E)} Începeți să vă gândiți la algoritmi cât mai buni, estimând în același
timp și cât v-ar lua ca să-i implementați. Faceți, pentru fiecare idee care vă
vine, calculul complexității. Nu trebuie neapărat să găsiți cel mai eficient
algoritm, ci numai unul suficient de bun. În general, trebuie ca, dintre toți
algoritmii care se încadrează în timpul de rulare, să-l alegeți pe cel care
este cel mai ușor de implementat. Iată un exemplu:

\begin{itemize}

\item Să presupunem că timpul de testare este de 5 secunde, lucrați pe un
  486DX4, algoritmul vostru are complexitatea $O(N^3)$, iar $N$ este maxim
  100. Un 486 face câteva milioane de operații elementare pe secundă, să zicem
  4.000.000. Aceasta înseamnă ceva mai puține operații mai costisitoare
  (atribuiri, comparații etc.) pe secundă. Să ne oprim deci la cifra de
  1.000.000. Programul vostru are timp de rulare cubic, iar $N^3$ este maxim
  1.000.000. De aici deducem că programul ar trebui să se încadreze într-o
  secundă. Calculul nostru este grosier, dar luând și o marjă de eroare
  arhisuficientă, rezultă că programul trebuie să meargă cu ușurință în 5
  secunde, deci algoritmul este acceptabil.

\end{itemize}

{\bf F)} Dacă algoritmul găsit este greu de implementat, mai căutați un altul
o vreme. Trebuie însă ca timpul petrecut pentru găsirea unui nou algoritm plus
timpul necesar pentru scrierea programului să nu depășească timpul necesar
pentru implementarea primului algoritm, altfel nu câștigați nimic. Deci nu
exagerați cu căutările și nu încercați să reduceți dincolo de limita
imposibilului complexitatea algoritmului. Mai ales, nu uitați că programul nu
poate avea o complexitate mai mică decât dimensiunea input-ului sau a
output-ului. De exemplu, dacă programul citește sau scrie matrice de
dimensiune $N \times N$, nu are sens să vă bateți capul ca să găsiți un
algoritm mai bun decât $O(N^2)$.

{\bf G)} Dintre toate ideile de implementare găsite (care se încadrează fără
probleme în timp), o veți alege pe cea mai scurtă ca lungime de cod. De
exemplu:

\begin{itemize}

\item Dacă $N \leq 1000$ și dispuneți de doi algoritmi, unul pe care îl
  estimați cam la 200 de linii de program, de complexitate $O(N \log N)$ și
  unul de 100 de linii de complexitate $O(N^2)$, cel de-al doilea este evident
  preferabil, pentru că nu pierdeți nimic din punctaj, sau cel mult pierdeți
  un test prin cine știe ce întâmplare, în schimb câștigați timp prețios pe
  care îl puteți folosi pentru alte probleme. Bineînțeles, primul program este
  mai eficient, dar în condiții de concurs el este {\bf prea} eficient. Este o
  mândrie să faceți o problemă perfect chiar dacă ratați o alta, dar este un
  câștig și mai mare să faceți amândouă problemele suficient de bine.

\end{itemize}

{\bf H)} În general, pentru orice problemă există cel puțin o soluție, fie și
una slabă. Sunt numeroase cazurile când nici nu vă vine altă idee de rezolvare
decât cea slabă. De regulă, când nu aveți în minte decât o rezolvare
neeficientă a problemei, care știți că nu o să treacă toate testele (un
backtracking, sau un $O(N^5)$, $O(N^6)$ etc.), e bine să încercați următorul
lucru:

\begin{itemize}

\item Să presupunem că v-a mai rămas o oră pentru rezolvarea acestei
  probleme. Calculați cam cât timp v-ar trebui ca să implementați rezolvarea
  slabă. Să zicem 40 de minute. În acest calcul trebuie să includeți și timpul
  de depanare a programului (care variază de la persoană la persoană) și pe
  cel de testare. Dacă sunteți foarte siguri pe voi, puteți să neglijați
  timpul de testare, dar orice program trebuie testat cel puțin pe exemplul de
  pe foaie.
  
\item Mai rămân deci 20 de minute, timp în care vă puteți gândi la altceva, la
  altă soluție. Pentru a avea șanse mai mari să găsiți o altă soluție, este
  indicat să încercați să ignorați complet soluția slabă, să nu o luați ca
  punct de plecare. Încercați să vă „goliți” mintea și să găsiți ceva nou,
  altfel vă veți învârti mereu în cerc.
  
\item Dacă vă vine vreo idee mai bună, ați scăpat de griji și mergeți la
  punctul {\bf (F)}. Altfel, la expirarea timpului de 20 de minute, vă apucați
  să implementați soluția pe care o aveți, oricât de ineficientă ar fi (de
  fapt, orice soluție, oricât de ineficientă, trebuie să ia măcar o treime sau
  o jumătate din punctaj, dacă nu apar erori de implementare).
  
\item Puteți, ca o măsură extremă, să depășiți cu {\bf maximum} 5 minute cele
  20 de minute planificate, dar de cele mai multe ori acesta e timp pierdut,
  deoarece intervine stresul și nu puteți să vă mai concentrați.

\end{itemize}

{\bf I)} Dacă ați ajuns până aici înseamnă că ați optat pentru o variantă de
implementare. Din acest moment, pentru această variantă veți scrie programul,
fără a vă mai gândi la altceva, chiar dacă pe parcurs vă vin alte idei. Iată
unele lucruri pe care e bine să le știți despre scrierea unui program:

\begin{itemize}

\item Datele de intrare se presupun a fi corecte. Aceasta este o regulă
  nescrisă (uneori) a concursului de informatică. Chiar dacă, prin absurd,
  știți sigur că datele de intrare trebuie verificate, mai bine n-o faceți,
  din mai multe motive. În primul rând, scopul cu care v-a fost dată problema
  este altul decât să se constate cine verifică mai bine datele de intrare. De
  aceea, cel mult un test sau două vor fi cu date greșite. În al doilea rând,
  nu se justifică să risipiți atâta timp numai pentru câteva puncte pe care
  le-ați putea pierde dacă nu faceți verificarea. În al treilea rând, e
  posibil să greșiți oricum problema în sine, caz în care nu mai contează dacă
  ați citit perfect datele de intrare. În sfârșit, legea lui Murphy spune că
  „oricâte teste ar efectua cineva asupra datelor de intrare, se va găsi
  cineva care să introducă date greșite”. Efortul este deci zadarnic...
  
\item Ultimul lucru, când sunteți convinși că programul este terminat și când
  v-ați hotărât să nu îl mai modificați, adăugați opțiunile de
  compilare. Puteți face aceasta apăsând {\tt Ctrl-O O}. La începutul
  programului vor apărea directivele de compilare. Setați \$B, \$I, \$R și \$S
  pe - (minus). Eventual, puteți include direct linia {\tt \{\$B-,I-,R-,S-\}}
  imediat după titlul programului. Aceasta va face compilatorul să nu mai
  evalueze complet expresiile booleene, să nu mai verifice operațiile de
  intrare/ieșire, domeniul de atribuire ({\it Range Checking}) și stiva ({\it
    Stack Checking}). Există două avantaje majore: în primul rând că programul
  merge mai repede (se câștigă câteva procente bune la viteză), iar în al
  doilea rând, psihologic vorbind, este preferabil ca un program să se
  blocheze decât să se oprească printr-un banal {\tt Range check error}. Nu vă
  grăbiți să puneți directivele de compilare încă de la început, deoarece nu
  veți mai primi mesajele corespunzătoare de eroare și vă va fi mai greu să
  depanați programul.
  
\item Pe cât este posibil, încercați să convingeți juriul să nu vă ruleze
  sursa (Pascal sau C/C++), ci direct executabilul. Merge simțitor mai
  repede. Asta numai în cazul în care vă temeți că programul ar putea să nu se
  încadreze în timp.
  
\item Dacă se poate, evitați lucrul cu pointeri. Programele care îi folosesc
  sunt mai greu de depanat și se pot bloca mult mai ușor.

\item Să presupunem că aveți de lucrat cu matrice de dimensiuni maxim $100
  \times 100$. Unii elevi au obiceiul să dimensioneze la început matricele de
  $5 \times 5$ sau $10 \times 10$, doarece sunt mai comod de evaluat cu {\it
    Evaluate} ({\tt Ctrl-F4}) sau {\it Watch} ({\tt Ctrl-F7}). Acest lucru
  este adevărat, dar există riscul ca la sfârșit să uitați să redimensionați
  matricele de $100 \times 100$. Decât să faceți o asemenea greșeală (care în
  mod sigur vă va compromite toată problema), mai bine setați dimensiunile
  corecte de la început. De altfel, ideal ar fi ca depanarea programelor să
  fie suprimată cu totul și programul să meargă din prima.
  
\item Evitați lucrul cu numere reale, dacă puteți. Operațiile în virgulă
  mobilă sunt mult mai lente. De exemplu, testul dacă un număr este prim nu va
  începe în nici un caz cu linia

  \begin{minted}{pascal}
    while i <= Sqrt(N) do
  \end{minted}

  ci cu linia

  \begin{minted}{pascal}
    while i * i <= N do
  \end{minted}

  Din punct de vedere logic, condițiile sunt perfect echivalente. Totuși,
  prima se evaluează de câteva zeci de ori mai încet decât a doua.

\item Dacă lucrați cu numere reale, nu folosiți testul

  \begin{minted}{pascal}
    R1=R2
  \end{minted}

  deoarece pot apărea erori, ci implementați o funcție:

  \begin{minted}{pascal}
    function Equal(R1,R2:Real):Boolean;
    begin
      Equal:=(Abs(R1-R2)<0.00001);
    end;
  \end{minted}

  Numărul de zerouri de după virgulă trebuie să fie suficient de mare astfel
  încât două numere diferite să nu fie tratate drept egale (se poate lucra de
  pildă cu 0.000001).

\item Tot în cazul numerelor reale, evitați pe cât posibil să faceți
  împărțiri, deoarece sunt foarte greoaie. De exemplu:

  $X/5 \leftrightarrow 0.2*X$\\
  $X/Y/Z \leftrightarrow X/(Y*Z)$

\item Optimizările de genul „{\tt X shl 1}” respectiv „{\tt X<<1}” în loc de
  „{\tt 2*X}” sunt niște artificii de cele mai multe ori neesențiale, care în
  schimb fac formulele mai lungi, greu de urmărit și pot crea complicații. Cel
  mai bine este să lucrați cu notațiile obișnuite și doar la sfârșit, dacă
  timpul de rulare trebuie redus cu orice preț, să faceți înlocuirile.

\item Alegeți-vă numele de variabile în așa fel încât programul să fie
  clar. Sunt permise mai mult de două litere! Numele fiecărei proceduri,
  funcții, variabile trebuie să-i explice clar utilitatea. E drept, lungimea
  programului crește, dar codul devine mult mai limpede și timpul de depanare
  scade foarte mult. Ca o regulă generală, claritatea programelor face mult
  mai ușoară înțelegerea lor chiar și după o perioadă mai îndelungată de timp
  (luni, ani). Nu trebuie nici să cădeți în cealaltă extremă. De exemplu, nu
  depășiți 10 caractere pentru un nume de variabilă.
  
\item Salvați programul cât mai des. Dacă vă obișnuiți, chiar la fiecare
  două-trei linii. După ce o să vă intre în reflex n-o să vă mai incomodeze cu
  nimic acest obicei, mai ales că în ziua de azi salvarea unui program de 2-3
  KB se face practic instantaneu. Au fost frecvente cazurile în care o pană de
  curent prindea pe picior greșit mulți concurenți, iar după aceea nu mai este
  absolut nimic de făcut, pentru că nimeni nu vă va crede pe cuvânt că ați
  făcut programul și că el mergea.
  
\item Obișnuiți-vă să programați modular. Faceți proceduri separate pentru
  citirea și inițializarea datelor, pentru sortare, pentru afișarea
  rezultatelor etc. În general nu se recomandă să scrieți proceduri în alte
  proceduri (adică e bine ca toate procedurile să aparțină direct de programul
  principal). Procedurile, acolo unde e posibil, nu trebuie să depășească un
  ecran, pentru a putea avea o viziune de ansamblu asupra fiecăreia în parte
  . Acest lucru ajută mult la depanare.
  
\item Rulați programul cât mai des. În primul rând după ce scrieți procedura
  de citire a datelor. Dacă e nevoie de sortarea datelor de intrare, nu strică
  să vă convingeți că programul sortează bine, rulând două-trei teste
  oarecare. E păcat să pierdeți puncte dintr-o greșeală copilărească.
  
\item O situație delicată apare când fișierul de intrare conține mai multe
  seturi de date (teste). În acest caz, atenția trebuie sporită, deoarece dacă
  la primul sau al doilea test programul vostru dă eroare și se oprește din
  execuție, veți pierde automat și toate celelalte teste care urmează. Dacă în
  fișierul de intrare exista un singur set de date, atunci pierderea din
  vedere a unui caz particular al problemei nu putea duce, în cel mai rău caz,
  decât la picarea unui test. Așa însă, picarea unui test poate atrage după
  sine picarea tuturor celor care îi urmează. Pe lângă corectitudinea strict
  necesară, programul trebuie să se încadreze și în timp pentru orice fel de
  test. Dacă la primul sau al doilea test din suită programul depășește timpul
  (sau, și mai rău, se blochează), e foarte probabil să fie oprit din execuție
  de către comisie, deci din nou veți pierde toate testele care au rămas
  neexecutate. Uneori comisia este binevoitoare și acceptă să modifice
  fișierul de date, tăind din el setul pe care programul vostru nu merge, dar
  acest lucru este greoi și nu tocmai cinstit față de ceilalți concurenți,
  deci nu vă bazați pe asta.
  
\item Tot în situația în care există mai multe seturi de date în fișierul de
  intrare, dacă ieșirea se face într-un fișier, este bine ca după afișarea
  rezultatului pentru fiecare test să actualizați fișierul de ieșire (fie prin
  comanda {\tt Flush}, fie prin două proceduri, {\tt Close} și {\tt
    Append}). În felul acesta, chiar dacă la unul din teste programul se
  blochează sau dă eroare, rezultatele deja scrise rămân scrise. Altfel, e
  posibil ca rezultatele de la testele anterioare să rămână într-un buffer în
  memorie, fără a fi „vărsate” pe disc.
  
\item Dacă fișierul de ieșire are dimensiuni foarte mari, de exemplu dacă vi
  se cer toate soluțiile, iar numărul acestora este de ordinul zecilor de mii,
  puteți avea surpriza ca timpul să nu vă ajungă pentru a le tipări pe toate
  în fișierul de ieșire. Și în acest caz, este recomandat ca după tipărirea
  fiecărei soluții să executați comanda {\tt Flush}, sau să închideți fișierul
  de ieșire și să-l redeschideți în modul {\tt Append}.
  
\end{itemize}

{\bf J)} Dacă ați trecut cu bine și de faza de scriere a programului, mai
aveți doar părțile de depanare și testare, care de multe ori se îmbină. Metoda
cea mai bună de depanare este următoarea:

\begin{itemize}

\item Începeți cu un test nici prea simplu, nici prea complicat (și ușor de
  urmărit cu creionul pe hârtie) și executați-l de la cap la coadă. Dacă merge
  perfect, treceți la teste mai complexe ({\bf minimum} 4 teste și maxim
  7-8). Dacă le trece și pe acestea, puteți fi mândri. Legile lui Murphy în
  programare se aplică în continuare: „Depanarea nu poate demonstra că un
  program merge; ea poate cel mult demonstra că un program nu merge”. Totuși,
  dacă programul vostru a mers perfect pe 7-8 teste date la întâmplare, există
  șanse foarte mari să meargă pe majoritatea testelor comisiei, sau chiar pe
  toate.
  
\item Exemplul dat în enunț nu are în general nici o semnificație deosebită
  (de fapt, are mai curând darul de a semăna confuzie printre concurenți), iar
  dacă programul merge pe acest test particular, nu înseamnă că o să meargă și
  pe alte teste. În culegere nu a fost explicat pe larg algoritmul decât pe
  exemplul din enunțul fiecărei probleme, dar aceasta s-a făcut numai pentru a
  nu supraîncărca materialul.
  
\item Dacă la unul din teste programul nu merge corespunzător, rulați din nou
  testul, dar de data aceasta procedură cu procedură. După fiecare procedură
  evaluați variabilele și vedeți dacă au valorile așteptate. În felul acesta
  puteți localiza cu precizie procedura, apoi linia unde se află
  eroarea. Corectați în această manieră toate erorile, până când testul este
  trecut.
  
\item În acest moment, luați de la capăt toate testele pe care programul le-a
  trecut deja. În urma depanării, s-ar putea ca alte greșeli să iasă la
  suprafață și programul să nu mai meargă pe vechile teste.
  
\item Repetați procedeul de mai sus până când toate testele merg. Dacă vă
  obișnuiți să programați modular și îngrijit, depanarea și testarea n-ar
  trebui să dureze mai mult de 5-10 minute. Din acest moment, nu mai
  modificați nici măcar o literă în program, sau dacă țineți să o faceți,
  păstrați-vă în prealabil o copie. Nu vă bazați pe faptul că puteți să țineți
  minte modificările făcute și să refaceți oricând forma inițială a
  programului în caz că noua versiune nu va fi bună.
  
\item Dacă totuși nu-i puteți „da de cap” programului, iar timpul alocat
  problemei respective expiră, aduceți programul la o formă în care să meargă
  măcar pe o parte din teste (pe jumătate, de exemplu) și treceți la problema
  următoare.

\end{itemize}

{\bf K)}

\begin{minted}{pascal}
    Dec(Nr_probleme_nerezolvate);
  end; { while }
\end{minted}

\begin{center}
  {\Huge \decofourleft \decofourright}
\end{center}

Probabil nu veți fi de acord cu toate sfaturile date mai sus. E bine însă să
le aplicați. Scopul pentru care ele au fost incluse în această carte este de a
ajuta concurenții să se acomodeze mai ușor cu atmosfera concursului. De multe
ori, primul an de participare la olimpiadă se soldează cu un rezultat cel mult
mediu, deoarece, oricât ar spune cineva „ei, nu-i așa mare lucru să mergi la
un concurs”, experiența acumulată contează mult. De aceea, abia de la a doua
participare și uneori chiar de mai târziu încep să apară rezultatele. Intenția
autorului a fost să vă ușureze misiunea și să vă dezvăluie câteva din
dificultățile de toate felurile care apar la orice concurs, pentru a nu vă da
ocazia să le descoperiți pe propria piele. Poate că aceste ponturi vă vor fi
de folos.
