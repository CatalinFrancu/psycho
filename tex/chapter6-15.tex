\section{Problema 15}

Problema următoare a fost propusă la proba de baraj de la Olimpiada Națională
de Informatică, Slatina 1995 și este un exemplu tipic de aplicare a
principiului lui Dirichlet.

{\bf ENUNȚ}: La un SHOP din Slatina se găsesc spre vînzare $P-1$ ($P$ este un
număr prim) produse unicat de costuri $X(1), X(2), \dots, X(P-1)$. Nici unul
din produse nu poate fi cumpărat prin plata exactă cu bancnote de $P$\$. În
SHOP intră un olimpic care are un număr nelimitat de bancnote de $P$\$ și o
singură bancnotă de $Q$\$ ($1 \leq Q \leq P-1$). Ce produse trebuie să cumpere
olimpicul pentru a putea plăti exact produsele cumpărate?

{\bf Intrarea}: Datele de intrare se dau în fișierul de intrare {\tt
  INPUT.TXT} ce conține două linii:

\begin{itemize}

\item pe prima linie valorile lui $P$ și $Q$;

\item pe a doua linie valorile costurilor produselor.

\end{itemize}

Ieșirea se va face în fișierul {\tt OUTPUT.TXT} unde se vor lista în ordine
crescătoare indicii produselor cumpărate de olimpic.

\texttt{
  \begin{tabular}{|l|l|}
    \hline
        {\bf INPUT.TXT} & {\bf OUTPUT.TXT} \\ \hline
        5 4 & 1 2 \\
        1 3 6 7 & \\
        \hline
  \end{tabular}
}

{\bf Timp de implementare}: la Slatina s-au acordat cam 90 de minute, dar 45
ar trebui să fie suficiente.

{\bf Timp de rulare pentru fiecare test}: 1 sec.

{\bf Complexitate cerută}: $O(P)$.

Enunțul original nu specifica nici o limită maximă pentru valoarea lui P. Vom
adăuga noi această limită, respectiv $P < 10.000$.

{\bf REZOLVARE}: Problema se reduce la a găsi un grup de obiecte pentru care
suma costurilor să fie divizibilă cu $P$ sau să fie congruentă cu $Q$ modulo
$P$. Am văzut deja în problema precedentă că dispunem de o soluție $O(N^2)$
pentru a găsi un număr de elemente care să se dividă cu $P$. În cazul nostru,
trebuie să observăm însă că nu avem $P$ obiecte, ci numai $P-1$; în schimb,
dispunem de o bancnotă suplimentară de valoare $Q$. Aceste diferențe vor fi
explicate mai târziu și se va vedea că ele nu schimbă cu nimic natura
problemei. Diferența esențială provine din faptul că nu se mai cere ca suma
numerelor să fie maximă, ca în problema precedentă. Orice combinație de numere
care dau o sumă potrivită este suficientă.

Să începem prin a explica principiul lui Dirichlet, care de altfel face apel
numai la intuiție și nu necesită cunoștințe speciale de matematică. Acest
principiu spune că dacă distribuim $N$ obiecte în $K$ cutii, atunci cel puțin
într-o cutie se vor afla minim $\lceil N/K \rceil$ obiecte (aici prin $\lceil
N/K \rceil$ se înțelege „cel mai mic întreg mai mare sau egal cu
$N/K$”). Demonstrația se face prin reducere la absurd: dacă în fiecare cutie
s-ar afla mai puțin decât $\lceil N/K \rceil$ obiecte, atunci numărul total de
obiecte ar fi mai mic decât $K \times \lceil N/K \rceil$, adică mai mic decât
$N$.

Spre exemplu, oricum am distribui 7 obiecte în 4 cutii, putem fi siguri că cel
puțin într-o cutie se vor afla minim $\lceil 7/4 \rceil = 2$
obiecte. Într-adevăr, dacă toate cutiile ar conține cel mult câte un obiect,
atunci numărul total de obiecte nu ar putea fi mai mare ca 4, ceea ce este
absurd.

Să vedem acum cum s-ar putea aplica acest principiu la problema de față. Să
facem notația

\begin{equation}
  S(k) = \sum_{i = 1}^{k} X(i)
\end{equation}

și convenția $S(0) = 0$. Prin urmare, putem scrie egalitatea

\begin{equation}
  S(k_2) - S(k_1) = \sum_{i = k_1 + 1}^{k_2} X(i)
\end{equation}

Dacă găsim în vectorul $S$ două valori $S(k_1)$ și $S(k_2)$ care dau același
rest la împărțirea prin $P$, înseamnă că diferența lor se divide cu $P$, deci
șirul de obiecte $k_1+1, k_1+2, \dots, k_2$ poate constitui o soluție.

Să presupunem pentru început că dispunem de $P$ obiecte. Se pune întrebarea:
ce valori poate lua restul împărțirii lui $S(k)$ prin $P$? Desigur, orice
valoare între 0 și $P-1$. Există deci în total $P$ resturi distincte. Pe de
altă parte, există $P+1$ elemente în vectorul $S$ (se consideră și elementul
$S(0)$). Începem să recunoaștem aici principiul lui Dirichlet, în care
„obiectele” sunt resturile $S(0) \bmod P, S(1) \bmod P, \dots, S(P) \bmod P$,
iar cutiile sunt clasele de resturi modulo $P$. Avem de distribuit P+1 obiecte
în P cutii, așadar cel puțin într-o cutie se vor afla $\lceil (P+1)/P \rceil =
2$ obiecte. Prin urmare, vor exista cu siguranță doi indici $k_1 < k_2$ astfel
încât $S(k_2) - S(k_1) \equiv 0 \pmod{P}$. Nu avem decât să tipărim secvența
$k_1+1, k_1+2, \dots, k_2$.

Să vedem acum ce se întâmplă dacă avem numai $P-1$ obiecte, așa cum este cazul
problemei. Atunci avem numai $P$ resturi posibile, deci se poate ca toate
elementele din $S$ să dea resturi distincte la împărțirea prin $P$. Dar în
acest caz, există un indice $k$ astfel încât $S(k) \equiv Q \pmod{N}$, deci
trebuie doar să tipărim secvența de indici $1, 2, \dots, k$.

Pentru a reuni aceste două cazuri într-unul singur, putem considera expresia
$S(k)$ drept un alt mod de a scrie expresia $S(k) - S(0)$. Problema se reduce
la a căuta doi indici $k_1, k_2 \in \{0, 1, 2, \dots, P\}$ astfel încât
$(S(k_2) - S(k_1)) \bmod N \in [0,Q]$. Să nu uităm că trebuie să efectuăm
această operație într-un timp liniar, deci nu avem voie să comparăm pur și
simplu două câte două elementele vectorului $S$. Vom prezenta modul în care se
pot găsi două elemente congruente modulo $P$, cazul celălalt tratându-se
analog. Metoda constă în crearea unui alt vector, $L$, în care $L(i) = j$
înseamnă că suma $S(j)$ dă restul $i$ la împărțirea prin $P$. Inițial, toate
elementele vectorului $L$ vor avea o valoare specială, eventual negativă. Apoi
se parcurge vectorul $S$ și pentru fiecare $S(j)$ se efectuează operația
$L(S(j) \bmod P) \leftarrow j$. În momentul în care se încearcă reatribuirea
unui element din $L$ care are deja o valoare dată, înseamnă că am găsit cei
doi indici pe care îi căutam.

Iată un exemplu. Dacă $P=7$ și $X=(8, 8, 2, 6, 13, 3)$, rezultă vectorul
$S$ = (8, 16, 18, 24, 37, 40). Resturile la împărțirea prin 7 sunt respectiv 1,
2, 4, 3, 2 și 5.

\centeredTikzFigure[
  mat/.style = {
    matrix of nodes,
    ampersand replacement=\&,
    nodes = {
      draw,
      rectangle,
      anchor=center,
      minimum width=2em,
      minimum height=2em,
    },
    row 1/.style = noborder,
    row 2/.style = noborder,
    column 1/.style = noborder,
  },
  noborder/.style = {
    nodes = {
      draw = none,
      font=\bf,
    },
  },
]{
  \matrix[mat] (m) {
    R \&[2em] \&   \&   \& L \&   \&   \&   \\[1em]
    \ \& 0 \& 1 \& 2 \& 3 \& 4 \& 5 \& 6 \\
    1 \& 0 \& 1 \& -1 \& -1 \& -1 \& -1 \& -1 \\[1em]
    2 \& 0 \& 1 \&  2 \& -1 \& -1 \& -1 \& -1 \\[1em]
    4 \& 0 \& 1 \&  2 \& -1 \&  3 \& -1 \& -1 \\[1em]
    3 \& 0 \& 1 \&  2 \&  4 \&  3 \& -1 \& -1 \\[1em]
    2 \\
  };

  \draw[<->] (m-7-1.east) -| (m-6-4.south);
}

După cum se vede, restul 2 poate fi obținut atât cu primele două obiecte, cât
și cu primele 5, deci suma prețurilor obiectelor 3, 4 și 5 este divizibilă cu
7.

Un ultim detaliu de implementare constă în aceea că nu este necesară memorarea
vectorului $S$, ci numai a elementului curent; orice alte informații care ne
trebuie la un moment dat le putem afla din vectorii $X$ și $L$. Pentru
memorarea elementului curent din $S$, se pornește cu valoarea 0 și la fiecare
pas se adaugă valoarea elementului corespunzător din $X$.

\begin{minted}{c}
#include <stdio.h>
#define NMax 10000
#define None -1

int X[NMax], L[NMax], P, Q;

void ReadData(void)
{ FILE *F = fopen("input.txt", "rt");
  int i;

  fscanf(F, "%d %d\n", &P, &Q);
  for (i=1; i<P; fscanf(F, "%d", &X[i++]));
  fclose(F);
}

void FindSum(void)
{ long S=0;
  FILE *F = fopen("output.txt", "wt");
  int i,j;

  for (i=1, L[0]=0; i<P; L[i++] = None);
  i=0;
  while ( L[ (S+=X[++i]) % P ]==None && // Restul 0
          L[ (S%P+P-Q) % P ]==None )    // Restul Q
    L[S%P]=i;
  for (j = 1 + ((L[S%P]!=None)? L[S%P]: L[ (S%P+P-Q) % P ]);
       j <= i; fprintf(F, "%d ", j++));
  fclose(F);
}

void main(void)
{
  ReadData();
  FindSum();
}
\end{minted}
