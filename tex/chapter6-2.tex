\section{Problema 2}

Problema următoare a fost propusă la a VI-a Olimpiadă Internațională de
Informatică, Stockholm 1994. Este și ea un bun exemplu de situație în care
putem cădea în plasa unei rezolvări „Branch and Bound” atunci când nu este
cazul.

{\bf ENUNȚ}: Se dă o configurație de $3 \times 3$ ceasuri, fiecare având un
singur indicator care poate arăta numai punctele cardinale (adică orele 3, 6,
9 și 12). Asupra acestor ceasuri se poate acționa în nouă moduri distincte,
fiecare acțiune însemnând mișcarea limbilor unui anumit grup de ceasuri în
sens orar cu 90°. În figura de mai jos se dă un exemplu de configurație
inițială a ceasurilor și se arată care sunt cele nouă tipuri de mutări (pentru
fiecare tip de mutare se mișcă numai ceasurile reprezentate hașurat).

\newcommand\threeByThree[7]{
  \fill[affected] (#1,#2) rectangle (#3,#4); 
  \draw[grid] (0,0) grid (2,2);
  \node[number] at (#5, #6) {#7};
}
\centeredTikzFigure[
  mat/.style = {
    matrix of nodes,
    ampersand replacement=\&,
    row sep=1em,
    column sep=1em,
    anchor=north,
  },
  grid/.style = {step=2/3,gray,very thin},
  affected/.style = {gray!40!white},
  number/.style = {font=\bf\large, anchor=center},
]{
  % clocks
  \matrix[mat] (m) {
    \draw (-1,0) -- (0,0) circle (1); \&
    \draw (-1,0) -- (0,0) circle (1); \&
    \draw (0,+1) -- (0,0) circle (1); \\
    \draw (0,-1) -- (0,0) circle (1); \&
    \draw (0,-1) -- (0,0) circle (1); \&
    \draw (0,-1) -- (0,0) circle (1); \\
    \draw (0,-1) -- (0,0) circle (1); \&
    \draw (+1,0) -- (0,0) circle (1); \&
    \draw (0,-1) -- (0,0) circle (1); \\
  };

  % moves
  \matrix[mat] (l) at (8,0) {
    %             SW corner |NE corner  |text pos |text
    \threeByThree {0/3}{2/3} {4/3}{6/3} {1/3}{5/3} {1} \&
    \threeByThree {0/3}{4/3} {6/3}{6/3} {3/3}{5/3} {2} \&
    \threeByThree {2/3}{2/3} {6/3}{6/3} {5/3}{5/3} {3} \\
    \threeByThree {0/3}{0/3} {2/3}{6/3} {1/3}{3/3} {4} \&
    % need one extra rectangle here -- just reuse the whole command
    \threeByThree {0/3}{2/3} {6/3}{4/3} {3/3}{3/3} {5}
    \threeByThree {2/3}{0/3} {4/3}{6/3} {3/3}{3/3} {5} \&
    \threeByThree {4/3}{0/3} {6/3}{6/3} {5/3}{3/3} {6} \\
    \threeByThree {0/3}{0/3} {4/3}{4/3} {1/3}{1/3} {7} \&
    \threeByThree {0/3}{0/3} {6/3}{2/3} {3/3}{1/3} {8} \&
    \threeByThree {2/3}{0/3} {6/3}{4/3} {5/3}{1/3} {9} \\
  };
}

Se cere ca, într-un număr minim de mutări, să aducem toate indicatoarele la
ora 12.

Intrarea se face din fișierul {\tt INPUT.TXT}, care conține configurația
inițială sub forma unei matrice $3 \times 3$. Pentru fiecare ceas se specifică
câte o cifră: 0 = ora 12, 1 = ora 3, 2 = ora 6, 3 = ora 9.

Ieșirea se va face în fișierul {\tt OUTPUT.TXT} sub forma unui șir de numere
între 1 și 9, pe un singur rând, separate prin spațiu, reprezentând șirul de
mutări care aduc ceasurile în starea finală. Se cere o singură soluție.

{\bf Exemplu}: Pentru figura de mai sus, fișierul {\tt INPUT.TXT} este

\begin{verbatim}
  3 3 0
  2 2 2
  2 1 2
\end{verbatim}

iar fișierul {\tt OUTPUT.TXT} ar putea fi:

\begin{verbatim}
  5 8 4 9
\end{verbatim}

{\bf Timp de implementare}: 1h - 1h 15min.

{\bf Timp de rulare}: o secundă.

{\bf Complexitate cerută}: $O(1)$ (timp constant).

{\bf REZOLVARE}: Din câte am văzut pe la concursuri, peste jumătate din elevi
s-ar apuca direct să implementeze o rezolvare Branch and Bound la această
problemă, fără să-și mai bată capul prea mult. Există argumente în favoarea
acestei inițiative:

\begin{itemize}

\item Mulți preferă să nu mai piardă timpul căutând o altă soluție, mai ales
  că problema seamănă mult cu „Lampa lui Dario Uri” (care de fapt este exact
  problema ceasurilor, dar în care ceasurile au doar două stări în loc de
  patru). În plus, se știe că pe cazul general al unei table $N \times N$,
  cele două probleme nu admit rezolvări polinomiale și atunci cea mai sigură
  soluție este prin tehnica Branch and Bound.

\item De asemenea, se observă că numărul total de configurații posibile pentru
  o tablă cu 9 ceasuri este de $4^9$, adică aproximativ un sfert de milion. Un
  algoritm Branch and Bound ar furniza așadar o soluție în timp
  rezonabil. Raționamentul multor elevi este „decât să pierd timpul căutând o
  soluție mai bună, fără să am certitudinea că o voi găsi, mai bine folosesc
  timpul implementând un Branch care măcar știu sigur că merge”.

\item Problema cere o soluție într-un număr minim de pași, lucru care îi cam
  descurajează pe cei care încă ar vrea să caute alte rezolvări. „Alte
  rezolvări” înseamnă de obicei un Greedy comod de implementat, iar asupra
  rezolvărilor Greedy se poartă întotdeauna discuții interminabile pe
  culoarele sălilor de concurs referitor la „cât de bune sunt” (adică în cât
  la sută din cazuri furnizează soluția optimă).

\end{itemize}

Se pierd însă din vedere unele lucruri esențiale. În primul rând, tabla nu
este de $N \times N$, ci are dimensiuni fixate, $3 \times 3$. În al doilea
rând, implementarea unui Branch and Bound în timp de concurs este o aventură
nu tocmai ușor de dus la bun sfârșit (personal mi-a fost frică să o încerc
vreodată). În sfârșit, după cum se va vedea mai jos, problema șirului minim de
transformări este o pseudo-problemă, deoarece soluția simplă este oricum
unică.

Ce se înțelege prin „soluție simplă”? Să remarcăm două lucruri:

\begin{enumerate}

\item Aplicarea de patru ori a aceleiași mutări nu schimbă nimic în
  configurația ceasurilor. Într-adevăr, mutarea va afecta de fiecare dată
  același grup de ceasuri, iar aplicarea de patru ori va roti fiecare
  indicator cu 360°, adică îl va aduce în poziția inițială. Din acest motiv,
  toate afirmațiile făcute în cele ce urmează vor fi valabile în algebra
  modulo 4.

\item Ordinea în care se aplică transformările nu contează.

\end{enumerate}

În consecință, prin „soluție simplă” se înțelege un șir de mutări ordonat
crescător în care nici o mutare nu apare de mai mult de trei ori. Să
demonstrăm acum că soluția simplă este unică.

Fie $A \in \mathbb{M}_3(\mathbb{Z}_4)$ matricea citită de la intrare, unde
$a_{i,j}$ arată de câte ori a fost rotit ceasul $C_{i,j}$ peste ora 12. Fie
matricea $B \in \mathbb{M}_3(\mathbb{Z}_4), b_{i,j}=4 - a_{i,j}$. Matricea $B$
arată de câte ori mai trebuie rotit fiecare ceas până la ora 12. O soluție
înseamnă a efectua fiecare din cele 9 mutări de un număr de ori, $p_1, p_2,
\dots, p_9$. Cum afectează aceste mutări ceasurile? Se poate deduce ușor:

\begin{table}[H]
  \centering
  \begin{tabular}{| l | l |}
    \hline
    Ceasul & Tipurile de mutări care îl afectează \\ \hline
    $C_{1,1}$ & 1, 2, 4 \\
    $C_{1,2}$ & 1, 2, 3, 5 \\
    $C_{1,3}$ & 2, 3, 6 \\
    $C_{2,1}$ & 1, 4, 5, 7 \\
    $C_{2,2}$ & 1, 3, 5, 7, 9 \\
    $C_{2,3}$ & 3, 5, 6, 9 \\
    $C_{3,1}$ & 4, 7, 8 \\
    $C_{3,2}$ & 5, 7, 8, 9 \\
    $C_{3,3}$ & 6, 8, 9 \\
    \hline
  \end{tabular}
\end{table}

Se obține deci un sistem de 9 ecuații cu 9 necunoscute:

\begin{equation}
  P = 
  \begin{pmatrix}
    p_1 + p_2 + p_4 & p_1 + p_2 + p_3 + p_5 & p_2 + p_3 + p_6 \\
    p_1 + p_4 + p_5 + p_7 & p_1 + p_3 + p_5 + p_7 + p_9 & p_3 + p_5 + p_6 + p_9 \\
    p_4 + p_7 + p_8 & p_5 + p_7 + p_8 + p_9 & p_6 + p_8 + p_9 \\
  \end{pmatrix}
  \equiv B
\end{equation}

Să presupunem că acest sistem admite două soluții $p_1, \dots, p_9$ și $q_1,
\dots, q_9$. Atunci $P \equiv B \pmod{4}$ și $Q \equiv B \pmod{4}$, deci $P
\equiv Q \pmod{4}$ și, prin diferite combinații liniare ale celor 9 ecuații,
se deduce $p_1 \equiv q_1$, $p_2 \equiv q_2$, ..., $p_9 \equiv q_9 \pmod{4}$,
adică cele două soluții sunt echivalente.

Odată ce am demonstrat că soluția este unică, algoritmul de găsire a ei este
foarte simplu: găsim o soluție oarecare, o ordonăm crescător și eliminăm orice
grup de 4 mutări identice. Pentru a găsi o soluție oarecare, avem nevoie de
niște mutări predefinite care să miște un singur ceas cu o singură poziție
înainte, fără a afecta celelalte ceasuri. Aceste mutări vor fi reținute sub
forma unui vector cu 9 componente, fiecare componentă indicând de câte ori se
efectuează fiecare din cele 9 tipuri de mutări. Deoarece avem nevoie de 9
asemenea mutări predefinite, câte una pentru fiecare ceas, rezultatul va fi o
matrice predefinită. De exemplu, pentru a determina secvența de mutări care
rotește ceasul $C_{1,1}$ cu o poziție, trebuie rezolvat sistemul

\begin{equation}
  \begin{pmatrix}
    p_1 + p_2 + p_4 & p_1 + p_2 + p_3 + p_5 & p_2 + p_3 + p_6 \\
    p_1 + p_4 + p_5 + p_7 & p_1 + p_3 + p_5 + p_7 + p_9 & p_3 + p_5 + p_6 + p_9 \\
    p_4 + p_7 + p_8 & p_5 + p_7 + p_8 + p_9 & p_6 + p_8 + p_9 \\
  \end{pmatrix}
  \equiv
  \begin{pmatrix}
    1 & 0 & 0 \\
    0 & 0 & 0 \\
    0 & 0 & 0
  \end{pmatrix}
\end{equation}

lucru care nu este foarte ușor, dar se poate duce la bun sfârșit în timp de
concurs. Soluția este $p_1=3, p_2=3, p_3=3, p_4=3, p_5=3, p_6=2, p_7=3, p_8=2,
p_9=0$, adică mutarea 1 trebuie efectuată de trei ori, mutarea 2 de trei ori
ș.a.m.d. Se obține prima linie din matricea predefinită, $(3, 3, 3, 3, 3, 2,
3, 2, 0)$. Mai trebuie rezolvate propriu-zis sistemele de ecuații pentru
ceasurile $C_{1,2}$ și $C_{2,2}$, soluțiile celorlalte sisteme decurgând ușor
prin simetrie. Soluțiile apar în textul sursă.

Odată determinate aceste șiruri elementare de mutări, vom lua pe rând fiecare
ceas, vom aplica șirul elementar corespunzător de atâtea ori cât e nevoie
pentru a-l aduce la ora 12 și vom aduna modulo 4 toate mutările
făcute. Vectorul sumă care rezultă este tocmai soluția noastră.

Pentru exemplul din enunț, folosind constantele din programul sursă, obținem:

\begin{equation}
  B =
  \begin{pmatrix}
    1 & 1 & 0 \\
    2 & 2 & 2 \\
    2 & 3 & 2
  \end{pmatrix}
\end{equation}

și

\begin{align*}
1 & \times (3,3,3,3,3,2,3,2,0) = & (3,3,3,3,3,2,3,2,0) & + \\
1 & \times (2,3,2,3,2,3,1,0,1) = & (2,3,2,3,2,3,1,0,1) & \\
0 & \times (3,3,3,2,3,3,0,2,3) = & (0,0,0,0,0,0,0,0,0) & \\
2 & \times (2,3,1,3,2,0,2,3,1) = & (0,2,2,2,0,0,0,2,2) & \\
2 & \times (2,3,2,3,1,3,2,3,2) = & (0,2,0,2,2,2,0,2,0) & \\
2 & \times (1,3,2,0,2,3,1,3,2) = & (2,2,0,0,0,2,2,2,0) & \\
2 & \times (3,2,0,3,3,2,3,3,3) = & (2,0,0,2,2,0,2,2,2) & \\
3 & \times (1,0,1,3,2,3,2,3,2) = & (3,0,3,1,2,1,2,1,2) & \\
2 & \times (0,2,3,2,3,3,3,3,3) = & (0,0,2,0,2,2,2,2,2) & \\
\cline{3-3}
  &                              & (0,0,0,1,1,0,0,1,1) &
\end{align*}

Prin urmare soluția simplă a exemplului este: 4 5 8 9.

\inputminted{c}{src/problem2.c}
