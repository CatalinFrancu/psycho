\section{Problema 5}

Această problemă a fost propusă la Olimpiada Națională de Informatică, Slatina
1995, la clasa a XI-a. Pe atunci programarea dinamică era o tehnică de
programare destul de puțin cunoscută de către majoritatea elevilor.

{\bf ENUNȚ}: O regiune deșertică este reprezentată printr-un tablou de
dimensiuni $M \times N$ ($1 \leq M \leq 100$, $1 \leq N \leq 100$). Elementele
tabloului sunt numere naturale mai mici ca 255, reprezentând diferențele de
altitudine față de nivelul mării (cota 0). Să se stabilească:

a) Un traseu pentru a traversa deșertul de la nord la sud (de la linia 1 la
linia $M$), astfel:

\begin{itemize}

\item Se pornește dintr-un punct al liniei 1;

\item Deplasarea se poate face în una din direcțiile: E, SE, S, SV, V;

\item Suma diferențelor de nivel (la urcare și la coborâre) trebuie să fie
  minimă.

\end{itemize}

b) Un traseu pentru a traversa deșertul de la nord la sud în condițiile
punctului (a), la care se adaugă condiția:

\begin{itemize}

\item Lungimea traseului să fie minimă.

\end{itemize}

{\bf Intrarea}: Fișierul de intrare {\tt INPUT.TXT} conține un singur set de
date cu următoarea structură:

\begin{tabular}{ll}
  linia 1: & $M$ $N$ \\
  liniile $2 \dots M + 1$: & elementele tabloului (pe linii) separate prin spații \\
\end{tabular}

{\bf Ieșirea}: Fișerul de ieșire {\tt OUTPUT.TXT} va conține rezultatele în
următorul format:

\begin{verbatim}
  (a)
  <suma diferențelor de nivel>
  TRASEU: (i_1, j_1)->(i_2, j_2)->...->(i_k, j_k)
  (b)
  <suma diferențelor de nivel> <lungime traseu>
  TRASEU: (i_1, j_1)->(i_2, j_2)->...->(i_p, j_p)
\end{verbatim}

unde $i_x$ și $j_x$ sunt linia și coloana fiecărei celule vizitate.

{\bf Exemplu}:

\texttt{
  \begin{tabular}{|l|l|}
    \hline
        {\bf INPUT.TXT} & {\bf OUTPUT.TXT} \\ \hline
        4 4 &             (a) \\
        10 7 2 5 &        5 \\
        13 20 25 3 &      TRASEU: (1,3)->(2,4)->(3,3)->(3,2)->(4,1) \\
        2 4 2 20 &        (b) \\
        5 10 9 11 &       9 3 \\
        &                 TRASEU: (1,3)->(2,4)->(3,3)->(4,2) \\
    \hline
  \end{tabular}
}

Acesta a fost enunțul original. Iată acum completările propuse și o precizare
importantă:

\begin{itemize}

\item {\bf Timpul de implementare}: 45 minute - 1h (la concurs a fost cam 1h
  30 min);

\item {\bf Timpul de rulare}: 2-3 secunde;

\item {\bf Complexitatea cerută}: $O(N^2)$;

\item La punctul (b), condiția nou adăugată este mai puternică decât cea de la
  punctul (a). Cu alte cuvinte, în primul rând contează lungimea drumului și
  abia apoi, dintre toate drumurile de lungime minimă, trebuie ales cel pentru
  care suma denivelărilor este minimă. Pentru a vă convinge că ordinea în care
  sunt impuse condițiile este importantă, să privim exemplul de mai sus. Dacă
  este mai importantă minimizarea sumei denivelărilor, atunci minimul este 5,
  iar drumul este soluția de la punctul (a). Dacă este mai importantă
  minimizarea lungimii drumului, atunci lungimea minimă este 3, iar din toate
  drumurile de lungime 3, cel mai puțin costisitor este cel indicat la punctul
  (b).

\end{itemize}

{\bf REZOLVARE}: Vom lăsa punctul (b) al acestei probleme în seama
cititorului, întrucât el nu este altceva decât o simplificare a punctului
(a). Să ne ocupăm acum de punctul (a). Vom numi efort diferența de altitudine
(în modul) la deplasarea cu un pas. Scopul este deci găsirea unor drumuri de
efort total minim. Matricea de altitudini o vom nota cu $Alt$.

O primă posibilitate de abordare a problemei este „greedy”, dar aceasta nu e
cea mai fericită alegere, chiar dacă este una comodă. Ideea de bază este
următoarea: Încercăm să pornim din colțul de NV și să ne deplasăm la fiecare
pas pe acea direcție pentru care efortul este minim, până ajungem la ultima
linie. Apoi pornim din a doua coloană a primei linii și aplicăm aceeași
tactică, apoi din a treia coloană și așa mai departe până la colțul de NE. În
final tipărim soluția cea mai bună găsită. Iată însă un exemplu pe care
această metodă dă greș:

\begin{equation}
  Alt =
  \begin{pmatrix}
    2 & 1 & 2 \\
    10 & 1 & 10 \\
    10 & 10 & 10
  \end{pmatrix}
\end{equation}

Pe această matrice, algoritmul greedy va găsi traseele (1,1) $\to$ (2,2) $\to$
(3,2) de efort total 10, (1,2) $\to$ (2,2) $\to$ (3,2) de efort total 9 și
(3,1) $\to$ (2,2) $\to$ (3,2) de efort total 10. Așadar, rezultatul optim ar
fi 9, ceea ce este fals deoarece alegerea traseului (1,1) $\to$ (2,1) $\to$
(3,1) ar duce la un efort total de 8, deci mai mic.

Motivul pentru care acest algoritm nu funcționează cum trebuie este că el nu
privește în perspectivă. În cazul de mai sus, coborârea în „văile” de
altitudine 1 era o primă mutare tentantă, dar fără nici un rezultat, deoarece
până la urmă tot era necesară suirea la altitudinea 10. Soluția corectă este
ca, pentru a afla efortul minim cu care se poate ajunge la o locație oarecare,
să analizăm toate drumurile care duc la acea locație. Dacă am cunoaște efortul
minim cu care se poate ajunge la fiecare din vecinii din E, NE, N, NV, și V ai
unei celule, atunci putem cu ușurință, pe baza unor comparații, să deducem din
ce parte este cel mai avantajos să venim în respectiva celulă și cu ce efort
minim.

\newcommand\EFF{\mathit{Eff}}

Mai concret, vom construi o matrice cu aceleași dimensiuni ca și matricea
$Alt$, pe care o vom denumi $\EFF$. În această matrice, $\EFF[i,j]$ reprezintă
efortul minim necesar pentru a ajunge de pe un punct oarecare de pe linia 1 în
celula $(i, j)$. Deducem că $\EFF[1,j] = 0, \forall 1 \leq j \leq N$. Noi
trebuie să completăm matricea $\EFF$, apoi să căutăm minimul dintre toate
elementele de pe linia $M$ (care este chiar efortul minim căutat) și să
reconstituim traseul de urmat.

Ca să vedem cum anume se face completarea matricei, facem mai întâi observația
că, odată ce am ajuns pe o linie, putem fie să coborâm direct pe linia imediat
inferioară, fie să ne deplasăm câțiva pași numai spre stânga sau numai spre
dreapta, apoi să coborâm pe linia următoare. În orice locație $(X,Y)$ a
matricei putem veni dinspre E, NE, N, NV, sau V. Pentru acești cinci vecini
presupunem deja calculate eforturile minime necesare, respectiv $\EFF[X,Y+1],
\EFF[X-1,Y+1], \EFF[X-1,Y], \EFF[X-1,Y-1], \EFF[X,Y-1]$. Atunci, în funcție de
direcția din care venim, efortul depus până la punctul $(X,Y)$ va fi:

\begin{tabular}{lll}
  dinspre est:       & $\EFF[X,Y+1]+ |Alt[X,Y+1] - Alt[X,Y]|$     & (1) \\
  dinspre nord-est:  & $\EFF[X-1,Y+1]+ |Alt[X-1,Y+1] - Alt[X,Y]|$ & (2) \\
  dinspre nord:      & $\EFF[X-1,Y]+ |Alt[X-1,Y] - Alt[X,Y]|$     & (3) \\
  dinspre nord-vest: & $\EFF[X-1,Y-1]+ |Alt[X-1,Y-1] - Alt[X,Y]|$ & (4) \\
  dinspre vest:      & $\EFF[X,Y-1]+ |Alt[X,Y-1] - Alt[X,Y]|$     & (5) \\
\end{tabular}

În principiu, nu avem decât să calculăm minimul dintre aceste expresii ca să
aflăm valoarea lui $\EFF[X,Y]$. În felul acesta, matricea $\EFF$ se va completa
pe linie, de sus în jos. Totuși, apare o problemă: pentru a-l afla pe
$\EFF[X,Y]$ avem nevoie de $\EFF[X,Y-1]$ (dacă ne deplasăm spre est), iar pentru
a-l afla pe $\EFF[X,Y-1]$ avem nevoie de $\EFF[X,Y]$ (dacă ne deplasăm spre
vest)! Bineînțeles, avem sentimentul că ne învârtim după propria
coadă. Totuși, dezlegarea nu e complicată, ținând cont de observația făcută
mai sus, că pe aceeași linie deplasarea se face într-o singură direcție. Este
suficient să parcurgem fiecare linie de două ori: prima oară o parcurgem de la
stânga la dreapta, în ipoteza că deplasarea pe linia respectivă se face spre
est, apoi încă o dată de la dreapta la stânga, în ipoteza că deplasarea pe
linia respectivă se face spre vest. La prima parcurgere, vom minimiza efortul
pentru fiecare căsuță cu expresia (5), iar la a doua - cu expresia
(1). Minimizarea cu expresiile (2), (3) și (4) se poate face la oricare din
parcurgeri, deoarece elementele liniei superioare nu se mai modifică.

După cum am spus, efortul minim se obține căutând minimul de pe ultima linie a
matricei $\EFF$ (aceasta deoarece nu contează în ce punct de pe ultima linie
este sosirea). Punctul în care se atinge acest minim este tocmai punctul de
sosire. Reconstituirea efectivă a drumului se face în sens invers: se pleacă
din punctul de sosire $(i_k,j_k)$ și se caută un punct vecin lui pe una din
cele cinci direcții permise, $(i_{k-1},j_{k-1})$ astfel încât

\begin{equation}
  \EFF[i_k,j_k] = \EFF[i_{k-1},j_{k-1}] + |Alt[i_k,j_k] - Alt[i_{k-1},j_{k-1}]|
\end{equation}

Cu alte cuvinte, se testează pentru care din expresiile (1) - (5) se verifică
egalitatea. Se reia, recursiv, același procedeu pentru locația
$(i_{k-1},j_{k-1})$.

Iată cum se completează matricea $\EFF$ pentru exemplul dat și cum se
reconstituie drumul:

\begin{equation}
  Alt =
  \begin{pmatrix}
    10 &  7 &  2 &  5 \\
    13 & 20 & 25 &  3 \\
    2 &  4 &  2 & 20 \\
    5 & 10 &  9 & 11
  \end{pmatrix}
\end{equation}

\begin{equation}
  \EFF =
  \begin{pmatrix}
    0 &  0 &  0 &  0 \\
    3 & 10 & 15 &  1 \\
    6 &  4 &  2 & 18 \\
    5 & 10 &  9 & 11
  \end{pmatrix}
\end{equation}

Minimul de pe linia a 4-a a matricei $\EFF$ este $\EFF[4,1]=5$, deci sosirea se
face în colțul de SV. Din ce parte am ajuns aici? Se testează toți vecinii și
se constată că $\EFF[4,1] = \EFF[3,2] + |Alt[4,1] - Alt[3,2]|$, deci s-a venit
de la locația (3,2). Apoi se constată că:

\begin{equation}
  \begin{split}
    \EFF[3,2] & = \EFF[3,3] + |Alt[3,2] - Alt[3,3]| \\
    \EFF[3,3] & = \EFF[2,4] + |Alt[3,3] - Alt[2,4]| \\
    \EFF[2,4] & = \EFF[1,3] + |Alt[2,4] - Alt[1,3]| \\
  \end{split}
\end{equation}

Din aceste relații rezultă că traseul urmat este $(1,3) \to (2,4) \to (3,3)
\to (3,2) \to (4,1)$.

Pentru o mai mare ușurință a implementării, se vor adăuga două coloane fictive
la matricea $Alt$: coloanele 0 și $N + 1$. Facem acest lucru pentru a ne putea
referi la celula $(X,Y-1)$ atunci când $(X, Y)$ este o celulă din prima
coloană (respectiv la celula $(X,Y+1)$ atunci când $(X,Y)$ este o celulă de pe
ultima coloană) fără a primi un mesaj de eroare. Trebuie însă să fim atenți ca
nu cumva noile coloane adăugate să perturbe datele de ieșire și să rezulte că
traseul optim trece prin coloana 0 sau $N+1$. Pentru a scăpa de grija
celulelor de pe aceste două coloane și a ne asigura că ele nu vor putea fi
selectate pentru traseul optim, le vom atribui altitudini foarte
mari. Deoarece diferența maximă de nivel la fiecare pas este 255, rezultă că
efortul total maxim ce se poate obține este $255 \times 99 = 25.245$. Așadar,
o altitudine a coloanelor laterale de 30.000 este suficientă.

\inputminted{c}{src/problem5.c}
