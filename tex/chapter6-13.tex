\section{Problema 13}

Probabil că orice elev care are cât de cât experiență în programare a auzit
despre {\bf problema celui mai lung subșir crescător}. Când este prezentată la
concurs, problema e „învelită” sub diverse forme. Aici o vom formaliza din
punct de vedere matematic.

{\bf ENUNȚ}: Se dă un vector cu $N$ elemente numere întregi. Se cere să se
determine cel mai lung subșir crescător.

{\bf Intrarea}: Fișierul de tip text {\tt INPUT.TXT} conține pe prima linie
numărul $N$ de elemente din vector ($N \leq 10.000$), iar pe fiecare din
următoarele $N$ linii se află câte un element al vectorului.

{\bf Ieșirea}: În fișierul de tip text {\tt OUTPUT.TXT} se vor lista pe prima
linie lungimea $L$ a celui mai lung subșir crescător, iar pe următoarele $L$
linii subșirul în sine. Dacă există mai multe soluții, se va tipări una
singură.

{\bf Exemplu}:

\texttt{
  \begin{tabular}{|l|l|}
    \hline
        {\bf INPUT.TXT} & {\bf OUTPUT.TXT} \\ \hline
        6 & 3 \\
        2 & 2 \\
        5 & 3 \\
        7 & 4 \\
        3 &   \\
        4 &   \\
        1 &   \\
        \hline
  \end{tabular}
}

{\bf Timp de implementare}: 45 minute.

{\bf Timp de rulare}: 5 secunde.

{\bf Complexitate cerută}: $O(N \log N)$.

{\bf REZOLVARE}: Începem prin a lămuri diferența dintre noțiunile de „subșir”
și „subsecvență”. Fie $V[1], V[2], \dots, V[N]$ vectorul citit. Prin {\bf
  subșir de lungime $L$} al vectorului $V$ se înțelege o succesiune nu
neapărat continuă de elemente $V[K_1]$, $V[K_2]$, $\dots, V[K_L]$, unde $K_1 < K_2
< \dots < K_L$. Prin {\bf subsecvență de lungime $L$} a vectorului, începând
de la poziția $K$, se înțelege succesiunea continuă de elemente $V[K], V[K+1],
\dots, V[K+L-1]$.

Rezolvarea prin metoda programării dinamice este în general cunoscută și nu
vom insista asupra ei. Probabil că aflarea celui mai lung subșir crescător
este punctul de plecare al oricărui elev în învățarea programării
dinamice. Totuși, această metodă de rezolvare are complexitatea $O(N^2)$. În
continuare prezentăm o metodă mai puțin cunoscută și ceva mai dificil de
implementat, dar mult mai eficientă. Ea are complexitatea $O(N \log N)$. Vom
enunța principiul de rezolvare, urmat de o schiță a demonstrației de
corectitudine.

Fie $V$ vectorul citit. Se parcurge vectorul de la stânga la dreapta și se
construiesc în paralel doi vectori $P$ și $Q$, astfel: inițial vectorul $Q$
este vid. Se ia fiecare element din $V$ și se suprascrie peste cel mai mic
element din $Q$ care este strict mai mare ca el. Dacă nu există un asemenea
element în $Q$, cu alte cuvinte dacă elementul analizat din $V$ este mai mare
ca toate elementele din $Q$, atunci el este adăugat la sfârșitul vectorului
$Q$. Concomitent, se notează în vectorul $P$ poziția pe care a fost adăugat în
vectorul $Q$ elementul din $V$. Iată cum se construiesc vectorii $P$ și $Q$
pentru exemplul din enunț:

\centeredTikzFigure[
  mat/.style = {
    matrix of nodes,
    ampersand replacement=\&,
    nodes = {
      draw,
      rectangle,
      minimum width=2em,
      minimum height=2em,
    },
    row 1/.style = {
      nodes = {
        draw = none,
        font=\bf,
      },
    }
  },
  emph/.style = {
    fill=gray!40,
  },
]{
  \matrix[mat] {
    V \&[2em] Q \&   \&   \&[2em] P \&   \&   \&   \&   \&   \\[1em]
    2 \& \node[emph] {2}; \&   \&   \& 1 \&   \&   \&   \&   \&   \\[1em]
    5 \& 2 \& \node[emph] {5}; \&   \& 1 \& 2 \&   \&   \&   \&   \\[1em]
    7 \& 2 \& 5 \& \node[emph] {7}; \& 1 \& 2 \& 3 \&   \&   \&   \\[1em]
    3 \& 2 \& \node[emph] {3}; \& 7 \& 1 \& 2 \& 3 \& 2 \&   \&   \\[1em]
    4 \& 2 \& 3 \& \node[emph] {4}; \& 1 \& 2 \& 3 \& 2 \& 3 \&   \\[1em]

    1 \& \node[emph] {1}; \& 3 \& 4 \&
    \node[emph] {1}; \& 2 \& 3 \& \node[emph] {2}; \& \node[emph] {3}; \& 1 \\
  };
}

Lungimea $L$ la care ajunge vectorul $Q$ la sfârșitul acestei prelucrări este
tocmai lungimea celui mai lung subșir crescător al vectorului $V$. Pentru a
afla exact și care sunt elementele subșirului crescător se procedează astfel:
se caută ultima apariție în vectorul $P$ a valorii $L$. Să spunem că ea este
găsită pe poziția $K_L$. Se caută apoi ultima apariție în vectorul $P$ a
valorii $L-1$, anterior poziției $K_L$. Ea va fi pe poziția $K_{L-1} <
K_{L}$. Analog se caută în vectorul $P$ valorile $L-2, L-3, \dots, 2,
1$. Subșirul crescător este $S=(V[K_1], V[K2], \dots, V[K_L])$.

În figura de mai sus au fost hașurate elementele respective găsite în vectorul
$P$. Vectorul $Q$ are la sfârșit lungimea $L=3$, deci cel mai lung subșir
crescător are trei elemente. Se caută în vectorul $P$ cifrele 3, 2 și 1 și se
găsesc pe pozițiile $K_1=1$, $K_2=4$ și $K_3=5$, ceea ce înseamnă că cel mai
lung subșir crescător este (2, 3, 4).

Demonstrația (care, fără a pierde din corectitudine, face apel la intuiție...)
folosește aceleași notații de mai sus și are următoarele etape:

\begin{proposition}
  Vectorul $Q$ este în permanență sortat crescător.
\end{proposition}

\begin{proof}
  Folosim inducția matematică. După primul pas (inserarea elementului $V[1]$),
  vectorul $Q$ are un singur element și este bineînțeles ordonat. Trebuie acum
  arătat că dacă vectorul Q este sortat după inserarea elementului $V[i-1]$,
  el rămâne sortat și după inserarea lui $V[i]$. Într-adevăr, pentru elementul
  $V[i]$ există două variante:

  \begin{enumerate}[label=\alph*)]

  \item $V[i]$ este mai mare decât toate elementele lui $Q$, caz în care este
    adăugat la sfârșitul lui $Q$, care rămâne sortat;

  \item Există $t>1$ astfel încât $Q[t-1] \leq V[i] <Q[t]$, caz în care $V[i]$
    se suprascrie peste $Q[t]$. Dar $Q[t] \leq Q[t+1] \implies Q[t-1] \leq
    V[i] < Q[t+1]$ și vectorul $Q$ rămâne sortat.

  \end{enumerate}
\end{proof}

Această deducție ne va fi utilă în calculul complexității algoritmului.

\begin{proposition}
  Odată ce au fost scrise pentru prima oară, elementele vectorului $Q$ nu mai
  pot decât să scadă sau să rămână constante. Ele nu pot crește niciodată.
\end{proposition}

\begin{proof}
  Afirmația este evidentă, decurgând din modul de construcție a lui $Q$.
\end{proof}

\begin{proposition}
  Elementele din vectorul $V$ de la poziția $K_{i-1}+1$ la poziția $K_{i}-1$
  sunt fie mai mici decât $V[K_{i-1}]$, fie mai mari decât $V[K_i]$.
\end{proposition}

\begin{proof}
  Acest lucru este natural, deoarece dacă ar exista $K_{i-1} < X <K_i$ astfel
  încât $V[K_{i-1}] \leq V[X] \leq V[K_i]$, atunci șirul $S$ ar putea fi
  extins cu elementul $V[X]$, deci nu ar fi subșir maximal, contradicție.
\end{proof}

\begin{proposition}
  Toate elementele care urmează în $V$ după $V[K_L]$ sunt mai mici decât
  $V[K_L]$.
\end{proposition}

\begin{proof}
  Dacă ar exista $X>K_L$ astfel încât $V[X] \geq V[K_L]$, atunci $S$ ar putea fi
  extins la dreapta cu $V[X]$, contradicție.
\end{proof}

\begin{proposition}
  Elementul $V[K_1]$ este suprascris peste poziția $Q[1]$.
\end{proposition}

\begin{proof}
  Dacă elementul $V[K_1]$ este inserat în $Q$ pe o poziție $t>1$, rezultă că,
  în momentul tratării lui $V[K_1]$, pe poziția $t-1$ în vectorul $Q$ exista
  deja un număr $X < V[K_1]$. Aceasta înseamnă că $S$ poate fi prelungit la
  stânga cu $X$, deci $S$ nu este un subșir crescător maxim, contradicție.
\end{proof}

\begin{proposition}
  Orice element $V[K_i]$ va fi suprascris în $Q$ pe poziția următoare celei pe
  care a fost scris $V[K_{i-1}]$
\end{proposition}

\begin{proof}
  Să presupunem că $V[K_{i-1}]$ a fost scris peste $Q[t]$. Înainte de a ajunge
  să tratăm elementul $V[K_i]$, a trebuit să tratăm elementele $V[K_{i-1}+1],
  V[K_{i-1}+2], \dots, V[K_{i}-1]$, care, după cum s-a stabilit la punctul
  (3), pot fi:

  \begin{enumerate}[label=\alph*)]

  \item mai mici decât $V[K_{i-1}]$, caz în care ele vor fi scrise în vector
    pe poziții mai mici sau egale cu $t$, iar $Q[t]$ va scădea, deci $Q[t]
    \leq V[K_{i-1}]$ (conform punctului 2);

  \item mai mari decât $V[K_i]$, caz în care vor fi scrise în $Q$ pe poziții
    mai mari decât $t$, așadar $Q[t+1]>V[K_i]$.
  \end{enumerate}

  Se obține lanțul de inegalități $Q[t] \leq V[K_{i-1}] \leq V[K_i] \leq
  Q[t+1]$, de unde rezultă conform modului de construcție că $V[K_i]$ va fi
  scris peste $Q[t+1]$. Cu aceasta, folosind și punctul (5), am demonstrat că
  $V[K_i]$ este scris pe poziția $Q[i], \forall i=1, 2, \dots, L$.

\end{proof}

După tratarea primelor $K_L$ elemente din $V$, vectorul $Q$ are lungimea
$L$. Mai rămâne de văzut ce se întâmplă cu elementele $V[K_L+1], \dots, V[N]$.

\begin{proposition}
  Elementele care îi urmează lui $V[K_L]$ se scriu în $Q$ pe poziții mai mici
  sau egale cu $L$.
\end{proposition}

\begin{proof}
  Acest fapt este intuitiv dacă se ține cont de punctul (4). Deducem că la
  final vectorul $Q$ are lungime $L$, adică aceeași cu a lui $S$.
\end{proof}

Modul de reconstituire a lui $S$ din vectorul $P$ este corect. Trebuie să avem
grijă ca, atunci când căutăm în vectorul $P$ o apariție a valorii $X$, să o
alegem pe ultima disponibilă, altfel pot apărea erori. Spre exemplu, pentru
vectorii dați în exemplu, $V=(2, 5, 7, 3, 4, 1)$ și $P=(1, 2, 3, 2, 3, 1)$,
dacă se alege prima apariție a lui 2 (pe poziția a doua), avem subșirul $S=(2,
5, 4)$ care nu este crescător. Dacă însă alegem ultima apariție a lui 2 (pe
poziția a patra), avem subșirul crescător maximal $S=(2, 3, 4)$.

Calculul complexității este ușor de făcut:

\begin{itemize}

\item Pentru construcția vectorilor $P$ și $Q$ sunt necesare $N$ inserții în
  vectorul $Q$, care este sortat; o inserție binară cere $O(\log N)$, așadar
  complexitatea primei părți este $O(N \log N)$.

\item Pentru reconstituirea lui $S$ se face o singură parcurgere a vectorului
  $P$, deci $O(N)$.

\end{itemize}

Complexitatea totală a algoritmului este $O(N \log N)$.

\begin{lstlisting}[language=C]
#include <stdio.h>
#include <stdlib.h>
#define Infinity 30000
typedef int Vector[10001];

Vector V,P,Q,*S;
int N,Len; /* Len = Lungimea vectorului Q */

void ReadData(void)
{ FILE *F=fopen("input.txt","rt");
  int i;

  fscanf(F,"%d",&N);
  for(i=1;i<=N;i++) fscanf(F,"%d",&V[i]);
  fclose(F);
}

int Insert(int K,int Lo,int Hi)
{ int Mid=(Lo+Hi)/2;

  if (Lo==Hi)
    { if (Hi>Len) Q[++Len+1]=Infinity;
      Q[Lo]=K;
      return Lo;
    }
    else if (K<Q[Mid]) return Insert(K,Lo,Mid);
                     else return Insert(K,Mid+1,Hi);
}

void BuildPQ(void)
{ int i,Place;

  Len=0;Q[1]=Infinity;
  for (i=1;i<=N;i++)
    P[i]=Insert(V[i],1,Len+1);
}

void BuildS(void)
{ int i,K=N;

  S=malloc(sizeof(*S));
  for (i=Len;i;i--)
    { while (P[K]!=i) K--;
      (*S)[i]=V[K];
    }
}

void WriteSolution(void)
{ FILE *F=fopen("output.txt","wt");
  int i;

  fprintf(F,"%d\n",Len);
  for(i=1;i<=Len;i++) fprintf(F,"%d\n",(*S)[i]);

  fclose(F);
}

void main(void)
{
  ReadData();
  BuildPQ();
  BuildS();
  WriteSolution();
}
\end{lstlisting}

Menționăm că în sursa de mai sus se putea construi vectorul $S$ peste vectorul
$Q$, deoarece pentru construirea lui $S$ nu avem nevoie decât de elementele
vectorului $P$. În acest fel, programul ar fi avut nevoie numai de trei
vectori, iar volumul total de date nu ar fi depășit un segment. Am preferat
totuși varianta în care vectorul $S$ este alocat dinamic pentru a evita
confuziile.
