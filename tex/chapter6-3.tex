\section{Problema 3}

Problema de mai jos este un exemplu de situație în care căutarea exhaustivă a
soluției este cea mai bună alegere. Ea a fost propusă spre rezolvare la a
VIII-a Olimpiadă Internațională de Informatică, Veszprem, Ungaria 1996.

{\bf ENUNȚ}: Văzând succesul cubului său magic, Rubik a inventat versiunea
plană a jocului, numit „pătrate magice”. Se folosește o tablă compusă din 8
pătrate de dimensiuni egale. Cele opt pătrate au culori distincte, codificate
prin numere de la 1 la 8, ca în figura următoare:

\centeredTikzFigure[
  mat/.style = {matrix of nodes, ampersand replacement=\&, nodes=cell},
  cell/.style = {rectangle, draw, minimum width=1.5em, minimum height=1.5em},
]{
  \matrix[mat] {
    1 \& 2 \& 3 \& 4 \\
    8 \& 7 \& 6 \& 5 \\
  };
}

Configurația tablei se poate reprezenta într-un vector cu 8 elemente citind
cele opt pătrate, începând din colțul din stânga sus și mergând în sens
orar. De exemplu, configurația din figură se reprezintă prin vectorul (1, 2,
3, 4, 5, 6, 7, 8). Aceasta este configurația inițială a tablei.

Unei configurații i se pot aplica trei transformări elementare, identificate
prin literele „A”, „B” și „C”:

\begin{itemize}

\item „A”  schimbă între ele cele două linii ale tablei;

\item „B”  rotește circular spre dreapta întregul dreptunghi (cu o poziție);

\item „C”  rotește în sens orar cele patru pătrate centrale (cu o poziție);

\end{itemize}

Efectele transformărilor elementare asupra configurației inițiale sunt
reprezentate în figura de mai jos:

\centeredTikzFigure[
  mat/.style = {matrix of nodes, ampersand replacement=\&, nodes=cell},
  cell/.style = {rectangle, draw, minimum width=1.5em, minimum height=1.5em},
  caption/.style = {font=\large\bf, yshift=-1em},
]{
  \matrix[mat] (m1) {
    8 \& 7 \& 6 \& 5 \\
    1 \& 2 \& 3 \& 4 \\
  };

  \matrix[mat] (m2) at (5, 0) {
    4 \& 1 \& 2 \& 3 \\
    5 \& 8 \& 7 \& 6 \\
  };

  \matrix[mat] (m3) at (10, 0) {
    1 \& 7 \& 2 \& 4 \\
    8 \& 6 \& 3 \& 5 \\
  };

  \node[caption] at (m1.south) {A};
  \node[caption] at (m2.south) {B};
  \node[caption] at (m3.south) {C};
}

Din configurația inițială se poate ajunge în orice configurație folosind doar
combinații de tranformări elementare. Trebuie să scrieți un program care
calculează o secvență de transformări elementare care să aducă tabla de la
configurația inițială la o anumită configurație finală cerută.

{\bf Intrarea}: Fișierul {\tt INPUT.TXT} conține 8 întregi pe aceeași linie,
separați prin spații, descriind configurația finală.

Ieșirea se va face în fișierul {\tt OUTPUT.TXT}. Pe prima linie a acestuia se
va tipări lungimea $L$ a secvenței de transformări, iar pe fiecare din
următoarele $L$ linii se va tipări câte un caracter „A”, „B” sau „C”,
corespunzător mutărilor care trebuie efectuate.

{\bf Exemplu}:

\texttt{
  \begin{tabular}{| l | l |}
    \hline
        {\bf INPUT.TXT} & {\bf OUTPUT.TXT} \\ \hline
        2 6 8 4 5 7 3 1 & 7 \\
        & B \\
        & C \\
        & A \\
        & B \\
        & C \\
        & C \\
        & B \\
    \hline
  \end{tabular}
}

{\bf Timp limită pentru un test}: 20 secunde.

{\bf Timp de implementare}: 1h 30min - 1h 45min

{\bf Note}:

\begin{enumerate}

\item La concurs s-au acordat, pentru fiecare test, două puncte dacă se
  furniza o soluție și încă două dacă lungimea ei nu depășea 300 de mutări.

\item Concurenților li s-a furnizat un program auxiliar, {\tt MTOOL.EXE}, cu
  care se puteau verifica soluțiile furnizate.

\end{enumerate}

{\bf REZOLVARE}: Și la această problemă se întrevăd două abordări, ca și în
problema ceasurilor: una bazată pe mutări predefinite, iar cealaltă pe o
căutare exhaustivă a soluției. De data aceasta însă, prima este
neinspirată. Să le analizăm pe rând pe fiecare, plecând de la următoarele
considerente:

\begin{itemize}

\item Dacă se aplică de două ori la rând mutarea A, tabla rămâne nemodificată;

\item Dacă se aplică de patru ori consecutiv una din mutările B sau C, tabla
  rămâne nemodificată;

\item Ordinea în care se efectuează mutările contează.

\end{itemize}

Soluția pe care autorul a prezentat-o la concurs avea predefinite mai multe
mutări care schimbau între ele oricare două pătrate vecine de pe
tablă. Mergând din aproape în aproape, fiecare pătrat era adus în poziția
corespunzătoare. Spre exemplu, succesiunea de mutări predefinite care duceau
la configurația din exemplu este:

\centeredTikzFigure[
  mat/.style = {matrix of nodes, ampersand replacement=\&, nodes=cell},
  cell/.style = {rectangle, draw, minimum width=1.5em, minimum height=1.5em},
  emph/.style = {fill=gray!50!white},
]{
  % the 8 boards
  \matrix[mat] (m1) {
    \node[emph]{1}; \& \node[emph]{2}; \& 3 \& 4 \\
    8 \& 7 \& 6 \& 5 \\
  };

  \matrix[mat] (m2) at (3.5, 0) {
    2 \& \node[emph]{1}; \& 3 \& 4 \\
    8 \& \node[emph]{7}; \& 6 \& 5 \\
  };

  \matrix[mat] (m3) at (7, 0) {
    2 \& 7 \& 3 \& 4 \\
    \node[emph]{8}; \& \node[emph]{1}; \& 6 \& 5 \\
  };

  \matrix[mat] (m4) at (10.5, 0) {
    2 \& 7 \& \node[emph]{3}; \& 4 \\
    1 \& 8 \& \node[emph]{6}; \& 5 \\
  };

  \matrix[mat] (m5) at (0, -2) {
    2 \& \node[emph]{7}; \& \node[emph]{6}; \& 4 \\
    1 \& 8 \& 3 \& 5 \\
  };

  \matrix[mat] (m6) at (3.5, -2) {
    2 \& 6 \& 7 \& 4 \\
    1 \& \node[emph]{8}; \& \node[emph]{3}; \& 5 \\
  };

  \matrix[mat] (m7) at (7, -2) {
    2 \& 6 \& \node[emph]{7}; \& 4 \\
    1 \& 3 \& \node[emph]{8}; \& 5 \\
  };

  \matrix[mat] (m8) at (10.5, -2) {
    2 \& 6 \& 8 \& 4 \\
    1 \& 3 \& 7 \& 5 \\
  };

  % arrows
  \draw[->] (m1.east) -- (m2.west);
  \draw[->] (m2.east) -- (m3.west);
  \draw[->] (m3.east) -- (m4.west);
  \draw[->] (m4.east) -- ++(0.75,0);
  \draw[->] (m5.east) -- (m6.west);
  \draw[->] (m6.east) -- (m7.west);
  \draw[->] (m7.east) -- (m8.west);
}

Această soluție funcționează instantaneu și este relativ ușor de
implementat. Ea are însă defectul că soluția furnizată este extrem de lungă,
ajungând frecvent la 500 de mutări. Din cele zece teste date, numai trei s-au
încadrat în limita de 300 de mutări. Iată mai jos și sursa Pascal prezentată
la concurs, care a câștigat numai 26 din cele 40 de puncte acordate pentru
problemă:

\begin{lstlisting}[language=Pascal]
program Magic;
{$B-,I-,R-,S-}
{ Tabla este 1 2 3 4
             A B C D }
const r12='BCBBB'; { Roteste in sens orar coloanele 12 }
      r23='C';     { Roteste in sens orar coloanele 23 }
      r34='BBBCB'; { Roteste in sens orar coloanele 34 }
      r14='BBCBB'; { Roteste in sens orar coloanele 41 }

      plBCD=r12+r23+r34+r14; { Permuta patratele BCD }
      PL234=plBCD+'A'+plBCD+'A'; { Permuta coloanele 234 }

      { SXY schimba intre ele coloanele X si Y }
      S14='B'+PL234;
      S12='BBB'+S14+'B';
      S23='BB'+S14+'BB';
      S34='B'+S14+'BBB';
      S13=r12+r12+r23+r23+r12+r12;
      S24='B'+S13+'BBB';

      { RevXY schimba intre ele patratele vecine X si Y }
      RevC3=plBCD+r23+r12+r12+r34+r34+'BBB'+plBCD+'A';
      RevD4='BBB'+RevC3+'B';
      RevB2='B'+RevC3+'BBB';
      RevA1='BB'+RevC3+'BB';

      Rev23='C'+RevC3+'CCC';
      Rev12='B'+Rev23+'BBB';
      Rev34='BBB'+Rev23+'B';
      Rev14='BB'+Rev23+'BB';

      RevAB='A'+Rev12+'A';
      RevBC='A'+Rev23+'A';
      RevCD='A'+Rev34+'A';
      RevAD='A'+Rev14+'A';

type Matrix=array[1..2,1..4] of Integer;
     Vector=array[1..60000] of Char;
var A,B:Matrix;
    V:Vector;
    N:Integer;

procedure MakeAMatrix;
begin
  A[1,1]:=1;
  A[1,2]:=2;
  A[1,3]:=3;
  A[1,4]:=4;
  A[2,1]:=8;
  A[2,2]:=7;
  A[2,3]:=6;
  A[2,4]:=5;
end;

procedure ReadBMatrix;
begin
  Assign(Input,'input.txt');
  Reset(Input);
  Read(B[1,1]);
  Read(B[1,2]);
  Read(B[1,3]);
  Read(B[1,4]);
  Read(B[2,4]);
  Read(B[2,3]);
  Read(B[2,2]);
  Read(B[2,1]);
  Close(Input);
end;

procedure AddString(S:String);
{ Adauga o secventa la sirul-solutie }
var i:Integer;
begin
  for i:=1 to Length(S) do
    begin
      Inc(N);
      V[N]:=S[i];
    end;
end;

procedure FindElement(K:Integer;var X,Y:Integer);
{ Cauta un element intr-o permutare }
var i,j:Integer;
begin
  for i:=1 to 2 do
    for j:=1 to 4 do
      if A[i,j]=K then begin
                         X:=i;
                         Y:=j;
                         Exit;
                       end;
end;

procedure Switch(var X,Y:Integer);
{ Schimba intre ele doua numere }
var IAux:Integer;
begin
  IAux:=X;X:=Y;Y:=IAux;
end;

procedure Process;
{ Transforma pozitia in pozitia B prin schimbari
  repetate ale elementelor vecine }
var i,j,k,l,m:Integer;
begin
  for j:=1 to 4 do
    for i:=1 to 2 do
      begin
        FindElement(B[i,j],k,l);
        { Gaseste elementul care trebuie adus
          pe pozitia (i,j) }
        if k<>i then begin
                       { Il aduce pe linia corecta }
                       case l of
                         1:AddString(RevA1);
                         2:AddString(RevB2);
                         3:AddString(RevC3);
                         4:AddString(RevD4);
                       end; {case}
                       Switch(A[k,l],A[i,l]);
                       k:=i;
                     end;
        for m:=l downto j+1 do
          { Il aduce pe coloana corecta }
          begin
            if k=1
              then case m of
                     2:AddString(Rev12);
                     3:AddString(Rev23);
                     4:AddString(Rev34);
                   end
              else case m of
                     2:AddString(RevAB);
                     3:AddString(RevBC);
                     4:AddString(RevCD);
                   end;
            Switch(A[k,m],A[k,m-1]);
          end;
      end;
end;

procedure Cut(K,D:Integer);
{ Taie din vectorul V D pozitii incepand cu K }
var i:Integer;
begin
  for i:=K to N-D do
    V[i]:=V[i+D];
  Dec(N,D);
end;

procedure Reduce;
{ Reduce secventele de mutari identice }
var i:Integer;
begin
  i:=1;
  repeat
    case V[i] of
      'A':if (i<=N-1) and (V[i+1]='A')
            then Cut(i,2)
            else Inc(i);
      'B':if (i<=N-3) and (V[i+1]='B')
            and (V[i+2]='B') and (V[i+3]='B')
            then Cut(i,4)
            else Inc(i);
      'C':if (i<=N-3) and (V[i+1]='C')
            and (V[i+2]='C') and (V[i+3]='C')
            then Cut(i,4)
            else Inc(i);
    end; {case}
  until i=N;
end;

procedure WriteSolution;
var i:Integer;
begin
  Assign(Output,'output.txt');
  Rewrite(Output);
  WriteLn(N);
  for i:=1 to N do WriteLn(V[i]);
  Close(Output);
end;

begin
  N:=0;
  MakeAMatrix;
  ReadBMatrix;
  Process;
  Reduce;
  WriteSolution;
end.
\end{lstlisting}

Singura soluție pare deci a fi una de tipul Branch and Bound, care nu este
tocmai la îndemână. Cu toate acestea, numărul total de configurații posibile
ale tablei este de numai 8! = 40.320. Într-adevăr, fiecare poziție de pe tablă
se reprezintă printr-o permutare a mulțimii {1,2,3,4,5,6,7,8}. Se poate face
deci cu ușurință o căutare exhaustivă a soluției. Aceasta simplifică mult
structurile de date folosite (implementarea Branch and Bound folosește
structuri destul de încâlcite). În plus, practica arată că se poate ajunge în
orice configurație în mai puțin de 25 de mutări.

Algoritmul de căutare este cunoscut sub numele de algoritmul lui Lee și are la
bază următoarea idee: Se pornește cu configurația inițială, care este depusă
într-o coadă. La fiecare pas se extrage prima configurație disponibilă din
coadă, se efectuează pe rând fiecare din cele trei mutări și se obțin trei
succesori. Aceștia sunt adăugați la sfârșitul cozii, dacă nu există deja în
coadă. Acest pas se numește {\bf expandare}. Expandarea continuă până când
elementul selectat spre expandare este tocmai configurația finală.

Figura următoare indică modul de expandare a cozii, cu mențiunea că printr-o
succesiune de litere ne-am referit la configurația care se obține efectuând
mutările respective:

\centeredTikzFigure[
  scale=0.8,
  mat/.style = {
    matrix of nodes,
    ampersand replacement=\&,
    row sep=1em,
    column sep=3em,
    nodes=cell,
  },
  cell/.style = {scale=0.8, anchor=west},
  conf/.style = {rectangle, draw, anchor=west, xshift=1em},
]{
  \matrix[mat] {
    \& {\bf Configurații expandate} \& {\bf Coada de configurații neexpandate}
    \\
    Pasul 0:
    \&
    \&
    \node[conf] {Inițială};
    \\
    Pasul 1:
    \&
    \node[conf] {Inițială};
    \&
    \node[conf] (a1) {A};
    \node[conf] (a2) at (a1.east) {B};
    \node[conf] (a3) at (a2.east) {C};
    \draw[-latex] (a1) -- (a2);
    \draw[-latex] (a2) -- (a3);
    \\
    Pasul 2:
    \&
    \node[conf] (b1) {Inițială};
    \node[conf] (b2) at (b1.east) {A};
    \&
    \node[conf] (b3) {B};
    \node[conf] (b4) at (b3.east) {C};
    \node[conf] (b5) at (b4.east) {AB};
    \node[conf] (b6) at (b5.east) {AC};
    \draw[-latex] (b3) -- (b4);
    \draw[-latex] (b4) -- (b5);
    \draw[-latex] (b5) -- (b6);
    \\
    Pasul 3:
    \&
    \node[conf] (c1) {Inițială};
    \node[conf] (c2) at (c1.east) {A};
    \node[conf] (c3) at (c2.east) {B};
    \&
    \node[conf] (c4) {C};
    \node[conf] (c5) at (c4.east) {AB};
    \node[conf] (c6) at (c5.east) {AC};
    \node[conf] (c7) at (c6.east) {BB};
    \node[conf] (c8) at (c7.east) {BC};
    \draw[-latex] (c4) -- (c5);
    \draw[-latex] (c5) -- (c6);
    \draw[-latex] (c6) -- (c7);
    \draw[-latex] (c7) -- (c8);
    \\
    Pasul 4:
    \&
    \node[conf] (d1) {Inițială};
    \node[conf] (d2) at (d1.east) {A};
    \node[conf] (d3) at (d2.east) {B};
    \node[conf] (d4) at (d3.east) {C};
    \&
    \node[conf] (d5) {AB};
    \node[conf] (d6) at (d5.east) {AC};
    \node[conf] (d7) at (d6.east) {BB};
    \node[conf] (d8) at (d7.east) {BC};
    \node[conf] (d9) at (d8.east) {CA};
    \node[conf] (d10) at (d9.east) {CB};
    \node[conf] (d11) at (d10.east) {CC};
    \draw[-latex] (d5) -- (d6);
    \draw[-latex] (d6) -- (d7);
    \draw[-latex] (d7) -- (d8);
    \draw[-latex] (d8) -- (d9);
    \draw[-latex] (d9) -- (d10);
    \draw[-latex] (d10) -- (d11);
    \\
  };
}

Se observă că, la pasul 2, în coadă au fost adăugate doar configurațiile „AB”
și „AC”, iar configurația „AA” nu, deoarece prin efectuarea de două ori a
mutării „A” se revine la configurația inițială, care a fost deja expandată. De
asemenea, la pasul 3, după expandarea configurației „B” au fost adăugate în
coadă numai configurațiile „BB” și „BC”, deoarece configurația „BA” este
echivalentă cu configurația „AB”, aflată deja în listă.

Pseudocodul algoritmului este:

\vspace{\algskip}
\begin{algorithmic}[1]
  \STATE citește datele de intrare
  \STATE inițializează coada cu configurația inițială
  \WHILE{primul element al cozii nu este configurația finală}
  \STATE expandează primul element al cozii
  \FOR{$i = A, B, C$}
  \IF{succesorul $i$ nu a fost deja pus în coadă}
  \STATE adaugă succesorul $i$ în coadă
  \ENDIF
  \ENDFOR
  \STATE șterge primul element al cozii
  \ENDWHILE
  \STATE reconstituie șirul de mutări
\end{algorithmic}

Algoritmul de mai sus garantează și găsirea soluției optime (în număr minim de
mutări). Rămân de lămurit două lucruri: (1) Cum ne dăm seama dacă o
configurație există deja în coadă și (2) cum se face reconstituirea soluției.

Pentru a afla dacă o configurație mai există în listă, cea mai simplă metodă
ar fi o căutare secvențială a listei. Totuși, această versiune ar fi extrem de
lentă, deoarece coada atinge rapid dimensiuni respectabile (de ordinul miilor
de elemente). În plus, un element al listei ar reține configurația
propriu-zisă (un vector cu opt elemente), ceea ce ar duce la un consum ridicat
de memorie. Testul de egalitate a doi vectori ar fi și el costisitor din punct
de vedere al timpului.

Există însă o altă metodă mai simplă. Am demonstrat că numărul de configurații
posibile ale tablei este 8! = 40.320.  Dacă am putea găsi o funcție bijectivă
$H: \mathbf{P_8} \to \{0, 1, \dots, 40.319\}$, unde $\mathbf{P_8}$ este
mulțimea permutărilor de 8 elemente, atunci ar fi suficient un vector
caracteristic cu 40.320 elemente. De îndată ce introducem în coadă o nouă
configurație $K$, nu avem decât să bifăm elementul corespunzător din vectorul
caracteristic. Înainte de a adăuga o configurație în coadă, testăm dacă nu
cumva elementul corespunzător ei a fost deja bifat, semn că nodul a mai fost
vizitat.

Cum se construiește funcția $H$? Pentru orice permutare $p \in \mathbf{P_8}$,
$H(p)$ este poziția lui $p$ în ordonarea lexicografică a lui $\mathbf{P_8}$
(începând de la 0):

\begin{align*}
H(1, 2, 3, 4, 5, 6, 7, 8) & = 0 \\
H(1, 2, 3, 4, 5, 6, 8, 7) & = 1 \\
H(1, 2, 3, 4, 5, 7, 6, 8) & = 2 \\
\cdots \\
H(8, 7, 6, 5, 4, 3, 1, 2) & = 40.318 \\
H(8, 7, 6, 5, 4, 3, 2, 1) & = 40.319 \\
\end{align*}

Se observă că primele 7! = 5.040 elemente din ordonare au pe prima poziție un
1, următoarele 5.040 au pe prima poziție un 2 etc. De asemenea, dintre
elementele care au pe prima poziție un 1, primele 6! = 720 au pe a doua
poziție un 2, următoarele 720 au pe a doua poziție un 3 etc.

Să calculăm de exemplu $H(2, 6, 8, 4, 5, 7, 3, 1)$. Prima cifră a permutării
este 2, deci se adaugă 7! = 5.040. Rămân cifrele 1, 3, 4, 5, 6, 7 și 8. A doua
cifră a permutării este 6, a cincea ca valoare dintre cifrele rămase, deci se
adaugă $4 \times 6! = 2.880$. Rămân cifrele 1, 3, 4, 5, 7 și 8 etc. Se aplică
procedeul până la ultima cifră și rezultă:

\begin{tabular}{|l|l|l|}
  \hline
  Cifre rămase           & Permutarea & Valoarea adăugată \\ \hline
  1, 2, 3, 4, 5, 6, 7, 8 & 2          & $1 \times 7! = 5.040$ \\
  1, 3, 4, 5, 6, 7, 8    & 6          & $4 \times 6! = 2.880$ \\
  1, 3, 4, 5, 7, 8       & 8          & $5 \times 5! = 600$ \\
  1, 3, 4, 5, 7          & 4          & $2 \times 4! = 48$ \\
  1, 3, 5, 7             & 5          & $2 \times 3! = 12$ \\
  1, 3, 7                & 7          & $2 \times 2! = 4$ \\
  1, 3                   & 3          & $1 \times 1! = 1$ \\
  1                      & 1          & $0 \times 0! = 0$ \\ \hline
                         &            & $H(p) = 8.585 $\\ \hline
\end{tabular}

Reciproc se construiește permutarea când i se cunoaște valoarea atașată:

\small{
  \begin{tabular}{|l|l|l|l|l|}
    \hline
    Cifre nefolosite       & $H(p)$ &                      & Cifra selectată & \\ \hline
    1, 2, 3, 4, 5, 6, 7, 8 & 8.585  & $8.585 \bdiv 7! = 1$ & 2 & $8.585 \bmod 7! = 3.545$ \\
    1, 3, 4, 5, 6, 7, 8    & 3.545  & $3.545 \bdiv 6! = 4$ & 6 & $3.545 \bmod 6! = 665$ \\
    1, 3, 4, 5, 7, 8       & 665    & $665 \bdiv 5! = 5$   & 8 & $665 \bmod 5! = 65$ \\
    1, 3, 4, 5, 7          & 65     & $65 \bdiv 4! = 2$    & 4 & $65 \bmod 4! = 17$ \\
    1, 3, 5, 7             & 17     & $17 \bdiv 3! = 2$    & 5 & $17 \bmod 3! = 5$ \\
    1, 3, 7                & 5      & $5 \bdiv 2! = 2$     & 7 & $5 \bmod 2! = 1$ \\
    1, 3                   & 1      & $1 \bdiv 1! = 1$     & 3 & $1 \bmod 1! = 0$ \\
    1                      & 0      & $0 \bdiv 0! = 0$     & 1 & \\ \hline
  \end{tabular}
}

Rezultă $p = (2, 6, 8, 4, 5, 7, 3, 1)$.

Această metodă de căutare are și avantajul că în listă se va ține un singur
număr pe doi octeți, făcându-se economie de memorie. Expandarea unui nod
constă din trei pași:

\begin{enumerate}

\item Se extrage primul număr din listă și se reconstituie configurația
  atașată;

\item Se fac cele trei mutări, obținându-se trei succesori;

\item Pentru fiecare succesor se calculează funcția $H$ și dacă configurația
  nu este găsită în listă, este adăugată.

\end{enumerate}

Pentru a face reconstituirea soluției avem nevoie de date
suplimentare. Respectiv, vectorul caracteristic atașat permutărilor nu va mai
reține doar dacă o poziție a fost „văzută” sau nu, ci și poziția din care ea
provine (prin valoarea funcției $H$). Trebuie de asemenea reținut tipul
mutării (A, B sau C) prin care s-a ajuns în acea configurație. Cei doi vectori
se numesc {\tt Father} și {\tt MoveKind}. Inițial, toate elementele vectorului
{\tt Father} au eticheta „Unknown”, semnificând că nodurile nu au fost încă
vizitate, cu excepția elementului atașat configurației inițiale, care poartă
eticheta specială „Root” (rădăcină).

Pseudocodul pentru expandarea unui nod arată cam așa:

\vspace{\algskip}
\begin{algorithmic}[1]
  \STATE $K \leftarrow$ primul număr din coadă
  \STATE $P \leftarrow H^{-1}(K)$
  \STATE află cei trei succesori $Q_A, Q_B, Q_C$
  \FOR{$i = A, B, C$}
  \IF{$Father[H(Q_i)] = Unknown$}
  \STATE $Father[H(Q_i)] \leftarrow K$
  \STATE $MoveKind[H(Q_i)] \leftarrow i$
  \STATE adaugă $H(Q_i)$ în coadă
  \ENDIF
  \ENDFOR
  \STATE șterge $K$ din coadă
\end{algorithmic}

Reconstituirea soluției se face recursiv: se pornește de la configurația
finală și se merge înapoi (folosind informația din vectorul {\tt Father}) până
la configurația inițială, măsurându-se astfel numărul de mutări. La revenire
se tipăresc toate mutările efectuate (folosind informația din vectorul {\tt
  MoveKind}).

\begin{lstlisting}[language=C]
#include <stdio.h>
#include <stdlib.h>
#define Unknown 0xFFFF
#define Root 0xFFFE

typedef huge unsigned Vector[40320];
typedef char CharVector[40320];
typedef int Perm[8];
typedef struct list { unsigned X; struct list * Next; } List;

Perm StartPerm={1,2,3,4,5,6,7,8}, EndPerm;
const Perm Moves[3]=
   {{7,6,5,4,3,2,1,0},
    {3,0,1,2,5,6,7,4},
    {0,6,1,3,4,2,5,7}};
/* Cele trei tipuri de mutari */
Vector Father; /* Legaturile de tip tata */
CharVector MoveKind;
unsigned StartValue,EndValue;
/* Valorile atasate configuratiilor initiala si finala */
List *Head, *Tail; /* Coada de expandat */

/**** Bijectia care asociaza un numar unei permutari ****/

unsigned Perm2Int(Perm P)
{ int i,j,k,Fact=5040;
  unsigned Sum=0;

  for (i=0;i<=6;i++)
    { k=P[i]-1;
      for (j=0;j<i;j++)
        if (P[j]<P[i]) k--;
      Sum+=k*Fact;
      Fact/=(7-i);
    }
  return Sum;
}

void Int2Perm(unsigned Sum, Perm P)
{ int i,j,k,Order,Fact=5040;
  Perm Used={0,0,0,0,0,0,0,0};

  for (i=0;i<=7;i++)
    { Order=Sum/Fact;
      j=-1;
      for (k=0;k<=Order;k++)
        do j++; while (Used[j]);
      Used[j]=1;
      P[i]=j+1;
      Sum%=Fact;
      if (i!=7) Fact/=(7-i);
    }
}

/**** Lucrul cu liste ****/

void InitList(void)
{
  Head=Tail=malloc(sizeof(List));
  Tail->X=StartValue;
  Tail->Next=NULL;
}

void AddToTail(unsigned K)
{
  Tail->Next=malloc(sizeof(List));
  Tail=Tail->Next;
  Tail->X=K;
  Tail->Next=NULL;
}

void Behead(void)
/* Sterge capul listei */
{ List *LCor=Head;

  Head=Head->Next;
  free(LCor);
}

/**** Cautarea propriu-zisa ****/

void MakeMove(Perm P,Perm Q,int Kind)
/* Kind = 0, 1 sau 2 */
{ int i;
  for (i=0;i<=7;i++)
    Q[i]=P[Moves[Kind][i]];
}

void Expand(void)
{ List *LCor;
  Perm P1,P2;
  unsigned i,XSon,Done;

  InitList();
  do {
    Int2Perm(Head->X,P1);
    for(i=0;i<=2;i++)
      { MakeMove(P1,P2,i);
        XSon=Perm2Int(P2);
        if (Father[XSon]==Unknown)
          { Father[XSon]=Head->X;
            MoveKind[XSon]=i+65;
            AddToTail(XSon);
          }
      }
    Done=(Head->X==EndValue);
    Behead(); }
  while (!Done);
}

/* Intrarea si iesirea */

void InitData(void)
{ FILE *F=fopen("input.txt","rt");
  unsigned i;

  for (i=0;i<=7;i++) fscanf(F,"%d",&EndPerm[i]);
  fclose(F);
  StartValue=Perm2Int(StartPerm);
  EndValue=Perm2Int(EndPerm);
  for (i=0;i<40320;) Father[i++]=Unknown;
  Father[StartValue]=Root;
}

void WriteMove(FILE *F,unsigned K,int Len)
{
  if (K!=StartValue)
    { WriteMove(F,Father[K],Len+1);
      fprintf(F,"%c\n",MoveKind[K]);
    }
    else fprintf(F,"%d\n",Len);
}

void Restore(void)
{ FILE *F=fopen("output.txt","wt");

  WriteMove(F,EndValue,0);
  fclose(F);
}

void main(void)
{
  InitData();
  Expand();
  Restore();
}
\end{lstlisting}
