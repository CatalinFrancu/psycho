\section{Problema 7}

Problema programării unui turneu de fotbal a fost dată spre rezolvare la a
III-a Balcaniadă de Informatică, Varna 1995. Vom prezenta mai întâi enunțul
nemodificat al problemei, după care vom adăuga câteva detalii care o vor face
mai „provocatoare”.

{\bf ENUNȚ}: Una din sarcinile Ministerului Sporturilor dintr-o țară balcanică
este de a organiza un campionat de fotbal cu $N$ echipe (numerotate de la 1 la
$N$). Campionatul constă din $N$ etape (dacă $N$ este impar) sau $N-1$ etape
(dacă $N$ este par); orice două echipe dispută între ele un joc și numai
unul. Scrieți un program care realizează o programare a campionatului.

{\bf Intrarea}: Numărul de echipe $N$ ($2 \leq N \leq 24$) va fi dat la
intrarea standard (tastatură).

{\bf Ieșirea} se va face în fișierul text {\tt OUTPUT.TXT} sub forma unui
tabel cu numere întregi avînd $N$ linii. Al $j$-lea element din linia $i$ este
numărul echipei care joacă cu echipa $i$ în etapa $j$ (evident, dacă $i$ joacă
cu $k$ în etapa $j$, atunci în tabel $k$ joacă cu $i$ în aceeași etapă). Dacă
echipa $i$ este liberă în etapa $j$, atunci al $j$-lea element al liniei $i$
este zero.

{\bf Exemple}:

\texttt{
  \begin{tabular}{|l|l|}
    \hline
        {\bf INPUT.TXT} & {\bf OUTPUT.TXT} \\ \hline
        3 &           2 3 0 \\
          &           1 0 3 \\
          &           0 1 2 \\ \hline
        4 &           2 3 4 \\
          &           1 4 3 \\
          &           4 1 2 \\
          &           3 2 1 \\
    \hline
  \end{tabular}
}

{\bf Timp de implementare}: circa 1h 45’

{\bf Timp de execuție}: 33 secunde

Iată și modificările pe care le propunem pentru a aduce cu adevărat problema
la nivel de concurs:

\begin{itemize}

\item Limita pentru numărul de echipe este $N \leq 200$;

\item {\bf Timpul de implementare} este de 45 minute;

\item {\bf Timpul de execuție} este de 2-3 secunde;

\item {\bf Complexitatea cerută} este $O(N^2)$.

\end{itemize}

{\bf REZOLVARE}: Vom lăsa pe seama cititorului conceperea, implementarea și
testarea unei rezolvări backtracking (adoptată de majoritatea în timpul
concursului). De altfel, la Balcaniadă concurenții au avut la dispoziție
calculatoare 486 / 50 Mhz, iar un backtracking îngrijit funcționa cam până la
$N = 18$ în timpul impus, ceea ce asigura cam 75\% din punctajul maxim. În
afară de aceasta, se mai puteau face și alte lucruri nu tocmai elegante, având
în vedere că datele de ieșire nu erau foarte mari. Pentru $N > 18$, mulți
concurenți lăsau programul să meargă până când termina (câteva minute bune sau
chiar mai mult), apoi scriau rezultatele într-un fișier temporar. Matricea
rezultată era inclusă ca o constantă în codul sursă. Programul care era dat
comisiei de corectare se prefăcea că „se gândește” timp de câteva secunde,
apoi scria pur și simplu matricea în fișierul de ieșire. Cu aceasta s-au luat
punctaje foarte apropiate de maxim. Totuși, pentru $N = 24$ un backtracking ar
fi stat foarte mult pentru a găsi soluția. Tot timpul acesta, calculatorul era
blocat, neputând fi folosit. De aceea, mulți concurenți nu au luat punctajul
maxim, preferând să renunțe la ultimele teste și să rezolve celelalte
probleme. Oricum, backtracking-ul este în acest caz o soluție pentru care
raportul punctaj obținut / timp consumat este foarte convenabil.

Există însă și o soluție care poate asigura un punctaj maxim fără bătăi de cap
și fără să folosească „date preprocesate”. Ea are complexitatea $O(N^2)$ și nu
face decât o singură parcurgere a matricei de ieșire. Ce-i drept, autorul a
pierdut cam trei ore pentru a o găsi și implementa în timp de concurs,
compromițând aproape rezolvarea celorlalte două probleme din ziua respectivă,
dar comisia de corectare a fost plăcut impresionată, programul fiind singurul
care mergea instantaneu. Rămâne ca voi să alegeți între eleganță și
eficiență...

Dacă ținem cont și de restricțiile suplimentare propuse, rezolvarea
backtracking nu mai este valabilă. În această situație, iată care este metoda
care stă la baza rezolvării în timp pătratic. În primul rând se reduce cazul
când $N$ este impar la un caz când $N$ este par, prin mărirea cu 1 a lui $N$
și introducerea unei echipe fictive. În fiecare etapă se consideră că echipa
care trebuia să joace cu echipa fictivă stă de fapt „pe bară”. Iată de exemplu
cum se rezolvă cazul $N = 3$:

\begin{enumerate}[label=(\alph*)]

\item Se programează un campionat cu 4 echipe, echipa 4 fiind echipa
  fictivă:

  \begin{table}[H]
    \rowcolors{1}{white}{gray!5}
    \centering
    \begin{tabular}{c|ccc}
      \hline
          {\bf Echipa} & {\bf Etapa 1} & {\bf Etapa 2} & {\bf Etapa 3}\\ \hline
          {\bf 1}      & 2             & 3             & 4 \\
          {\bf 2}      & 1             & 4             & 3 \\
          {\bf 3}      & 4             & 1             & 2 \\
          {\bf 4}      & 3             & 2             & 1 \\
          \hline
    \end{tabular}
  \end{table}

\item Se neglijează ultima linie din tabel, meciurile echipei fictive nefiind
  importante:

  \begin{table}[H]
    \rowcolors{1}{white}{gray!5}
    \centering
    \begin{tabular}{c|ccc}
      \hline
          {\bf Echipa} & {\bf Etapa 1} & {\bf Etapa 2} & {\bf Etapa 3}\\ \hline
          {\bf 1}      & 2             & 3             & 4 \\
          {\bf 2}      & 1             & 4             & 3 \\
          {\bf 3}      & 4             & 1             & 2 \\
          \hline
    \end{tabular}
  \end{table}

\item Peste tot unde apare cifra 4, ea este înlocuită cu 0:

  \begin{table}[H]
    \rowcolors{1}{white}{gray!5}
    \centering
    \begin{tabular}{c|ccc}
      \hline
          {\bf Echipa} & {\bf Etapa 1} & {\bf Etapa 2} & {\bf Etapa 3}\\ \hline
          {\bf 1}      & 2             & 3             & 0 \\
          {\bf 2}      & 1             & 0             & 3 \\
          {\bf 3}      & 0             & 1             & 2 \\
          \hline
    \end{tabular}
  \end{table}

\end{enumerate}

Mai rămâne să vedem cum se tratează cazul când $N$ este par. E destul de greu
de dat o demonstrație matematică metodei care urmează; de fapt, nici nu există
una, problema fiind rezolvată în timp de concurs prin inducție incompletă
(adică s-a constatat cu creionul pe hârtie că metoda merge pentru $N = 4, 6,
8$ și 10, apoi s-a scris programul care să facă același lucru și s-a constatat
că merge și pentru valori mai mari). Să tratăm și aici cazul $N = 8$, apoi să
generalizăm procedeul.

Vor fi șapte etape și, fără a reduce cu nimic generalitatea problemei, putem
presupune că echipa 1 joacă pe rând cu echipele 2, 3, 4, 5, 6, 7 și
8. Deocamdată tabelul arată astfel:

\begin{table}[H]
  \setlength{\tabcolsep}{5pt}
  \rowcolors{1}{white}{gray!5}
  \centering
  \begin{tabular}{c|ccccccc}
    \hline
        {\bf Echipa} & {\bf Etapa 1} & {\bf Etapa 2} & {\bf Etapa 3} &
        {\bf Etapa 4} & {\bf Etapa 5} & {\bf Etapa 6} & {\bf Etapa 7} \\ \hline
        {\bf 1} & 2 & 3 & 4 & 5 & 6 & 7 & 8 \\
        {\bf 2} & 1 &   &   &   &   &   &   \\
        {\bf 3} &   & 1 &   &   &   &   &   \\
        {\bf 4} &   &   & 1 &   &   &   &   \\
        {\bf 5} &   &   &   & 1 &   &   &   \\
        {\bf 6} &   &   &   &   & 1 &   &   \\
        {\bf 7} &   &   &   &   &   & 1 &   \\
        {\bf 8} &   &   &   &   &   &   & 1 \\
        \hline
  \end{tabular}
\end{table}

Echipa 2 are deja programat un meci cu echipa 1 în prima etapă și mai are de
programat meciurile cu echipele 3, 4, 5, 6, 7 și 8 în celelalte
etape. Așezarea se poate face oricum, cu condiția ca echipele 1 și 2 să nu își
aleagă același partener în aceeași etapă. De exemplu, putem programa meciul
2-3 în etapa a 3-a, meciul 2-4 în etapa a 4-a, ..., iar meciul 2-8 în etapa a
2-a. Astfel am completat linia a doua a tabloului:

\begin{table}[H]
  \setlength{\tabcolsep}{5pt}
  \rowcolors{1}{white}{gray!5}
  \centering
  \begin{tabular}{c|ccccccc}
    \hline
        {\bf Echipa} & {\bf Etapa 1} & {\bf Etapa 2} & {\bf Etapa 3} &
        {\bf Etapa 4} & {\bf Etapa 5} & {\bf Etapa 6} & {\bf Etapa 7} \\ \hline
        {\bf 1} & 2 & 3 & 4 & 5 & 6 & 7 & 8 \\
        {\bf 2} & 1 & 8 & 3 & 4 & 5 & 6 & 7 \\
        {\bf 3} &   & 1 & 2 &   &   &   &   \\
        {\bf 4} &   &   & 1 & 2 &   &   &   \\
        {\bf 5} &   &   &   & 1 & 2 &   &   \\
        {\bf 6} &   &   &   &   & 1 & 2 &   \\
        {\bf 7} &   &   &   &   &   & 1 & 2 \\
        {\bf 8} &   & 2 &   &   &   &   & 1 \\
        \hline
  \end{tabular}
\end{table}

Echipa 3 are programate meciurile cu 1 și 2 și mai are de programat meciurile
cu echipele 4, 5, 6, 7 și 8. Pentru a nu crea conflicte (adică două echipe să
nu-și aleagă același adversar), putem aplica același procedeu: pornim de la
etapa a 5-a și completăm toate celulele goale ale liniei a 3-a cu numerele de
la 4 la 8, mergând circular spre dreapta:

\begin{table}[H]
  \setlength{\tabcolsep}{5pt}
  \rowcolors{1}{white}{gray!5}
  \centering
  \begin{tabular}{c|ccccccc}
    \hline
        {\bf Echipa} & {\bf Etapa 1} & {\bf Etapa 2} & {\bf Etapa 3} &
        {\bf Etapa 4} & {\bf Etapa 5} & {\bf Etapa 6} & {\bf Etapa 7} \\ \hline
        {\bf 1} & 2 & 3 & 4 & 5 & 6 & 7 & 8 \\
        {\bf 2} & 1 & 8 & 3 & 4 & 5 & 6 & 7 \\
        {\bf 3} & 7 & 1 & 2 & 8 & 4 & 5 & 6 \\
        {\bf 4} &   &   & 1 & 2 & 3 &   &   \\
        {\bf 5} &   &   &   & 1 & 2 & 3 &   \\
        {\bf 6} &   &   &   &   & 1 & 2 & 3 \\
        {\bf 7} & 3 &   &   &   &   & 1 & 2 \\
        {\bf 8} &   & 2 &   & 3 &   &   & 1 \\
        \hline
  \end{tabular}
\end{table}

Se folosește aceeași metodă pentru a completa și celelalte linii ale
matricei. Pentru fiecare linie $i$ ($i \leq N-1$):

\begin{itemize}

\item Se caută pe linia $i$ apariția valorii $i$-1;

\item Se caută primul spațiu liber (mergând circular spre dreapta). Primul
  spațiu liber înseamnă prima etapă în care se poate programa meciul dintre
  echipele $i$ și $i+1$ (trebuie ca ambele să fie libere în acea etapă). Se
  constată experimental că trebuie început de la a doua poziție liberă de după
  apariția valorii $i-1$;

\item Se merge circular spre dreapta și, în căsuțele libere întâlnite se trec
  valorile $i+1, i+2, \dots, N$. Concomitent, pe aceleași coloane ale liniilor
  $i+1, i+2, \dots, N$ se trece valoarea $i$.

\end{itemize}

\begin{table}[H]
  \setlength{\tabcolsep}{5pt}
  \rowcolors{1}{white}{gray!5}
  \centering
  \begin{tabular}{c|ccccccc}
    \hline
        {\bf Echipa} & {\bf Etapa 1} & {\bf Etapa 2} & {\bf Etapa 3} &
        {\bf Etapa 4} & {\bf Etapa 5} & {\bf Etapa 6} & {\bf Etapa 7} \\ \hline
        {\bf 1} & 2 & 3 & 4 & 5 & 6 & 7 & 8 \\
        {\bf 2} & 1 & 8 & 3 & 4 & 5 & 6 & 7 \\
        {\bf 3} & 7 & 1 & 2 & 8 & 4 & 5 & 6 \\
        {\bf 4} & 6 & 7 & 1 & 2 & 3 & 8 & 5 \\
        {\bf 5} & 8 & 6 & 7 & 1 & 2 & 3 & 4 \\
        {\bf 6} & 4 & 5 & 8 & 7 & 1 & 2 & 3 \\
        {\bf 7} & 3 & 4 & 5 & 6 & 8 & 1 & 2 \\
        {\bf 8} & 5 & 2 & 6 & 3 & 7 & 4 & 1 \\
        \hline
  \end{tabular}
\end{table}

Fiecare linie a matricei poate fi parcursă in acest fel de maxim 2 ori (o dată
pentru găsirea coloanei de start și o dată pentru completarea liniei), deci
numărul total de celule vizitate este cel mult $2 \times N^2$. Complexitatea
pătratică este cea optimă, deoarece programul trebuie în orice caz să
tipărească la ieșire circa $N^2$ numere.

\begin{minted}{c}
#include <stdio.h>
#define NMax 200
unsigned char A[NMax+1][NMax+1];
int N,RealN;

void FindNextFree(int Line,int *K)
/* Cauta (circular) urmatorul spatiu liber pe linia Line */
{ do *K=*K%(N-1)+1;
  while (A[Line][*K]);
}

void MakeMatrix(void)
{ int i,j,k;

  for (i=1;i<=N;i++)
    for (j=1;j<=N;j++) A[i][j]=0;
  for (i=1;i<N;i++)  /* Pentru fiecare echipa */
    { if (i==1)      /* Alege coloana de start */
        j=1;
        else { j=0;
               do j++; while (A[i][j]!=i-1);
               FindNextFree(i,&j);
               FindNextFree(i,&j);
             }
      for (k=i+1;k<=N;k++) /* Completeaza circular linia */
        { A[i][j]=k;
          A[k][j]=i;
          if (k<N) FindNextFree(i,&j);
        }
    }
}

void WriteMatrix(void)
{ int i,j;
  FILE *F=fopen("output.txt","wt");

  for (i=1;i<=RealN;i++)
    { for (j=1;j<N;j++)
        fprintf(F,"%4d",A[i][j]>RealN ? 0 : A[i][j]);
      fprintf(F,"\n");
    }
  fclose(F);
}

void main(void)
{
  printf("N=");scanf("%d",&RealN);
  N=RealN+RealN%2; /* Se adauga 1 daca RealN e impar */
  MakeMatrix();
  WriteMatrix();
}
\end{minted}
