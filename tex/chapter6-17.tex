\section{Problema 17}

Iată un nou exemplu de problemă care admite două rezolvări: una evidentă, dar
neeficientă și una mai puțin evidentă, dar cu mult mai eficientă.

{\bf ENUNȚ}: Fie $V$ un vector. Arborele cartezian atașat vectorului $V$ este
un arbore binar care se obține astfel: Dacă vectorul $V$ este vid (are 0
elemente), atunci arborele cartezian atașat lui este de asemenea vid. Altfel,
se selectează elementul de valoare minimă din vector și se pune în rădăcina
arborelui, iar arborii cartezieni atașați fragmentelor de vector din stânga
(respectiv din dreapta) elementului minim se pun în subarborele stâng,
respectiv drept al rădăcinii. Iată, de exemplu, care este arborele cartezian
al următorului vector cu 8 elemente:

\tikzset{
  level/.style={sibling distance=10em/#1},
  circ/.style = {circle, draw, minimum size=2em},
  mat/.style = {
    matrix of nodes,
    ampersand replacement=\&,
    nodes = {
      draw,
      anchor=center,
      minimum size=2em,
    },
  },
}
\newcommand\cartesianStack[3] {
  \matrix[mat] (m) at (0,2) {
    #1 \& #2 \& #3 \& \ \\
  };

  \node at ([xshift=-1em]m.west) {$S$};
}

\centeredTikzFigure[]{
  \node[circ] (n1) at (-3,-2) {8};
  \node[circ] (n2) at (-2,-1) {2};
  \node[circ] (n3) at (-1,-2) {4};
  \node[circ] (n4) at (0,0) {1};
  \node[circ] (n5) at (1,-2) {5};
  \node[circ] (n6) at (2,-1) {3};
  \node[circ] (n7) at (3,-3) {6};
  \node[circ] (n8) at (4,-2) {4};
  \draw (n1) -- (n2) -- (n3);
  \draw (n2) -- (n4) -- (n6) -- (n5);
  \draw (n6) -- (n8) -- (n7);

  \matrix[mat, column sep=0.38em] (v) at (0.5, 2) {
    8 \& 2 \& 4 \& 1 \& 5 \& 3 \& 6 \& 4 \\
  };
  \node at ([xshift=-1em] v.west) {$V$};

  % no easy anchors; just use hardcoded coordinates
  \draw[densely dotted, semithick] (-3.52,-2.53) rectangle (-0.49,2.55);
  \draw[densely dotted, semithick] (0.49,-3.53) rectangle (4.51,2.55);
}

În figură au fost încadrate prin dreptunghiuri punctate porțiunile din stânga,
respectiv din dreapta elementului minim, împreună cu subarborii
atașați. Trebuie observat că arborele cartezian atașat unui vector poate să nu
fie unic, în cazul în care există mai multe elemente de valoare minimă. Vom
impune ca o condiție suplimentară ca elementul care va fi trecut în rădăcină
să fie cel mai din stânga dintre minime (cel cu indicele cel mai mic). Astfel,
arborele cartezian este unic.

Cerința problemei este ca, dându-se un vector, să i se construiască arborele
cartezian.

{\bf Intrarea}: Fișierul de intrare {\tt INPUT.TXT} conține pe prima linie
valoarea lui $N$ ($N \leq 10.000$), iar pe a doua $N$ numere naturale mai mici
ca 30.000, separate prin spații.

Ieșirea se va face în fișierul text {\tt OUTPUT.TXT} sub următoarea formă:

$T_1 \quad T_2 \quad T_3 \quad \dots \quad T_N$

unde $T_i$ este indicele în vector al elementului care este părintele lui
$V[i]$ în arborele cartezian. Dacă $V[i]$ este rădăcina arborelui, atunci $T_i
= 0$.

{\bf Exemplu}: Pentru exemplul dat mai sus, fișierul {\tt INPUT.TXT} este:

\begin{verbatim}
  8
  8 2 4 1 5 3 6 4
\end{verbatim}

După cum reiese din figură, tatăl elementului 8 este elementul 2, adică al
doilea în vector; tatăl elementului 2 este elementul 1, adică al 4-lea în
vector; tatăl elementului 5 este elementul 3, adică al 6-lea în vector
ș.a.m.d. Fișierul de ieșire este deci:

\begin{verbatim}
  2 4 2 0 6 4 8 6
\end{verbatim}

{\bf Complexitate cerută}: $O(N)$.

{\bf Timp de implementare}: 45 minute - 1h.

{\bf Timp de rulare}: 2 secunde.

{\bf REZOLVARE}: Nu întâmplător s-a impus o complexitate liniară pentru
rezolvarea acestei probleme. Altfel, ea ar fi trivială în $O(N^2)$, prin
următoarea metodă: scriem o procedură care parcurge vectorul și caută minimul,
apoi se reapelează pentru bucățile de vector aflate în stânga, respectiv în
dreapta minimului. Pentru a demonstra că această variantă de rezolvare are
complexitate pătratică, să ne imaginăm cum s-ar comporta ea pe cazul:

\begin{equation}
  V = (N \quad N-1 \quad N-2 \quad \dots \quad 2 \quad 1)
\end{equation}

La primul apel, procedura ar face $N$ comparații pentru a parcurge vectorul
(deoarece elementul minim este ultimul în vector) și s-ar reapela pentru
porțiunea din vector care cuprinde primele $N-1$ elemente. La al doilea apel,
ar face $N-1$ comparații și s-ar reapela pentru primele $N-2$ elemente etc. În
concluzie, numărul total de comparații făcute este

\begin{equation}
  N + (N-1) + (N-2) + \cdots + 1 = \frac{N(N + 1)}{2}
\end{equation}

de unde rezultă complexitatea. O problemă interesantă, pe care îi vom lăsa
plăcerea cititorului să o rezolve, este de a demonstra că această versiune
{\bf nu poate atinge o complexitate mai bună decât $O(N \log N)$} și de a
arăta care sunt cazurile cele mai favorabile pe care se obține această
complexitate.

A doua metodă este și ea destul de ușor de înțeles și de implementat. Ceea
este mai greu de acceptat este că ea are complexitate liniară, așa cum vom
încerca să explicăm la sfârșit. Iată mai întâi principiul de rezolvare: vom
porni cu un arbore cartezian vid și, la fiecare pas, vom adăuga câte un
element al vectorului $V$ la acest arbore, astfel încât structura obținută să
rămână un arbore cartezian. La al $k$-lea pas, vom adăuga elementul $V[k]$ în
arbore și vom restructura arborele în așa fel încât să obținem arborele
cartezian atașat primelor $k$ elemente din $V$. Trebuie să ne concentrăm
atenția asupra a două lucruri:

\begin{enumerate}

\item Cunoscând arborele cartezian atașat primelor $k$-1 vârfuri și elementul
  $V[k]$, cum se obține arborele cartezian atașat primelor $k$ vârfuri?

\item Cum reușim să actualizăm de $N$ ori arborele astfel încât timpul total
  consumat să fie liniar?

\end{enumerate}

Pentru a răspunde la prima întrebare, pe lângă vectorii $V$ și $T$, mai este
necesară o stivă $S$, în care vom stoca elemente ale vectorului $V$. Inițial,
stiva este vidă. Atunci când un nou element $X$ sosește, el va fi introdus în
stivă imediat după ultimul număr din stivă care are o valoare mai mică sau
egală cu $X$. Toate elementele care se aflau înainte în stivă pe poziții mai
mari sau egale cu poziția pe care a fost inserat $X$ vor fi eliminate din
stivă, iar elementul care se afla exact pe poziția lui $X$ va deveni fiul
stâng al lui $X$. $X$ însuși va deveni fiul drept al predecesorului său în
stivă. La fiecare moment, primul element din stivă este rădăcina arborelui
cartezian.

Pentru a înțelege mai bine principiul de funcționare a stivei, să analizăm mai
de aproape exemplul din enunț.

La început stiva este vidă. Primul element din V are valoarea 8, drept care îl
vom pune în stivă, iar arborele cartezian va avea un singur nod:

\centeredTikzFigure[]{
  \node[circ] {8};

  \cartesianStack{8}{\ }{\ }
}

Următorul element sosit este 2. Acesta este mai mic decât 8, deci trebuie
introdus înaintea lui în stivă. El va fi deci primul element din stivă și
rădăcina arborelui cartezian la acest moment. Concomitent, 8 va fi eliminat
din stivă și va deveni fiul stâng al lui 2:

\centeredTikzFigure[]{
  \node[circ] {2}
  child { node[circ] {8} }
  child[missing] {node[circ] {}};

  \cartesianStack{2}{\ }{\ }
}

Se observă că arborele obținut este tocmai arborele cartezian atașat secvenței
$(V[1], V[2])$. Următorul element este 4, care este mai mare decât 2, deci
trebuie adăugat în vârful stivei. Nici un element nu este eliminat din stivă,
iar 4 devine fiul drept al lui 2:

\centeredTikzFigure[]{
  \node[circ] {2}
  child { node[circ] {8} }
  child { node[circ] {4} };

  \cartesianStack{2}{4}{\ }
}

Următorul element sosit este 1, care este mai mic decât toate numerele din
stivă. Stiva se va goli, iar numărul 2 (cel peste care se va scrie 1) va
deveni fiul stâng al lui 1:

\centeredTikzFigure[]{
  \node[circ] {1}
  child { node[circ] {2}
    child { node[circ] {8} }
    child { node[circ] {4} }
  }
  child[missing] {node[circ] {}};

  \cartesianStack{1}{\ }{\ }
}

Deja arborele începe să semene cu forma sa finală. Urmează elementul 5, care
va fi adăugat în stivă și „atârnat” în dreapta lui 1:

\centeredTikzFigure[]{
  \node[circ] {1}
  child { node[circ] {2}
    child { node[circ] {8} }
    child { node[circ] {4} }
  }
  child { node[circ] {5} };

  \cartesianStack{1}{5}{\ }
}

Elementul 3 este mai mare ca 1, căruia îi va deveni fiu drept, dar mai mic ca
5, pe care îl va elimina din stivă:

\centeredTikzFigure[]{
  \node[circ] {1}
  child { node[circ] {2}
    child { node[circ] {8} }
    child { node[circ] {4} }
  }
  child { node[circ] {3}
    child { node[circ] {5} }
    child[missing] {node[circ] {}}
  };

  \cartesianStack{1}{3}{\ }
}

Următorul număr, 6, va fi adăugat la extremitatea dreaptă a arborelui și în
vârful stivei:

\centeredTikzFigure[]{
  \node[circ] {1}
  child { node[circ] {2}
    child { node[circ] {8} }
    child { node[circ] {4} }
  }
  child { node[circ] {3}
    child { node[circ] {5} }
    child { node[circ] {6} }
  };

  \cartesianStack{1}{3}{6}
}

În sfârșit, elementul 4 va fi fiul drept al lui 3 și îl va elimina din stivă
pe 6, care îi va deveni fiu stâng:

\centeredTikzFigure[]{
  \node[circ] {1}
  child { node[circ] {2}
    child { node[circ] {8} }
    child { node[circ] {4} }
  }
  child { node[circ] {3}
    child { node[circ] {5} }
    child { node[circ] {4}
      child { node[circ] {6} }
      child[missing] {node[circ] {}}
    }
  };

  \cartesianStack{1}{3}{4}
}

Se observă că arborele a ajuns tocmai la forma sa corectă. Trebuie acum să ne
ocupăm de un detaliu de implementare. Pentru a afla poziția pe care trebuie
inserat un element în stivă avem două metode:

\begin{enumerate}

\item Putem să căutăm în stivă de la dreapta la stânga (ar fi mai corect spus
  „de la vârf spre bază”) până dăm de un element mai mic decât cel de inserat;
  programul folosește această metodă și îi vom discuta în final eficiența.

\item Putem face o căutare binară în stivă, întrucât elementele din stivă au
  valori crescătoare de la bază spre vârf (lăsăm demonstrația acestei
  afirmații în seama cititorului). O căutare binară într-un vector de $k$
  elemente poate necesita, în cazul cel mai nefavorabil, $\log k$
  comparații. În cazul cel mai nefavorabil, când vectorul $V$ este sortat
  crescător, elementele vor fi introduse pe rând în stivă și nu vor mai fi
  scoase, deci la fiecare pas se vor face $\log k$ comparații, unde $k$ ia
  valori de la 1 la $N$. Complexitatea care rezultă este mai slabă decât cea
  cerută:

  \begin{equation}
    \begin{split}
      O(\log 1 + \log 2 + \cdots + \log N) = O(\sum_{k = 1}^{N} \log k) = \\
      = O(\log \prod_{k = 1}^{N} k) = O(\log N!) = O(N \cdot \log N)
    \end{split}
  \end{equation}

\end{enumerate}

Acesta este unul din puținele cazuri în care căutarea binară este mai
ineficientă decât cea secvențială.

Pentru ușurința programării, sursa C de mai jos reține în stiva $S$ nu
valorile elementelor, ci indicii lor în vectorul $V$ (deoarece aceștia sunt
ceruți pentru construcția vectorului $T$).

\begin{lstlisting}[language=C]
#include <stdio.h>
#define NMax 10001

int V[NMax], /* Vectorul */
    T[NMax], /* Vectorul de tati */
    S[NMax], /* Stiva */
    N;       /* Numarul de elemente */

void ReadData(void)
{ FILE *F=fopen("input.txt","rt");
  int i;

  fscanf(F,"%d\n",&N);
  for (i=1; i<=N;)
    fscanf(F, "%d", &V[i++]);
}

void BuildTree(void)
{ int i,k,LenS=0;

  S[0]=0; /* Pentru ca initial T[1] sa fie 0 */
  for (i=1; i<=N; i++)
    { /* Cauta pozitia pe care va fi inserat V[i] */
      k=LenS+1;
      while (V[S[k-1]]>V[i]) k--;
      /* Face corecturile in S si T */
      T[i]=S[k-1];
      if (k<=LenS) T[S[k]]=i;
      /* i este ultimul element din stiva, deci... */
      S[LenS=k]=i;
    }
}

void WriteSolution(void)
{ FILE *F=fopen("output.txt","wt");
  int i;
  for (i=1; i<=N;)
    fprintf(F,"%d ",T[i++]);
  fprintf(F,"\n");
}

void main(void)
{
  ReadData();
  BuildTree();
  WriteSolution();
}
\end{lstlisting}

Acum să analizăm și complexitatea acestui algoritm. În primul rând, ea nu
poate fi mai bună decât $O(N)$, pentru că aceasta este complexitatea
funcțiilor de intrare și ieșire. Procedura {\tt BuildTree} se compune dintr-un
ciclu {\tt for} în care se execută patru operații în timp constant și o
instrucțiune repetitivă {\tt while}. Numărul total de operații în timp
constant care se execută în procedură este prin urmare $O(N)$. Problema este:
care este numărul total maxim de evaluări ale condiției logice din bucla {\tt
  while}? Aparent, bucla while se execută de $O(N)$ ori, deci numărul total de
evaluări ar fi $O(N^2)$. Să aruncăm totuși o privire mai atentă.

Fiecare evaluare a condiției din bucla {\tt while} are ca efect decrementarea
lui $k$ și, implicit, eliminarea unui element deja existent în stivă. Pe de
altă parte, fiecare element este introdus în stivă o singură dată și deci nu
poate fi eliminat din stivă decât cel mult o dată. Așadar numărul maxim de
elemente ce pot fi eliminate din stivă pe parcursul executării procedurii {\tt
  BuildTree} este $N-1$, deci numărul total de evaluări ale condiției este
$O(N)$. De aici rezultă că programul are complexitate liniară.
