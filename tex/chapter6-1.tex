\chapter{Probleme de concurs}

\section{Problema 1}

{\bf ENUNȚ}: Se consideră următorul joc: Pe o tablă liniară cu $2N+1$ căsuțe
sunt dispuse $N$ bile albe (în primele $N$ căsuțe) și $N$ bile negre (în
ultimele $N$ căsuțe), căsuța din mijloc fiind liberă. Bilele albe se pot mișca
numai spre dreapta, iar cele negre numai spre stânga. Mutările posibile sunt:

\begin{enumerate}

\item O bilă albă se poate deplasa o căsuță spre dreapta, numai dacă aceasta
  este liberă;

\item O bilă albă poate sări peste bila aflată imediat în dreapta ei
  (indiferent de culoarea acesteia), așezându-se în căsuța de dincolo de ea,
  numai dacă aceasta este liberă;

\item O bilă neagră se poate deplasa o căsuță spre stânga, numai dacă aceasta
  este liberă;

\item O bilă neagră poate sări peste bila aflată imediat în stânga ei
  (indiferent de culoarea acesteia), așezându-se în căsuța de dincolo de ea,
  numai dacă aceasta este liberă.

\end{enumerate}

Trebuie schimbat locul bilelor albe cu cele negre. Se mai cere în plus ca
prima mutare să fie făcută cu o bilă albă.

{\bf Intrarea}: De la tastatură se citește numărul $N \leq 1.000$.

{\bf Ieșirea}: În fișierul {\tt OUTPUT.TXT} se vor tipări două linii terminate
cu <tt><Enter></tt>. Pe prima se va tipări numărul de mutări efectuate, iar pe
a doua o succesiune de cifre cuprinse între 1 și 4, nedespărțite prin spații,
corespunzătoare mutărilor ce trebuie făcute.

{\bf Exemple}:

\begin{itemize}

\item $N=1 \implies$ Ieșirea 141

\item $N=2 \implies$ Ieșirea 14322341

\end{itemize}

{\bf Complexitate cerută}: $O(N^2)$.

{\bf Timp de implementare}: 1h.

{\bf Timp de rulare}: 10 secunde pentru un test.

{\bf REZOLVARE}: La prima vedere, problema pare să se preteze la o rezolvare
în timp exponențial, prin metoda „Branch and Bound”. Un neajuns al enunțului
pare să fie faptul că nu se specifică dacă numărul de mutări efectuate trebuie
sau nu să fie minim. Pentru a ne lămuri, să privim în detaliu soluțiile pentru
$N=1$ și $N=2$:

\begin{itemize}

\item Pentru $N = 1$, toate mutările sunt forțate ((a) - se mută bila albă,
  (b) - se sare cu cea neagră peste ea, (c) - se mută din nou bila albă);
  trebuie remarcat că după mutările (a) și (b) se obțin două configurații
  simetrice una în raport cu cealaltă (oglindite).

\item Pentru $N = 2$, se poate începe sărind cu bila albă de la margine peste
  cealaltă, dar această mutare ar duce la blocarea jocului. Este deci
  obligatoriu să se înceapă prin a împinge bila albă centrală (a). Următoarea
  mutare este forțată ((b) - se sare cu bila neagră peste cea albă), apoi
  toate mutările sunt obligate (în sensul că dacă la orice pas se face altă
  mutare decât cea care conduce la soluție, jocul se blochează în câteva
  mutări): (c) - se împinge bila neagră, (d), (e) - se sare de două ori cu
  bilele albe, (f) - se împinge bila neagră, (g) - se sare cu bila neagră, (h)
  - se împinge bila albă. Trebuie din nou remarcat că după mutările (c) și (e)
  se obțin două configurații simetrice.

\end{itemize}

Așadar în ambele cazuri, soluția este unică. De fapt, există două soluții
asemănătoare, una dacă se începe cu o mutare a bilei albe și una dacă se
începe cu o mutare a bilei negre. Fiindcă enunțul impune ca prima mutare să se
facă cu o bilă albă, soluția este unică. Se mai observă și că, atât pentru
$N=1$ cât și pentru $N=2$ șirul de mutări este simetric. Pentru a indica
efectiv modul de determinare a soluției (care va sugera și ideea de scriere a
programului) și pentru a explica observațiile de mai sus, să generalizăm
observațiile făcute pentru un $N$ oarecare.

\tikzset{
  board/.style = {
    matrix of nodes,
    ampersand replacement=\&,
    nodes=cell,
    column 5/.style=noborder,
    column 15/.style=noborder,
  },
  cell/.style = {
    rectangle,
    draw,
    minimum height=1em,
    minimum width=1em,
    font=\huge,
  },
  noborder/.style={
    nodes={draw=none, font=\normalsize},
  },
  empty/.style={text=white},
}

\begin{itemize}

\item Configurația inițială este:

  \centeredTikzFigure[]{
    \matrix[board] (m) {
      $\circ$ \& $\circ$ \& $\circ$ \& $\circ$ \&
      $\cdots$ \&
      $\circ$ \& $\circ$ \& $\circ$ \& $\circ$ \&
      \node[empty] {$\circ$}; \&
      $\bullet$ \& $\bullet$ \& $\bullet$ \& $\bullet$ \&
      $\cdots$ \&
      $\bullet$ \& $\bullet$ \& $\bullet$ \& $\bullet$
      \\
    };
  }

\item Se împinge bila albă și se sare cu cea neagră peste ea (șirul de mutări
  14):

  \centeredTikzFigure[]{
    \matrix[board] (m) {
      $\circ$ \& $\circ$ \& $\circ$ \& $\circ$ \&
      $\cdots$ \&
      $\circ$ \& $\circ$ \& $\circ$ \& $\bullet$ \&
      $\circ$ \&
      \node[empty] {$\circ$}; \& $\bullet$ \& $\bullet$ \& $\bullet$ \&
      $\cdots$ \&
      $\bullet$ \& $\bullet$ \& $\bullet$ \& $\bullet$
      \\
    };
  }

\item Se împinge bila neagră și se sare de două ori cu cele albe peste ea
  (șirul de mutări 322):

  \centeredTikzFigure[]{
    \matrix[board] (m) {
      $\circ$ \& $\circ$ \& $\circ$ \& $\circ$ \&
      $\cdots$ \&
      $\circ$ \& $\circ$ \& \node[empty] {$\circ$}; \& $\bullet$ \&
      $\circ$ \&
      $\bullet$ \& $\circ$ \& $\bullet$ \& $\bullet$ \&
      $\cdots$ \&
      $\bullet$ \& $\bullet$ \& $\bullet$ \& $\bullet$
      \\
    };
  }

\item Se împinge bila albă și se sare de trei ori cu cele negre peste ea
  (șirul de mutări 1444):

  \centeredTikzFigure[]{
    \matrix[board] (m) {
      $\circ$ \& $\circ$ \& $\circ$ \& $\circ$ \&
      $\cdots$ \&
      $\circ$ \& $\bullet$ \& $\circ$ \& $\bullet$ \&
      $\circ$ \&
      $\bullet$ \& $\circ$ \& \node[empty] {$\circ$}; \& $\bullet$ \&
      $\cdots$ \&
      $\bullet$ \& $\bullet$ \& $\bullet$ \& $\bullet$
      \\
    };
  }

\item Se împinge bila neagră și se sare de patru ori cu cele albe peste ea
  (șirul de mutări 32222):

  \centeredTikzFigure[]{
    \matrix[board] (m) {
      $\circ$ \& $\circ$ \& $\circ$ \& $\circ$ \&
      $\cdots$ \&
      \node[empty] {$\circ$}; \& $\bullet$ \& $\circ$ \& $\bullet$ \&
      $\circ$ \&
      $\bullet$ \& $\circ$ \& $\bullet$ \& $\circ$ \&
      $\cdots$ \&
      $\bullet$ \& $\bullet$ \& $\bullet$ \& $\bullet$
      \\
    };
  }

  \hfil\hdashrule{6cm}{1pt}{1pt 4pt}\hfil

\item Se împinge bila albă (mutarea 1)

  \centeredTikzFigure[]{
    \matrix[board] (m) {
      \node[empty] {$\circ$}; \& $\circ$ \& $\bullet$ \& $\circ$ \&
      $\cdots$ \&
      $\circ$ \& $\bullet$ \& $\circ$ \& $\bullet$ \&
      $\circ$ \&
      $\bullet$ \& $\circ$ \& $\bullet$ \& $\circ$ \&
      $\cdots$ \&
      $\circ$ \& $\bullet$ \& $\circ$ \& $\bullet$
      \\
    };
  }

\item Se sare de $N$ ori cu cele negre peste ea (șirul de mutări 44..44):

  \centeredTikzFigure[]{
    \matrix[board] (m) {
      $\bullet$ \& $\circ$ \& $\bullet$ \& $\circ$ \&
      $\cdots$ \&
      $\circ$ \& $\bullet$ \& $\circ$ \& $\bullet$ \&
      $\circ$ \&
      $\bullet$ \& $\circ$ \& $\bullet$ \& $\circ$ \&
      $\cdots$ \&
      $\circ$ \& $\bullet$ \& $\circ$ \& \node[empty] {$\circ$};
      \\
    };
  }

\end{itemize}

Ultimele două configurații sunt simetrice. În acest moment șirul de mutări se
inversează:

\begin{itemize}

\item Se împinge bila albă (mutarea 1):

  \centeredTikzFigure[]{
    \matrix[board] (m) {
      $\bullet$ \& $\circ$ \& $\bullet$ \& $\circ$ \&
      $\cdots$ \&
      $\circ$ \& $\bullet$ \& $\circ$ \& $\bullet$ \&
      $\circ$ \&
      $\bullet$ \& $\circ$ \& $\bullet$ \& $\circ$ \&
      $\cdots$ \&
      $\circ$ \& $\bullet$ \& \node[empty] {$\circ$}; \& $\circ$
      \\
    };
  }

  \hfil\hdashrule{6cm}{1pt}{1pt 4pt}\hfil

\item Se sare de patru ori cu bilele albe și se împinge bila neagră (șirul
  de mutări 22223):

  \centeredTikzFigure[]{
    \matrix[board] (m) {
      $\bullet$ \& $\bullet$ \& $\bullet$ \& $\bullet$ \&
      $\cdots$ \&
      $\bullet$ \& \node[empty] {$\circ$}; \& $\circ$ \& $\bullet$ \&
      $\circ$ \&
      $\bullet$ \& $\circ$ \& $\bullet$ \& $\circ$ \&
      $\cdots$ \&
      $\circ$ \& $\circ$ \& $\circ$ \& $\circ$
      \\
    };
  }

\item Se sare de trei ori cu bilele negre și se împinge bila albă (șirul de
  mutări 4441):

  \centeredTikzFigure[]{
    \matrix[board] (m) {
      $\bullet$ \& $\bullet$ \& $\bullet$ \& $\bullet$ \&
      $\cdots$ \&
      $\bullet$ \& $\bullet$ \& $\circ$ \& $\bullet$ \&
      $\circ$ \&
      $\bullet$ \& \node[empty] {$\circ$}; \& $\circ$ \& $\circ$ \&
      $\cdots$ \&
      $\circ$ \& $\circ$ \& $\circ$ \& $\circ$
      \\
    };
  }

\item Se sare de două ori cu bilele albe și se împinge bila neagră (șirul de
  mutări 223):

  \centeredTikzFigure[]{
    \matrix[board] (m) {
      $\bullet$ \& $\bullet$ \& $\bullet$ \& $\bullet$ \&
      $\cdots$ \&
      $\bullet$ \& $\bullet$ \& $\bullet$ \& \node[empty] {$\circ$}; \&
      $\circ$ \&
      $\bullet$ \& $\circ$ \& $\circ$ \& $\circ$ \&
      $\cdots$ \&
      $\circ$ \& $\circ$ \& $\circ$ \& $\circ$
      \\
    };
  }

\item Se sare cu bila neagră și se împinge bila albă (șirul de mutări 41),
  obținându-se configurația finală:

  \centeredTikzFigure[]{
    \matrix[board] (m) {
      $\bullet$ \& $\bullet$ \& $\bullet$ \& $\bullet$ \&
      $\cdots$ \&
      $\bullet$ \& $\bullet$ \& $\bullet$ \& $\bullet$ \&
      \node[empty] {$\circ$}; \&
      $\circ$ \& $\circ$ \& $\circ$ \& $\circ$ \&
      $\cdots$ \&
      $\circ$ \& $\circ$ \& $\circ$ \& $\circ$
      \\
    };
  }

\end{itemize}

În concluzie, șirul de mutări este: o împingere - un salt - o împingere - două
salturi - o împingere - trei salturi - ... - o împingere - $N-1$ salturi - o
împingere - $N$ salturi - o împingere - $N-1$ salturi - ... - o împingere -
trei salturi - o împingere - două salturi - o împingere - un salt - o
împingere, culorile alternând la fiecare pas.

Pentru a calcula numărul de mutări, putem să le numărăm pe măsură ce le
efectuăm, dar deoarece se cere afișarea mai întâi a numărului de mutări și
după aceea a mutărilor în sine, trebuie fie să stocăm toate mutările în
memorie, fie să lucrăm cu fișiere temporare, ambele variante putând duce la
complicații nedorite. Din fericire, numărul de mutări se poate calcula cu
ușurință astfel: fiecare piesă albă trebuie mutată în medie cu $N$ pași către
dreapta și fiecare piesă neagră trebuie mutată cu $N$ pași către stânga. Deci
numărul total de pași este $2N(N+1).$ Din secvența generală de mutări expusă
mai sus se observă că nu se fac decât $2N$ împingeri de piese (mutări de un
singur pas), restul fiind salturi (mutări de câte doi pași). Deci numărul de
mutări este:

\begin{equation}
  2N + \frac{2N(N + 1) - 2N}{2} = 2N + N^2 = N(N + 2)
\end{equation}

De aici deducem că nu există un algoritm mai bun decât $O(N^2)$, deoarece
numărul de mutări este $O(N^2)$. Propunem ca temă cititorului să demonstreze
că nu există decât două succesiuni de mutări care rezolvă problema, din care
una începe cu mutarea unei piese albe, iar cealaltă este oglindirea ei și
începe cu mutarea unei piese negre, deci nu poate constitui o soluție
corectă. Demonstrația începe prin a arăta că sunt necesare cel puțin $2N$
împingeri de piese. Această demonstrație explică de ce nu se cere un număr
minim de mutări în enunț - cerința nu ar avea sens întrucât soluția este
oricum unică.  Acestea fiind zise, programul arată astfel:

\begin{minted}{c}
#include <stdio.h>
int N;
FILE *OutF;

void Jump(int Level, char A, char B)
{ int i;
  putc(A,OutF);
  for (i=1;i<=Level;i++) putc(B,OutF);
  if (Level<N)
    {
      Jump(Level+1,'1'+'2'-A,'4'+'3'-B);
      for (i=1;i<=Level;i++) putc(B,OutF);
    }
  putc(A,OutF);
}

void main(void)
{
  printf("N=");scanf("%d",&N);
  OutF=fopen("output.txt","wt");
  fprintf(OutF,"%ld\n",(long)N*(N+2));
  Jump(1,'1','4');
  fprintf(OutF,"\n");
  fclose(OutF);
}
\end{minted}
