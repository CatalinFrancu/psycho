\section{Problema 14}

Continuăm cu două probleme foarte asemănătoare. Atât de asemănătoare, încât
diferența dintre ele pare - la o privire superficială - neglijabilă. Totuși,
algoritmul de rezolvare se schimbă fundamental.

{\bf ENUNȚ}: $N$ grămezi de mere trebuie împărțite la $N$ copii. Deoarece
copiii sunt buni prieteni, trebuie ca împărțirea să se facă în mod echitabil,
fiecare primind același număr de mere. Spiritul de dreptate al copiilor este
atât de puternic, încât ei preferă ca unele mere să nu fie date nici unui
copil, decât ca unii să primească mai multe mere ca alții. O condiție
suplimentară este ca fie toate merele dintr-o grămadă să fie împărțite
copiilor, fie grămada să nu mai fie împărțită deloc. Desigur, interesul este
ca fiecare copil să primească un număr cât mai mare de mere.

Să se selecteze un număr de grămezi din cele $N$ astfel încât numărul total de
mere să se dividă cu $N$, iar suma selectată să fie maximă. Dacă există mai
multe soluții, se cere una singură.

{\bf Intrarea}: Fișierul de intrare {\tt INPUT.TXT} conține pe prima linie
numărul $N$ de grămezi ($1 \leq N \leq 100$). Pe a doua linie se dau
cantitățile de mere din cele $N$ grămezi ($N$ numere naturale pozitive, toate
mai mici ca 200, separate prin spații).

{\bf Ieșirea}: În fișierul {\tt OUTPUT.TXT} se va scrie pe prima linie numărul
maxim de mere găsit. Pe a doua linie se vor tipări indicii grămezilor
selectate, în ordine crescătoare.

{\bf Exemplu}:

\texttt{
  \begin{tabular}{|l|l|}
    \hline
        {\bf INPUT.TXT} & {\bf OUTPUT.TXT} \\ \hline
        4 & 12 \\
        3 2 5 7 & 1 2 4 \\
        \hline
  \end{tabular}
}

{\bf Timp de implementare}: 45 minute - maxim o oră.

{\bf Timp de rulare}: 1 secundă.

{\bf Complexitate cerută}: $O(N^2)$.

{\bf REZOLVARE}: O primă modalitate, de altfel foarte comodă, este să
verificăm toate posibilitățile de a selecta grămezi. Aceasta presupune să
generăm toate submulțimile mulțimii de grămezi, iar pentru fiecare submulțime
să calculăm suma merelor. În felul acesta putem afla submulțimea pentru care
suma merelor se divide cu $N$ și este maximă. Din nefericire, această soluție,
banal de implementat, are o complexitate exponențială, mai precis $O(N \times
2^N)$, deoarece există $2^N$ submulțimi ale mulțimii grămezilor și pentru
fiecare submulțime putem calcula suma merelor în timp liniar. Prin urmare,
suntem departe de complexitatea cerută în enunț.

Punctul de plecare pentru rezolvarea corectă a problemei este din nou
principiul de optimalitate al programării dinamice. Să notăm cu $M[1], M[2],
\dots, M[N]$ cantitățile de mere din fiecare grămadă. Să considerăm că
submulțimea optimă conține în total $S$ mere, iar grămada cu numărul $N$ face
parte din ea. Fie $K$ restul împărțirii lui $M[N]$ la $N$, deci

\begin{equation}
  M[N] \equiv K \pmod{N}
\end{equation}

Atunci suma $S - M[N]$ dă restul $(N-K) \bmod N$ la împărțirea prin $N$, de
unde deducem că

\begin{equation}
  S - M[N] \equiv N - K \pmod{N}
\end{equation}

De asemenea, putem afirma că $S - M[N]$ este cea mai mare dintre toate sumele
care se pot obține folosind grămezile $1, 2, \dots, N-1$ și care dau același
rest la împărțirea prin $N$. Demonstrația nu este grea: dacă ar exista o sumă
mai mare decât $S - M[N]$ congruentă cu $N - K$ modulo $N$, am putea folosi
această sumă și grămada $M[N]$ pentru a obține o sumă $S'>S$ astfel încât

\begin{equation}
  S' \equiv 0 \pmod{N}
\end{equation}

Această observație ne sugerează și metoda de rezolvare a problemei. Pentru a
afla care este submulțimea de sumă maximă, avem două variante:

\begin{itemize}

\item Grămada $N$ face parte din această submulțime, caz în care trebuie să
  descoperim cea mai mare sumă care se poate obține adunând grămezi dintre
  primele $N-1$ și care dă restul $N - K$ la împărțirea prin $N$;

\item Grămada $N$ nu face parte din această submulțime, caz în care trebuie să
  descoperim cea mai mare sumă care se poate obține adunând grămezi dintre
  primele $N-1$ și care se împarte exact la $N$.

\end{itemize}

Pentru că nu putem ști de la început dacă grămada a $N$-a face sau nu parte
din submulțimea maximală, trebuie să avem răspunsul pregătit pentru ambele
situații. Mai mult, pentru a putea afla care sunt sumele maxime formate cu
primele $N-1$ grămezi care dau diferite resturi (în cazul nostru 0 sau $N -
K$) la împărțirea prin $N$, trebuie să reluăm exact aceeași problemă: grămada
cu numărul $N-1$ poate face sau nu parte din submulțime. În concluzie, putem
formaliza problema astfel: avem nevoie să putem răspunde la toate întrebările
de forma „Care este suma maximă care se poate forma cu grămezi din primele $P$
astfel încât restul la împărțirea prin $N$ să fie $Q$?”. Vom nota răspunsul la
această întrebare cu $R[P,Q]$ (unde $1 \leq P \leq N$ și $0 \leq Q <
N$). Răspunsurile tuturor întrebărilor se pot deci dispune într-o matrice $R$,
iar scopul nostru este să-l aflăm pe $R[N,0]$. Am observat că pentru a-l putea
afla pe $R[P,Q]$ avem nevoie de două valori din linia $P$-1 a matricei (după
cum grămada $P$ face sau nu parte din submulțimea optimă). Pentru a afla toate
elementele liniei $P$ a matricei, este deci foarte probabil să avem nevoie de
întreaga linie $P-1$.

Astfel, problema se reduce la a compune o linie a matricei din cea
precedentă. Dacă presupunem că grămada $P$ nu intră în componența submulțimii
optime, atunci linia $P$ este identică cu linia $P-1$:

\begin{equation}
  R[P,Q] = R[P - 1, Q], \quad \forall 0 \leq Q < N
\end{equation}

Dacă presupunem că grămada $P$ face parte din submulțime, atunci avem
egalitatea:

\begin{equation}
  R[P,Q] = R[P - 1, (Q - M[P]) \bmod N] + M[P], \quad \forall 0 \leq Q < N
\end{equation}

Alegând dintre aceste variante pe cea care ne convine mai mult, obținem:

\begin{equation}
  R[P,Q] =  \max
  \begin{Bmatrix}
    R[P - 1, Q] \\
    R[P - 1, (Q - M[P]) \bmod N] + M[P]
  \end{Bmatrix}
  \quad \forall 0 \leq Q < N
\end{equation}

În felul acesta se completează fără nici un fel de probleme matricea $R$. După
cum am spus, $R[N,0]$ indică suma maximă divizibilă cu $N$. Mai rămâne de
văzut cum se face reconstituirea soluției. De fapt, nu avem decât să parcurgem
aceleași etape ale raționamentului, dar în sens invers. Respectiv: dacă
$R[N,0] = R[N-1,0]$, atunci deducem că grămada a $N$-a nu a fost folosită
pentru a se obține suma maximă și avem nevoie să obținem suma maximă
divizibilă cu $N$ din primele $N-1$ grămezi (adică $R[N-1,0]$). Dacă $R[N,0]
\neq R[N-1,0]$, atunci grămada a $N$-a a fost folosită și avem nevoie să
obținem suma maximă de rest $N-M[N]$ modulo $P$ folosind primele $N-1$
grămezi. Pe cazul general, pentru a obține suma maximă de rest $Q$ folosind
primele $P$ grămezi, avem două posibilități:

\begin{itemize}

\item Dacă $R[P,Q] = R[P-1,Q]$, atunci grămada $P$ nu este folosită și trebuie
  să obținem același rest $Q$ folosind doar primele $P-1$ grămezi;

\item Dacă $R[P,Q] \neq R[P-1,Q]$, atunci grămada $P$ este folosită și trebuie
  să obținem restul $Q- M[P]$ modulo $N$ folosind primele $P-1$ grămezi.

\end{itemize}

Menționăm că programul este puțin diferit de ceea ce s-a explicat până acum,
în sensul că prima linie din matricea $R$ corespunde ultimei grămezi de mere,
a doua linie corespunde penultimei grămezi de mere ș.a.m.d. Cu alte cuvinte,
grămezile de mere sunt procesate în ordine inversă. Am făcut acest lucru
pentru a ușura procedura de aflare a soluției; se observă că prima oară se
decide dacă ultima grămadă face parte din submulțime, apoi penultima etc. Deci
și la găsirea soluției se generează grămezile în ordine inversă. Prin această
dublă inversiune, indicii grămezilor de mere selectate vor fi listați în
ordine crescătoare. Dacă am fi prelucrat grămezile de mere în ordinea lor din
fișier, s-ar fi impus scrierea unei proceduri recursive pentru afișarea
soluției.

Să facem și calculul complexității acestui program. Citirea datelor se face în
$O(N)$, la fel și reconstituirea soluției. Pentru a completa o linie din
matrice, avem nevoie să parcurgem linia precedentă, adică $O(N)$. Pentru
compunerea întregii matrice, timpul necesar este pătratic.

Mai trebuie făcută o singură observație referitoare la modul de inițializare a
matricei. Vom adăuga în matricea $R$ linia cu numărul 0, care va conține
sumele maxime ce se pot obține fără a folosi nici o grămadă. Desigur, se poate
obține numai suma 0, care dă restul 0 la împărțirea prin $N$, iar alte sume nu
se pot obține. Vom pune deci $R[0,0]=0$ și $R[0,i]=\mathbf{Impossible}$ pentru
orice $1 \leq i < N$, unde $\mathbf{Impossible}$ este o constantă specială
(preferabil negativă, pentru a nu se confunda cu valorile obișnuite din
matrice).

\begin{minted}{c}
#include <stdio.h>
#define NMax 101
#define Impossible -30000

int M[NMax], R[NMax][NMax], N;
void ReadData(void)
{ FILE *F = fopen("input.txt", "rt");
  int i;

  fscanf(F, "%d\n", &N);
  for (i=1; i<=N; fscanf(F, "%d", &M[i++]));
  fclose(F);
}

void Share(void)
{ int i,j;

  R[0][0]=0;
  for (i=1; i<N; R[0][i++]=Impossible);

  for (i=1; i<=N; i++)
    {
      for (j=0; j<N; j++)
        R[i][j] = R[i-1][j];
      for (j=0; j<N; j++)
        if (R[i-1][j] != Impossible
           && R[i-1][j] + M[N+1-i] > 
              R[i][ (R[i-1][j]+M[N+1-i]) % N ])
         R[i][ (R[i-1][j]+M[N+1-i]) % N ] =
         R[i-1][j] + M[N+1-i];
    }
}

void WriteSolution(void)
{ FILE *F = fopen("output.txt", "wt");
  int i,j;

  fprintf(F, "%d\n", R[N][0]);
  j=0;
  for (i=N; i; i--)
    if (R[i][j] != R[i-1][j])
      {
        fprintf(F, "%d ", N+1-i);
        j = (j + N - M[N+1-i]%N) % N;
      }
  fprintf(F, "\n");
  fclose(F);
}

void main(void)
{
  ReadData();
  Share();
  WriteSolution();
}
\end{minted}
