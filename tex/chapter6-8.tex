\section{Problema 8}

Problema {\bf codului lui Prüfer} pentru arborii generali, mai exact cea a
decodificării acestui cod, este genul de problemă care nu este excesiv de
grea, dar care necesită un artificiu fără de care nu se poate ajunge la
complexitatea optimă.

{\bf ENUNȚ}: Un arbore general neorientat conex cu $N+2$ vârfuri se poate
codifica eficient printr-un vector cu $N$ numere, astfel:

\begin{itemize}

\item Numerotăm nodurile de la 1 la $N+2$ într-o ordine oarecare;

\item Eliminăm cea mai mică frunză (nod de grad 1) și adăugăm în vector
  numărul nodului de care ea aparținea;

\item Reluăm procedeul pentru arborele rămas: tăiem cea mai mică frunză și
  adăugăm în vector numărul nodului de care ea aparținea;

\item Repetăm procedeul până mai rămân doar două noduri.

\end{itemize}

Vectorul rezultat se numește codificare Prüfer a arborelui dat. Iată un
exemplu de construcție a codului Prüfer atașat arborelui din figura de mai
jos:

\centeredTikzFigure[
  mat/.style = {
    matrix of nodes,
    ampersand replacement=\&,
    nodes=cell,
    column sep=2em,
    row sep=2em,
  },
  cell/.style={
    draw,
    circle,
    minimum height=2.5em,
  }
]{
  \matrix[mat] (m) {
    3 \& 2 \& 6 \\
    1 \& 5 \&   \\
    7 \& 4 \&   \\
  };

  \draw (m-1-1) edge (m-2-1);
  \draw (m-2-1) edge (m-3-1);
  \draw (m-2-1) edge (m-2-2);
  \draw (m-2-2) edge (m-1-2);
  \draw (m-2-2) edge (m-3-2);
  \draw (m-1-2) edge (m-1-3);
}

Codul lui Prüfer se obține astfel:

\begin{itemize}

\item Îl tăiem pe 3 și scriem 1;

\item Îl tăiem pe 4 și scriem 5;

\item Îl tăiem pe 6 și scriem 2;

\item Îl tăiem pe 2 și scriem 5;

\item Îl tăiem pe 5 și scriem 1;

\end{itemize}

Deci codificarea este (1, 5, 2, 5, 1) (și mai rămâne muchia 1-7, lucru care
este evident, deoarece 1 și 7 sunt singurele noduri care nu au fost tăiate).

Așadar fiecărui arbore oarecare cu $N+2$ noduri i se poate atașa un vector cu
$N$ componente numere naturale cuprinse între 1 și $N+2$. Se poate demonstra
că funcția definită între cele două mulțimi (mulțimea arborilor și mulțimea
vectorilor) este bijectivă. De aici rezultă două lucruri:

\begin{enumerate}

\item Există $(N+2)^N$ arbori generali cu $N+2$ noduri.

\item Codificarea Prüfer admite și decodificare (deoarece funcția de
  codificare este bijectivă). Tocmai aceasta este problema de rezolvat. Se dă
  un vector de $N$ numere întregi, fiecare cuprins între 1 și $N+2$. Se cere
  să se tipărească cele $N+1$ muchii ale arborelui decodificat.

\end{enumerate}

{\bf Intrarea} se face din fișierul text {\tt INPUT.TXT} care conține două
linii. Pe prima linie se dă $N$ ($1 \leq N \leq 10000$), pe a doua cele $N$
numere separate prin spații.

{\bf Ieșirea} se va face în fișierul text {\tt OUTPUT.TXT}. Acesta va conține
muchiile arborelui, câte una pe linie, o muchie fiind indicată prin vârfurile
adiacente separate printr-un spațiu.

{\bf Exemplu}:

\texttt{
  \begin{tabular}{|l|l|}
    \hline
        {\bf INPUT.TXT} & {\bf OUTPUT.TXT} \\ \hline
        5         & 5 2 \\
        1 5 2 5 1 & 1 3 \\
        & 4 5 \\
        & 7 1 \\
        & 6 2 \\
        & 1 5 \\
    \hline
  \end{tabular}
}

{\bf Timp de implementare}: 45 minute.

{\bf Timp de rulare}: 2-3 secunde.

{\bf Complexitate cerută}: $O(N)$.

{\bf REZOLVARE}: Să pornim de la exemplul particular prezentat mai sus, urmând
ca apoi să generalizăm algoritmul.

Primim la intrare $N=5$ (deducem că arborele are 7 noduri) și codificarea (1,
5, 2, 5, 1). Primul element din vector este 1. Știm deci că, la primul pas, a
fost eliminată o frunză al cărei părinte este nodul 1. Întrebarea este: ce
număr purta respectiva frunză? În nici un caz 1, deoarece numărul 1 îl avea
tatăl ei. Nici 2, nici 5 nu ar putea fi, deoarece aceste numere apar mai
târziu în codificare, deci „mai este nevoie” de ele și nu pot fi încă
eliminate. Rămân 3, 4, 6 și 7. Dintre acestea noi știm că a fost tăiată cea
mai mică {\bf frunză}. Este însă ușor de văzut că toate nodurile enumerate
sunt frunze în arborele inițial (deoarece nu mai apar în vector, adică nici un
alt nod nu mai este legat de ele), deci îl vom alege pe cel mai mic cu
putință, adică pe 3. Rezultă că prima muchie tăiată a fost (1,3).

Pe poziția a doua în codificare apare numărul 5. Ce număr ar putea avea frunza
tăiată? 1 și 2 mai apar ulterior în codificare deci nu pot fi eliminate încă,
5 este chiar tatăl frunzei necunoscute, iar 3 a fost deja eliminat. Dintre 4,
6 și 7, frunza cu numărul cel mai mic este 4, deci următoarea muchie tăiată
este (4,5).

În continuare apare un 2. Cel mai mic nod care nu mai apare în vector și nici
nu a fost deja eliminat este 6, deci muchia este (6,2). Se observă că mai
departe nu mai apare nici un 2 în codificare, de unde deducem că după tăierea
nodului 6, nodul 2 a devenit frunză și va putea fi la rândul său eliminat. Cu
un raționament analog, următoarele muchii eliminate sunt (2,5) și (5,1), iar
singurele noduri rămase sunt 1 și 7. Arborele a fost reconstituit corect.

Deci procedeul general este: pentru fiecare număr dintre cele $N$ din vector,
nodul care aparținea de el poartă cel mai mic număr care nu intervine ulterior
în codificare și nu a fost tăiat deja. Astfel se generează $N$ muchii. A
$N+1$-a muchie are drept capete ultimele noduri rămase netăiate.

Acesta este algoritmul. Trebuie să ne ocupăm acum de partea de implementare.

Pentru a testa la orice moment dacă un nod mai apare sau nu în vector, este
bine să se creeze la citirea datelor un vector care să rețină numărul de
apariții în codificare al fiecărui nod. Pentru exemplul dat, vectorul de
apariții va fi $Apar=(2,1,0,0,2,0,0)$, semnificând că 1 și 5 apar de câte două
ori, 2 apare o singură dată, iar 3, 4, 6 și 7 nu apar deloc. La fiecare pas,
când se „restaurează” o muchie, se decrementează poziția corespunzătoare
tatălui în vectorul de apariții. Un număr nu mai apare ulterior în codificare
dacă pe poziția sa din vectorul de apariții se află un 0.

Astfel, o primă versiune de program ar putea fi:

\vspace{\algskip}
\begin{algorithmic}[1]
  \REQUIRE $N$, $V[1..N]$
  \STATE construiește vectorul $Apar[1..N+2]$
  \FOR{$i = 1$ la $N$}
  \STATE caută primul $j$ pentru care $Apar[j]=0$ și $j$ nu a fost tăiat
  \PRINT muchia $(j,V[i])$
  \STATE $Apar[V[i]] \leftarrow Apar[V[i]]-1$
  \STATE marchează nodul $j$ ca fiind tăiat
  \ENDFOR
\end{algorithmic}

După cum se observă, căutarea celui mai mic nod $j$ se face în timp liniar,
ceea ce înseamnă că algoritmul complet va necesita un timp pătratic. Pentru
a-l reduce la o complexitate liniară, începem prin a observa că la fiecare pas
alegem frunza cu numărul cel mai mic. Aceasta înseamnă că, în general, numărul
frunzei alese spre a fi tăiată va crește la fiecare pas. Astfel, prima oară am
tăiat nodul 4, apoi nodul 7. Singurul caz când va fi tăiată o frunză mai mică
decât cea dinaintea ei este atunci când unui nod cu număr mic i se taie toate
frunzele și devine el însuși o frunză, putănd fi tăiat. În exemplul nostru,
nodul 2 nu putea fi eliminat de la început, deși avea un număr mic, deoarece
mai avea atașată frunza 6. După eliminarea muchiei (6,2), nodul 2 a devenit
frunză și a fost eliminat imediat.

Putem deci păstra într-o variabilă (care în program se numește $Next$)
următoarea frunză care care trebuie eliminată. Dacă prin decrementarea
numărului de apariții la pasul curent nu s-a creat nici un zero în vectorul
$Apar$, sau s-a creat un zero, dar pe o poziție $K>Next$, atunci totul este
bine și la pasul următor se va elimina nodul $Next$. Dacă s-a creat un zero pe
o poziție $K$ mai mică decât $Next$, rezultă că în arbore există acum două
frunze, $K$ și $Next$, $K$ fiind mai mică, deci prioritară. Atunci la pasul
următor se va elimina frunza de pe poziția $K$, urmând ca peste doi pași să se
revină la frunza $Next$. Dacă prin eliminarea acestei frunze $K$ s-a creat un
alt zero în vectorul de apariții, tot pe o poziție $K'$ mai mică decât Next,
se va trece mai întâi la acea poziție, urmând ca după aceea să se revină la
poziția $Next$ etc.

La prima vedere, pare necesară menținerea unei stive în care să depunem
numerele $Next, K, K'$... și să le scoatem din stivă pe măsură ce nodurile
respective sunt eliminate. Acest lucru ar presupune în continuare un algoritm
pătratic. Totuși nu este așa deoarece, odată ce am tăiat un nod și l-am marcat
ca atare, putem fi siguri că nu ne vom mai intâlni cu el până la sfârșitul
decodificării, deci nu mai este necesară stocarea lui. Există o pseudo-stivă
care are înălțimea 2: frunza asupra căreia se operează curent și nodul care
urmează, $Next$.

În acest fel, numărul total de incrementări al variabilei $Next$ nu depășește
$N$ (iar ea nu este niciodată decrementată), deci programul este rezolvat în
timp liniar. Facem observația că algoritmul $O(N)$ este optim; nu poate exista
unul mai bun deoarece există $O(N)$ muchii care trebuie tipărite.

\begin{minted}{c}
#include <stdio.h>
#define NMax 10002
typedef int Vector[NMax];

Vector V,Apar;
int N;

void ReadData(void)
{ int i;
  FILE *InF=fopen("input.txt","rt");

  fscanf(InF,"%d",&N);
  for (i=1;i<=N+2;Apar[i++]=0);
  for (i=1;i<=N;i++)
    { fscanf(InF,"%d",&V[i]);
      Apar[V[i]]++;
    }
  fclose(InF);
}

void Decode(void)
{ int Current=0,Next,i;
  FILE *OutF=fopen("output.txt","wt");

  do; while (Apar[++Current]);    /* Se cauta prima frunza */
  Next=Current;
  for (i=1;i<=N;i++)
    { fprintf(OutF,"%d %d\n",Current,V[i]);
      if (Current==Next) do; while (Apar[++Next]);
      /* Daca am ajuns la ultimul 0, mai caut unul */
      Apar[V[i]]--;
      Current=(V[i]<Next) && (Apar[V[i]]==0) ? V[i] : Next;
      /* Daca exista o frunza mai mica decat Next, */
      /* ea este prioritara, altfel revin la Next */
    }
  fprintf(OutF,"%d %d\n",Current,Next);
  fclose(OutF);
}

void main(void)
{
  ReadData();
  Decode();
}
\end{minted}
