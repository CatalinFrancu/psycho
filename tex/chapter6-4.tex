\section{Problema 4}

Continuăm cu o problemă care a fost de asemenea dată spre rezolvare la a
VIII-a Olimpiadă Internațională de Informatică, Veszprem 1996. Problema în
sine nu a fost foarte grea și mulți elevi au luat punctaj maxim. Totuși,
enunțul permite unele modificări interesante care practic schimbă cu totul
problema.

{\bf ENUNȚ}: Să considerăm următorul joc de două persoane. Tabla de joc constă
într-o secvență de întregi pozitivi. Cei doi jucători mută pe rând. Mutarea
fiecărui jucător constă în alegerea unui număr de la unul din cele două capete
ale secvenței. Numărul ales este șters de pe tablă. Jocul se termină când
toate numerele au fost selectate. Primul jucător câștigă dacă suma numerelor
alese de el este mai mare sau egală cu cea a numerelor alese de cel de-al
doilea jucător. În caz contrar, câștigă al doilea jucător.

Dacă tabla conține inițial un număr par de elemente, atunci primul jucător are
o strategie de câștig. Trebuie să scrieți un program care implementează
strategia cu care primul jucător câștigă jocul. Răspunsurile celui de-al
doilea jucător sunt date de un program rezident. Cei doi jucători comunică
prin trei proceduri ale modulului {\tt Play} care v-a fost pus la
dispoziție. Procedurile sunt {\tt StartGame}, {\tt MyMove} și {\tt
  YourMove}. Primul jucător începe jocul apelând procedura fără parametri {\tt
  StartGame}. Dacă alege numărul de la capătul din stânga, el va apela
procedura {\tt MyMove('L')}. Analog, apelul de procedură {\tt MyMove('R')}
trimite un mesaj celui de-al doilea jucător prin care îl informează că a ales
numărul de la capătul din dreapta. Cel de-al doilea jucător, deci computerul,
mută imediat, iar primul jucător poate afla mutarea acestuia executând
procedura {\tt YourMove(C)}, unde {\tt C} este o variabilă de tip {\tt Char}
(în C/C++ apelul este {\tt YourMove(\&C)}). Valoarea lui {\tt C} este {\tt
  'L'} sau {\tt 'R'}, după cum numărul ales este de la capătul din stânga sau
din dreapta.

{\bf Intrarea}: Prima linie din fișierul {\tt INPUT.TXT} conține dimensiunea
inițială $N$ a tablei. $N$ este par și $2 \leq N \leq 100$. Următoarele $N$
linii conțin fiecare câte un număr, reprezentând conținutul tablei de la
stânga la dreapta. Fiecare număr este cel mult 200.

{\bf Ieșirea}: Când jocul se termină, programul trebuie să scrie rezultatul
final în fișierul text {\tt OUTPUT.TXT}. Fișierul conține două numere pe prima
linie, reprezentând suma numerelor alese de primul, respectiv de cel de-al
doilea jucător. Programul trebuie să joace un joc corect și ieșirea trebuie să
corespundă jocului jucat.

{\bf Exemplu}:

\texttt{
  \begin{tabular}{| l | l |}
    \hline
        {\bf INPUT.TXT} & {\bf OUTPUT.TXT} \\ \hline
        6 & 15 14 \\
        4 & \\
        7 & \\
        2 & \\
        9 & \\
        5 & \\
        2 & \\
    \hline
  \end{tabular}
}

{\bf Timp limită de execuție}: 20 secunde pentru un test.

Acesta a fost enunțul original, la care va trebui să facem câteva modificări,
în parte deoarece nu putem folosi modulul {\tt Play}, în parte pentru a face
problema mai restrictivă:

\begin{itemize}

\item Mutările vor fi anunțate pe ecran prin tipărirea unui caracter {\tt 'L'}
  sau {\tt 'R'};

\item Mutările celui de-al doilea jucător vor fi comunicate de un partener
  uman, prin introducerea de la tastatură a unui caracter {\tt 'L'} sau {\tt
    'R'};

\item Rezultatul final se va tipări pe ecran, sub aceeași formă (pereche de
  numere).

\item Timpul de gândire pentru fiecare mutare trebuie să fie cât mai mic
  (practic răspunsul să fie instantaneu);

\item {\bf Complexitatea totală} a calculelor efectuate să fie $O(N)$.

\item {\bf Timpul de implementare} a fost cam de 1h 40 min. Propunem reducerea
  lui la 30 minute.

\end{itemize}

{\bf REZOLVARE}: Este ușor de demonstrat că o rezolvare „greedy” a problemei
(la fiecare mutare jucătorul 1 alege numărul mai mare) nu atrage întotdeauna
câștigul. Iată un contraexemplu:

\centeredTikzFigure[
  mat/.style = {
    matrix of nodes,
    ampersand replacement=\&,
    column sep = 0.5em,
    nodes=cell,
  },
  cell/.style = {rectangle, draw, minimum width=1.5em, minimum height=1.5em},
]{
  \matrix[mat] {
    7 \& 10 \& 1 \& 2 \\
  };
}

La prima mutare, jucătorul 1 poate să aleagă fie numărul 7, fie numărul
2. Dacă se va „lăcomi” la 7, jucătorul 2 va lua numărul 10 și inevitabil va
câștiga. Soluția pentru primul jucător este să ia numărul 2, apoi, indiferent
de ce va juca partenerul său, va putea lua numărul 10 și va câștiga.

Iată o soluție izbitor de simplă de complexitate $O(N)$: La citirea datelor se
face suma elementelor aflate pe poziții pare și a celor aflate pe poziții
impare. Să presupunem că suma elementelor de ordin par este mai mare sau egală
cu cea a elementelor de ordin impar (cazul invers se tratează analog). Atunci,
dacă primul jucător ar putea să aleagă toate elementele de ordin par (care
sunt într-adevăr $N/2$, adică atâtea câte are el dreptul să aleagă), ar
câștiga jocul. Jucătorul 1 poate începe jocul prin a lua primul sau ultimul
element din secvență, deci îl va alege pe ultimul, care are număr de ordine
par. Al doilea jucător are de ales între primul și al $N-1$-lea element,
ambele având număr de ordine impar. Indiferent ce variantă o va adopta, primul
jucător va avea din nou acces la un element de pe o poziție pară. Dacă
jucătorul 2 alege elementul din stânga (primul), atunci jucătorul 1 va putea
lua elementul de după el (al doilea), iar dacă jucătorul 2 alege elementul din
dreapta (al $N-1$-lea), atunci jucătorul 1 va putea lua elementul dinaintea el
(al $N-2$-lea). Deci primul jucător nu are altceva de făcut decât să repete
mutările făcute de cel de-al doilea. Să privim de exemplu desfășurarea jocului
pe tabla dată în enunț:

\centeredTikzFigure[
  mat/.style = {
    matrix of nodes,
    ampersand replacement=\&,
    row sep=0.5em,
    column sep=0.5em,
    nodes=cell,
  },
  cell/.style = {draw, minimum width=1.5em, minimum height=1.5em},
  row 1/.style = {draw=white, nodes={minimum height=2em}},
  noborder/.style={column #1/.style={nodes={draw=none}}},
  noborder/.list={1, 8},
]{
  \matrix[mat] (m) {
    Jucătorul 1 \& 1 \& 2 \& 3 \& 4 \& 5 \& 6 \& Jucătorul 2 \\
    0           \& 4 \& 7 \& 2 \& 9 \& 5 \& 2 \& 0  \\
    2           \& 4 \& 7 \& 2 \& 9 \& 5 \&   \& 0  \\
    2           \& 4 \& 7 \& 2 \& 9 \&   \&   \& 5  \\
    11          \& 4 \& 7 \& 2 \&   \&   \&   \& 5  \\
    11          \&   \& 7 \& 2 \&   \&   \&   \& 9  \\
    18          \&   \&   \& 2 \&   \&   \&   \& 9  \\
    18          \&   \&   \&   \&   \&   \&   \& 11 \\
  };

  \node at ([yshift=-1em]m.south) {$7 + 9 + 2 > 4 + 2 + 5$};
}

Programul în sine nici nu are nevoie să mai rețină vectorul de numere în
memorie, din moment ce primul jucător nu are altceva de făcut decât să imite
mutările celui de-al doilea. Un calcul al sumelor la citirea datelor este
suficient. Complexitatea $O(N)$ este optimă, deoarece vectorul trebuie parcurs
cel puțin o dată pentru citirea configurației inițiale a tablei.

\begin{minted}{c}
#include <stdio.h>

void main(void)
{ FILE *F=fopen("input.txt","rt");
  int SEven,SOdd,N,i,K;

  fscanf(F,"%d\n",&N);
  for (i=1, SEven=SOdd=0; i<=N; i++)
    { fscanf(F, "%d\n", &K);
      if (i&1) SOdd+=K;
        else SEven+=K;
    }
  fclose(F);

  printf("Mutarea mea: %c\n", SEven>=SOdd ? 'R' : 'L');
  for (i=1; i<N/2; i++)
    { printf("Mutarea dvs. (L/R) ? ");
      printf("Mutarea mea: %c\n", getchar());
      getchar(); /* Caracterul newline */
    }
  printf("Mutarea dvs. (L/R) ? ");
  getchar();

  if (SEven>=SOdd)
    printf("%d %d\n", SEven, SOdd);
    else printf("%d %d\n", SOdd, SEven);
}
\end{minted}

O a doua variantă a enunțului aduce unele condiții suplimentare:

\begin{itemize}

\item Se cere să se tipărească numai diferența maximă de scor pe care o poate
  obține primul jucător, considerând că ambii parteneri joacă perfect;

\item {\bf Complexitatea cerută} este $O(N^2)$.

\item {\bf Timpul de implementare} este de 45 minute, maxim 1h.

\end{itemize}

{\bf REZOLVARE}: Trebuie mai întâi să lămurim ce se înțelege prin „joc
perfect”. Jucătorul 1 are întotdeauna victoria la îndemână (metoda este
arătată mai sus), dar nu la orice scor. Jucătorul 2 urmărește să minimizeze
diferența de scor. Fie $D$ diferența de scor cu care se termină un joc. $D$
poate lua diferite valori pentru aceeași configurație inițială a tablei, în
funcție de mutările făcute de cei doi jucători. Fie $D_{MAX}$ diferența maximă
de scor pe care o poate obține primul jucător indiferent de mutările celui
de-al doilea. Exact această valoare trebuie aflată. $D_{MAX}$ nu este
propriu-zis o diferență maximă. Jucătorul 1 poate să câștige și la diferențe
mai mari decât $D_{MAX}$, dar trebuie ca jucătorul 2 să-l „ajute”. Să reluăm
exemplul cu 4 numere:

\centeredTikzFigure[
  mat/.style = {
    matrix of nodes,
    ampersand replacement=\&,
    column sep = 0.5em,
    nodes=cell,
  },
  cell/.style = {rectangle, draw, minimum width=1.5em, minimum height=1.5em},
]{
  \matrix[mat] {
    7 \& 10 \& 1 \& 2 \\
  };
}

În acest caz, primul jucător are asigurat scorul 12-8 (deci diferența
4). Pentru aceasta, el începe prin a lua numărul 2, apoi, orice ar replica
celălalt, va lua numărul 10, jucătorului 2 revenindu-i așadar numerele 1 și
7. El poate obține și scorul 17-3 (jucătorul 1 ia numărul 7, celălalt ia 2,
jucătorul 1 ia 10, iar celălalt ia 1), dar aceasta se întâmplă numai dacă
jucătorul 2 face o greșeală. După cum am arătat mai sus, dacă primul jucător
începe luând numărul 7, el pierde în mod normal partida. Iată deci că în acest
caz $D_{MAX}=4$.

Pentru a putea afla diferența maximă de scor, este bine să privim mereu în
adâncime. Există patru variante în care ambii parteneri pot face câte o
mutare:

\begin{enumerate}

\item Ambii jucători aleg numere din partea stângă;

\item Ambii aleg numere din partea dreaptă;

\item Primul jucător alege numărul din stânga, iar celălalt pe cel din
  dreapta;

\item Primul jucător alege numărul din dreapta, iar celălalt pe cel din
  stânga;

\end{enumerate}

În urma oricărei variante de mutare, secvența se scurtează cu două
elemente. Dacă am putea cunoaște dinainte care este rezultatul jocului pentru
fiecare din secvențele scurte, am putea să decidem care variantă de joc este
cea mai convenabilă pentru secvența inițială, ținând cont și de modul de joc
al jucătorului al doilea. Tocmai de aici vine și ideea de rezolvare. Să notăm
cu $A[1], A[2], ..., A[N]$ secvența citită la intrare. Vom construi o
matrice $D$ cu $N$ linii și $N$ coloane, unde $D[i,j]$ este diferența maximă
pe care o poate obține jucătorul 1 pentru secvența $A[i], A[i+1], \dots,
A[j]$. Bineînțeles, sunt luate în considerare numai secvențele de lungime
pară. Scopul nostru este să-l aflăm pe $D[1,N]$.

Elementele matricei pe care le putem afla fără multă bătaie de cap sunt
$D[1,2]$, $D[2,3]$, $\dots$, $D[N-1,N]$. Într-adevăr, dintr-o secvență de
numai două numere, primului jucător îi revine cel mai mare, iar celui de-al
doilea - cel mai mic. Așadar

\begin{equation}
  D[i,i + 1] = |A[i] - A[i + 1]|
\end{equation}

Cum calculăm $D[i,j]$ dacă cunoaștem valorile matricei $D$ pentru toate
subsecvențele incluse în secvența $A[i], A[i+1], ..., A[j]$? După cum am mai
spus, avem patru variante:

\centeredTikzFigure[
  scale=0.7,
  every node/.style={scale=0.7, anchor=base west},
  mat/.style = {
    matrix of nodes,
    ampersand replacement=\&,
  },
  header/.style={text width=9em, yshift=4em},
]{
  \matrix[mat] (m) {
    \node[header] {Elementul selectat de primul jucător}; \&
    \node[header] {Elementul selectat de al doilea jucător}; \&
    \node[header] {Rezultatul pentru secvența rămasă}; \&
    \node[header] {Rezultatul final}; \\
    $A[i]$ \&
    $A[i + 1]$ \&
    $D[i + 2, j]$ \&
    $R_1 = D[i + 2, j] + A[i] - A[i + 1]$ \\
    \&
    $A[j]$ \&
    $D[i + 1, j - 1]$ \&
    $R_2 = D[i + 1, j - 1] + A[i] - A[j]$ \\
    $A[j]$ \&
    $A[i]$ \&
    $D[i + 1, j - 1]$ \&
    $R_3 = D[i + 1, j - 1] + A[j] - A[i]$ \\
    \&
    $A[j - 1]$ \&
    $D[i, j - 2]$ \&
    $R_4 = D[i, j - 2] + A[j] - A[j - 1]$ \\
  };

  \draw[->] (m-2-1.east) -- (m-2-2.west);
  \draw[->] (m-2-2.east) -- (m-2-3.west);
  \draw[->] (m-2-3.east) -- (m-2-4.west);
  \draw[->] (m-2-1.east) -- (m-3-2.west);
  \draw[->] (m-3-2.east) -- (m-3-3.west);
  \draw[->] (m-3-3.east) -- (m-3-4.west);

  \draw[->] (m-4-1.east) -- (m-4-2.west);
  \draw[->] (m-4-2.east) -- (m-4-3.west);
  \draw[->] (m-4-3.east) -- (m-4-4.west);
  \draw[->] (m-4-1.east) -- (m-5-2.west);
  \draw[->] (m-5-2.east) -- (m-5-3.west);
  \draw[->] (m-5-3.east) -- (m-5-4.west);
}

Trebuie să ținem minte că, dacă primul jucător optează să-l aleagă pe $A[i]$
(una din primele două variante), atunci jucătorul 2 va juca în așa fel încât
pierderea să fie minimă, iar scorul final va fi $\min(R_1,R_2)$. Dacă
jucătorul 1 alege varianta 3 sau 4, scorul final va fi $\min(R_3,R_4)$. Dar
jucătorul 1 este primul la mutare, deci va alege varianta care îi maximizează
profitul. Rezultatul este

\begin{equation}
  D[i,j] = \max(\min(R_1, R_2), \min(R_3, R_4))
\end{equation}

adică

\begin{align}
  \begin{split}
    D[i,j] = \max( & A[i] + \min(D[i + 2, j] - A[i + 1], D[i + 1, j - 1] - A[j]), \\
    & A[j] + \min(D[i + 1, j - 1] - A[i], D[i, j - 2] - A[j - 1]))
  \end{split}
\end{align}

Matricea $D$ se completează pe diagonală, pornind de la diagonala principală
și mergând până în colțul de N-E. Iată cum arată matricea atașată datelor de
intrare din enunț:

\begin{equation}
  D =
  \begin{pmatrix}
    X & 3 & X & 10 & X & 7 \\
    X & X & 5 & X & 9 & X \\
    X & X & X & 7 & X & 4 \\
    X & X & X & X & 4 & X \\
    X & X & X & X & X & 3 \\
    X & X & X & X & X & X
  \end{pmatrix}
\end{equation}

Pentru exemplul din enunț, răspunsul este deci $D_{MAX}=7$. Cum elementele
matricei sunt parcurse cel mult o dată, rezultă o complexitate de $O(N^2)$.

\begin{minted}{c}
#include <stdio.h>
#include <stdlib.h>
#define NMax 101

int D[NMax][NMax], A[NMax], N;

void ReadData(void)
{ FILE *F=fopen("input.txt","rt");
  int i;

  fscanf(F,"%d\n",&N);
  for (i=1; i<=N;)
    fscanf(F, "%d\n", &A[i++]);
  fclose(F);
}

int Min(int A, int B)
{
  return A<B ? A : B;
}

int Max(int A, int B)
{
  return A>B ? A : B;
}

void FindMax(void)
{ int i,j,k;

  for (i=1;i<N;i++)
    D[i][i+1]=abs(A[i]-A[i+1]);
  for (k=3;k<=N-1;k++)
    for (i=1;i+k<=N;i++)
      { j=i+k;
        D[i][j]=Max(A[i]+Min(D[i+2][j]-A[i+1],
                             D[i+1][j-1]-A[j]),
                    A[j]+Min(D[i+1][j-1]-A[i],
                             D[i][j-2]-A[j-1]));
      }
  printf("Diferenta maxima este %d\n",D[1][N]);
}

void main(void)
{
  ReadData();
  FindMax();
}
\end{minted}

Programul prezentat mai sus poate fi optimizat, dacă timpul o permite și dacă
acest lucru este necesar. Lăsăm cititorul să încerce să rezolve aceeași
problemă folosind o cantitatate de memorie direct proporțională cu $N$.
