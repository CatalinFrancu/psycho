\section{Problema 16}

Următoarele probleme aparțin categoriei de probleme pe care, dacă ne grăbim,
le putem clasifica drept „ușoare”. Într-adevăr, ele au soluții vizibile și
foarte la îndemână, dar și soluții mai subtile și mult mai performante. Pentru
a obliga cititorul să se gândească și la aceste soluții, am ales limite pentru
datele de intrare suficient de mari încât să facă nepractice rezolvările „la
minut”.

{\bf ENUNȚ}: (Generarea unui arbore oarecare când i se cunosc gradele) Se dă
un vector cu $N$ numere întregi. Se cere să se construiască un arbore cu $N$
noduri numerotate de la 1 la $N$ astfel încât gradele celor $N$ noduri să fie
exact numerele din vector. Dacă acest lucru nu este posibil, se va da un mesaj
de eroare corespunzător.

{\bf Intrarea}: Datele de intrare se află în fișierul {\tt INPUT.TXT}. Pe
prima linie se află numărul de noduri $N$ ($N \leq 10.000$), iar pe a doua
linie se află cele $N$ numere separate prin spații. Toate numerele sunt strict
pozitive și mai mici ca 10.000.

Ieșirea se va face în fișierul {\tt OUTPUT.TXT}. Dacă problema are soluție, se
va tipări arborele prin muchiile lui. Fiecare muchie se va lista pe câte o
linie, prin nodurile adiacente separate printr-un spațiu. Dacă problema nu are
soluție, se va afișa un mesaj corespunzător.

\texttt{
  \begin{tabular}{|l|l|}
    \hline
        {\bf INPUT.TXT} & {\bf OUTPUT.TXT} \\ \hline
        6 & 1 4 \\
        1 2 3 2 1 1 & 2 5 \\
        & 3 6 \\
        & 4 6 \\
        & 5 6 \\ \hline
        3 & Problema nu are solutie! \\
        2 2 1 & \\
        \hline
  \end{tabular}
}

{\bf Timp de implementare}: 30 minute - 45 minute.

{\bf Timp de rulare}: 2-3 secunde.

{\bf Complexitate cerută}: $O(N \log N)$; dacă vectorul citit la intrare se
presupune sortat, se cere o complexitate $O(N)$.

{\bf REZOLVARE}: Să începem prin a ne pune întrebarea: când are problema
soluție și când nu?

Se știe că un arbore oarecare cu $N$ noduri are $N-1$ muchii. Fiecare din
aceste muchii va contribui cu o unitate la gradele nodurilor
adiacente. Deducem de aici că suma gradelor tuturor nodurilor este egală cu
dublul numărului de muchii, adică, notând cu $G[1], G[2], \dots, G[N]$ gradele
nodurilor,

\begin{equation}
  \sum_{i = 1}^{N} G[i] = 2 \cdot (N - 1)
\end{equation}

Am aflat deci o condiție necesară pentru ca problema să aibă soluție. O a doua
condiție este ca toate nodurile să aibă grade cuprinse între 1 și
$N-1$. Totuși, ținând cont de afirmația enunțului că toate numerele din vector
sunt strict pozitive, rezultă că a doua condiție nu mai trebuie
verificată. Iată de ce: să presupunem că am verificat prima condiție și am
constatat că suma celor $N$ numere este $2(N-1)$, iar unul dintre numere este
cel puțin $N$. Atunci ar rezulta că suma celorlalte $N-1$ numere este cel mult
$N-2$, de unde rezultă că există cel puțin un nod de grad 0, ceea ce
contrazice informația din enunț. Prin urmare, numai prima condiție este
importantă, cea de-a doua fiind redundantă.

Vom demonstra că această condiție este și suficientă indicând efectiv modul de
construcție a arborelui în cazul în care ea este satisfăcută. Începem prin a
sorta vectorul de numere. Acest lucru era oricum de așteptat, deoarece
complexitatea $N \log N$ ne-o permite. Trebuie numai să avem grijă să alegem
un algoritm de sortare de complexitate $N \log N$. Programul care urmează
folosește heapsort-ul. Odată ce am sortat vectorul, trebuie să reconstituim
muchiile în timp liniar, și iată cum:

\begin{itemize}

\item Se poate demonstra că primele două elemente din vectorul sortat au
  valoarea 1. Într-adevăr, dacă toate elementele ar fi mai mari sau egale cu
  2, atunci suma lor ar fi mai mare sau egală cu $2N$, ori noi știm că suma
  trebuie să fie $2N-2$, adică există cel puțin două elemente egale cu 1 în
  vector. Acest lucru rezultă imediat dacă ne gândim că orice arbore are cel
  puțin două frunze.

\item Vom căuta în vector primul număr mai mare sau egal cu 2. Se pune
  întrebarea: există întotdeauna acest număr? Nu cumva există un arbore în
  care toate nodurile au grad 1? Să aplicăm condiția precedentă și să vedem ce
  se întâmplă. Dacă toate nodurile au grad 1, atunci suma gradelor este $N$,
  ceea ce conduce la ecuația:

  \begin{equation}
    N = 2 \cdot (N - 1) \implies N = 2
  \end{equation}

\item Iată deci că există un singur arbore în care toate nodurile sunt frunze,
  anume cel cu 2 noduri unite printr-o muchie. Vom reveni mai târziu la acest
  caz particular. Deocamdată presupunem că există în vector un număr mai mare
  ca 1, pe poziția $K$ în vector. Atunci vom uni nodul 1 din arbore (care știm
  că are gradul 1) cu nodul $K$. În felul acesta, nodul 1 și-a completat
  numărul necesar de vecini și poate fi neglijat pe viitor, iar $G[K]$ va fi
  decrementat cu o unitate, întrucât nodul $K$ și-a completat unul din
  vecini. Astfel, problema s-a redus la un arbore cu $N-1$ noduri numerotate
  de la 2 la $N$.

\item Vectorul $G$ este în continuare sortat, deoarece $G[K-1] = 1 < G[K] \leq
  G[K+1]$ înainte de decrementarea lui $G[K]$, deci după decrementare vom avea
  $G[K-1] = 1 \leq G[K] < G[K+1]$, adică dubla relație de ordonare se
  păstrează.

\item Întrucât secvența $G[2], G[3], \dots, G[N]$ reprezintă gradele unui
  arbore, putem aplica același raționament ca mai înainte pentru a deduce că
  $G[2]=1$. Cu ce nod vom uni nodul 2? Dacă $G[K] > 1$, îl vom uni cu nodul
  $K$. Dacă prin decrementare, $G[K]$ a ajuns la valoarea 1, vom trece la
  nodul $K+1$ (despre care știm că are gradul mai mare ca 1) și vom trasa
  muchia $2 \leftrightarrow (K+1)$.

\item Procedeul acesta se repetă până când au fost trasate $N-2$
  muchii. Aceasta înseamnă că a mai rămas o singură muchie de trasat. Iată
  deci că, mai devreme sau mai târziu, este oricum inevitabil să ajungem la
  cazul particular de arbore de care am amintit mai devreme. Deoarece la
  primul pas am unit nodul 1 cu nodul $K$, la al doilea pas am unit nodul 2 cu
  un alt nod ($K$ sau $K+1$) ș.a.m.d., rezultă că în $N-2$ iterații, toate
  nodurile de la 1 la $N-2$ și-au completat numărul de vecini. De aici rezultă
  că ultima muchie pe care o vom trasa este $(N-1) \leftrightarrow N$; putem
  să tipărim această muchie „cu ochii închiși”, fără nici un fel de teste
  suplimentare. Ultima muchie trasată este diferită de celelalte și necesită o
  operație separată de trasare din cauză că, în timp ce primele $N-2$ iterații
  uneau o frunză cu un nod intern, această ultimă iterație are de unit două
  frunze, deci nu are sens să mai căutăm un nod de grad mai mare ca 1.
\end{itemize}

Aceasta este metoda de lucru. Calculul complexității este simplu: Avem nevoie
doar de doi indici: Unul care marchează frunza curentă (în program el se
numește pur și simplu {\tt i}) și care avansează la fiecare pas, și unul care
marchează primul număr mai mare ca 1 din vector (în program se numește {\tt
  First}) și care se incrementează cu cel mult 1 la fiecare pas (deci de mai
puțin de $N$ ori în total). De aici rezultă complexitatea liniară a
algoritmului.

Să vedem cum se comportă această metodă pe cazul particular al exemplului 1:

\centeredTikzFigure[
  mat/.style = {
    matrix of nodes,
    ampersand replacement=\&,
    anchor=north,
    nodes = {
      draw,
      rectangle,
      anchor=center,
      minimum width=2em,
      minimum height=2em,
    },
  },
  pad/.style = {
    ->,
    shorten >= 3pt,
    shorten <= 3pt,
  },
  index/.style = {
    <-,
    shorten >= 3pt,
    font=\footnotesize,
  },
  edge/.style = {
    xshift=80,
    yshift=-10.5,
  },
]{
  \node (g) at (0,1) {$G$};

  \matrix[mat] (m) at (0,0) {
    1 \& 2 \& 3 \& 2 \& 1 \& 1 \\[2em]
    1 \& 1 \& 1 \& 2 \& 2 \& 3 \\[2em]
    1 \& 1 \& 1 \& 1 \& 2 \& 3 \\[2em]
    1 \& 1 \& 1 \& 1 \& 1 \& 3 \\[2em]
    1 \& 1 \& 1 \& 1 \& 1 \& 2 \\[2em]
    1 \& 1 \& 1 \& 1 \& 1 \& 1 \\[2em]
  };

  \draw[pad] (m-1-3.south) -- node[right] {sortare} (m-2-3.north);

  \draw[index] (m-2-1.south) -- node[right] {i} ++(0,-0.7);
  \draw[index] (m-2-4.south) -- node[right] {First} ++(0,-0.7);

  \draw[index] (m-3-2.south) -- node[right] {i} ++(0,-0.7);
  \draw[index] (m-3-5.south) -- node[right] {First} ++(0,-0.7);

  \draw[index] (m-4-3.south) -- node[right] {i} ++(0,-0.7);
  \draw[index] (m-4-6.south) -- node[right] {First} ++(0,-0.7);

  \draw[index] (m-5-4.south) -- node[right] {i} ++(0,-0.7);
  \draw[index] (m-5-6.south) -- node[right] {First} ++(0,-0.7);

  \node[edge] at (m-2-6.south east) { $\implies$ muchia (1,4)};
  \node[edge] at (m-3-6.south east) { $\implies$ muchia (2,5)};
  \node[edge] at (m-4-6.south east) { $\implies$ muchia (3,6)};
  \node[edge] at (m-5-6.south east) { $\implies$ muchia (4,6)};
  \node[edge] at (m-6-6.north east) { $\implies$ muchia (5,6)};
}

Mai trebuie remarcat că soluția nu este unică. Propunem ca temă cititorului să
scrie un program care să verifice în timp $O(N \log N)$ dacă soluția furnizată
de un alt program este corectă.

\begin{lstlisting}[language=C]
#include <stdio.h>
int G[10001], N;
long Sum;
FILE *OutF;

void ReadData(void)
{ FILE *F=fopen("input.txt","rt");
  int i;

  fscanf(F,"%d\n",&N);
  for (i=1, Sum=0; i<=N; i++)
    { fscanf(F,"%d",&G[i]);
      Sum+=G[i];
    }
  fclose(F);
}

void Sift(int K, int N)
/* Cerne al K-lea element dintr-un heap de N elemente */
{ int Son;

  /* Alege un fiu mai mare ca tatal */
  if (K<<1<=N)
    { Son=K<<1;
      if (K<<1<N && G[(K<<1)+1]>G[(K<<1)])
        Son++;
      if (G[Son]<=G[K]) Son=0;
    }
    else Son=0;
  while (Son)
    { /* Schimba G[K] cu G[Son] */
      G[K]=(G[K]^G[Son])^(G[Son]=G[K]);
      K=Son;
      /* Alege un alt fiu */
      if (K<<1<=N)
        { Son=K<<1;
          if (K<<1<N && G[(K<<1)+1]>G[(K<<1)])
            Son++;
          if (G[Son]<=G[K]) Son=0;
        }
        else Son=0;
    }
}

void HeapSort(void)
{ int i;

  /* Construieste heap-ul */
  for (i=N>>1; i;) Sift(i--,N);
  /* Sorteaza vectorul */
  for (i=N; i>=2;)
    { G[1]=(G[1]^G[i])^(G[i]=G[1]);
      Sift(1,--i);
    }
}

void Match(void)
{ int i, First=1;

  while (G[First]==1 && First<N) First++;
  /* Trebuie adaugata si conditia First<N
     pentru a acoperi cazul particular N=2 */
  for (i=1; i<=N-2; i++)
    { fprintf(OutF,"%d %d\n", i, First);
      First+=(--G[First]==1);
    }
  fprintf(OutF,"%d %d\n", N-1, N);
}

void main(void)
{
  ReadData();
  OutF=fopen("output.txt","wt");
  if (Sum==(N-1)<<1)
    { HeapSort();
      Match();
    }
    else fputs("Problema nu are solutie!\n",OutF);
  fclose(OutF);
}
\end{lstlisting}
