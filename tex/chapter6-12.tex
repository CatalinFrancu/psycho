\section{Problema 12}

Această problemă a fost propusă la a IV-a Balcaniadă de Informatică, Nicosia
1996.

{\bf ENUNȚ}: Grupul de rock U2 va da un concert în Nicosia. Un grup de $N \leq
200$ fani U2 așteaptă la coadă în scopul de a cumpăra bilete de la singura
caserie deschisă. Fiecare persoană vrea să cumpere numai un bilet, iar
casierul poate vinde unei persoane cel mult două bilete.

Casierul folosește $T[i]$ unități de timp pentru a servi al $i$-lea fan ($1
\leq i \leq N$). Este posibil totuși ca doi fani așezați la coadă unul după
altul (de exemplu, al $j$-lea și al $j+1$-lea) să convină ca numai unul din ei
să rămână la coadă, iar celălalt să plece, dacă timpul $R[j]$ ($1 \leq j \leq
N-1$) în care casierul servește al $j$-lea și al $j+1$-lea fan este mai mic
decât $T[j]+T[j+1]$. Deci, pentru a minimiza timpul de lucru al casierului,
fiecare persoană din coadă încearcă să negocieze cu predecesorul și cu
succesorul său, ceea ce va duce în final la o servire mai rapidă.

Fiind date numerele întregi pozitive $N$, $T[i]$ ($1 \leq i \leq N$) și $R[j]$
($1 \leq j \leq N-1$), se cere să se minimizeze timpul total al
casierului. Acest lucru va fi realizat grupând într-un mod optim perechi de
persoane consecutive. Atenție! Nu este necesar ca un anumit fan să se cupleze
neapărat cu predecesorul sau cu succesorul său.

{\bf Intrarea}: În fișierul {\tt INPUT.TXT}, datele de intrare sunt date pe
trei linii:

\begin{itemize}

\item prima linie conține numărul întreg $N$;

\item a doua linie conține $N$ întregi: valorile $T[i]$, separate prin câte un
  spațiu;

\item a treia linie conține $N$-1 întregi: valorile $R[j]$, separate de
  asemenea prin câte un spațiu;

\end{itemize}

Ieșirea se va face în fișierul {\tt OUTPUT.TXT}, astfel:

\begin{itemize}

\item prima linie conține un întreg care reprezintă timpul total (minim) al
  casierului;

\item pe fiecare din următoarele linii se află un singur număr sau două numere
  separate prin caracterul '+'. Mai exact, fiecare linie conține numărul $i$
  dacă al $i$-lea fan este servit singur, sau $i+(i+1)$ dacă cei doi fani sunt
  serviți ca o pereche.

\end{itemize}

{\bf Exemplu}:

\texttt{
  \begin{tabular}{|l|l|}
    \hline
        {\bf INPUT.TXT} & {\bf OUTPUT.TXT} \\ \hline
        7              & 14 \\
        5 4 3 2 1 4 4  & 1 \\
        7 3 4 2 2 4    & 2+3 \\
        \              & 4+5 \\
        \              & 6+7 \\
        \hline
  \end{tabular}
}

{\bf Timp de rulare}: 15 secunde pentru un test.

Acesta este enunțul în forma lui de la Nicosia. Iată acum și completările „din
studio”:

\begin{itemize}

\item Numărul de fani este $N \leq 5.000$;

\item {\bf Complexitatea cerută} este $O(N)$;

\item {\bf Timpul de rulare} este de o secundă.

\end{itemize}

{\bf REZOLVARE}: O primă metodă de rezolvare o vom lăsa în seama cititorului,
întrucât ea este foarte asemănătoare cu rezolvarea problemei înmulțirii optime
a unui șir de matrice. Ideea de pornire este de a defini o matrice $D$ cu $N$
linii și $N$ coloane, în care $D[X,Y]$ reprezintă timpul minim în care pot fi
serviți fanii $X, X+1, \dots, Y-1, Y$. Scopul este de a afla valoarea
$D[1,N]$.

După cum se știe, însă, înmulțirea optimă a unui șir de matrice se poate
determina în timp $O(N^3)$. Pentru condițiile inițiale ale problemei ($N \leq
200$), metoda se încadrează în timp. Ea putea fi deci folosită la concurs, pe
câtă vreme dacă se impun condițiile suplimentare, rezolvarea în timp cubic nu
mai dă rezultate. Iată ideea de rezolvare a problemei în timp liniar.

Vom denumi o {\bf cuplare de ordin $K$} modul în care primii $K$ fani sunt
serviți câte unul sau câte doi. Fiecare cuplare are atașat un cost, respectiv
timpul consumat de casier pentru a servi toți fanii. Dintre toate cuplările de
ordin $K$, cele pentru care costul este minim (în cazul general pot fi mai
multe) se vor numi {\bf cuplări optime de ordinul $K$}. Cerința problemei
exprimată cu noua terminologie este: să se găsească o cuplare optimă de
ordinul $N$.

Să presupunem acum că am găsit cumva această cuplare optimă de ordinul $N$, pe
care o vom nota cu $C_N$. Dacă în această cuplare al $K$-lea fan este servit
de unul singur, atunci modul de servire al primilor $K-1$ fani reprezintă o
cuplare optimă de ordinul $K-1$, pe care o vom nota cu $C_{K-1}$. Demonstrația
nu este grea: dacă fanii $1, 2, \dots, K-1$ nu ar fi cuplați în mod optim în
cadrul lui $C_N$, atunci ar exista o cuplare a lor $C'_{K-1}$ de cost mai mic
decât $C_{K-1}$. Dar această cuplare mai bună ar putea fi folosită pentru a
obține o cuplare mai bună a tuturor celor $N$ fani (servind primii $K-1$ fani
conform cuplării $C'_{K-1}$, iar pe ceilalți conform cuplării $C_N$). S-ar
obține astfel o cuplare $C'_N$ de cost mai mic decât $C_N$, ceea ce este
absurd, deoarece am presupus $C_N$ ca fiind optimă.

În mod absolut identic se poate demonstra că dacă $C_N$ este o cuplare optimă
de ordinul $N$ în care fanii $K$ și $K+1$ sunt serviți împreună, atunci modul
de servire al primilor $K-1$ fani reprezintă o cuplare optimă de ordinul
$K-1$. Recunoaștem în aceste afirmații principiul programării dinamice:
optimul global presupune optime locale. De aici deducem că, pentru a realiza o
cuplare optimă a primilor $K$ fani avem nevoie de câte o cuplare optimă pentru
primii $K-1$ fani, respectiv pentru primii $K-2$ fani, urmând ca apoi să aflăm
care este costul minim al unei cuplări de ordin $K$, atât în ipoteza că fanul
al $K$-lea este servit singur, cât și în ipoteza că el este servit împreună cu
al $K-1$-lea fan.

Vom crea așadar doi vectori $T_1$ și $T_2$, ambii cu câte $N$ elemente, în
care:

\begin{itemize}

\item $T_1[K]$ este timpul minim în care pot fi serviți fanii $1, 2, \dots,
  K$, astfel încât fanul al $K$-lea să rămână singur;

\item $T_2[K]$ este timpul minim în care pot fi serviți fanii $1, 2, \dots,
  K$, astfel încât fanii $K$ și $K-1$ să fie cuplați.

\end{itemize}

Datele inițiale pe care le putem trece în cei doi vectori sunt:

\begin{itemize}

\item $T_1[1] = T[1]$;

\item $T_2[1] = \infty$ (un singur fan nu poate fi cuplat cu nimeni);

\item $T_1[2] = T[1]+T[2]$ (dacă al doilea fan rămâne singur, atunci și primul
  rămâne singur);

\item $T_2[2] = R[1]$.

\end{itemize}

Relațiile de recurență se deduc fără prea multă bătaie de cap:

\begin{itemize}

\item $T_1[K] = T[K] + \min(T_1[K-1], T_2[K-1])$ (dacă fanul $K$ rămâne
  singur, atunci el este servit în timpul $T[K]$, iar fanul $K-1$ se va cupla
  cu $K-2$ sau va rămâne singur, după cum este mai convenabil);

\item $T_2[K] = R[K-1] + \min(T_1[K-2], T_2[K-2])$ (dacă fanul $K$ se cuplează
  cu $K-1$, atunci ei sunt serviți în timpul $R[K-1]$, iar fanul $K-2$ se va
  cupla cu $K-3$ sau va rămâne singur, după cum este mai convenabil);

\end{itemize}

Putem deci să completăm vectorii de la stânga la dreapta. Odată ce am făcut
aceasta, costul minim al cuplării de ordinul $N$ este $\min(T_1[N],
T_2[N])$. Pentru a reconstitui și așezarea fanilor, vom proceda recurent,
astfel: dacă $T_2[N]<T_1[N]$, înseamnă că este mai avantajos ca fanii $N$ și
$N-1$ să fie cuplați și reluăm reconstituirea pentru fanii $1, 2, \dots,
N-2$. În caz contrar, înseamnă că este mai avantajos ca fanul $N$ să rămână
singur și reluăm reconstituirea pentru fanii $1, 2, \dots, N-1$. De exemplu,
iată modul în care se completează vectorii $T_1$ și $T_2$ pentru exemplul din
enunț:

\centeredTikzFigure[
  scale = 0.75,
  mat/.style = {
    matrix of nodes,
    ampersand replacement=\&,
    nodes = {
      scale = 0.75,
      draw,
      rectangle,
      anchor=center,
      font=\Large,
      minimum width=3.2em,
      minimum height=3.2em,
    },
  },
  double/.style = {
    nodes={
      draw = none,
      font = \tiny,
    },
  },
]{
  \matrix[mat] at (0,0) {
    5 \& 4 \& 3 \& 2 \& 1 \& 4 \& 4 \\
  };

  \matrix[mat] at (0,-2) {
    7 \& 3 \& 4 \& 2 \& 2 \& 4 \\
  };

  \matrix[mat,double] at  (0, -3.6) {
    \ \& 4 + 5 = \& 3 + 7 = \& 2 + 8 = \& 1 + 10 = \& 4 + 10 = \& 4 + 12 = \\
  };
  \matrix[mat] at (0,-4) {
    5 \& 9 \& 10 \& 10 \& 11 \& 14 \& 16 \\
  };

  \matrix[mat,double] at  (0, -5.6) {
    \ \& \ \& 3 + 5 = \& 4 + 7 = \& 2 + 8 = \& 2 + 10 = \& 4 + 10 = \\
  };
  \matrix[mat] at (0,-6) {
    $\infty$ \& 7 \& 8 \& 11 \& 10 \& 12 \& 14 \\
  };

  \node at (-6,0) {\Large $T$};
  \node at (-6,-2) {\Large $R$};
  \node at (-6,-4) {\Large $T_1$};
  \node at (-6,-6) {\Large $T_2$};
}

\begin{itemize}

\item $T_2[7]<T_1[7] \implies$ Fanii 6 și 7 sunt cuplați. Reconstituim
  așezarea primilor 5 fani.

\item $T_2[5]<T_1[5] \implies$ Fanii 4 și 5 sunt cuplați. Reconstituim
  așezarea primilor 3 fani.

\item $T_2[3]<T_1[3] \implies$ Fanii 2 și 3 sunt cuplați, iar primul fan este
  singur.

\end{itemize}

\inputminted{c}{src/problem12.c}
