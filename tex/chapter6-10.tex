\section{Problema 10}

Această problemă a fost dată la unul din barajele pentru selecționarea lotului
restrâns al României pentru Olimpiada Internațională din 1996.

{\bf ENUNȚ}: Fie un număr prim $P$. Pe mulțimea $\{0, 1, \dots, P-1\}$ se
definesc operațiile binare $+$, $-$, $\times$, $/$ modulo $P$, în felul
următor:

\begin{enumerate}

\item $a + b \bmod P$ este restul împărțirii lui $a+b$ la $P$. Analog pentru
  „$\times$”.

\item Expresia $a-b$ este definită ca fiind soluția ecuației $b+x \equiv a
  \pmod{P}$. Analog pentru „/”.

\end{enumerate}

Se știe că ecuația $b + x = a$ are întotdeauna soluție unică, iar $b \times x
= a$ are soluție unică pentru orice $b \neq 0$. Pentru $b = 0$, operația $a/b$
nu e definită. De exemplu, dacă $P=11$, atunci $6+7=2$, $6-7=10$, $6 \times
7=9$, $6/7=4$. Se știe că adunarea și înmulțirea modulo $P$ sunt comutative,
asociative, posedă elemente neutre (pe 0, respectiv pe 1), iar adunarea este
distributivă față de înmulțire. În plus, pentru orice $a$ există $b$ astfel
încât $a+b=0$; notând $b$ cu ($-a$) avem $c-a=c+(-a)=c+b$ pentru orice $c$. De
asemenea, pentru orice $a \neq 0$ există $b$ astfel încât $a \times b = 1$;
notând $b$ cu $(1/a)$ avem că $c/a=c \times (1/a)=c \times b$.

Dându-se un număr prim $P$, un întreg $D$ între 0 și $P-1$ și un șir de $N$
numere, cuprinse fiecare între 0 și $P-1$, se cere să se introducă între
elementele șirului operatorii $+$, $-$, $\times$, $/$ și parantezele
corespunzătoare, astfel încât să se obțină o expresie corectă a cărei valoare
să fie $D$ (lucrând în aritmetica modulo $P$). În caz că acest lucru nu este
posibil, se va afișa un mesaj corespunzător.

{\bf Intrarea}: Fișierul {\tt INPUT.TXT} conține două linii:

\begin{itemize}

\item pe prima linie se găsesc trei numere întregi: $P$, $N$ și $D$, separate
  prin spații ($2 \leq P \leq 23$, $P$ prim, $1 \leq N \leq 30$, $0 \leq D
  \leq P-1$);

\item pe următoarea linie se găsește șirul de numere ce formează expresia ($N$
  numere întregi cuprinse între 0 și $P-1$).

\end{itemize}

{\bf Ieșirea}: Fișierul text {\tt OUTPUT.TXT} va conține o singură linie pe
care se va găsi expresia generată sau mesajul „Nu există soluție.”. Expresia
va fi parantezată complet (fiecare operator va avea o pereche de paranteze
atașate).

{\bf Exemple}:

\texttt{
  \begin{tabular}{|l|l|}
    \hline
        {\bf INPUT.TXT} & {\bf OUTPUT.TXT} \\ \hline
        11 3 6 & (4+(7/9)) \\
        4 7 9  &           \\ \hline
        11 3 7 & Nu exista solutie. \\
        1 1 1  &           \\
        \hline
  \end{tabular}
}

{\bf Timp de execuție} pentru un test: 1 minut.

Modificările și completările propuse de autor sunt:

\begin{itemize}

\item {\bf Timp de execuție}: 10 secunde

\item {\bf Timp de implementare}: 1h 30 minute, maxim 1h 45 minute.

\item {\bf Complexitate cerută}: $O(N^3 \times P^2)$.

\end{itemize}

{\bf REZOLVARE}: Problema în sine nu este foarte complicată. Ea face apel la
programarea dinamică, dar este foarte asemănătoare cu una din problemele bine
cunoscute de elevi - înmulțirea optimă a unui șir de matrice. Ceea ce o face
mai „provocatoare” este timpul alocat implementării. De fapt, la respectiva
probă au fost propuse trei probleme, cam de același nivel de dificultate, iar
timpul total permis a fost de 4 ore. De asemenea, structurile de date impuse
și modul de utilizare a lor sunt mai rar întâlnite.

Ideea de la care se pornește în rezolvarea acestei probleme este următoarea:
Parantezarea și completarea cu operatori a unui șir de $N$ numere, $V=(V(1),
V(2), \dots, V(N))$, astfel încât să se obțină rezultatul $D$ este posibilă
dacă și numai dacă există două valori $D_1$ și $D_2$, un număr întreg $K$
cuprins între 1 și $N-1$ și un operator $\oplus \in \{ +, -, \times, / \}$
astfel încât următoarele condiții să fie îndeplinite simultan:

\begin{enumerate}

\item Este posibilă parantezarea și completarea cu operatori a șirului $V' =
  (V(1),$ $V(2),$ $\dots, V(K))$ astfel încât să se obțină rezultatul $D_1$;

\item Este posibilă parantezarea și completarea cu operatori a șirului
  $V''=(V(K+1), V(K+2), \dots, V(N))$ astfel încât să se obțină rezultatul
  $D_2$;

\item $D_1 \oplus D_2 = D$

\end{enumerate}

Cu alte cuvinte, trebuie să găsim un loc în care să „spargem” expresia noastră
în așa fel încât, luând două valori care se pot obține pentru partea din
stânga, respectiv din dreapta, și inserând între ele operatorul potrivit, să
obținem valoarea $D$.

Pentru aflarea operatorului, a locului de împărțire a expresiei în două și a
celor două valori necesare, ar fi de ajuns patru instrucțiuni repetitive {\tt
  for}. Totuși, rămâne o singură întrebare: cum se poate verifica dacă se
poate sau nu obține valoarea $D_1$ pentru subexpresia din stânga, respectiv
valoarea $D_2$ pentru subexpresia din dreapta? Aceasta este o subproblemă
similară cu problema în sine, dar redusă la dimensiuni mai mici. O abordare
directă ar fi comodă: se scrie o procedură care efectuează cele patru
instrucțiuni for și, pentru fiecare combinație posibilă de operatori și
operanzi, se reapelează recursiv ca să afle dacă valorile operanzilor se pot
obține. Totuși, este intuitiv că această variantă va avea o complexitate
uriașă, care o face inutilizabilă.

Motivul principal al nerentabilității acestei implementări este că ea reface
de nenumărate ori exact aceleași calcule. Spre exemplu, dacă $N=4$, programul
va încerca să spargă vectorul $(V(1), V(2), V(3), V(4))$ în două părți. Există
trei moduri posibile:

\begin{enumerate}[label=(\alph*)]

\item $(V(1))$ și $(V(2), V(3), V(4))$;

\item $(V(1), V(2))$ și $(V(3), V(4))$;

\item $(V(1), V(2), V(3))$ și $(V(4))$;

\end{enumerate}

Pentru a studia cazul ({\bf c}), expresia stângă va trebui la rândul ei ruptă
în două bucăți, lucru care poate fi făcut în două moduri:

\begin{enumerate}[label=(\alph*)]
  \setcounter{enumi}{3}

\item $(V(1))$ și $(V(2), V(3))$;

\item $(V(1), V(2))$ și $(V(3))$;

\end{enumerate}

Se observă că deja secvența $(V(1), V(2))$ a fost studiată de două ori (în
cazurile ({\bf b}) și ({\bf e})), iar secvența $(V(1))$ tot de două ori (în
cazurile ({\bf a}) și ({\bf d})). Exemplele pot continua. Dacă însă am reuși
să nu mai evaluăm de două ori aceeași secvență, complexitatea programului s-ar
reduce foarte mult. Pentru aceasta, trebuie să pornim cu secvențe foarte
scurte (întâi cele de un singur număr, apoi cele de două numere), și să trecem
la secvențe mai lungi, bazându-ne pe faptul că secvențele mai lungi se
descompun în secvențe mai scurte care au fost deja analizate. Facem mențiunea
că o secvență cu un singur număr poate fi parantezată într-un singur fel
(practic nu este nevoie de paranteze și operatori) și poate produce un singur
rezultat, egal cu valoarea numărului. O secvență de două numere poate produce
maximum patru rezultate distincte, prin folosirea pe rând a celor patru
operatori disponibili.

Pentru a stoca rezultatele obținute, vom folosi o matrice tridimensională $A$
de dimensiuni $N \times N \times P$. $A[i,j,r]$ indică dacă există vreo
parantezare corespunzătoare a secvenței $(V(i), V(i+1), \dots, V(j))$ astfel
încât să se obțină rezultatul $r$. Dacă o asemenea parantezare există,
$A[i,j,r]$ va indica punctul în care trebuie spartă în două expresia
(printr-un număr între $i$ și $j-1$). Dacă nu există o asemenea parantezare,
$A[i,j,r]$ va lua o valoare specială (0 de exemplu).

De aici decurge modul de inițializare al matricei:

\begin{equation}
  \begin{cases}
    A[i,i,V(i)] = i, & \quad \forall 1 \leq i \leq N \\

    A[i,j,r] = 0, & \quad \text{pentru orice alte valori ale lui } i, j \text{
      și } r
  \end{cases}
\end{equation}

Matricea se va completa pe diagonală, începând de la diagonala principală și
terminând în colțul de NE. Pentru a afla toate valorile care se pot obține
prin parantezarea secvenței $(V(i), V(i+1), \dots, V(j))$, se va împărți
această expresie în două părți disjuncte, în toate modurile posibile: $(V(i))$
și $(V(i+1), \dots, V(j))$, apoi $(V(i), V(i+1))$ și $(V(i+2), \dots, V(j))$
și așa mai departe până la $(V(i), V(i+1), \dots, V(j-1))$ și
$(V(j))$. Fiecăreia din părțile obținute îi va corespunde în matrice un
element de forma $A[i,k]$ sau $A[k+1,j]$, cu $i \leq k < j$. În orice caz,
toate elementele de această formă se vor afla în matrice dedesubtul diagonalei
din care face parte elementul $A[i,j]$, deci pentru secvențele respective se
cunosc deja toate valorile pe care le pot lua prin parantezare și introducerea
operatorilor. Tot ce avem de făcut este să combinăm în toate modurile aceste
valori prin inserarea fiecăruia din cei patru operatori pentru a obține toate
valorile posibile ale expresiei $(V(i), V(i+1), \dots, V(j))$.

Dacă în final $A[1,N,D] \neq 0$, atunci problema are soluție. Vom vedea
imediat și cum se reconstituie ea. Iată modul de compunere a matricei pentru
exemplul din enunț (cu deosebirea că, în figurile de mai jos, $A[i,j]$ nu mai
este un vector cu $P$ elemente, ci o mulțime de valori între 0 și $P-1$,
această reprezentare fiind mai comodă):

\begin{equation}
  A[1,1] = \{4\} \quad A[2,2] = \{7\} \quad A[3,3] = \{9\}
\end{equation}

\begin{equation}
  A =
  \begin{pmatrix}
    \{4\} & ? & ? \\
    ? & \{7\} & ? \\
    ? & ? & \{9\}
  \end{pmatrix}
\end{equation}

Între numerele 4 și 7 plasăm cei patru operatori și obținem:

\begin{align*}
  4 + 7 & \equiv 0 \pmod{11} \\
  4 - 7 & \equiv 4 + 4 \equiv 8 \pmod{11} \\
  4 \times 7 & \equiv 6 \pmod{11} \\
  4\ /\ 7 & \equiv 4 \times 8 \equiv 10 \pmod{11} \\
  A[1,2] & = \{0, 6, 8, 10\}
\end{align*}

Analog se procedează pentru numerele 7 și 9:

\begin{align*}
  7 + 9 & \equiv 5 \pmod{11} \\
  7 - 9 & \equiv 7 + 2 \equiv 9 \pmod{11} \\
  7 \times 9 & \equiv 8 \pmod{11} \\
  7\ /\ 9 & \equiv 7 \times 5 \equiv 2 \pmod{11} \\
  A[2,3] & = \{2, 5, 8, 9\}
\end{align*}

\begin{equation}
  A =
  \begin{pmatrix}
    \{4\} & \{0, 6, 8, 10\} & ? \\
    ? & \{7\} & \{2, 5, 8, 9\} \\
    ? & ? & \{9\}
  \end{pmatrix}
\end{equation}

Pentru a calcula $A[1,3]$, putem grupa termenii în două moduri: Fie primul
separat și ultimii doi separat, fie primii doi separat și ultimul separat. În
primul caz, ultimii doi termeni - după cum s-a văzut - pot produce patru
rezultate distincte (2, 5, 8 și 9). Combinând oricare din aceste rezultate cu
primul termen (4) și adăugând orice operator, vor rezulta 16 valori posibile,
din care evident unele vor coincide. Analog se procedează și pentru celălalt
caz:

\begin{table}[H]
  \centering
  \begin{tabular}{l@{\hspace{1in}}l}
    \hline
    Cazul I & Cazul II \\ \hline
    {$\begin{aligned}
        4 + 2 & \equiv 6 \pmod{11} \\
        4 - 2 & \equiv 2 \pmod{11} \\
        4 \times 2 & \equiv 8 \pmod{11} \\
        4\ /\ 2 & \equiv 2 \pmod{11} \\
      \end{aligned}$}
    &
    {$\begin{aligned}
        0 + 9 & \equiv 9 \pmod{11} \\
        0 - 9 & \equiv 2 \pmod{11} \\
        0 \times 9 & \equiv 0 \pmod{11} \\
        0\ /\ 9 & \equiv 0 \pmod{11} \\
      \end{aligned}$}
    \\ \hline
    {$\begin{aligned}
        4 + 5 & \equiv 9 \pmod{11} \\
        4 - 5 & \equiv 10 \pmod{11} \\
        4 \times 5 & \equiv 9 \pmod{11} \\
        4\ /\ 5 & \equiv 3 \pmod{11} \\
      \end{aligned}$}
    &
    {$\begin{aligned}
        6 + 9 & \equiv 4 \pmod{11} \\
        6 - 9 & \equiv 8 \pmod{11} \\
        6 \times 9 & \equiv 10 \pmod{11} \\
        6\ /\ 9 & \equiv 8 \pmod{11} \\
      \end{aligned}$}
    \\ \hline
    {$\begin{aligned}
        4 + 8 & \equiv 1 \pmod{11} \\
        4 - 8 & \equiv 7 \pmod{11} \\
        4 \times 8 & \equiv 10 \pmod{11} \\
        4\ /\ 8 & \equiv 6 \pmod{11} \\
      \end{aligned}$}
    &
    {$\begin{aligned}
        8 + 9 & \equiv 6 \pmod{11} \\
        8 - 9 & \equiv 10 \pmod{11} \\
        8 \times 9 & \equiv 6 \pmod{11} \\
        8\ /\ 9 & \equiv 7 \pmod{11} \\
      \end{aligned}$}
    \\ \hline
    {$\begin{aligned}
        4 + 9 & \equiv 2 \pmod{11} \\
        4 - 9 & \equiv 6 \pmod{11} \\
        4 \times 9 & \equiv 3 \pmod{11} \\
        4\ /\ 9 & \equiv 9 \pmod{11} \\
      \end{aligned}$}
    &
    {$\begin{aligned}
        10 + 9 & \equiv 8 \pmod{11} \\
        10 - 9 & \equiv 1 \pmod{11} \\
        10 \times 9 & \equiv 2 \pmod{11} \\
        10\ /\ 9 & \equiv 6 \pmod{11} \\
      \end{aligned}$}
    \\ \hline
  \end{tabular}
\end{table}

\begin{align*}
  A[1,3] = \{0, 1, 2, 3, 4, 6, 7, 8, 9, 10\}
\end{align*}

\begin{equation}
  A =
  \begin{pmatrix}
    \{4\} & \{0, 6, 8, 10\} & \{0 \dots 4, 6 \dots 10\} \\
    ? & \{7\} & \{2, 5, 8, 9\} \\
    ? & ? & \{9\}
  \end{pmatrix}
\end{equation}

Se observă că singura valoare care nu se poate obține prin parantezarea
șirului (4, 7, 9) este 5. Să vedem acum și care este metoda de reconstituire a
soluției. Fie $A[1,N,D]=X$. Dacă $X=0$, atunci nu există soluție. Dacă $X \neq
0$, atunci $X$ indică poziția din vector după care trebuie inserat semnul. Nu
se indică însă ce semn trebuie inserat, nici care sunt valorile care trebuie
obținute pentru partea stângă, respectiv dreaptă. De aceea, vom căuta o
combinație oarecare de valori $D_1$ și $D_2$ care se pot obține și un operator
oarecare astfel încât $D_1 \oplus D_2 = D$. Odată ce le găsim, vom căuta, prin
aceeași metodă, o modalitate de a obține valoarea $D_1$ în partea stângă a
vectorului (știm sigur că această modalitate există) și o modalitate de a
obține valoarea $D_2$ în partea dreaptă a vectorului.

Iată în continuare câteva detalii de implementare. Pentru efectuarea
operațiilor matematice modulo $P$ s-a scris o funcție separată, $Expr$, care
primește două numere $X$ și $Y$ între 0 și $P-1$ și un număr $Op$ între 1 și 4
reprezentând codificarea operatorului și întoarce un număr între 0 și $P$,
reprezentând valoarea operației ($X\,Op\,Y$). De fapt, ar trebui ca această
funcție să întoarcă un număr între 0 și $P-1$ dacă operația este posibilă și
să nu întoarcă nimic dacă operația este imposibilă, respectiv dacă se încearcă
o împărțire la 0. Cum acest lucru nu este posibil în Pascal, am asimilat
valoarea $P$ cu un cod de eroare, iar funcția va întoarce această valoare
(care nu poate fi atinsă prin operații obișnuite) în cazul unei împărțiri prin
0. Mai departe, pentru a nu face un test separat dacă rezultatul funcției
reprezintă o adresă valabilă în cea de-a treia dimensiune a tabloului, am
preferat să supradimensionăm tabloul cu o unitate și să ignorăm tot ceea ce se
scrie în coloana $P$.

Să ne ocupăm acum de operațiile aritmetice. Adunarea, scăderea și înmulțirea
se fac în timp constant. O problemă apare în cazul împărțirii, deoarece ea nu
mai seamănă deloc cu cea învățată pe mulțimea $\mathbb{R}$. Pentru a calcula
$X/Y$, trebuie găsit acel număr $Z$ care, înmulțit cu $Y$, să dea $X$. Acest
lucru se poate face într-o primă fază prin căutare secvențială (se încearcă
valoarea 0, apoi 1, apoi 2 și așa mai departe; trebuie să existe un cât
deoarece $P$ este prim). Tehnica are însă influențe neplăcute asupra
complexității, supărătoare asupra timpului de rulare și dezastruoase asupra
punctajului obținut. De aceea, este bine ca, în măsura în care timpul o
permite, să se construiască o tabelă predefinită de calculare a
inverșilor. Atunci, în loc să se efectueze împărțirea $X/Y$, se efectuează
înmulțirea $X \times Y^{-1}$. Inversul unui element depinde și de modulul
ales. Programul care urmează construiește un tabel care a fost importat în
programul sursă ca o constantă, matricea {\tt Invers} ({\tt Invers[A,B]} este
inversul lui $B$ modulo $A$). Liniile corespunzătoare unor numere neprime au
fost totuși inserate, pentru ușurința implementării.

\begin{minted}{pascal}
program Invert;
const NMax=30;
      PMax=23;
var i,P:Integer;

function Invers(P,K:Integer):Integer;
var i:Integer;
begin
  i:=0;
  repeat Inc(i) until (K*i) mod P=1;
  Invers:=i;
end;

begin
  Assign(Output,'invers.txt');Rewrite(Output);
  for P:=2 to PMax do
    begin
      Write('{',P:2,'} (');
      for i:=1 to NMax do
        begin
          if (P in [2,3,5,7,11,13,17,19,23]) and (i<P)
            then Write(Invers(P,i):2)
            else Write(99);
          if i<>NMax then Write(',');
          if i=NMax div 2 then Write(#13#10'      ');
        end;
      WriteLn('),');
    end;
  Close(Output);
end.
\end{minted}

Să analizăm în sfârșit complexitatea: Trebuie completată o matrice, deci
$O(N^2)$ elemente. Pentru fiecare element din matrice, secvența
corespunzătoare din vector trebuie spartă în două în toate modurile posibile,
deci încă $O(N)$. Pentru fiecare descompunere, trebuie combinate în toate
felurile toate valorile disponibile pentru partea stângă, respectiv
dreaptă. Cum numărul de valori este $O(P)$, rezultă o complexitate de
$O(P^2)$. Înmulțind toți acești factori rezultă o complexitate totală de
$O(N^3 \times P^2)$. Dacă nu am fi făcut împărțirea a două numere modulo P în
timp constant, ci în $O(P)$, atunci complexitatea totală ar fi fost $O(N^3
\times P^3)$, deci timpul de rulare putea fi și de 20 de ori mai mare.

Programul de mai jos pare îngrozitor de lung, dar, dacă avem în vedere faptul
că o bună bucată o reprezintă constanta {\tt Invers}, care este generată,
putem avea speranțe să-l scriem în timpul alocat.

\begin{minted}{pascal}
program ParaNT;
{$B-,I-,R-,S-}
const NMax=30;
      PMax=23;
      NoWay=0;
      OpNames:String[4]='+-*/';
      Invers:array[2..PMax,1..NMax] of Integer=
{ 2}(( 1,99,99,99,99,99,99,99,99,99,99,99,99,99,99,
      99,99,99,99,99,99,99,99,99,99,99,99,99,99,99),
{ 3} ( 1, 2,99,99,99,99,99,99,99,99,99,99,99,99,99,
      99,99,99,99,99,99,99,99,99,99,99,99,99,99,99),
{ 4} (99,99,99,99,99,99,99,99,99,99,99,99,99,99,99,
      99,99,99,99,99,99,99,99,99,99,99,99,99,99,99),
{ 5} ( 1, 3, 2, 4,99,99,99,99,99,99,99,99,99,99,99,
      99,99,99,99,99,99,99,99,99,99,99,99,99,99,99),
{ 6} (99,99,99,99,99,99,99,99,99,99,99,99,99,99,99,
      99,99,99,99,99,99,99,99,99,99,99,99,99,99,99),
{ 7} ( 1, 4, 5, 2, 3, 6,99,99,99,99,99,99,99,99,99,
      99,99,99,99,99,99,99,99,99,99,99,99,99,99,99),
{ 8} (99,99,99,99,99,99,99,99,99,99,99,99,99,99,99,
      99,99,99,99,99,99,99,99,99,99,99,99,99,99,99),
{ 9} (99,99,99,99,99,99,99,99,99,99,99,99,99,99,99,
      99,99,99,99,99,99,99,99,99,99,99,99,99,99,99),
{10} (99,99,99,99,99,99,99,99,99,99,99,99,99,99,99,
      99,99,99,99,99,99,99,99,99,99,99,99,99,99,99),
{11} ( 1, 6, 4, 3, 9, 2, 8, 7, 5,10,99,99,99,99,99,
      99,99,99,99,99,99,99,99,99,99,99,99,99,99,99),
{12} (99,99,99,99,99,99,99,99,99,99,99,99,99,99,99,
      99,99,99,99,99,99,99,99,99,99,99,99,99,99,99),
{13} ( 1, 7, 9,10, 8,11, 2, 5, 3, 4, 6,12,99,99,99,
      99,99,99,99,99,99,99,99,99,99,99,99,99,99,99),
{14} (99,99,99,99,99,99,99,99,99,99,99,99,99,99,99,
      99,99,99,99,99,99,99,99,99,99,99,99,99,99,99),
{15} (99,99,99,99,99,99,99,99,99,99,99,99,99,99,99,
      99,99,99,99,99,99,99,99,99,99,99,99,99,99,99),
{16} (99,99,99,99,99,99,99,99,99,99,99,99,99,99,99,
      99,99,99,99,99,99,99,99,99,99,99,99,99,99,99),
{17} ( 1, 9, 6,13, 7, 3, 5,15, 2,12,14,10, 4,11, 8,
      16,99,99,99,99,99,99,99,99,99,99,99,99,99,99),
{18} (99,99,99,99,99,99,99,99,99,99,99,99,99,99,99,
      99,99,99,99,99,99,99,99,99,99,99,99,99,99,99),
{19} ( 1,10,13, 5, 4,16,11,12,17, 2, 7, 8, 3,15,14,
       6, 9,18,99,99,99,99,99,99,99,99,99,99,99,99),
{20} (99,99,99,99,99,99,99,99,99,99,99,99,99,99,99,
      99,99,99,99,99,99,99,99,99,99,99,99,99,99,99),
{21} (99,99,99,99,99,99,99,99,99,99,99,99,99,99,99,
      99,99,99,99,99,99,99,99,99,99,99,99,99,99,99),
{22} (99,99,99,99,99,99,99,99,99,99,99,99,99,99,99,
      99,99,99,99,99,99,99,99,99,99,99,99,99,99,99),
{23} ( 1,12, 8, 6,14, 4,10, 3,18, 7,21, 2,16, 5,20,
      13,19, 9,17,15,11,22,99,99,99,99,99,99,99,99));

type Matrix=array[1..NMax, 1..NMax, 0..PMax] of Integer;
       { S-a inclus si valoarea PMax, care nu poate fi
         atinsa, pentru a se depozita "deseurile" }
     Vector=array[1..NMax] of Integer;
var A:Matrix;       { Matricea de calcul }
    V:Vector;       { Numerele }
    N,P,D:Integer;

procedure ReadData;
var i:Integer;
begin
  Assign(Input,'input.txt');Reset(Input);
  ReadLn(P,N,D);
  for i:=1 to N do Read(V[i]);
  Close(Input);
end;

function Expr(X,Y,Op:Integer):Integer;
{ Calculeaza expresia (X Op Y) unde Op=1 ('+'),
  Op=2 ('-'), Op=3 ('*'), Op=4 ('/'). Daca Op=4
  si Y=0 se returneaza valoarea P (care nu poate
  fi atinsa prin alte operatii corecte). }
begin
  case Op of
    1:Expr:=(X+Y) mod P;
    2:Expr:=(X+P-Y) mod P;
    3:Expr:=(X*Y) mod P;
    4:if Y=0 then Expr:=P  { = imposibil }
             else Expr:=(X*Invers[P,Y]) mod P
      { S-a creat o tabela predefinita de inversi,
        deoarece altfel impartirea se efectua numai
        in O(P) }
  end; {case}
end;

procedure Combine(i,j,k:Integer);
{ Urmeaza a se combina toate valorile posibile
  pentru A[i,k] si A[k+1,j] pentru a se afla
  toate valorile posibile pentru A[i,j] }
var p1,p2,Op:Integer;
begin
  for p1:=0 to P-1 do
    if A[i,k,p1]<>NoWay
      then for p2:=0 to P-1 do
             if A[k+1,j,p2]<>NoWay
               then { Am gasit doua valori posibile
                      si aplicam cei patru operatori }
                    for Op:=1 to 4 do
                      A[i,j,Expr(p1,p2,Op)]:=k;
end;

procedure ComposeMatrix;
var i,j,k,l:Integer;
begin
  { Initializarea matricei }
  for i:=1 to N do
    for j:=1 to N do
      for k:=0 to P-1 do A[i,j,P]:=NoWay;

  for i:=1 to N do
    A[i,i,V[i]]:=1; { sau orice <> NoWay }

  for l:=2 to N do { Lungimea intervalelor }
    for i:=1 to N-l+1 do
      begin
        j:=i+l-1; { S-au fixat [i,j] capetele intervalului }
        for k:=i to j-1 do { Se alege locul de impartire }
          Combine(i,j,k);
      end;
end;

procedure SeekValues(Lo,Hi,Mid,Value:Integer;
                     var v1,v2:Integer; var Op:Char);
{ Se stie unde e "sparta" expresia in doua; se cauta
  valorile care trebuie obtinute pentru partea stanga,
  respectiv dreapta, si pentru operator }
var i,j,k:Integer;
begin
  for i:=0 to P-1 do
    if A[Lo,Mid,i]<>NoWay
      then for j:=0 to P-1 do
             if A[Mid+1,Hi,j]<>NoWay
               then for k:=1 to 4 do
                      if Expr(i,j,k)=Value
                        then begin
                               v1:=i;
                               v2:=j;
                               Op:=OpNames[k];
                               Exit;
                             end;
end;

procedure WriteExpression(Lo,Hi,Value:Integer);
var v1,v2,Place:Integer;
    Op:Char;
begin
  if Lo=Hi
    then Write(V[Lo])
    else begin
           Place:=A[Lo,Hi,Value];
           SeekValues(Lo,Hi,Place,Value,v1,v2,Op);
           Write('(');
           WriteExpression(Lo,Place,v1);
           Write(Op);
           WriteExpression(Place+1,Hi,v2);
           Write(')');
         end;
end;

procedure WriteSolution;
begin
  Assign(Output,'output.txt');Rewrite(Output);
  if A[1,N,D]=NoWay
    then Write('Nu exista solutie')
    else WriteExpression(1,N,D);
  WriteLn;
  Close(Output);
end;

begin
  ReadData;
  ComposeMatrix;
  WriteSolution;
end.
\end{minted}

O posibilă îmbunătățire a programului de sus ar fi să reținem în $A[i,j,r]$ nu
numai locul unde se face secționarea expresiei, ci și operatorul introdus și
eventual și valorile care trebuie obținute pe partea stângă, respectiv
dreaptă. Totuși, volumul de date ar fi crescut corespunzător și ar fi devenit
greu de manipulat. În schimb, în versiunea prezentă, programul merge puțin mai
lent, dar nesesizabil. Să vedem de ce. Complexitatea reconstituirii expresiei
în sine se află astfel: avem de reconstituit $O(N)$ operatori. Pentru fiecare
din ei, trebuie să căutăm valorile stângă și dreaptă și operatorul în sine,
deci $O(4 \times P^2)$. Complexitatea totală a reconstituirii datelor este
$O(N \times P^2)$, adică oricum mult mai mică față de cea a compunerii
matricei. Este preferabil să nu complicăm structurile de date și codul scris,
mai ales că diferența ca timp de rulare este infimă.

Propunem ca temă cititorului o versiune a acestei probleme, în care nu se va
mai lucra în inelul $\mathbb{Z}_P$, ci într-un grup cu elementele $a, b, c, d,
\dots,$ pentru care se cunoaște tabela de compoziție. În acest caz avem un
singur operator $\oplus$, iar cerința este să se parantezeze expresia $x_1
\oplus x_2 \oplus \cdots \oplus x_k$ astfel încât rezultatul să fie $y$.
