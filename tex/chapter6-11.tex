\section{Problema 11}

Problema celui mai lung prefix a fost dată la a VIII-a Olimpiadă
Internațională de Informatică, Veszprem 1996. Iată enunțul nemodificat al
problemei:

{\bf ENUNȚ}: Structura unor compuși biologici este reprezentată prin
succesiunea con\-sti\-tu\-en\-ți\-lor lor. Acești constituenți sunt notați cu
litere mari. Biologii sunt interesați să descompună o secvență lungă în altele
mai scurte, numite primitive. Spunem că o secvență $S$ poate fi compusă
dintr-un set de primitive $P$ dacă există $N$ primitive $p_1, \dots, p_N$ în
$P$ astfel încât concatenarea $p_1 p_2 \dots p_N$ a primitivelor să fie egală
cu $S$. Aceeași primitivă poate interveni de mai multe ori în concatenare și
nu trebuie neapărat ca toate primitivele să fie prezente.

Primele $M$ caractere din $S$ se numesc prefixul lui $S$ de lungime
$M$. Scrieți un program care primește la intrare un set de primitive $P$ și o
secvență de constituenți $T$. Programul trebuie sa afle lungimea celui mai
lung prefix al lui $T$ care se poate compune din primitive din $P$.

{\bf Datele de intrare} apar în două fișiere. Fișierul {\tt INPUT.TXT} descrie
setul de primitive $P$, iar fișierul {\tt DATA.TXT} conține secvența de
examinat. Pe prima linie din {\tt INPUT.TXT} se află $N$, numărul de primitive
din $P$ ($1 \leq N \leq 100$). Fiecare primitivă se dă pe două linii
consecutive: pe prima lungimea $L$ a primitivei ($1 \leq L \leq 20$), iar pe a
doua un șir de litere mari de lungime $L$. Toate cele $N$ primitive sunt
distincte.

Fiecare linie din fișierul {\tt DATA.TXT} conține o literă mare pe prima
poziție. El se termină cu o linie conținând un punct („.”). Lungimea secvenței
este cuprinsă între 1 si 500.000.

{\bf Ieșirea}: Pe ecran se va tipări lungimea celui mai lung prefix din $T$
care poate fi compus din primitive din $P$.

{\bf Exemplu}:

\texttt{
  \begin{tabular}{|l|l|}
    \hline
        {\bf INPUT.TXT} & {\bf OUTPUT.TXT} \\ \hline
        5   & A \\
        1   & B \\
        A   & A \\
        2   & B \\
        AB  & A \\
        3   & C \\
        BBC & A \\
        2   & B \\
        CA  & A \\
        2   & A \\
        BA  & B \\
        \   & C \\
        \   & B \\
        \   & . \\
        \hline
  \end{tabular}
}

Pe ecran se va tipări numărul 11.

{\bf Timp de rulare}: 30 secunde pentru un test.

{\bf Timp de implementare}: 1h.

{\bf Complexitate cerută}: $O(S \times L \times N)$, unde $S$ este lungimea
secvenței.

{\bf REZOLVARE}: Menționăm de la început că datele problemei sunt
supradimensionate. Nici unul din cele zece teste cu care au fost verificate
programele nu a depășit în realitate 12 primitive. În schimb, fișierul {\tt
  DATA.TXT} a fost unic pentru toate testele și a conținut o secvență de
lungime 500.000.

Problema se rezolvă și în acest caz prin reducerea ei la una similară, dar cu
date de intrare mai mici. Respectiv, un prefix $S$ al lui $T$ se poate
descompune în primitive dacă există o primitivă $p_i$ astfel încât $S=S'+p_i$
și $S'$ se poate descompune în primitive. Am redus împărțirea în primitive a
lui $S$ la despărțirea în primitive a lui $S'$, care are o lungime mai mică
decât $S$. O primă modalitate, pur teoretică, de a rezolva problema, este să
reținem toate prefixele lui $T$ care se pot descompune în primitive; în felul
acesta, putem studia prefixe din ce în ce mai lungi, bazându-ne pe prefixe mai
scurte deja studiate. Totuși, este imposibil să ținem minte toate prefixele
lui $T$, deoarece lungimea medie a unui prefix poate atinge 250.000 caractere.

O îmbunătățire care poate fi adusă acestui algoritm este următoarea: deoarece
toate prefixele aparțin aceleiași secvențe de constituenți, $T$, este
suficient să reținem în întregime secvența $T$ și, pentru fiecare prefix ce se
poate descompune în primitive, păstrăm doar lungimea sa. Făcând abstracție de
limitările de memorie, putem crea un vector $V$ cu $S$ variabile booleene
(unde $S \leq 500.000$), iar $V[i]$ va indica dacă subșirul de lungime $i$ din
$T$ se poate descompune în primitive. $V[i]$ va primi valoarea {\bf True} dacă
și numai dacă există o primitivă $p$ în $P$ de lungime $L$ astfel încât să fie
îndeplinite simultan condițiile:

\begin{itemize}

\item $V[i-L] = \mathbf{True}$;

\item secvența de caractere {\tt T[i-L+1]T[i-L+2]$\dots$T[i]} este egală cu
  $p$.

\end{itemize}

Iată o primă variantă (în pseudocod) a algoritmului:

\vspace{\algskip}
\begin{algorithmic}[1]
  \REQUIRE $T$, primitivele
  \FOR{$i = 1$ la $S$}
  \IF{există o primitivă $p$ astfel încât $V[i-L]$ și $T[i-L+1]T[i-L+2] \dots T[i] =p$}
  \STATE $V[i] \leftarrow \mathbf{True}$
  \ELSE
  \STATE $V[i] \leftarrow \mathbf{False}$
  \ENDIF
  \ENDFOR
  \STATE caută cel mai mare $i$ pentru care $V[i] = \mathbf{True}$
  \PRINT $i$
\end{algorithmic}

Chiar și în acest caz, apare o problemă, deoarece avem nevoie de doi vectori,
unul de caractere și altul de variabile booleene, ambii de lungime maxim
500.000. Necesarul de memorie este deci cam de 1MB. Sigur, pentru
calculatoarele de astăzi această sumă este ușor de alocat, dar problema admite
oricum o soluție la fel de rapidă și mult mai economică. Iată care este
principiul:

Pentru a vedea dacă un prefix $S$ de lungime $L$ se poate descompune în
primitive, noi avem nevoie să cunoaștem dacă prefixele de lungime mai mică se
pot descompune. Dar avem oare nevoie de toate prefixele? Nu, deoarece noi vom
concatena unul din prefixele de lungime mai mică cu o primitivă pentru a
obține noul prefix $S$. Însă primitivele au lungime de maxim 20
caractere. Așadar, noi nu trebuie să cunoaștem decât dacă prefixele de lungime
$L-1, L-2, \dots, L-20$ se pot descompune; restul nu ne interesează. În felul
acesta am eliminat vectorul $V$ și l-am redus la un vector de numai 20 de
elemente (care în program se numește {\tt CanGet}). La fiecare moment, când se
prelucrează un nou caracter din secvența de constituenți $T$, primul element
din {\tt CanGet} se pierde (deoarece informația pe care el o stochează este
învechită), iar următoarele 19 elemente se deplasează spre stânga cu câte o
poziție. Al 20-lea element, care acum a rămas disponibil, va fi calculat la
pasul curent.

O altă modificare pornește de la observația că, datorită aceleiași limitări a
lungimii primitivelor la 20 de caractere, nu avem nevoie nici măcar să reținem
întregul vector $T$, ci numai ultimele 20 de litere ale lui. La fiecare pas,
litera cea mai „veche” din $T$ (adică de indice minim) se va pierde, iar la
celelalte 19 litere se va adăuga litera nou citită. Avem așadar nevoie doar de
un string de 20 de caractere, pe care în program l-am numit {\tt Last}. Pentru
a deplasa spre stânga vectorii {\tt CanGet} și {\tt Last}, se pot folosi fie
atribuirile succesive, fie rutinele de acces direct la memorie.

Deoarece prefixele care se pot descompune sunt identificate în ordinea
crescătoare a lungimii, ultimul asemenea prefix găsit este tocmai cel de
lungime maximă. Dar pentru că, în momentul în care găsim un prefix, nu putem
ști dinainte că el este ultimul, trebuie să reținem într-o variabilă lungimea
celui mai lung prefix găsit până la momentul respectiv, variabilă pe care o
actualizăm de fiecare dată când găsim un nou prefix.

În felul acesta, am reușit să reducem memoria folosită aproape la strictul
necesar, adică numai la dicționarul de primitive și la doi vectori de câte
douăzeci de caractere. Se recomandă totuși să se aloce un buffer cât mai mare
pentru citirea datelor din fișierul {\tt DATA.TXT} pentru mărirea vitezei de
citire. Repartizarea buffer-ului se face cu procedura Pascal {\tt SetTextBuf}.

O optimizare care nu a fost inclusă în program, fiind lăsată ca temă
cititorului, este următoarea: dacă la un moment dat, în timpul examinării
secvenței de constituenți, este întâlnit un șir de cel puțin 20 de prefixe
consecutive, din care nici unul nu se poate descompune în primitive, atunci
nici mai departe nu vom mai întâlni vreun prefix care să se poată
descompune. Explicați de ce. Această optimizare poate să nu aducă uneori nimic
nou în evoluția programului, dar alteori poate să reducă la zero timpul de
rulare.

Prezentăm mai jos codul sursă al programului. A fost preferat limbajul Pascal,
deoarece pune la dispoziție rutine mai comode de manevrare a șirurilor de
caractere.

\inputminted{pascal}{src/problem11.pas}
