\chapter{Lucrul cu numere mari}

De multe ori, în probleme, apar situații când este nevoie să memorăm numere
întregi foarte mari (de ordinul sutelor de cifre), iar uneori trebuie să
efectuăm și operații aritmetice cu aceste numere. Iată un asemenea exemplu:

{\bf ENUNȚ}: Se dă un număr natural cu $N \leq 1000$ cifre. Se cere să se
extragă rădăcina cubică a numărului. Se garantează că numărul citit este cub
perfect.

{\bf Intrarea}: Fișierul {\tt INPUT.TXT} conține un singur rând, terminat cu
{\tt EOF}, pe care se află numărul, cifrele fiind nedespărțite.

{\bf Ieșirea}: În fișierul {\tt OUTPUT.TXT} se va tipări rădăcina cubică a
numărului, pe o singură linie terminată cu {\tt EOF}.

{\bf Exemplu:}

\texttt{
  \begin{tabular}{| l | l |}
    \hline
    {\bf INPUT.TXT} & {\bf OUTPUT.TXT} \\ \hline
    2097152 & 128 \\
    \hline
  \end{tabular}
}

{\bf Timp de implementare}: 1h 30’.

{\bf Timp de rulare}: 10 secunde.

{\bf Complexitate cerută}: $O(N^2)$.

{\bf REZOLVARE}: Problema este cât se poate de simplă din punct de vedere
matematic; dificultatea apare la implementare, atât datorită structurilor de
date necesare, cât mai ales datorită complexității cerute. Despre lucrul cu
numere întregi (chiar și {\tt Longint}) nici nu poate fi vorba, iar la lucrul
cu numere reale apar erori de calcul.

Structura de date propusă pentru abordarea acestui gen de probleme este
următoarea: numerele vor fi reprezentate printr-un vector de cifre
zecimale. Prima poziție (poziția 0) din vector este rezervată pentru a memora
numărul de cifre. Definiția C a tipului de date este:

\begin{minted}{c}
typedef int Huge[1001];
\end{minted}

Iată cum s-ar memora numărul „1997” într-un asemenea vector:

\centeredTikzFigure[
  frame/.style = {rectangle, draw=black, line width=0.2pt, minimum size=20pt},
  label/.style = {rectangle, minimum size=20pt, font=\scriptsize},
]{
  \node (L1) {H};
  \node[frame] (r1)  at ([xshift=1em] L1.east) {4};
  \node[frame, anchor=west] (r2) at (r1.east) {7};
  \node[frame, anchor=west] (r3) at (r2.east) {9};
  \node[frame, anchor=west] (r4) at (r3.east) {9};
  \node[frame, anchor=west] (r5) at (r4.east) {1};

  \node[label, anchor=south] at (r1.north) {H[0]};
  \node[label, anchor=south] at (r2.north) {H[1]};
  \node[label, anchor=south] at (r3.north) {H[2]};
  \node[label, anchor=south] at (r4.north) {H[3]};
  \node[label, anchor=south] at (r5.north) {H[4]};
}  

Se observă că vectorul este oarecum „întors pe dos”. Totuși, această formă
este cea mai convenabilă, pentru că ea permite implementarea cu o mai mare
ușurință a operațiilor aritmetice.

Mai trebuie remarcat că pe fiecare poziție $K$ în vector se află cifra care îl
reprezintă pe $10^{K-1}$ în numărul reprezentat: în $H[1]$ se află cifra
unităților ($10^0$), în $H[2]$ se află cifra zecilor ($10^1$), în $H[3]$ se
află cifra sutelor ($10^2$) ș.a.m.d. Formatul zecimal nu este cea mai fericită
(a se citi „eficientă”) alegere. El ocupă doar patru biți din cei opt ai unui
octet, deci face risipă de memorie. Dacă am alege baza de numerație 256, am
folosi la maximum memoria, și în plus operațiile ar fi cu mult mai rapide
(deoarece 256 este o putere a lui 2, înmulțirile și împărțirile la 256 sunt
simple deplasări la stânga și la dreapta ale reprezentărilor binare). Să facem
următorul calcul ca să vedem câte cifre are în baza 256 un număr care are 1000
de cifre în baza 10:

\begin{equation}
  10^{1000} =
  10 \cdot (10^3)^{333} \approx
  10 \cdot (2^{10})^{333} \approx
  10 \cdot 2^2 \cdot (2^8)^{416} =
  40 \cdot 256^{416}
\end{equation}
	
Așadar, numărul de cifre s-a redus la sub jumătate. Un algoritm liniar ar
funcționa de două ori mai repede pe reprezentări în baza 256, iar unul
pătratic ar funcționa de patru ori mai repede. Inconvenientul major este
dificultatea depanării unui program care operează într-o bază aritmetică atât
de străină nouă. Vom rămâne deci la baza 10, cu mențiunea că acei mai temerari
dintre voi pot încerca folosirea bazei 256.

Numerele reprezentate pe vectori le vom numi pur și simplu „vectori”, iar
numerele reprezentate printr-un tip ordinal de date le vom numi, printr-o
analogie ușor forțată cu matematica, „scalari”. Să vedem acum cum se
efectuează operațiile elementare pe aceste numere.

\section{Inițializarea}

Un vector poate fi inițializat în trei feluri: cu 0, cu un scalar sau cu un
alt vector.

La inițializarea cu 0, singurul lucru pe care îl avem de făcut este să setăm
numărul de cifre pe 0. De aceea, este practic inutil să implementăm această
funcție ca atare; putem folosi în loc singura instrucțiune pe care ea o
conține.

\begin{minted}{c}
void Atrib0(Huge H)
{ H[0]=0; }
\end{minted}

La inițializarea cu un scalar nenul, trebuie să așezăm fiecare cifră pe
poziția corespunzătoare, aflând în paralel și numărul de cifre. Se începe cu
cifra unităților, și la fiecare pas se pune în vector cifra cea mai puțin
semnificativă, după care numărul de reprezentat se împarte la 10
(neglijându-se restul), iar numărul de cifre se incrementează.

\begin{minted}{c}
void AtribValue(Huge H, unsigned long X)
{ H[0]=0;
  while (X)
    { H[++H[0]]=X%10;
      X/=10;
    }
}
\end{minted}

Iată, de exemplu, cum se pune pe vector numărul 195:

\centeredTikzFigure[
  col1/.style = {rectangle, minimum height=2em, text width=5em},
  col2/.style = {rectangle, minimum height=2em, text width=7em},
  frame/.style = {rectangle, draw=black, line width=0.2pt, minimum size=20pt},
]{
  % row 0
  \node[col1] (c01) {};
  \node[col2, anchor=west] (c02) at (c01.east) {};
  \node[frame] (c03)  at ([xshift=2em] c02.east) {0};
  \node[frame, anchor=west] (c04) at (c03.east) {0};
  \node[frame, anchor=west] (c05) at (c04.east) {0};
  \node[frame, anchor=west] (c06) at (c05.east) {0};

  % row 1
  \node[col1, anchor=north] (c11) at (c01.south) {$X = 195$};
  \node[col2, anchor=west] (c12) at (c11.east) {$X\,\bmod\,10 = 5$};
  \node[frame] (c13)  at ([xshift=2em] c12.east) {1};
  \node[frame, anchor=west] (c14) at (c13.east) {5};
  \node[frame, anchor=west] (c15) at (c14.east) {0};
  \node[frame, anchor=west] (c16) at (c15.east) {0};
  \node[anchor=west] (c17)  at ([xshift=2em] c16.east) {$X / 10 = 19$};

  % row 2
  \node[col1, anchor=north] (c21) at (c11.south) {$X = 19$};
  \node[col2, anchor=west] (c22) at (c21.east) {$X\,\bmod\,10 = 9$};
  \node[frame] (c23)  at ([xshift=2em] c22.east) {2};
  \node[frame, anchor=west] (c24) at (c23.east) {5};
  \node[frame, anchor=west] (c25) at (c24.east) {9};
  \node[frame, anchor=west] (c26) at (c25.east) {0};
  \node[anchor=west] (c27)  at ([xshift=2em] c26.east) {$X / 10 = 1$};

  % row 3
  \node[col1, anchor=north] (c31) at (c21.south) {$X = 1$};
  \node[col2, anchor=west] (c32) at (c31.east) {$X\,\bmod\,10 = 1$};
  \node[frame] (c33)  at ([xshift=2em] c32.east) {3};
  \node[frame, anchor=west] (c34) at (c33.east) {5};
  \node[frame, anchor=west] (c35) at (c34.east) {9};
  \node[frame, anchor=west] (c36) at (c35.east) {1};
  \node[anchor=west] (c37)  at ([xshift=2em] c36.east) {$X / 10 = 0$};

  % top label
  \node[minimum height=2em, anchor=south] (L1) at ([xshift=1em] c04.north) {H};
}  

În sfârșit, inițializarea unui vector cu altul se face printr-o simplă copiere
(se pot folosi cu succes rutine de lucru cu memoria, cum ar fi {\tt FillChar}
în Pascal sau {\tt memmove} în C). Pentru eleganță, poate fi folosită și
atribuirea cifră cu cifră:

\begin{minted}{c}
void AtribHuge(Huge H, Huge X)
{ int i;

  for (i=0;i<=X[0];i++) H[i]=X[i];
}
\end{minted}

\section{Compararea}

Pentru a compara două numere „uriașe”, începem prin a afla numărul de cifre
semnificative (deoarece, în urma anumitor operații pot rezulta zerouri
nesemnificative care „atârnă” totuși la numărul de cifre). Aceasta se face
decrementând numărul de cifre al fiecărui număr atâta timp cât cifra cea mai
semnificativă este 0. După ce ne-am asigurat asupra acestui punct, comparăm
numărul de cifre al celor două cifre. Numărul cu mai multe cifre este cel mai
mare. Dacă ambele numere au același număr de cifre, pornim de la cifra cea mai
semnificativă și comparăm cifrele celor două numere până la prima diferență
întâlnită. În acest moment, numărul a cărui cifră este mai mare este el însuși
mai mare. Dacă toate cifrele numerelor sunt egale două câte două, atunci în
mod evident numerele sunt egale.

După cum se vede, algoritmul seamănă foarte bine cu ceea ce s-a învățat la
matematică prin clasa a II-a (doar că atunci nu ni s-a spus că este vorba de
un „algoritm”). Rutina de mai jos compară două numere uriașe $H_1$ și $H_2$ și
întoarce -1, 0 sau 1, după $H_1$ este mai mic, egal sau mai mare decât $H_2$.

\begin{minted}{c}
int Sgn(Huge H1,Huge H2)
{ int i;

  while (!H1[H1[0]] && H1[0]) H1[0]--;
  while (!H2[H2[0]] && H2[0]) H2[0]--;
  if (H1[0]!=H2[0])
    return H1[0]<H2[0] ? -1 : 1;
  i=H1[0];
  while (H1[i]==H2[i] && i) i--;
  return H1[i]<H2[i] ? -1 : H1[i]==H2[i] ? 0 : 1;
}
\end{minted}

\section{Adunarea a doi vectori}

Fiind dați doi vectori, $A$ cu $M$ cifre și $B$ cu $N$ cifre, adunarea lor se
face în mod obișnuit, ca la aritmetică. Pentru a evita testele de depășire, se
recomandă să se completeze mai întâi vectorul cu mai puține cifre cu zerouri
până la dimensiunea vectorului mai mare. La sfârșit, vectorul sumă va avea
dimensiunea vectorului mai mare dintre $A$ și $B$, sau cu 1 mai mult dacă
apare transport de la cifra cea mai semnificativă. Procedura de mai jos adaugă
numărul $B$ la numărul $A$.

\begin{minted}{c}
void Add(Huge A, Huge B)
/* A <- A+B */
{ int i,T=0;

  if (B[0]>A[0])
    { for (i=A[0]+1;i<=B[0];) A[i++]=0;
      A[0]=B[0];
    }
    else for (i=B[0]+1;i<=A[0];) B[i++]=0;
  for (i=1;i<=A[0];i++)
    { A[i]+=B[i]+T;
      T=A[i]/10;
      A[i]%=10;
    }
  if (T) A[++A[0]]=T;
}
\end{minted}

\section{Scăderea a doi vectori}

Se dau doi vectori $A$ și $B$ și se cere să se calculeze diferența $A - B$. Se
presupune $B \leq A$ (acest lucru se poate testa cu funcția {\tt Sgn}). Pentru
aceasta, se pornește de la cifra unităților și se memorează la fiecare pas
„împrumutul” care trebuie efectuat de la cifra de ordin imediat superior
(împrumutul poate fi doar 0 sau 1). Deoarece $B \leq A$, avem garanția că
pentru a scădea cifra cea mai semnificativă a lui $B$ din cifra cea mai
semnificativă a lui $A$ nu e nevoie de împrumut.

\begin{minted}{c}
void Subtract(Huge A, Huge B)
/* A <- A-B */
{ int i, T=0;

  for (i=B[0]+1;i<=A[0];) B[i++]=0;
  for (i=1;i<=A[0];i++)
    A[i]+= (T=(A[i]-=B[i]+T)<0) ? 10 : 0;
    /* Adica A[i]=A[i]-(B[i]+T);
       if (A[i]<0) T=1; else T=0;
       if (T) A[i]+=10; */
  while (!A[A[0]]) A[0]--;
}
\end{minted}

\section{Înmulțirea și împărțirea cu puteri ale lui 10}

Aceste funcții sunt uneori utile. Ele pot folosi și funcțiile de înmulțire a
unui vector cu un scalar, care vor fi prezentate mai jos, dar se pot face și
prin deplasarea întregului număr spre stânga sau spre dreapta. De exemplu,
înmulțirea unui număr cu 100 presupune deplasarea lui cu două poziții înspre
cifra cea mai semnificativă și adăugarea a două zerouri la coadă. Principalul
avantaj al scrierii unor funcții separate pentru înmulțirea cu 10, 100, ...,
este că se pot folosi rutinele de acces direct al memoriei ({\tt FillChar},
respectiv {\tt memmove}). Iată funcțiile care realizează deplasarea
vectorilor, atât prin mutarea blocurilor de memorie, cât și prin atribuiri
succesive.

\begin{minted}{c}
void Shl(Huge H, int Count)
/* H <- H*10^Count */
{ 
  /* Shifteaza vectorul cu Count pozitii */
  memmove(&H[Count+1],&H[1],sizeof(int)*H[0]);
  /* Umple primele Count pozitii cu 0 */
  memset(&H[1],0,sizeof(int)*Count);
  /* Incrementeaza numarul de cifre */
  H[0]+=Count;
}

void Shl2(Huge H, int Count)
/* H <- H*10^Count */
{ int i;

  /* Shifteaza vectorul cu Count pozitii */
  for (i=H[0];i;i--) H[i+Count]=H[i];
  /* Umple primele Count pozitii cu 0 */
  for (i=1;i<=Count;) H[i++]=0;
  /* Incrementeaza numarul de cifre */
  H[0]+=Count;
}

void Shr(Huge H, int Count)
/* H <- H/10^Count */
{ 
  /* Shifteaza vectorul cu Count pozitii */
  memmove(&H[1],&H[Count+1],sizeof(int)*(H[0]-Count));
  /* Decrementeaza numarul de cifre */
  H[0]-=Count;
}

void Shr2(Huge H, int Count)
/* H <- H/10^Count */
{ int i;

  /* Shifteaza vectorul cu Count pozitii */
  for (i=Count+1;i<=H[0];i++) H[i-Count]=H[i];
  /* Decrementeaza numarul de cifre */
  H[0]-=Count;
}
\end{minted}

\section{Înmulțirea unui vector cu un scalar}

Și această operație este o simplă implementare a modului manual de efectuare a
calculului. La înmulțirea „de mână” a unui număr mare cu unul de o singură
cifră, noi parcurgem deînmulțitul de la sfârșit la început, și pentru fiecare
cifră efectuăm următoarele operații:

\begin{itemize}

\item Înmulțim cifra respectivă cu înmulțitorul;

\item Adăugăm „transportul” de la înmulțirea precedentă;

\item Separăm ultima cifră a rezultatului și o trecem la produs;

\item Celelalte cifre are rezultatului constituie transportul pentru
  următoarea înmulțire;

\item La sfârșitul înmulțirii, dacă există transport, acesta are o singură
  cifră, care se scrie înaintea rezultatului.

\end{itemize}

Exact același procedeu se poate aplica și dacă înmulțitorul are mai mult de o
cifră. Singura deosebire este că transportul poate avea mai multe cifre (poate
fi mai mare ca 9). Din această cauză, la sfârșitul înmulțirii, poate rămâne un
transport de mai multe cifre, care se vor scrie înaintea rezultatului. Iată de
exemplu cum se calculează produsul $312 \times 87$:

\begin{center}
  \begin{tabular}{lll@{\hspace{1in}}llllll}
    & & & & & {\large\bf 3} & {\large\bf 1} & {\large\bf 2} & $\times$ \\
    & & & & & & {\large\bf 8} & {\large\bf 7} & \\ \cline{4-8}

    $87 \times 2$ & $= 174$ & $= 17 \times 10 + 4$ & & & & & {\large\bf 4} & \\
    $87 \times 1 + 17$ & $= 104$ & $= 10 \times 10 + 4$ & & & & {\large\bf 4} & & \\
    $87 \times 3 + 10$ & $= 271$ & $= 27 \times 10 + 1$ & & & {\large\bf 1} & & & \\
    $87 \times 0 + 27$ & $= 27$ & $= 2 \times 10 + 7$ & & {\large\bf 7} & & & & \\
    $87 \times 0 + 2$ & $= 2$ & $= 0 \times 10 + 2$ & {\large\bf 2} & & & & &
  \end{tabular}
\end{center}

Procedura este descrisă mai jos:

\begin{minted}{c}
void Mult(Huge H, unsigned long X)
/* H <- H*X */
{ int i;
  unsigned long T=0;

  for (i=1;i<=H[0];i++)
    { H[i]=H[i]*X+T;
      T=H[i]/10;
      H[i]=H[i]%10;
    }
  while (T) /* Cat timp exista transport */
    { H[++H[0]]=T%10;
      T/=10;
    }
}
\end{minted}

\section{Înmulțirea a doi vectori}

Dacă ambele numere au dimensiuni mari și se reprezintă pe tipul de date {\tt
  Huge}, produsul lor se calculează înmulțind fiecare cifră a deînmulțitului
cu fiecare cifră a înmulțitorului și trecând rezultatul la ordinul de mărime
(exponentul lui 10) cuvenit. De fapt, același lucru îl facem și noi pe
hârtie. Considerând același exemplu, în care ambele numere sunt „uriașe”,
produsul lor se calculează de mână astfel:

\begin{center}
  \begin{tabular}{llllll}
    & & 3 & 1 & 2 & $\times$ \\
    & & & 8 & 7 \\ \cline{1-5}
    & 2 & 1 & 8 & 4 \\
    2 & 4 & 9 & 6 & \\ \cline{1-5}
    2 & 7 & 1 & 4 & 4
  \end{tabular}
\end{center}

S-a luat deci fiecare cifră a înmulțitorului și s-a efectuat produsul parțial
corespunzător, corectând la fiecare pas rezultatul prin calculul
transportului. Rezultatul pentru fiecare produs parțial s-a scris din ce în ce
mai în stânga, pentru a se alinia corect ordinele de mărime. Acest procedeu
este oarecum incomod de implementat. Se pot face însă unele observații care
ușurează mult scrierea codului:

\begin{itemize}

\item Prin înmulțirea cifrei cu ordinul de mărime $10^i$ din primul număr cu
  cifra cu ordinul de mărime $10^j$ din al doilea număr se obține o cifră
  corespunzătoare ordinului de mărime $10^{i+j}$ în rezultat (sau se obține un
  număr cu mai mult de o singură cifră, caz în care transportul merge la cifra
  corespunzătoare ordinului de mărime $10^{i + j + 1}$).
      
\item Dacă numerele au $M$ și respectiv $N$ cifre, atunci produsul lor va avea
  fie $M + N$ fie $M + N - 1$ cifre. Într-adevăr, dacă numărul $A$ are $M$
  cifre, atunci $10^{M - 1} \leq A < 10^M$ și $10^{N - 1} \leq B < 10^N$, de
  unde rezultă $10^{M + N - 2} \leq A \times B < 10^{M + N}$.

\item La calculul produselor parțiale se poate omite calculul transportului,
  acesta urmând a se face la sfârșit. Cu alte cuvinte, într-o primă fază putem
  pur și simplu să înmulțim cifră cu cifră și să adunăm toate produsele de
  aceeași putere, obținând un număr cu „cifre” mai mari ca 9, pe care îl
  aducem la forma normală printr-o singură parcurgere. Să reluăm același
  exemplu:

\end{itemize}

\begin{center}
  \begin{tabular}{l@{\hspace{0.5in}}rrrrrl}
    \multirow{2}{*}{Intrarea: Vectorii $A$ și $B$}
    & & & {\large\bf 3} & {\large\bf 1} & {\large\bf 2} & $\times$ \\
    & & & & {\large\bf 8} & {\large\bf 7} & \\ \cline{2-6}

    \multirow{2}{*}{Etapa I: Efectuarea produselor intermediare}
    & & & 21 & 7 & 14 & \\
    & & 24 & 8 & 16 & & \\ \cline{2-6}

    \multirow{2}{*}{Etapa a II-a: Adunarea produselor intermediare}
    & & 24 & 29 & 23 & 14 & \\
    & & $\overleftarrow{2}$ & $\overleftarrow{3}$ & $\overleftarrow{2}$ & $\overleftarrow{1}$ & \\
    \cline{2-6}

    Etapa a III-a: Corectarea rezultatului
    & {\large\bf 2} & {\large\bf 7} & {\large\bf 1} & {\large\bf 4} & {\large\bf 4} &
  \end{tabular}
\end{center}

Această operație efectuează $M \times N$ produse de cifre și $M + N$ (sau $M +
N - 1$, după caz) „transporturi” pentru aflarea rezultatului, deci are
complexitatea $O(M \times N)$. Iată și implementarea:

\begin{minted}{c}
void MultHuge(Huge A, Huge B, Huge C)
/* C <- A x B */
{ int i,j,T=0;

  C[0]=A[0]+B[0]-1;
  for (i=1;i<=A[0]+B[0];) C[i++]=0;
  for (i=1;i<=A[0];i++)
    for (j=1;j<=B[0];j++)
      C[i+j-1]+=A[i]*B[j];
  for (i=1;i<=C[0];i++)
    { T=(C[i]+=T)/10;
      C[i]%=10;
    }
  if (T) C[++C[0]]=T;
}
\end{minted}

Mai există o altă modalitate de a înmulți două numere de câte $N$ cifre
fiecare, care are complexitatea $O(N^{\log_2 3}) \approx O(N^{1,58}) \approx
O(N \sqrt{N})$. Ea derivă de un algoritm propus de Strassen în 1969 pentru
înmulțirea matricelor. Diferența se face simțită, ce-i drept pentru valori
mari ale lui $N$, dar constanta multiplicativă crește destul de mult și, în
plus, soluția e mai greu de implementat; de aceea nu recomandăm implementarea
ei în timpul concursului. Ideea de bază este să se micșoreze numărul de
înmulțiri și să se mărească numărul de adunări, deoarece adunarea a doi
vectori se face în $O(N)$, pe când înmulțirea se face în $O(N^2)$. Să
considerăm întregii $A$ și $B$, fiecare de câte $N$ cifre. Trebuie să-i
înmulțim într-un timp $T(N)$ mai bun decât $O(N^2)$. Să împărțim numărul $A$
în două „bucăți” $C$ și $D$, fiecare de câte $N$ / 2 cifre, iar întregul $B$
în două bucăți $E$ și $F$, tot de câte $N$ / 2 cifre (presupunem că $N$ este
par):

\centeredTikzFigure[
  frame/.style = {rectangle, draw=black, line width=0.2pt, minimum width=6em},
]{
  % row 0
  \node[minimum width=2em] (a) at (0,0) {\bf $A$};
  \node[frame] (c) at ([xshift=5em] a.east) {$C$};
  \node[frame, anchor=west] (d) at (c.east) {$D$};

  % row 0
  \node[minimum width=2em] (b) at (0,-2em) {\bf $B$};
  \node[frame] (e) at ([xshift=5em] b.east) {$E$};
  \node[frame, anchor=west] (f) at (e.east) {$F$};

  % top arrow
  \draw [<->] (3em,.5) -- node[above]{$N$} (15em,.5);

  % bottom arrows
  \draw [<->] (3em,-1.35) -- node[below]{$N/2$} (9em,-1.35);
  \draw [<->] (9em,-1.35) -- node[below]{$N/2$} (15em,-1.35);
}  

Atunci se poate scrie relația

\begin{align}
  \begin{split}
    A \times B & = (C \times 10^{N / 2} + D) \times (E \times 10^{N / 2} + F) \\
    & = CE \times 10^{N} + (CF + DE) \times 10^{N / 2} + DF
  \end{split}
\end{align}

Pentru a putea calcula produsul $A \times B$ avem, prin urmare, nevoie de
patru produse parțiale, de trei adunări și de două înmulțiri cu puteri ale lui
10. Adunările și înmulțirile cu puteri ale lui 10 se fac în timp liniar. Dacă
efectuăm cele patru produse parțiale prin patru înmulțiri, rezultă formula
recurentă de calcul

\begin{equation}
  T(N) = 4 T(N / 2) + O(N)
\end{equation}

care duce prin eliminarea recurenței la $T(N) \in O(N^2)$. Cu alte cuvinte,
încă n-am câștigat nimic. Trebuie să reușim cumva să reducem numărul de
înmulțiri de la 4 la 3, chiar dacă prin aceasta vom mări numărul de adunări
necesare. Să definim produsul

\begin{align}
  \begin{split}
    G & = (C + D) \times (E + F) \\
    & = CE + CF + DE + DF \\
    & = CE + DF + (CF + DE)
  \end{split}
\end{align}

Atunci putem scrie:

\begin{equation}
  A \times B = CE \times 10^{N} + (G - CE - DF) \times 10^{N / 2} + DF
\end{equation}
	
Pentru această variantă, folosim doar trei înmulțiri, și chiar dacă avem
nevoie de șase adunări și scăderi și două înmulțiri cu puteri ale lui 10,
complexitatea se va reduce la $O(N^{\log_2 3})$. În cazul în care $N$ este o
putere a lui 2, împărțirea în două a numerelor se poate face fără probleme la
fiecare pas, până se ajunge la numere de o singură cifră, care se înmulțesc
direct. În cazul în care $N$ nu este o putere a lui 2, este comod să se
completeze numerele cu zerouri până la o putere a lui 2. În funcțiile descrise
mai jos, {\tt MultRec} nu face decât înmulțirea recursivă, pe când {\tt
  MultHuge2} se ocupă și de corectarea numărului de cifre (incrementarea până
la o putere a lui 2). Pentru calculul produselor $C \times E$ și $D \times F$,
procedura {\tt MultRec} se autoapelează; pentru calcularea produsului $(C+D)
\times (E+F)$, însă, este nevoie să fie apelată procedura {\tt MultHuge2},
deoarece prin cele două adunări poate să apară o creștere a numărului de cifre
al factorilor, care în acest caz trebuie readuși la un număr de cifre egal cu
o putere a lui 2.

\begin{minted}{c}
void MultHuge2(Huge A, Huge B, Huge P);

void MultRec(Huge A, Huge B, Huge P)
{ Huge C,D,E,F,CE,DF;

  if (A[0]==1)
    { P[1]=A[1]*B[1];
      P[0]=(P[2]=P[1]/10)>0 ? 2 : 1;
      P[1]%=10;
    }
    else { P[0]=0;
           AtribHuge(C,A);Shr(C,A[0]/2);
           AtribHuge(D,A);D[0]=A[0]/2;
           AtribHuge(E,B);Shr(E,B[0]/2);
           AtribHuge(F,B);F[0]=B[0]/2;
           MultRec(C,E,CE);MultRec(D,F,DF);
           Add(C,D);Add(E,F);
           MultHuge2(C,E,P);
           Subtract(P,CE);Subtract(P,DF);
           Shl(P,A[0]/2);
           Shl(CE,A[0]);Add(P,CE);
           Add(P,DF);
         }
}

void MultHuge2(Huge A, Huge B, Huge P)
/* P <- A x B, varianta N^(lg 3) */
{ int i,j,NDig=A[0]>B[0] ? A[0] : B[0],Needed=1,SaveA,SaveB;

  /* Corecteaza numarul de cifre */
  while (Needed<NDig) Needed<<=1;
  SaveA=A[0];SaveB=B[0];A[0]=B[0]=Needed;
  for (i=SaveA+1;i<=Needed;) A[i++]=0;
  for (i=SaveB+1;i<=Needed;) B[i++]=0;
  MultRec(A,B,P);

  /* Restaureaza numarul de cifre */
  A[0]=SaveA;B[0]=SaveB;
  while (!P[P[0]] && P[0]>1) P[0]--;
}
\end{minted}

\section{Împărțirea unui vector la un scalar}

Ne propunem să scriem o funcție care să împartă numărul $A$ de tip {\tt Huge}
la scalarul $B$, să rețină valoarea câtului tot în numărul $A$ și să întoarcă
restul (care este o variabilă scalară). Să pornim de la un exemplu particular
și să generalizăm apoi procedeul: să calculăm câtul și restul împărțirii lui
1997 la 7. Cu alte cuvinte, să găsim acele numere $C$ de tip {\tt Huge} și $R
\in \{0, 1, 2, 3, 4, 5, 6\}$ cu proprietatea că $1997 = 7 \times C + R$.

\begin{center}
  \begin{tabular}{
      llll
      llllll}
    1 & 9 & 9 & 7 & & \multicolumn{1}{|l}{7} & & & & \\ \cline{6-9}
    0 & & & & & \multicolumn{1}{|l}{0} & 2 & 8 & 5 & $\leftarrow$ câtul \\ \cline{1-1}
    1 & 9 & & & & & & & & \\
    1 & 4 & & & & & & & & \\ \cline{1-2}
    & 5 & 9 & & & & & & & \\
    & 5 & 6 & & & & & & & \\ \cline{2-3}
    & & 3 & 7 & & & & & & \\
    & & 3 & 5 & & & & & & \\ \cline{3-4}
    & & & 2 & $\leftarrow$ restul & & & & &
  \end{tabular}
\end{center}

La fiecare pas se coboară câte o cifră de la deîmpărțit alături de numărul
deja existent (care inițial este 0), apoi rezultatul se împarte la împărțitor
(7 în cazul nostru). Câtul este întotdeauna o cifră și se va depune la
sfârșitul câtului împărțirii, iar restul va fi folosit pentru următoarea
împărțire. Restul care rămâne după ultima împărțire este tocmai $R$ pe care îl
căutăm. Procedeul funcționează și atunci când deîmpărțitul are mai multe
cifre. La sfârșit trebuie să decrementăm corespunzător numărul de cifre al
câtului, prin neglijarea zerourilor create la începutul numărului. Numărul
maxim de cifre al câtului este egal cu cel al deîmpărțitului.

\begin{minted}{c}
unsigned long Divide(Huge A, unsigned long X)
/* A <- A/X si intoarce A%X */
{ int i;
  unsigned long R=0;

  for (i=A[0];i;i--)
    { A[i]=(R=10*R+A[i])/X;
      R%=X;
    }
  while (!A[A[0]] && A[0]>1) A[0]--;
  return R;
}
\end{minted}

Dacă dorim numai să aflăm restul împărțirii, nu mai avem nevoie decât să
recalculăm restul la fiecare pas, fără a mai modifica vectorul $A$:

\begin{minted}{c}
unsigned long Mod(Huge A, unsigned long X)
/* Intoarce A%X */
{ int i;
  unsigned long R=0;

  for (i=A[0];i;i--)
    R=(10*R+A[i])%X;
  return R;
}
\end{minted}

\section{Împărțirea a doi vectori}

Dacă se dau doi vectori $A$ și $B$ și se cere să se afle câtul $C$ și restul
$R$, etapele de parcurs sunt aceleași ca la punctul precedent, numai că
operatorii „/” și „\%” trebuie implementați de utilizator, ei nefiind definiți
pentru vectori. Cu alte cuvinte, după ce „coborâm” la fiecare pas următoarea
cifră de la deîmpărțit, trebuie să aflăm cea mai mare cifră $X$ astfel încât
împărțitorul să se cuprindă de $X$ ori în restul de la momentul
respectiv. Acest lucru se face cel mai comod prin adunări repetate: pornim cu
cifra $X$ = 0 și o incrementăm, micșorând concomitent restul, până când restul
care rămâne este prea mic. Să efectuăm aceeași împărțire, $1997:7$, considerând
că ambele numere sunt reprezentate pe tipul {\tt Huge}.

\centeredTikzFigure[
  scale=0.8,
  every node/.style={scale=0.8},
  frame/.style = {rectangle, draw=black, line width=0.2pt, minimum width=22em},
  digit/.style = {rectangle, minimum width = 2em, font=\bf\Large}
]{
  % box 1
  \node[frame] (box1) { 
    \begin{tabular}{ll}
      \multicolumn{2}{l}{Restul înainte de coborârea cifrei: 0} \\
      \multicolumn{2}{l}{Restul după coborârea cifrei: $10 \times 0 + 1 = 1$} \\
      $7 \times 1 = 7$ & prea mare \\
      Restul rămas: 1
    \end{tabular}
  };

  % box 2
  \node[frame] (box2) at (0,-4) { 
    \begin{tabular}{ll}
      \multicolumn{2}{l}{Restul înainte de coborârea cifrei: 1} \\
      \multicolumn{2}{l}{Restul după coborârea cifrei: $10 \times 1 + 9 = 19$} \\
      $7 \times 1 = 7$ & $19 - 7 = 12$ \\
      $7 \times \mathbf{2} = 14$ & $12 - 7 = 5$ \\
      $7 \times 3 = 21$ & prea mare \\
      Restul rămas: 5
    \end{tabular}
  };

  % box 3
  \node[frame] (box3) at (10.5,-6.2) { 
    \begin{tabular}{ll}
      \multicolumn{2}{l}{Restul înainte de coborârea cifrei: 5} \\
      \multicolumn{2}{l}{Restul după coborârea cifrei: $10 \times 5 + 9 = 59$} \\
      $7 \times 1 = 7$ & $59 - 7 = 52$ \\
      $\cdots$ \\
      $7 \times \mathbf{8} = 56$ & $10 - 7 = 3$ \\
      $7 \times 9 = 63$ & prea mare \\
      Restul rămas: 3
    \end{tabular}
  };

  % box 4
  \node[frame] (box4) at (10.5,-1) { 
    \begin{tabular}{ll}
      \multicolumn{2}{l}{Restul înainte de coborârea cifrei: 3} \\
      \multicolumn{2}{l}{Restul după coborârea cifrei: $10 \times 3 + 7 = 37$} \\
      $7 \times 1 = 7$ & $37 - 7 = 30$ \\
      $\cdots$ \\
      $7 \times \mathbf{5} = 35$ & $9 - 7 = 2$ \\
      $7 \times 6 = 42$ & prea mare \\
      Restul rămas: 2
    \end{tabular}
  };

  % digits
  \node[digit] (d1) at (4, 3.5) {1};
  \node[digit, anchor=west] (d2) at (d1.east) {9};
  \node[digit, anchor=west] (d3) at (d2.east) {9};
  \node[digit, anchor=west] (d4) at (d3.east) {7};
  \node[digit, anchor=west] (quotient) at (d4.east) {: 7 =};
  \node[digit, anchor=north] (d5) at (d1.south) {0};  
  \node[digit, anchor=west] (d6) at (d5.east) {2};
  \node[digit, anchor=west] (d7) at (d6.east) {8};
  \node[digit, anchor=west] (d8) at (d7.east) {5};
  \node[digit, anchor=west] (remainder) at (d8.east) {rest 2};

  %arrows
  \draw[-] (d5.south) -- (d5 |- box1.north);
  \draw[-] (d6.south) |- (box2.east);
  \draw[-] (d7.south) |- (box3.west);
  \draw[-] (d8.south) -- (d8 |- box4.north);
}

Cazul cel mai defavorabil (când $X = 9$) presupune 9 scăderi și 10 comparații,
cazul cel mai favorabil (când $X = 0$) presupune numai o comparație, deci
cazul mediu presupune 4 scăderi și 5 comparații. Căutarea lui $X$ se poate
face și binar, prin înjumătățirea intervalului, ceea ce reduce timpul mediu de
căutare la aproximativ 3 comparații și trei înmulțiri, dar codul se complică
nejustificat de mult (de cele mai multe ori).

\begin{minted}{c}
void DivideHuge(Huge A, Huge B, Huge C, Huge R)
/* A/B = C rest R */
{ int i;

  R[0]=0;C[0]=A[0];
  for (i=A[0];i;i--)
    { Shl(R,1);R[1]=A[i];
      C[i]=0;
      while (Sgn(B,R)!=1)
        { C[i]++;
          Subtract(R,B);
        }
    }
  while (!C[C[0]] && C[0]>1) C[0]--;
}
\end{minted}

\section{Extragerea rădăcinii cubice}

Vom sări peste prezentarea algoritmului de extragere a rădăcinii pătrate, pe
care îl vom lăsa ca temă cititorului, și ne vom îndrepta atenția asupra celui
de extragere a rădăcinii cubice, care este puțin mai complicat, dar care poate
fi ușor extins pentru rădăcini de orice ordin. Problema este exact cea din
enunț, așa că vom porni de la exemplul dat. Să notăm $A = 2.097.152$ și $X =
\sqrt[3]{A}$. Cum se află $X$?

O primă variantă ar fi căutarea binară a rădăcinii, prin înjumătățirea
intervalului. Inițial se pornește cu intervalul (1,$A$), deoarece rădăcina
cubică se află undeva între 1 și A (evident, încadrarea este mai mult decât
acoperitoare; ea ar putea fi mai limitativă, dar nu ar reduce timpul de lucru
decât cu câteva iterații). La fiecare pas, intervalul va fi înjumătățit. Cum,
probabil că știți deja; se ia jumătatea intervalului, se ridică la puterea a
treia și se compară cu $A$. Dacă este mai mare, înseamnă că rădăcina trebuie
căutată în jumătatea inferioară a intervalului. Dacă este mai mică, vom
continua căutarea în jumătatea superioară a intervalului. Dacă cele două
numere sunt egale, înseamnă că am găsit tocmai ce ne interesa. Prima variantă
a pseudocodului este:

\vspace{\algskip}
\begin{algorithmic}[1]
  \STATE {\bf citește} $A$ cu $N$ cifre
  \STATE $Lo \leftarrow 1, Hi \leftarrow A, X \leftarrow 0$
  \WHILE{$X = 0$}
  \STATE $Mid \leftarrow (Lo+Hi)/2$
  \IF {$Mid3 < A$}
  \STATE $Lo \leftarrow Mid+1$
  \ELSIF{$Mid3 > A$}
  \STATE $Hi \leftarrow Mid-1$
  \ELSE
  \STATE $X \leftarrow Mid$
  \ENDIF
  \ENDWHILE
\end{algorithmic}

În cazul cel mai rău, algoritmul de mai sus efectuează $\log_2 A$ înjumătățiri
de interval, fiecare din ele presupunând o adunare, o împărțire la 2 și o
ridicare la cub. Dintre aceste operații, cea mai costisitoare este ridicarea
la cub, $O(N^2)$. Complexitatea totală este prin urmare $O(N^2 \log_2
A)$. Deoarece $A$ are ordinul de mărime $10^N$, rezultă complexitatea $O(N^3
\log 10)= O(N^3)$, adică mai proastă decât cea cerută (de altfel, un algoritm
cu această complexitate nici nu s-ar încadra în timp pentru $N$ = 1000). Dacă
timpul ne permite, trebuie să căutăm altă metodă.

În exemplul ales, să observăm că $10^6 \leq A < 10^9$, de unde deducem că
$10^2 \leq X < 10^3$. Cu alte cuvinte, $X$ are 3 cifre. În cazul general, dacă
$A$ are $N$ cifre, atunci $X$ are $\lceil N / 3 \rceil$ cifre (prin $\lceil N
/ 3 \rceil$ se înțelege „cel mai mic întreg mai mare sau egal cu $N /
3$”). Care ar putea fi prima cifră a lui $X$ ? Dacă $X$ începe cu cifra 2,
atunci $200 \leq X < 300 \implies 8.000.000 \leq 2.097.152 < 27.000.000$, ceea
ce este fals. Cu atât mai puțin poate prima cifră a lui $X$ să fie mai mare ca
2. Rezultă că prima cifră a lui $X$ este 1. De altfel, pentru a afla acest
lucru, putem să și neglijăm ultimele 6 cifre ale lui $A$. Ne interesează doar
prima cifră, cea a milioanelor, iar prima cifră a lui $X$ o alegem în așa fel
încât cubul ei să fie mai mic sau egal cu 2.

Ce putem spune despre a doua cifră? Dacă ar fi 3, atunci $130 \leq X < 140
\implies 2.197.000 \leq 2.097.152 < 2.744.000$, fals (deci cifra este cel mult
2). Dacă ar fi 1, atunci $110 \leq X < 120 \implies 1.331.000 \leq 2.097.152 <
1.728.000$, fals. Rezultă că a doua cifră este obligatoriu 2. Analog, putem
neglija ultimele trei cifre ale lui $A$, iar a doua cifră a lui $X$ este cel
mai mare $C$ pentru care $\overline{1C}^3 \leq 2097$. Pentru a afla ultima
cifră, aplicăm același raționament: Dacă ar fi 9, atunci $X$ = 129 și ar
rezulta $2.146.688 = X^3 = 2.097.152$, absurd. Dacă considerăm că cifra este
8, atunci calculul se verifică. Am aflat așadar că $X$ = 128.

Procedeul general este următorul: dându-se un număr $A$ cu $N$ cifre, îl
completăm cu zerouri nesemnificative până când $N$ se divide cu 3 (poate fi
necesar să adăugăm maxim două zerouri). Numărul de cifre semnificative ale
rădăcinii cubice este $M = N / 3$. Aflăm pe rând fiecare cifră, începând cu
cea mai semnificativă. Să presupunem că am aflat cifrele $X_{M}, X_{M-1},
\dots, X_{K+1}$. Cifra $X_K$ este cea mai mare cifră pentru care numărul
$\overline{X_{M}X_{M-1}{\dots}X_{K+1}X_{K}00{\dots}00}^3 \leq A$. Cifra $X_K$
este unică, deoarece există, în general, mai multe cifre care verifică
proprietatea cerută, dar una singură este „cea mai mare”. O a doua versiune a
pseudocodului este deci:

\vspace{\algskip}
\begin{algorithmic}[1]
\STATE {\bf citește} $A$ cu $N$ cifre
\STATE $X \leftarrow 0, T \leftarrow 0$
\WHILE{$N \bmod 3 \neq 0$}
\STATE adaugă un 0 nesemnificativ
\ENDWHILE
\FOR {$i = 1$ \TO $N/3$}
\STATE adaugă la $T$ următoarele 3 cifre din $A$
\STATE adaugă la $X$ cea mai mare cifră astfel încât $X^3 \leq T$
\ENDFOR
\end{algorithmic}

Să evaluăm complexitatea acestei versiuni. Linia 1 se execută în timp liniar,
$O(N)$. Liniile 2-5 se execută în timp constant. Linia 7 presupune adăugarea
unei cifre ($O(N)$), iar linia 8 un număr constant de ridicări la cub, așadar
înmulțiri ($O(N^2)$). Deoarece liniile 7 și 8 se execută de $N/3$ ori (linia
6), rezultă o complexitate de $O(N^3)$. Iată că nici acest algoritm nu a adus
îmbunătățiri și pare și ceva mai greu de implementat. El poate fi totuși
modificat pentru a-i scădea complexitatea la $O(N^2)$.

Principalul neajuns al său este efectuarea ridicării la cub, care se face în
$O(N^2)$. Dacă am putea să-l aflăm la fiecare pas pe $X^3$ fără a efectua
înmulțiri, adică în timp liniar, atunci întregul algoritm ar avea complexitate
pătratică. Bineînțeles, prima întrebare care vine pe buzele cititorului este
„cum să ridicăm la cub fără să facem înmulțiri?”. Să nu uităm însă ceva: că
noi nu-l cunoaștem numai pe $X$. Îl cunoaștem și pe $X$ de la pasul anterior,
care avea o cifră mai puțin (îl vom boteza $OldX$). Să presupunem că, printr-o
metodă oarecare, am reușit să-l ridicăm pe $OldX$ la puterile a doua și a
treia (și am obținut numerele $OldX2$ și $OldX3$). Cum putem, cunoscând aceste
trei numere, precum și noua cifră ce se va adăuga la sfârșitul lui $X$ (să-i
spunem $C$), să-l aflăm pe $X$, pătratul și cubul său? Nu e prea greu:

\begin{equation}
  X = 10 \cdot OldX + C
\end{equation}

\begin{align}
  \begin{split}
    X^2 & = (10 \cdot OldX + C)^2 = 100 \cdot OldX^2 + 20 \cdot C \cdot OldX + C^2 \\
    & = 100 \cdot OldX^2 + (20 \cdot C) \cdot OldX + C^2
  \end{split}
\end{align}

\begin{align}
  \begin{split}
    X^3 & = (10 \cdot OldX + C)^3 = 1000 \cdot OldX^3 + 300 \cdot C \cdot OldX^2 + 30 \cdot C^2 \cdot OldX + C^3 \\
    & = 1000 \cdot OldX^3 + (300 \cdot C) \cdot OldX^2 + (30 \cdot C^2) \cdot OldX + C^3
  \end{split}
\end{align}

Iată așadar că pentru a afla noile valori ale puterilor 1, 2 și 3 ale lui $X$,
folosindu-le pe cele vechi, nu avem nevoie decât de adunări și de înmulțiri cu
numere mici (de ordinul miilor). Toate aceste operații se fac în timp liniar,
deci am reușit să găsim un algoritm pătratic. Iată mai jos sursa C:

\begin{minted}{c}
void FindDigit(Huge L,Huge NewL2,Huge NewL3,
     Huge OldL,Huge OldL2,Huge OldL3,Huge Target)
{ Huge Aux;

  L[1]=10;
  do
    { L[1]--;
      /* Trebuie calculat L^3. Se stiu OldL (L/10)
         si noua cifra L[1]. Deci (OldL*10+L[1])^3=?
         Se aplica binomul lui Newton. */
      AtribHuge(NewL3,OldL3);Shl(NewL3,3);
      AtribHuge(Aux,OldL2);Mult(Aux,300*L[1]);
      Add(NewL3,Aux);
      AtribHuge(Aux,OldL);Mult(Aux,30*L[1]*L[1]);
      Add(NewL3,Aux);
      AtribValue(Aux,L[1]*L[1]*L[1]);
      Add(NewL3,Aux);
    }
  while (Sgn(NewL3,Target)==1);
  /* Aceeasi operatie pentru L^2 */
  AtribHuge(NewL2,OldL2);Shl(NewL2,2);
  AtribHuge(Aux,OldL);Mult(Aux,20*L[1]);
  Add(NewL2,Aux);
  AtribValue(Aux,L[1]*L[1]);
  Add(NewL2,Aux);
  /* Noile valori devin 'vechi' */
  AtribHuge(OldL2,NewL2);
  AtribHuge(OldL,L);
  AtribHuge(OldL3,NewL3);
}

void CubeRoot(Huge A, Huge X)
{ Huge Target,OldX,OldX2,OldX3,NewX2,NewX3;
  int i;

  /* Se initializeaza vectorii cu 0 (nici o cifra) */
  OldX[0]=OldX2[0]=OldX3[0]=X[0]=0;
  for (i=1;i<=(A[0]+2)/3;i++)
    { AtribHuge(Target,A);
      Shr(Target,3*((A[0]+2)/3-i));
      Shl(X,1);
      FindDigit(X,NewX2,NewX3,OldX,OldX2,OldX3,Target);
    }
}
\end{minted}

Acum nu mai avem decât să scriem rutinele de intrare/ieșire și programul
principal:

\begin{minted}{c}
#include <stdio.h>
#include <mem.h>
#define NMax 1000
typedef int Huge[NMax+3];
Huge A,X; /* A[0] si X[0] indica numarul de cifre */

void ReadData(void)
{ FILE *F=fopen("input.txt","rt");
  int C,i;

  A[0]=0;
  do A[++A[0]]=(C=fgetc(F))-'0';
  while (C!=EOF);
  A[0]--;
  fclose(F);
  /* Intoarce vectorul pe dos */
  for (i=1;i<=A[0]/2;i++)
    A[i]=(A[i]^A[A[0]+1-i])^(A[A[0]+1-i]=A[i]);
}

void WriteSolution(void)
{ FILE *F=fopen("output.txt","wt");
  int i=X[0];

  while (!X[i]) i--;
  while (i) fputc(X[i--]+'0',F);
  fclose(F);
}

void main(void)
{
  ReadData();
  CubeRoot(A,X);
  WriteSolution();
}
\end{minted}

Pentru a extinde această metodă la rădăcini de orice ordin $K$, trebuie numai
să ținem cont de expresia binomului lui Newton:

\begin{align}
  \begin{split}
    X^p & = (10 \cdot OldX + C)^p \\
    & = \sum_{i=0}^{p} \mathbf{C}_p^i \cdot 10^i \cdot OldX^i \cdot C^{p-i}
  \end{split}
\end{align}

Presupunând că avem calculate toate puterile de la 1 la $p$ ale lui $OldX$, se
poate calcula noua valoare a lui $X^p$ folosind numai adunări și înmulțiri cu
scalari. În felul acesta se pot calcula în timp liniar valorile lui $X, X^2,
X^3, \dots, X^K.$
