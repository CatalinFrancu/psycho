\section{Problema 18}

{\bf ENUNȚ}: Se dă un vector nesortat cu elemente numere reale
oarecare. Considerând că vectorul ar fi sortat, se cere să se găsească
distanța maximă între două elemente consecutive ale sale, fără însă a sorta
efectiv vectorul.

{\bf Intrarea}: Fișierul {\tt INPUT.TXT} conține pe prima linie numărul $N$ de
elemente din vector ($N \leq 5.000$). Pe următoare linie se dau numerele
separate prin spații.

{\bf Ieșirea}: Pe ecran se va tipări un mesaj de forma:

Distanța maximă este $D$

{\bf Exemplu}: Pentru fișierul de intrare cu conținutul

\begin{verbatim}
  4
  5 3.2 2 3.7
\end{verbatim}

răspunsul trebuie să fie

\begin{verbatim}
  Distanta maxima este 1.3
\end{verbatim}

{\bf Timp de implementare}: 30 minute.

{\bf Timp de rulare}: 2-3 secunde.

{\bf Complexitate cerută}: $O(N)$.

{\bf REZOLVARE}: Desigur că primul lucru la care ne gândim este să sortăm
vectorul și să îl parcurgem apoi de la stânga la dreapta, căutând distanța
maximă între două elemente consecutive. Complexitatea unui asemenea algoritm
este $O(N \log N)$. Nici această soluție nu este rea, iar la un concurs,
comisiei de corectare i-ar veni destul de greu să găsească teste care să
departajeze un algoritm în $O(N \log N)$ de unul liniar, chiar și pentru $N =
5.000$. Totuși, vom arăta care este algoritmul liniar; în primul rând de
dragul „artei”, iar în al doilea rând pentru că nu este cu mult mai greu de
implementat decât o sortare.

Primul lucru care trebuie făcut este găsirea maximului și a minimului din
vector; să notăm aceste valori cu $V_{max}$ și $V_{min}$. Aceste operații se
fac în timp liniar, eventual chiar la citirea datelor din fișier. Apoi se
împarte intervalul $[V_{min}, V_{max}]$ de pe axa reală în $N-1$ intervale
egale. Iată cazul exemplului din enunț, unde $V_{min} = 2$ și $V_{max} = 5$:

\tikzset{
  axisLabel/.style = {
    font=\bf,
    text depth=0,
  },
}
\newcommand\axisNumber[3] {
  \draw[ultra thick] (#1,0.2) -- (#1,-0.2);
  \node[axisLabel] at (#3,0.6) {#2};
}
\centeredTikzFigure[
]{
  %axis
  \draw[->] (-2.5,0) -- (8,0);

  % ticks
  \draw (0,0.1) -- (0,-0.1);
  \draw (3,0.1) -- (3,-0.1);
  \draw (4,0.1) -- (4,-0.1);

  % numbers
  \node at (0,0.6) {0};
  \axisNumber{2}{2}{2};
  \axisNumber{3.2}{3,2}{3.0};
  \axisNumber{3.7}{3,7}{3.9};
  \axisNumber{5}{5}{5};

  % variables
  \node at (2, -0.8) {$V_{min}$};
  \node at (5, -0.8) {$V_{max}$};
}

Lungimea fiecărui interval va fi deci de

\begin{equation}
  D = \frac{V_{max} - V_{min}}{N - 1} \qquad
  (\text{în cazul nostru } D = 1)
\end{equation}

De ce s-a făcut această împărțire? Dacă notăm cu $D_{max}$ distanța maximă pe
axă între două numere vecine, adică tocmai valoarea pe care o căutăm, se poate
demonstra că $D_{max} \geq D$. Într-adevăr, între cele $N$ numere de pe axă se
formează $N-1$ intervale. Dacă presupunem că $D_{max} < D$, rezultă că
distanța între oricare două numere consecutive de pe axă este mai mică decât
$D$. De aici deducem că distanța dintre primul și ultimul număr, adică
$V_{max} - V_{min}$, este mai mică decât $(N-1) \times D$. Dar aceasta duce
la relația:

\begin{equation}
  V_{max} - V_{min} < (N - 1) \cdot \frac{V_{max} - V_{min}}{N - 1}
  \implies
  V_{max} - V_{min} < V_{max} - V_{min}
\end{equation}

relație care este absurdă; demonstrația afirmației $D_{max} \geq D$ este
completă.

Următorul pas pe care îl avem de făcut este să parcurgem încă o dată vectorul
de numere și să aflăm pentru fiecare element căruia dintre intervalele de
lungime îi aparține. Și această operație se poate face în $O(N)$. Convenim ca
dacă un număr $X$ se află exact la limita dintre două intervale, adică

\begin{equation}
  X = V_{min} + k \cdot D, \quad 0 \leq k < N
\end{equation}

el să fie considerat ca aparținând intervalului din dreapta. Aceasta înseamnă
că elementul de valoare $V_{max}$ nu aparține nici unuia din cele $N-1$
intervale, ci celui de-al $N$-lea interval, $[V_{max}, V_{max} + D]$. Să vedem
la ce ne ajută acest lucru. Din moment ce $D_{max} \geq D$, rezultă că este
imposibil ca distanța maximă să se producă între două numere din același
interval, deoarece distanța în cadrul aceluiași interval nu poate atinge
valoarea $D$. Este deci obligatoriu ca distanța maximă să apară între două
elemente din intervale distincte. Să urmărim în figura următoare ce alte
proprietăți mai au aceste numere:

\newcommand\axisPair[3] {
  \draw[ultra thick] (#1,0.2) -- (#1,-0.2);
  \node[axisLabel] at (#1,0.6) {#2};
  \node[axisLabel] at (#1,-0.6) {#3};
}
\centeredTikzFigure[
]{
  %axis
  \draw[->] (-5,0) -- (9,0);

  % ticks
  \foreach \x in {0,...,4} {
    \draw (\x,0.1) -- (\x,-0.1);
  };

  % values
  \axisPair{-3}{}{$V_{min}$}
  \axisPair{0.25}{}{$X$}
  \axisPair{0.75}{?}{$X'$}
  \axisPair{3.25}{?}{$Y'$}
  \axisPair{3.75}{}{$Y$}
  \axisPair{7}{}{$V_{max}$}

  \node[axisLabel] at (-1.5, 0.3) {$\cdots$};
  \node[axisLabel] at (5.5, 0.3) {$\cdots$};
}

Dacă $X$ și $Y$ sunt valorile între care diferența este maximă, este de la
sine înțeles că între ele nu mai există nici un număr, deoarece se prespune că
$X$ și $Y$ sunt consecutive în vectorul sortat. Aceasta înseamnă însă că $X$
este cel mai mare număr din intervalul său, iar numărul $X'$ nu poate exista
acolo unde a fost el figurat. Analog, $Y$ este cel mai mic număr din
intervalul său, iar numărul $Y'$ nu poate exista. De fapt, în nici unul din
intervalele dintre cele care le cuprind pe $X$ și $Y$ nu poate exista nici un
număr.

Prin urmare, diferența maximă se poate produce numai între maximul unui
interval și minimul imediat următor. Următorul pas în găsirea soluției
presupune aflarea pentru fiecare din cele $N-1$ intervale (sau $N$ dacă îl
considerăm și pe ultimul, cel care nu îl conține decât pe $V_{max}$) a
minimului și a maximului. Și acest pas se execută în $O(N)$, deoarece
procesarea fiecărui element din vector se reduce la numai două comparări, cu
minimul și cu maximul intervalului în care se încadrează el. Vor rezulta doi
vectori care în program se vor numi $Lo$ și $Hi$. Iată care sunt valorile lor
pentru exemplul nostru:

\begin{table}[H]
  \centering
  \begin{tabular}{ccccc}
    {\bf Intervalul}: & [2, 3) & [3, 4) & [4, 5) & [5, 6) \\
            $Lo$: & 2 & 3,2 & $-$ & 5 \\
            $Hi$: & 2 & 3,7 & $-$ & 5 \\
  \end{tabular}
\end{table}

Deoarece în intervalul [4,5) nu se află elemente, rezultă că elementele
  corespunzătoare din vectorii $Lo$ și $Hi$ trebuie să aibă o valoare specială
  care să informeze programul asupra acestui lucru. De exemplu, sursa oferită
  mai jos folosește următorul artificiu: inițializează vectorul $Lo$ cu valori
  foarte mari ($V_{max} + 1$), astfel încât orice număr „repartizat” într-un
  interval să modifice această valoare. Similar, vectorul $Hi$ este
  inițializat cu $V_{min} - 1$. Dacă pentru un interval aceste valori se
  păstrează până la sfârșit, putem trage concluzia că în respectivul interval
  nu se află nici un număr.

În continuare, elementele vectorilor $Lo$ și $Hi$ se amestecă formând un nou
vector $W$ care de data aceasta este sortat. Sortarea este foarte ușoară,
pentru că nu avem decât să așezăm numerele în ordinea $Lo[1]$, $Hi[1]$,
$Lo[2]$, $Hi[2]$, $\dots$, $Lo[N]$, $Hi[N]$. Deși la prima vedere pare că noul
vector rezultat are $2N$ elemente, de fapt el are numai $N$ elemente, pentru
că:

\begin{itemize}

\item Dacă într-un interval $K$ există un singur număr, (cazul intervalelor
  [2,3) și [5,6)) sau există numai numere egale, atunci $Lo[K] = Hi[K]$ și
      este suficient să copiem în $W$ una singură dintre aceste două valori;

\item Dacă într-un interval nu există nici un număr, putem să nu copiem nici o
  valoare în vectorul $W$.

\end{itemize}

Astfel, construcția vectorului $W$ se poate face în timp liniar, mai exact în
$O(2N)$. Se observă că la această construcție se poate întâmpla ca unele
numere să „dispară”, adică să nu fie trecute în vectorul $W$. De exemplu, dacă
între numerele 3,2 și 3,7 ar mai fi existat un număr, 3,5, el nu ar fi fost
nici minim, nici maxim pentru intervalul său, deci nu ar fi fost
copiat. Totuși, trierea în acest fel a elementelor nu afectează în nici un fel
soluția. În cazul nostru, nu se întâmplă să dispară nici un element, deci $W =
(2, 3,2, 3,7, 5)$.

După ce am construit vectorul $W$, nu mai avem decât să-l parcurgem de la
stânga la dreapta și să tipărim diferența maximă întâlnită între două numere
consecutive (repetăm, vectorul $W$ este sortat), această ultimă etapă
necesitând și ea un timp liniar. Nu se poate obține o complexitate inferioară
celei liniare, întrucât citirea datelor presupune ea însăși $N$ operații.

Ca un detaliu de implementare, odată ce au fost construiți vectorii $Lo$ și
$Hi$, vectorul $V$ nu mai este necesar, deci putem construi chiar în el
vectorul $W$, pentru a economisi memorie.

\begin{minted}{c}
#include <stdio.h>
#define NMax 5000
float V[NMax+1], Lo[NMax+1], Hi[NMax+1];
float Delta, Max, Min;
int N;

void ReadData(void)
/* Citeste vectorul si afla maximul si minimul */
{ FILE *F=fopen("input.txt", "rt");
  int i;

  fscanf(F, "%d\n", &N);
  fscanf(F, "%f", &V[1]);
  Max=Min=V[1];

  for (i=2; i<=N; i++)
    {
      fscanf(F, "%f", &V[i]);
      if (V[i]>Max) Max=V[i];
        else if (V[i]<Min) Min=V[i];
    }
  fclose(F);
}

void Split(void)
{ int i, K;

  Delta = (Max-Min)/(N-1);
  /* Se initializeaza vectorii Lo si Hi */
  for (i=0; i<=N; Lo[i]=Max+1, Hi[i++]=Min-1);

  /* Se construiesc intervalele */
  for (i=1; i<=N; i++)
    {
      K = (V[i]-Min)/Delta;
      if (V[i]<Lo[K]) Lo[K]=V[i];
      if (V[i]>Hi[K]) Hi[K]=V[i];
    }
}

void Rebuild(void)
/* Rescrie vectorul V, pentru a economisi memorie,
   pastrand numai capetele intervalelor */
{ int i, M=0;

  for (i=0;i<N; i++)
    {
      if (Lo[i] != Max+1) V[++M]=Lo[i];
      if (Hi[i] != Min-1 && Hi[i] != Lo[i]) V[++M]=Hi[i];
    }
  N=M;
}

void FindGap(void)
/* Acum cautarea distantei maxime se face
   secvential, vectorul fiind sortat */
{ int i;
  float Gap=0;

  for (i=2; i<=N; i++)
    if (V[i]-V[i-1] > Gap)
      Gap = V[i]-V[i-1];
  printf("Distanta maxima este %0.3f\n", Gap);
}

void main(void)
{
  ReadData();
  if (Max==Min)
    puts("0");
  else
    {
      Split();
      Rebuild();
      FindGap();
    }
}
\end{minted}
