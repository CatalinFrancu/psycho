\chapter{Lucrul cu structuri mari de date}

De multe ori în practică, atât la concursuri cât și atunci când scriem
programe care vehiculează un volum mai mare de date, avem nevoie de structuri
de date de mari dimensiuni. După cum se știe, însă, compilatorul Borland
Pascal nu permite definirea de structuri de date mai mari de 64 KB. Ce facem
dacă, spre exemplu, avem nevoie de un vector cu sute de mii de elemente sau de
o matrice pătratică de $400 \times 400$ elemente ?

O primă soluție este să schimbăm limbajul de programare folosit și să ne
orientăm spre C / C++ sau alte limbaje în care gestiunea datelor voluminoase
se face mai comod pentru utilizator. De fapt, programele scrise „la domiciliu”
se scriu mai degrabă în C decât în Pascal, deoarece codul generat este mai
eficient. Din nefericire, compilatorul de C este destul de lent, cel puțin în
comparație cu cel de Pascal și, deși există destui concurenți care lucrează în
C la olimpiadă, Pascal-ul mi se pare o alegere mai bună atunci când timpul de
implementare contează decisiv.

În aceste condiții, se impune găsirea unor modalități de a ne supune rigorilor
limbajului Pascal și de a „înghesui” cumva datele în memorie. Chiar dacă
dispunem de memorie extinsă, segmentarea la 64 KB a datelor ridică destul de
multe probleme. Vom trata pe rând câteva cazuri.

\section{Vectori de tip boolean de mari dimensiuni}

În Borland Pascal, variabilele de tip logic ({\tt Boolean}) se reprezintă pe
un octet. Se știe însă că variabilele booleene pot lua doar două valori, deci
un singur bit ar fi suficient pentru a le reprezenta. Motivul acestei aparente
„risipe” de memorie este viteza de rulare a programului. Regiștrii lucrează la
nivel de octet, iar operațiile la nivel de bit sunt mai costisitoare din punct
de vedere al timpului. În plus, cei șapte biți care ar fi economisiți nu ar
putea fi folosiți decât cel mult pentru alte variabile booleene.

În unele cazuri, însă, comprimarea variabilelor logice la un singur bit este
necesară, această misiune revenindu-i programatorului. Iată un exemplu de
problemă de concurs:

{\bf ENUNȚ}: Se dau $N - 1$ numere naturale distincte cuprinse între 1 și
$N$. Să se tipărească cel de-al $N$-lea număr (cel care lipsește).

{\bf Intrarea} se face din fișierul {\tt INPUT.TXT}, care conține pe prima
linie valoarea lui $N$ ($N \leq 500.000$), iar pe următoarele $N - 1$ linii
câte un număr cuprins între 1 și $N$.

{\bf Ieșirea}: pe ecran se va tipări numărul care lipsește.

{\bf Exemplu}: Pentru fișierul de intrare:

\begin{verbatim}
  5
  3
  2
  5
  1
\end{verbatim}

rezultatul tipărit pe ecran va fi 4.

{\bf Complexitate cerută}: $O(N)$.

{\bf Timp de implementare}: 30 minute.

{\bf REZOLVARE}: O soluție foarte elegantă a problemei este următoarea: se
știe că suma primelor $N$ numere naturale este

\begin{equation}
  S_N = \frac{N(N + 1)}{2}
\end{equation}

Se calculează suma $S'$ a numerelor din fișierul de intrare, iar numărul lipsă
este $K = S_N - S'$. Această metodă are un dezavantaj care o face
inutilizabilă: $S_{500.000}$ este aproximativ $125 \cdot 10^9$, adică un număr
mult prea mare chiar și pentru tipul de date {\tt Longint}. Ar trebui să se
memoreze numerele foarte mari pe un vector de cifre, apoi ar trebui scrise
funcțiile pentru adunarea și scăderea unor asemenea numere (vezi capitolul al
II-lea), lucru destul de incomod dacă se ține seama de timpul de implementare.

Pentru a memora tot vectorul în memorie este nevoie de $500.000 \times 4 =
2.000.000$ octeți, adică foarte mult, iar găsirea valorii care lipsește ar
presupune în cel mai bun caz o sortare în $O(N \log N)$, care ar depăși
complexitatea cerută. Ar mai fi soluția de a căuta pe rând fiecare număr în
fișier și de a tipări primul număr pe care nu-l găsim. În cazul cel mai rău,
însă, algoritmul ar face $N$ parcurgeri ale fișierului de intrare și $N^2$
comparații, deci ar depăși de asemenea complexitatea cerută, ca să nu mai
vorbim de timpul de rulare.

Rezolvarea cea mai la îndemână este de a construi un vector de variabile
booleene cu $N$ elemente. Se face apoi citirea datelor și se bifează în vector
fiecare număr citit. Apoi se parcurge vectorul și se caută singurul element
nebifat. Această versiune face o singură parcurgere a fișierului, adică
minimul posibil.

Mai rămâne de văzut cum operăm cu un vector de 500.000 de variabile
booleene. Dacă fiecare variabilă ar ocupa un octet, necesarul de memorie ar fi
de 500.000 de octeți, care sunt greu de găsit în memoria de bază. Dacă însă
alocăm câte un bit pentru fiecare element, necesarul de memorie este 500.000 /
8 = 62.500 octeți, adică o sumă rezonabilă care, mai mult, încape într-un
singur segment de memorie și poate fi alocată static fără probleme. Cum se
poate accesa și modifica valoarea unui bit din acest vector comprimat? Vom
numerota vectorul nostru începând de la 0, deci ultimul său element va fi
$V[62.499]$. Primii săi opt biți (biții 0...7, adică octetul $V[0]$) vor fi
variabilele logice atașate numerelor 1...8, următorii opt biți (biții 8...15,
adică octetul $V[1]$) vor fi variabilele logice atașate numerelor 9...16, și
așa mai departe. Ultimii opt biți (biții 499.992...499.999, adică octetul
$V[62.499]$) vor fi variabilele logice atașate numerelor 499.993...500.000. În
general, bitul cu numărul $X$ indică dacă numărul $X + 1$ a fost găsit sau
nu. Iată cum ar putea arăta vectorul la un moment oarecare al citirii din
fișier:

\centeredTikzFigure[
  byte/.style = {matrix of nodes, ampersand replacement=\&, anchor=west, nodes=bit},
  bit/.style = {rectangle, draw, minimum width=1em, minimum height=1em},
  bitLabel/.style = {font=\scriptsize, anchor=north, yshift=-2.5em},
]{
  % bytes and \cdots
  \matrix[byte] (m1) {
    1 \& 0 \& 0 \& 0 \& 1 \& 0 \& 1 \& 0 \\
  };
  \node[anchor=west] (byteDots) at (m1.east) {$\cdots$};
  \matrix[byte] (m2) at (byteDots.east) {
    0 \& 1 \& 0 \& 1 \& 0 \& 0 \& 1 \& 1 \\
  };
  \matrix[byte] (m3) at ([xshift=1em]m2.east) {
    1 \& 0 \& 0 \& 1 \& 0 \& 0 \& 0 \& 1 \\
  };

  % byte labels
  \node[anchor=south] at (m1.north) {$V[62.499]$};
  \node[anchor=south] at (m2.north) {$V[1]$};
  \node[anchor=south] at (m3.north) {$V[0]$};

  % bit labels
  \node[bitLabel] (l1) at (m1-1-1.south) {bitul 499.999};
  \node[bitLabel] at (m1-1-4.south east) {$\cdots$};
  \node[bitLabel] (l2) at (m1-1-8.south) {bitul 499.992};

  \node[bitLabel] (l3) at (m2-1-1.south) {bitul 15};
  \node[bitLabel] at (m2-1-4.south east) {$\cdots$};
  \node[bitLabel] (l4) at (m2-1-8.south) {bitul 8};

  \node[bitLabel] (l5) at (m3-1-1.south) {bitul 7};
  \node[bitLabel] at (m3-1-4.south east) {$\cdots$};
  \node[bitLabel] (l6) at (m3-1-8.south) {bitul 0};

  % arrows
  \draw[->] (l1.north) -- (m1-1-1.south);
  \draw[->] (l2.north) -- (m1-1-8.south);
  \draw[->] (l3.north) -- (m2-1-1.south);
  \draw[->] (l4.north) -- (m2-1-8.south);
  \draw[->] (l5.north) -- (m3-1-1.south);
  \draw[->] (l6.north) -- (m3-1-8.south);
}

Vectorul este reprezentat „pe dos”, ceea ce ar putea să pară ciudat la prima
vedere. Am făcut însă acest lucru deoarece, în cadrul octetului, biții sunt
numerotați în ordine crescătoare de la dreapta spre stânga și am ținut să
păstrăm aceeași ordine și pentru numerotarea octeților în vector.

Primul octet semnifică că numărul 1 a fost întâlnit, numerele 2, 3 și 4 nu au
fost întâlnite încă, numărul 5 a fost găsit etc. Trebuie acum să vedem cum se
face efectiv modificarea vectorului. Inițial toți biții au valoarea 0. Se
citește din fișier un număr $X$ și trebuie ca al $X-1$-lea bit să fie setat pe
1.

Mai întâi trebuie aflat în ce octet se află al $X-1$-lea bit. Se observă
imediat că în octetul $Oct = (X-1) \bdiv 8$. De exemplu, biții 0..7 se află în
octetul 0, biții 7..15 în octetul 1 ș.a.m.d. Mai e nevoie să știm al câtulea
bit este bitul nostru în cadrul octetului. El va fi pe poziția $B = (X-1)
\bmod 8$ (numărătoarea începe de la 0). În sfârșit, trebuie să setăm bitul
respectiv la valoarea 1. În problema de mai sus, singura operație necesară
este modificarea unui bit din 0 în 1. Vom trata însă cazul cel mai general,
când se cere setarea unui bit la o anumită valoare (0 sau 1) fără a se ști ce
valoare a avut el înainte.

Să pornim de la un exemplu particular urmând ca apoi să generalizăm
rezultatul. Se cere să se seteze bitul 2 al octetului $A = 00110010$ la
valoarea 1. Pentru aceasta, putem folosi o mască logică în care numai bitul 2
este setat pe 1, iar ceilalți sunt 0, adică masca $M = 00000100$. Între
octeții $A$ și $M$ se poate face acum un SAU logic:

\begin{verbatim}
  A  00110010 SAU
  M  00000100
     --------
  B  00110110
\end{verbatim}

Se observă că $B$ = $A$ cu excepția bitului 2, care a luat valoarea 1. Acest
fapt este ușor de explicat: 0 este element neutru în raport cu operația SAU
($0\; \mathrm{SAU}\; X = X,\; \forall X$) și deci biții de valoare 0 din $M$
nu modifică biții corespunzători din $A$. În schimb, $1\; \mathrm{SAU}\; X =
1,\; \forall X$, deci bitul 2 din $B$ va lua valoarea 1 indiferent de valoarea
bitului corespunzător din $A$.

Revenind la cazul nostru, octetul {\tt Oct} are o valoare oarecare cuprinsă
între 0 și 255, iar noi trebuie să-i setăm bitul $B$ ($0 \leq B \leq 7$) la
valoarea 1. Masca $M$ va avea toți biții de valoare 0, cu excepția bitului $B$
care va avea valoarea 1. Aceasta înseamnă ca masca $M$ are valoarea $M = 2^B$
= {\tt 1 shl B}. Operația pe care o avem de făcut este:

\begin{minted}{pascal}
V[Oct] := V[Oct] or (1 shl B)
\end{minted}

Să luăm acum un exemplu pentru a vedea cum se setează un bit oarecare la
valoarea 0. Fie $A = 01101101$. Se cere să setăm bitul 5 la valoarea 0. De
data aceasta, vom folosi o mască în care toți biții sunt 1, mai puțin al 5-lea
și vom aplica operația ȘI logic:

\begin{verbatim}
  A 01101101 ȘI
  M 11011111
    --------
  B 01001101
\end{verbatim}

Se observă că $B = A$ cu excepția bitului 5, care a trecut de la valoarea 1 la
valoarea 0. Aceasta deoarece 1 este element neutru în raport cu operația ȘI
($1\; \mathrm{\text{ȘI}}\; X = X,\; \forall X$) și deci biții de valoare 1 din
$M$ nu modifică biții corespunzători din $A$. În schimb, $0\;
\mathrm{\text{ȘI}}\; X = 0,\; \forall X$, deci bitul 5 din $B$ va lua valoarea
0 indiferent de valoarea bitului corespunzător din $A$.

Să ne întoarcem la problema noastră: trebuie setat bitul $B$ din octetul {\tt
  Oct} la valoarea 0. Pentru a construi masca $M$ remarcăm că un octet cu toți
biții de valoare 1 este egal cu 255. Din 255 trebuie să scădem ({\tt 1 shl
  B}), deci operația necesară este:

\begin{minted}{pascal}
V[Oct] := V[Oct] and (255 - (1 shl Bit))
\end{minted}

Mai avem nevoie și să testăm valoarea unui bit. Să luăm de exemplu $A =
00101111$ și să aflăm ce valoare au biții 3 și 7. Folosim măștile $M_3 =
00001000$ și $M_7 = 10000000$, aplicând de fiecare dată operația ȘI logic.

\begin{verbatim}
  A  00101111 ȘI                A   00101111 ȘI
  M3 00001000                   M7  10000000
     --------                       --------
     00001000                       00000000  
\end{verbatim}

În ce caz rezultatul poate fi diferit de 0? Șapte dintre biții măștii sunt 0,
deci biții corespunzători din B vor fi oricum 0. Cel de-al optulea depinde de
bitul respectiv din $A$: dacă acesta este 0, rezultatul va fi 0, dacă este 1,
rezultatul va fi diferit de 0. Revenind la problemă, testul care trebuie făcut
este:

\begin{minted}{pascal}
V[Oct] and (1 shl Bit) <>  0
\end{minted}

Cel mai bine este să se creeze o „interfață” care să aibă implementate cele
două funcții (setarea și testarea unui bit). După aceasta nu se va mai accesa
direct vectorul, ci numai prin intermediul acestor funcții. În felul acesta,
dacă vor apărea erori de funcționare din cauza lucrului cu vectorul, vom ști
unde să le căutăm.

Inițializarea vectorului se poate face prin două metode: una, mai elegantă,
constă în folosirea funcției de atribuire pentru fiecare element al său în
parte. A doua, mai rapidă, constă în suprascrierea cu valoarea 0 a tuturor
celor 62.500 de elemente ale vectorului $V$, printr-un acces direct al
memoriei (procedura {\tt FillChar} din Pascal).

În program, pentru creșterea vitezei, s-au făcut următoarele modificări:

\begin{itemize}

\item $(X-1) \bdiv 8$ înseamnă $(X-1) \bdiv 2^3$, adică {\tt (X-1) shr 3};

\item $(X-1) \bmod 8$ reprezintă ultimii 3 biți din reprezentarea binară a
  lui $X - 1$, adică {\tt (X-1) and 7} (deoarece 7 în binar este 00000111).

\item {\tt 255 - (1 shl A) = 255 xor (1 shl A)}. Cu alte cuvinte, schimbarea
  unui singur bit din 1 în 0 se poate face atât prin scădere, cât și printr-un
  SAU exclusiv, a doua variantă fiind mai rapidă.

\end{itemize}

\inputminted{pascal}{src/chapter3-1.pas}

\section{Vectori de dimensiuni mari cu elemente de valori mici}

Metoda descrisă la punctul anterior se poate aplica și în alte două cazuri
particulare. Dacă elementele vectorului de care avem nevoie pot lua nu doar
două valori, ci patru valori (0, 1, 2, 3), atunci ele se pot reprezenta pe doi
biți. În concluzie, într-un octet putem „înghesui” patru elemente ale
vectorului, iar pe un segment de memorie se pot reprezenta peste 250.000
elemente. În cazul în care elementele vectorului iau valori între 0 și 15, ele
se reprezintă pe 4 biți (în limba engeză, grupul de 4 biți poartă un nume
special - {\it nibble}), adică două elemente pe un octet, iar pe un segment
încap peste 125.000 elemente. Vom trata numai al doilea caz, lăsându-l pe
primul ca temă.

Vom folosi de asemenea un vector $V$ cu 62.500 elemente numerotate cu începere
de la 0. Octetul 0 va fi folosit pentru a memora primele două elemente din
vectorul dat, octetul 1 va memora al treilea și al patrulea element
etc. Octetul 62.499 va memora elementele 124.999 și 125.000. Să vedem de
exemplu cum se memorează vectorul (5, 12, 4, 11, 0, 15).

\centeredTikzFigure[
  byte/.style = {matrix of nodes, ampersand replacement=\&, anchor=west, nodes=bit},
  bit/.style = {rectangle, draw, minimum width=1em, minimum height=1em},
  byteLabel/.style = {anchor=south, yshift=0.5em},
  accolade/.style = {decorate, decoration={brace, amplitude=0.5em, raise=0.8em, mirror}},
  nlabel/.style = {below=1.5em, font=\scriptsize},
]{
  % bytes
  \matrix[byte] (m1) {
    1 \& 1 \& 1 \& 1 \& 0 \& 0 \& 0 \& 0 \\
  };
  \matrix[byte] (m2) at (m1.east) {
    1 \& 0 \& 1 \& 1 \& 0 \& 1 \& 0 \& 0 \\
  };
  \matrix[byte] (m3) at (m2.east) {
    1 \& 1 \& 0 \& 0 \& 0 \& 1 \& 0 \& 1 \\
  };

  % byte labels
  \node[byteLabel] at (m1.north) {$V[2]$};
  \node[byteLabel] at (m2.north) {$V[1]$};
  \node[byteLabel] at (m3.north) {$V[0]$};

  % nibble dividers
  \draw[line width=1pt] (m1.north) -- (m1.south);
  \draw[line width=1pt] (m2.north) -- (m2.south);
  \draw[line width=1pt] (m3.north) -- (m3.south);

  % nibble labels and decorations
  \draw[accolade] (m1-1-1.south west) to node[nlabel] {nibble 5} (m1-1-4.south east);
  \draw[accolade] (m1-1-5.south west) to node[nlabel] {nibble 4} (m1-1-8.south east);
  \draw[accolade] (m2-1-1.south west) to node[nlabel] {nibble 3} (m2-1-4.south east);
  \draw[accolade] (m2-1-5.south west) to node[nlabel] {nibble 2} (m2-1-8.south east);
  \draw[accolade] (m3-1-1.south west) to node[nlabel] {nibble 1} (m3-1-4.south east);
  \draw[accolade] (m3-1-5.south west) to node[nlabel] {nibble 0} (m3-1-8.south east);
}

Avem nevoie de aceleași operații ca și în cazul precedent:

\begin{itemize}

\item Setarea valorii unui element, indiferent de valoarea sa precedentă;

\item Aflarea valorii unui element;

\item Inițializarea vectorului.

\end{itemize}

Pentru a seta valoarea elementului cu numărul $X$ ($1 \leq X \leq 125.000$),
trebuie alterat nibble-ul cu numărul $X-1$. În ce octet se află acest nibble?
În octetul $Oct = (X-1) \bdiv 2$. Ce poziție ocupă nibble-ul în cadrul
octetului? Poziția $Nib = (X-1) \bmod 2$. Să luăm acum un caz particular și să
vedem cum se face modificarea propriu-zisă, urmând ca apoi să generalizăm.

Se dă octetul $A = 01110010$ și se cere ca nibble-ul 1 (cel din stânga) să fie
setat la valoarea 13 (în binar 1101). Se observă că nibble-ul stâng are deja o
altă valoare, deci în primul rând trebuie „curățată zona”, respectiv nibble-ul
trebuie adus la valoarea 0. Cum? Probabil ați învățat deja, cu o mască
logică. Cum toți patru biții trebuie puși pe 0, iar ceilalți patru trebuie să
rămână nealterați, folosim masca $M_1 = 00001111$ și aplicăm operatorul ȘI
logic:

\begin{verbatim}
  A   01110010 ȘI
  M1  00001111
      --------
  A’  00000010.
\end{verbatim}

Așadar nibble-ul 0 a rămas nemodificat, iar nibble-ul 1 are valoarea 0. Acum
putem aduna pur și simplu nibble-ul de valoare 13. Pentru aceasta, luăm
numărul 13 (în binar 1101) și îl deplasăm spre stânga cu 4 poziții, pentru a-l
alinia în dreptul nibble-ului 1, după care efectuăm un SAU logic (se poate
face și adunarea, dar ea este mai lentă din punct de vedere al
calculatorului).

\begin{verbatim}
  A’  00000010 SAU
  M2  11010000
      --------
  B   11010010
\end{verbatim}

Să presupunem acum că voiam să setăm nibble-ul drept la aceeași valoare,
13. Ce aveam de făcut? „Curățam” jumătatea dreaptă a octetului printr-un ȘI
logic cu masca $M_1 = 11110000$, apoi făceam un SAU logic cu nibble-ul 13:

\begin{verbatim}
  A   01110010 ȘI
  M1  11110000
      --------
  A’  01110000 SAU
  M2  00001101
      --------
  B   01111101
\end{verbatim}

Metoda generală este deci următoarea. Se construiește masca $M_1$ cu care se
setează pe 0 nibble-ul dorit. Masca se obține scăzând din octetul „plin”
11111111 (zecimal 255, hexazecimal \$FF) valoarea 1111 (zecimal 15,
hexazecimal \$0F), deplasată la stânga cu 4 poziții dacă {\tt Nib = 1}. O
primă formă a instrucțiunii de construire a măștii ar fi:

\begin{minted}{pascal}
if Nib=1 then M1:=$FF - ($0F shl 4)
         else M1:=$FF - $0F
\end{minted}

Pentru a evita instrucțiunea {\tt if}, destul de mare consumatoare de timp, se
poate calcula direct

\begin{minted}{pascal}
M1:=$FF xor ($0F shl (Nib shl 2))
\end{minted}

deoarece {\tt Nib shl 2} este 0 dacă $Nib = 0$ și 4 dacă $Nib = 1$. S-a
înlocuit scăderea cu operația SAU exclusiv, pentru motivul arătat la punctul
1.

Apoi se adună, după aceeași metodă, valoarea $K$ dorită ($0 \leq K \leq 15$):

\begin{minted}{pascal}
V[Oct] = (V[Oct] and ($FF xor ($0F shl (Nib shl 2))))
         or (K shl (Nib shl 2))
\end{minted}

Să vedem cum se află valoarea unui nibble. Să presupunem că se dă octetul $A =
01111010$ și se cere să se afle nibble-ul 1 (cel din stânga). Pentru aceasta,
se face un ȘI logic cu masca $M = 11110000$ (deoarece ultimii patru biți nu
interesează), iar rezultatul se deplasează spre dreapta cu 4 biți:

\begin{verbatim}
  A   01111010 ȘI
  M   11110000
      --------
  A’  01110000 >>>>
      --------
  B   00000111
\end{verbatim}

Deci nibble-ul stâng are valoarea 7. Revenind la cazul nostru, valoarea $K$ a
nibble-ului $Nib$ din octetul $Oct$ este:

\begin{minted}{pascal}
K = (V[Oct] and ($0F shl (Nib shl 2))) shr (Nib shl 2)
\end{minted}

Se recomandă și în acest caz scrierea unor funcții și folosirea vectorului $V$
numai prin intermediul acestor funcții. În cazul inițializării vectorului,
dacă toate elementele trebuie puse la valoarea 0, se poate folosi totuși
procedura {\tt FillChar}, care e mult mai rapidă decât apelarea funcțiilor
noastre pentru fiecare element în parte.

Prezentăm mai jos numai un program demonstrativ despre modul de lucru cu
aceste funcții:

\inputminted{pascal}{src/chapter3-2.pas}

\section{Alocarea dinamică a matricelor de mari dimensiuni}

Să presupunem că avem nevoie de o matrice cu 400 linii și 400 coloane cu
elemente de tip {\tt Integer}. Necesarul de memorie este deci de $400 \times
400 \times 2 = 320.000$ octeți, adică mult mai mult decât un segment. O cale
foarte comodă de a rezolva această dificultate este de a declara matricea
drept un vector de pointeri la vectori, ca în figura de mai jos:

% Outputs a row. Overwrites object IDs.
\newcommand{\ptrToArray}[2]{
  \node[leader] (h1) at (#1) {#2};
  \node[cell] (c1) at (h1.east) {};
  \matrix[mat] (m1) at (c1.east) {
    \ \& \ \& \ \\
  };
  \draw[->] (c1.center) -- (m1-1-1.west);
  \node[anchor=west] (dots1) at (m1.east) {$\cdots$};
  \node[cell] (t1) at (dots1.east) {};
}
\centeredTikzFigure[
  leader/.style = {minimum width=4em, anchor=north, yshift=-1em},
  mat/.style = {matrix of nodes, ampersand replacement=\&, anchor=west, xshift=2em, nodes=cell},
  cell/.style = {rectangle, draw, minimum width=2em, minimum height=2em, anchor=west},
  cellLabel/.style = {font=\scriptsize, anchor=west},
]{
  % first row
  \node (placeholder) {};
  \ptrToArray{placeholder}{$A[1]$}

  % column headers
  \node[anchor=south] at (m1-1-1.north) {1};
  \node[anchor=south] at (m1-1-2.north) {2};
  \node[anchor=south] at (m1-1-3.north) {3};
  \node[anchor=south] at (t1.north) {400};

  % more rows
  \ptrToArray{h1.south}{$A[2]$}
  \ptrToArray{h1.south}{$A[3]$}

  % skip
  \node[leader] (h1) at (h1.south) {$\vdots$};

  % last row
  \ptrToArray{h1.south}{$A[400]$}
}

Dimensiunea vectorului de pointeri este de $400 \times 4 = 1.600$ octeți (un
pointer se reprezintă pe 4 octeți). Dimensiunea unei linii din matrice este de
$400 \times 2 = 800$ octeți. Așadar, fiecare structură de date încape pe un
segment. Vectorii sunt alocați dinamic la intrarea în program.

Metoda este comod de implementat (practic singura diferență este că elementul
de la coordonatele $(i,j)$ din matrice nu va mai fi adresat cu {\tt A[i,j]},
ci cu {\tt A[i]\^{}[j]}) și nu este consumatoare de timp (adresarea unui
element mai presupune, pe lângă cele două incrementări datorate indicilor, și
o indirectare datorată pointerului).

Iată un exemplu demonstrativ de folosire a acestei structuri de date, care
calculează suma numerelor naturale de la 1 la 100.

\inputminted{pascal}{src/chapter3-3.pas}

\section{Fragmentarea matricelor de mari dimensiuni}

O altă soluție pentru a reprezenta în memorie structuri de date care depășesc
un segment este de a le fragmenta în bucăți mai mici, fiecare din acestea
nedepășind un segment. Pentru a accesa un element oarecare al structurii,
depistăm întâi fragmentul din care el face parte, apoi îl localizăm în cadrul
fragmentului.

Să considerăm același exemplu al unei matrice $A$ de dimensiuni $400 \times
400$ cu elemente de tip {\tt Integer}. Spațiul necesar este de 320.000 octeți,
adică mai mult de 4 segmente de câte 64 KB și mai puțin de 5. Să spunem că am
dori să fragmentăm această matrice în 5 matrice $B[0], B[1], B[2], B[3],
B[4]$, fiecare având câte 400 linii și 80 de coloane (vom numerota liniile de
la 0 la 399 și coloanele de la 0 la 79).  Cele 5 matrice se vor aloca dinamic,
fiecare încăpând pe un segment (așadar vectorul $B$ este un vector de pointeri
la matrice).

Unde se va regăsi elementul $A[i,j]$ ? Primele 80 de coloane din matricea $A$
se vor afla în matricea $B[0]$, următoarele 80 de coloane din $A$ se vor afla
în matricea $B[1]$ ș.a.m.d. Coloana $j$ a matricei $A$ se va afla prin urmare
în matricea $B[j \bdiv 80]$. Linia $i$ va fi aceeași, iar coloana pe care se
află elementul $A[i,j]$ în cadrul matricei $B[j \bdiv 80]$ este $j \bmod 80$.

Acum se vede de ce, pentru a calcula mai rapid câtul și restul împărțirilor,
este bine ca numărul de coloane la care se face fragmentarea să fie o putere a
lui 2, astfel încât operațiile $\bdiv$ și $\bmod$ să se poată înlocui
printr-un {\tt shr} și un {\tt and}. Pentru problema noastră, cea mai
rezonabilă cifră este 64, ceea ce înseamnă că avem nevoie de $\lceil 400/64
\rceil = 7$ fragmente (prin rotunjire în sus a rezultatului). De fapt, prin
alocarea a 7 segmente (numerotate de la 0 la 6) se creează $7 \times 64 = 448$
de coloane, cu 48 mai mult decât era necesar. Se pierde deci o cantitate de
memorie de $48 \times 400 \times 2 = 38.400$ octeți. În cazul în care această
memorie nu este vitală, se poate face „risipă”, câștigându-se în schimb
viteză.

Iată cum ar arăta matricea fragmentată la 64 de coloane:

\centeredTikzFigure[
  % array style
  arrayCell/.style = {rectangle, draw, minimum width=2.5em, minimum height=2.5em},
  array/.style = {
    matrix of nodes,
    ampersand replacement=\&,
    nodes=arrayCell,
  },
  % matrix style
  cell/.style = {rectangle, draw, minimum width=2.5em, minimum height=2.5em, anchor=west},
  noBorder/.style = {nodes={draw=none}},
  tall/.style = {minimum height=4em},
  wide/.style = {nodes={minimum width=4em}},
  hspace1/.style = {nodes={minimum width=4em}},
  hspace2/.style = {nodes={minimum width=8em}},
  mat/.style = {
    matrix of nodes,
    ampersand replacement=\&,
    scale=0.5,
    every node/.style={scale=0.5},
    nodes=cell,
    row 1/.style=noBorder,
    row 5/.style={noBorder, tall},
    row 7/.style=noBorder,
    column 1/.style=noBorder,
    column 5/.style={noBorder, wide},
    column 7/.style={noBorder, hspace1},
    column 11/.style={noBorder, wide},
    column 13/.style={noBorder, hspace2},
    column 17/.style={noBorder, wide},
    column 20/.style={noBorder, wide},
  },
]{
  % array
  \matrix[array] (a) {
    $B[0]$ \& $B[1]$ \& $B[2]$ \& $B[3]$ \& $B[4]$ \& $B[5]$ \& $B[6]$ \\
  };

  % matrix -- all chunks in one matrix
  \matrix[mat, anchor=north] (m) at ([yshift=-3em]a.south) {
    \  \& 0 \& 1 \& 2 \& $\cdots$ \& 63 \& \  \&
    0 \& 1 \& 2 \& $\cdots$ \& 63 \& \  \&
    0 \& 1 \& 2 \& $\cdots$ \& 15 \& 16 \& $\cdots$ \& 63 \\

    0 \& \  \& \  \& \  \& $\cdots$ \& \  \& \  \&
    \  \& \  \& \  \& $\cdots$ \& \  \& \  \&
    \  \& \  \& \  \& $\cdots$ \& \  \& \  \& $\cdots$ \& \ \\

    1 \& \  \& \  \& \  \& $\cdots$ \& \  \& \  \&
    \  \& \  \& \  \& $\cdots$ \& \  \& \  \&
    \  \& \  \& \  \& $\cdots$ \& \  \& \  \& $\cdots$ \& \ \\

    2 \& \  \& \  \& \  \& $\cdots$ \& \  \& \  \&
    \  \& \  \& \  \& $\cdots$ \& \  \& \  \&
    \  \& \  \& \  \& $\cdots$ \& \  \& \  \& $\cdots$ \& \ \\

    $\vdots$ \& $\vdots$ \& $\vdots$ \& $\vdots$ \& \  \& $\vdots$ \& \  \&
    $\vdots$ \& $\vdots$ \& $\vdots$ \& \  \& $\vdots$ \& \  \&
    $\vdots$ \& $\vdots$ \& $\vdots$ \& \  \& $\vdots$ \& $\vdots$ \& \  \& $\vdots$ \\

    399 \& \  \& \  \& \  \& $\cdots$ \& \  \& \  \&
    \  \& \  \& \  \& $\cdots$ \& \  \& \  \&
    \  \& \  \& \  \& $\cdots$ \& \  \& \  \& $\cdots$ \& \ \\

    \  \& 0 \& 1 \& 2 \& $\cdots$ \& 63 \& \  \&
    64 \& 65 \& 66 \& $\cdots$ \& 127 \& \  \&
    384 \& 385 \& 386 \& $\cdots$ \& 399 \& 400 \& $\cdots$ \& 447 \\
  };

  % some matrix borders in the background
  \begin{scope}[on background layer]
    \draw[densely dotted, fill=gray!20] (m-1-19.north west) rectangle (m-7-21.south east);
    \draw (m-2-2.north west) rectangle (m-6-6.south east);
    \draw (m-2-8.north west) rectangle (m-6-12.south east);
    \draw (m-2-14.north west) rectangle (m-6-18.south east);
    \draw (m-2-19.north west) rectangle (m-6-21.south east);
  \end{scope}

  % arrows
  \draw[->] (a-1-1.south) -- (m-1-5.north west);
  \draw[->] (a-1-2.south) -- (m-1-11.north west);
  \draw[->] (a-1-7.south) -- (m-1-17.north west);
}

Prezentăm mai jos un scurt exemplu de fragmentare, care face același lucru ca
și programul de la punctul anterior. S-au scris, ca și în cazurile precedente,
două funcții, una pentru modificarea unui element și una pentru aflarea
valorii lui. De asemenea, {\tt X div 64} a fost înlocuit peste tot cu {\tt X
  shr 6}, iar {\tt X mod 64} cu {\tt X and 63}.

\inputminted{pascal}{src/chapter3-4.pas}
